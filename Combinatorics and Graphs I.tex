\documentclass[math]{amznotes}
\usepackage[utf8]{inputenc}
\usepackage{amsmath}
\usepackage{amsfonts}
\usepackage{graphicx}
\usepackage{tikz}
\usepackage{etoolbox}

\graphicspath{ {./images/} }
\geometry{
    a4paper,
    headheight = 1.5cm
}

\patchcmd{\chapter}{\thispagestyle{plain}}
{\thispagestyle{fancy}}{}{}

\theoremstyle{remark}
\newtheorem*{claim}{Claim}
\newtheorem*{remark}{Remark}
\newtheorem{case}{Case}

\newcommand{\map}[3]{#1: #2 \rightarrow #3} % Mapping
\newcommand{\image}[2]{#2\left[#1\right]} % Image
\newcommand{\preimage}[2]{#2\left[#1\right]^{-1}} % Pre-image
\newcommand{\eval}[3]{\left. #1\right\rvert_{#2 = #3}} % Evaluated at

\newenvironment{solution}
    {\let\oldqedsymbol=\qedsymbol
    \renewcommand{\qedsymbol}{\ }
    \begin{proof}[Solution]
    }
    {\end{proof}
    \renewcommand{\qedsymbol}{\oldqedsymbol}
    }

\begin{document}
\fancyhead[L]{
    Combinatorics and Graphs I 
}
\fancyhead[R]{
    Lecture Notes
}
\tableofcontents

\chapter{Permutations and Combinations}
\section{Basic Counting Principles}
An important motivation to study combinatorics is to count the \textbf{number of ways} in which an event may occur. Intuitively, we have two approaches to count.

The first approach is to categorise the event into \textbf{non-overlapping cases}. This means that we break an event into mutually exclusive sub-events, after which we can count the number of ways for each sub-event to occur. The agregate of these counts is the total number of ways for the original event to occur.

Those familiar with basic set theory may consider $E$ to be the set containing all distinct ways for an event to occur. By breaking up the event, we essentially establish a \textbf{partition} of~$E$, so that the sum of cardinalities of all the elements in that partition equals the cardinality of $E$.

This motivates us to write the following principle using set notations.
\begin{thmbox}{Addition Principle (AP)}{AP}
    Let $k \in \N^+$ and let $A_1, A_2, \cdots, A_k$ be $k$ finite sets which are pairwise disjoint, i.e. $A_i \cap A_j = \varnothing$ whenever~$i \neq j$, then
   \begin{equation*}
        \abs{\bigcup_{i = 1}^k A_i} = \sum_{i = 1}^{k}\abs{A_i}.
   \end{equation*} 
   \tcblower
   \begin{proof}
        The case where $k = 1$ is trivial.

        Suppose that when $k = n$, we have
        \begin{equation*}
            \abs{\bigcup_{i = 1}^n A_i} = \sum_{i = 1}^{n}\abs{A_i}
        \end{equation*} 
        for any $n$ finite sets which are pairwise disjoint. Let $A_{n + 1}$ be an arbitrary finite set which is disjoint with any of the $A_i$'s from the $n$ sets. So we have:
        \begin{align*}
            \abs{\bigcup_{i = 1}^{n + 1} A_i} & = \abs{\left(\bigcup_{i = 1}^n A_i\right) \cup A_{n + 1}} \\
            & = \abs{\bigcup_{i = 1}^n A_i} + \abs{A_{n + 1}} - \abs{\left(\bigcup_{i = 1}^n A_i\right) \cap A_{n + 1}} \\
            & = \left(\sum_{i = 1}^{n}\abs{A_i}\right) + \abs{A_{n + 1}} - \abs{\varnothing} \\
            & = \sum_{i = 1}^{n + 1}\abs{A_i}.
        \end{align*}
        Therefore, the original statement holds for all $k \in \N^+$.
   \end{proof}
\end{thmbox}
\begin{notebox}
    \begin{remark}
        In more casual language, this means that if an event $E_k$ has $n_k$ distinct ways to occur, then there is $\sum_{i = 1}^{k}n_k$ ways for at least one of the events $E_1, E_2, \cdots, E_k$ to occur, provided that $E_i$ and $E_j$ can never occur concurrently whenever $i \neq j$.
    \end{remark}
\end{notebox}
Given an event $E$, the other approach to count the number of ways for it to occur is to break~$E$ up internally into \textbf{non-overlapping stages}.

With set notations, we can write the $i$-th stage for $E$ to occur as $e_i$, and so a way for $E$ to occur can be represented by an ordered tuple $(e_1, e_2, \cdots, e_k)$, where $k$ is the total number of stages to undergo for $E$ to occur.

Let $E_i$ denote the set of all distinct ways to undergo the $i$-th stage of $E$, then it is easy to see that $E$ is just the \textbf{Cartesian product} of all the $E_i$'s. Hence, we derive the following principle:
\begin{thmbox}{Multiplication Principle (MP)}{MP}
    Let $k \in \N^+$ and let $A_1, A_2, \cdots, A_k$ be $k$ pairwise disjoint finite sets, then
    \begin{equation*}
        \abs{\prod_{i = 1}^{k}A_i} = \prod_{i = 1}^{k}\abs{A_i}.
    \end{equation*}
    \tcblower
    \begin{proof}
        The case where $k = 1$ is trivial.

        Suppose that when $k = n$, we have
        \begin{equation*}
            \abs{\prod_{i = 1}^{n}A_i} = \prod_{i = 1}^{n}\abs{A_i}
        \end{equation*} 
        for any $n$ finite sets which are pairwise disjoint. Let $A_{n + 1}$ be an arbitrary finite set which is disjoint with any of the $A_i$'s from the $n$ sets. Take $a_i, a_j \in A_{n + 1}$. Note that for all $\mathbfit{a} \in \prod_{i = 1}^{n}A_i$, $(\mathbfit{a}, a_i) \neq (\mathbfit{a}, a_j)$ whenever $a_i \neq a_j$. This means that
        \begin{align*}
            \abs{\prod_{i = 1}^{n + 1}A_i} & = \abs{\prod_{i = 1}^{n}A_i \times A_{n + 1}} \\
            & = \abs{\prod_{i = 1}^{n}A_i}\abs{A_{n + 1}} \\
            & = \left(\prod_{i = 1}^{n}\abs{A_i}\right)\abs{A_{n + 1}} \\
            & = \prod_{i = 1}^{n + 1}\abs{A_i}
        \end{align*}
        Therefore, the original statement holds for all $k \in \N^+$.
    \end{proof}
\end{thmbox}
\begin{notebox}
    \begin{remark}
        In more casual language, this means that if an event $E$ requires $k$ stages to be undergone before it occurs and the $i$-th stage has $n_i$ ways to complete, then there is $\prod_{i = 1}^{k}n_k$ ways for $E$ to occur, provided that no two different stages complete concurrently.
    \end{remark}
\end{notebox}
Often times, it is not straight-forward to count directly due to the presence of restrictions. We shall consider the following:

Let $E$ be the set of all possible ways for an event to occur. Let $p$ be some predicate representing some restriction and let $E(p)$ denote the set of all possible ways for the event to occur while $p$ holds. Note that:
\begin{displaymath}
    E(p) \cup E(\neg p) = E \quad \textrm{and} \quad E(p) \cap E(\neg p) = \varnothing,
\end{displaymath}
i.e. $\left\{E(p), E(\neg p)\right\}$ is a partition of $E$. Therefore, to count the number of ways for the event to occur while $p$ holds, it suffices to compute $E(\neg p)$, i.e. find the number of ways for the event to occur while $p$ does not hold.
\begin{thmbox}{Principle of Complementation}{PC}
    Let $U$ be a set and let $E \subseteq U$, then 
    \begin{equation*}
        \abs{E} = \abs{U} - \abs{U - E}.
    \end{equation*}
\end{thmbox}
\begin{notebox}
    \begin{remark}
        It may also help to think the Principle of Complementation as an inverse of the Addition Principle, where
        \begin{displaymath}
            E = \bigcup_{i = 1}^n E_i
        \end{displaymath}
        is the total number of ways for an event to occur and $E_p$ is the number of ways for the event to occur with restriction $p$.
    \end{remark}
\end{notebox}
In some cases, it is difficult to count the objects directly. However, note that the events for which we are counting are just sets, so we can treat ``number of occurrences'' of an event as the \textbf{cardinality} of a set. Therefore, we have the following principles:
\begin{thmbox}{Injection Principle}{injPrinciple}
    Let $A$, $B$ be finite sets. If there exists an injection
    \begin{displaymath}
        f \colon A \hookrightarrow B,
    \end{displaymath}
    then $\abs{A} \leq \abs{B}$.
\end{thmbox}
\begin{thmbox}{Bijection Principle}{bijPrinciple}
    Let $A$, $B$ be finite sets. If there exists a bijection
    \begin{displaymath}
        f \colon A \to B,
    \end{displaymath}
    then $\abs{A} = \abs{B}$.
\end{thmbox}
\section{Permutations}
A fundamental problem in combinatorics is described as follows: given a set $S$, how many ways are there to arrange $r$ elements in $S$, i.e. how many \textbf{distinct sequences} can be formed using the elements in $S$ without repetition? The process of selecting elements from $S$ and arranging them as a sequence is known as \textbf{permutation}.

Note that forming a sequence using $r$ elements from a set $S$ is an event consisting of $r$ stages, as we need to select an element for each of the $r$ terms of the sequence. Suppose $S$ has $n$ elements. For the first term of the sequence, we can choose any of the elements in $S$, so there is $n$ ways to do it. For the second term, since we cannot repeat the elements, we are left with $n - 1$ choices. 

Continue choosing elements in this way, we realise that if we choose the terms sequentially, when we reach the $k$-th term we will be left with $n - k + 1$ options as the previous~$(k - 1)$ terms have taken away $(k - 1)$ elements. By Theorem \ref{thm:MP}, we know that the number of sequences which can be formed is given by $\prod_{i = 1}^{r}(n - r + i)$.
\begin{dfnbox}{Permutations}{permutations}
    Let $A$ be a finite set such that $\abs{A} = n$, an $r$-permutation of $A$ is a way to arrange $r$ elements of $A$, denoted as $P^n_r$ and given by
    \begin{equation*}
        P^n_r = \prod_{i = 1}^{r}(n - r + i) = \frac{n!}{(n - r)!}.
    \end{equation*}
\end{dfnbox}
With some algebraic manipulations, it is easy to derive the following formula, which we, however, will prove in a combinatorial manner.
\begin{thmbox}{}{permutationId}
    Let $n, r \in \N$ with $r \leq n$, then $P^{n + 1}_r = P^n_r + rP^n_{r - 1}$.
    \tcblower
    \begin{proof}
        Let $S = \left\{x \in \N^+ \colon x \leq n + 1\right\}$ represent $(n + 1)$ distinct objects. Consider a permutation of $S$:
        \\\\
        If $n + 1$ is not inside the permutation, this is equivalent to an $r$-permutation of~$S - \{n + 1\}$, so there are $P^n_r$ such permutations.
        \\\\
    If $n + 1$ is inside the permutation, it means we need to first find an $(r - 1)$-permuation of $S - \{n + 1\}$, which has $P^n_{r - 1}$ ways to do. After that, we need to insert~$n + 1$ into each of these $(r - 1)$-permutations. Note that for each of such permutations, there are $r$ positions into which we can place $n + 1$. Therefore, the total number of $r$-permuations of $S$ derived in this manner is $rP^n_{r - 1}$.
        \\\\
        Therefore, there are $P^n_{r} + rP^n_{r - 1}$ $r$-permutations of $S$, i.e. $P^{n + 1}_r = P^n_r + rP^n_{r - 1}$. 
    \end{proof}
\end{thmbox}
\subsection{Circular Permutations}
Consider arranging $n$ distinct objects around a circle. If the slots around the circle are uniquely labelled, this is exactly the same as permutations along a straight line.

However, if the slots are identical, i.e. we are arranging $n$ distinct objects around a circle with identical slots, only the \textbf{relative positions} of the objects matter.

Let $\mathbfit{x}_i$ be an arbitrary straight-line permutations of the $n$ objects and let $\mathbfit{y}_i$ be the corresponding circular permutation of the $n$ objects.

Note that if we translate every element in $\mathbfit{x}_i$ by $k$ positions, this will result in a different straight-line permutation $\mathbfit{x}_j$ but does not change the corresponding circular permutation because the relative positions of the objects remain unchanged.

Notice that $k$ can take the values $0, 1, 2, \cdots, n - 1$, so for the same set of $n$ distinct objects, every circular permutation is mapped to $n$ straight-line permutations.
\begin{dfnbox}{Circular Permutations}{circPermutation}
    Let $A$ be a finite set such that $\abs{A} = n$, a circular $r$ permutation of $A$ is a way to arrange~$r$ elements of $A$ around a circular locus, denoted as $Q^n_r$ and given by
    \begin{equation*}
        Q^n_r = \frac{P^n_r}{r} = \frac{n!}{r(n - r)!}.
    \end{equation*}
\end{dfnbox}
\subsection{Permutations with Idential Objects}
\begin{thmbox}{Generalised Formula for Permutations}{genPnr}
    Let $k \in \N^+$ and let $A_1, A_2, \cdots, A_k$ be $k$ distinct objects, where $A_i$ occurs $n_i > 0$ times for~$i = 1, 2, \cdots, k$, then the number of permutations for these $k$ objects are given by
    \begin{equation*}
        \frac{\left(\sum_{i = 1}^{k}n_i\right)!}{\prod_{i = 1}^{k}\left(n_i!\right)}.
    \end{equation*}
\end{thmbox}
\section{Combinations}
Beside permutations, there are also occasions where we only care about which elements from a particular set are selected instead of the order of selection.

Note that if we want to find a selection of $r$ elements from a set $A$ where the order of selected elements does not matter, it is equivalent to finding a subset of $A$ containing $r$ elements. This motivates us to give the following definition:
\begin{dfnbox}{Combinations}{combinations}
    Let $A$ be a finite set such that $\abs{A} = n$, an $r$-combination of $A$ is a set $B \subseteq A$ with~$\abs{B} = r$. The number of combinations of $A$ is given by
    \begin{equation*}
        C^n_r = \frac{P^n_r}{P^r_r} = \frac{n!}{r!(n - r)!} = \begin{pmatrix}
            n \\
            r
        \end{pmatrix}.
    \end{equation*}
\end{dfnbox}
\begin{notebox}
    \begin{remark}
        Two obvious results:
        \begin{enumerate}
            \item If $r > n$ or $r < 0$, $C^n_r = 0$;
            \item $C^n_r = C^n_{n - r}$ (By Theorem \ref{thm:PC}).
        \end{enumerate}
    \end{remark}
\end{notebox}
Similar to permutations, we have the following important identity:
\begin{thmbox}{Pascal's Triangle}{pascalTri}
    Let $n$ be an integer with $n \geq 2$ and let $r$ be an integer with $0 \leq r \leq n$, then
    \begin{equation*}
        C^{n + 1}_r = C^{n}_{r - 1} + C^{n}_r.
    \end{equation*}
    \tcblower
    \begin{proof}
        Let $S = \left\{x \in \N^+ \colon x \leq n + 1\right\}$ represent $(n + 1)$ distinct objects. Consider an $r$-combination $T$ of $S$:
        \\\\
        If $n + 1 \notin T$, this is equivalent to an $r$-combination of $S - \{n + 1\}$, so there are $C^n_r$ such permutations.
        \\\\
        If $n + 1 \in T$, it suffices to find an $(r - 1)$-combination of $S - \{n + 1\}$, which has $C^n_{r - 1}$ ways to do.
        \\\\
        Therefore, there are $C^n_{r} + C^n_{r - 1}$ $r$-combinations of $S$, i.e. $C^{n + 1}_r = C^n_r + C^n_{r - 1}$. 
    \end{proof}
\end{thmbox}
\subsection{Counting Subsets}
A useful application of combinations, derived directly from the definition, is to count the number of subsets for a given set which is finite. In other words, given a set $A$ with $\abs{A} = n \in \N$, we wish to find a general formula for $\abs{\mathcal{P}(A)}$.

Let $A_i$ be the set of all subsets of $A$ whose cardinality is $i$, then clearly
\begin{equation*}
    \abs{\mathcal{P}(A)} = \sum_{i = 0}^{n}\abs{A_i} = \sum_{i = 0}^{n}C^n_i.
\end{equation*}
We can expand the above expression algebraically and realise that it simpifies to $2^n$. However, in a combinatorial perspective, we are able to prove this result in a more succint manner:
\begin{thmbox}{General Formula for $\abs{\mathcal{P}(A)}$}{cardPowerset}
    Let $A$ be a finite set. If $\abs{A} = n$, then $\abs{\mathcal{P}(A)} = 2^n$.
    \tcblower
    \begin{proof}
        Let $S$ be an arbitrary subset of $A$. Consider an arbitrary element $a \in A$, then either $a \in S$ or $a \notin S$. 
        \\\\
        Let $a_i \in A$ for $i = 1, 2, \cdots, n$. For all $S \in \mathcal{P}(A)$, We replace $a_i$ by $1$ if $a_i \in S$, and by $0$ otherwise. Let $B$ be the set of all binary sequences of length $n$. It is clear that there exists a bijection between $\mathcal{P}(A)$ and $B$, and so $\abs{\mathcal{P}(A)} = \abs{B}$.
        \\\\
        For each binary sequence of length $n$, each of its digits is either $0$ or $1$. By Theorem \ref{thm:MP}, this means that there are in total $2^n$ such binary sequences. Therefore,
        \begin{equation*}
            \abs{\mathcal{P}(A)} = \abs{B} = 2^n.
        \end{equation*}
    \end{proof}
\end{thmbox}

\section{Fun Problems Collection}
\begin{genbox}{No Consecutive}
    Let $X = \left\{x \in \N^+ \colon x \leq n\right\}$. Find the number of $r$-combinations of $X$ such that~$\abs{x_i - x_j} \geq 2$ whenever $i \neq j$.
    \tcblower   
    \begin{solution}
        Note that for each of the $r$-combinations of $X$, we can re-write it into a strictly increasing sequence $\{x_i\}_{i = 1}^r$ with $x_i \in X$ and $x_{i + 1} - x_i \geq 2$. Let the set of all such sequences be $S_X$. Consider the function
        \begin{displaymath}
            f \colon S_X \to Y
        \end{displaymath}
        where $f\left(\{x_i\}_{i = 1}^r\right) = \{x_i - i + 1\}_{i = 1}^r$ and $Y$ is the set of all strictly increasing sequences whose terms are integers bounded between $1$ and $n - r + 1$ inclusive.
        \\\\
        Take two different sequences $s_1, s_2 \in S_X$. Let the $k$-th terms of $s_1$ and $s_2$ be $k_1$ and $k_2$ respectively such that $k_1 \neq k_2$. Let $t_1 = f(s_1)$ and $t_2 = f(s_2)$, whose $k$-th terms are $h_1$ and $h_2$ respectively. Note that $h_1 = k_1 - k + 1 \neq k_2 - k + 1 = h_2$, so~$t_1 \neq t_2$, which means $f$ is injective.
        \\\\
        Take an arbitrary $\{y_i\}_{i = 1}^r \in Y$ and consider the sequence $\{y_i + i - 1\}_{i = 1}^r$. Note that~$1 \leq y_i \leq n - r + 1$, so $1 \leq y_i + i - 1 \leq n$ for $i = 1, 2, \cdots, r$, which means~$\{y_i + i - 1\}_{i = 1}^r \in S_X$. This means that for all $y \in Y$, there exists some $s \in S_X$ such that $f(s) = y$, so $f$ is surjective.
        \\\\
        Therefore, $f$ is a bijection and so
        \begin{equation*}
            \abs{S_X} = \abs{Y} = C^{n - r + 1}_r.
        \end{equation*}
    \end{solution}
\end{genbox}
\begin{genbox}{Top Secret}
    Six scientists are working on a secret project. They wish to lock up the documents in a cabinet so that the cabinet can be opened when and only when three or more of the scientists are present. Suppose that each lock has one and only one key with the locks being pairwise distinct.
    \begin{enumerate}
        \item What is the smallest number of locks needed?
        \item What is the smallest number of keys that each scientist needs to carry?
    \end{enumerate}
    \tcblower   
    \begin{solution}
        Let $S = \left\{s_1, s_2, s_3, s_4, s_5, s_6\right\}$ be the set of scientists and $L$ be the set of locks. Define $T$ to be the set of all subsets of $S$ with $2$ elements, then for any $P \in T$, $P$ is a pair of distinct scientists.
        \\\\
        Let $L_P \subseteq L$ be the set of locks which cannot be opened by the pair of scientists in $P$. Note that for all $P \in T$, $L_P \neq \varnothing$. 
        \\\\
        Consider $P, Q \in T$ with $P \neq Q$, then $\abs{P \cup Q} \geq 3$. Thus, $L_P \cap L_Q = \varnothing$. Now, define the function
        \begin{displaymath}
            f \colon T \to L
        \end{displaymath}
        such that $f(P) = \ell_P$ where $\ell_P$ is an arbitrary element from $L_P$ (note that we can do this without using Choice as each of the $L_P$'s is finite).
        \\\\
        Suppose that there exist $P, Q \in T$ with $P \neq Q$ such that $f(P) = f(Q) = \ell_0$. Notice that this would mean that $\ell_0 \in L_P \cap L_Q$, which is a contradiction. Therefore,~$f$ must be injective. Hence, 
        \begin{equation*}
            \abs{L} \geq \abs{T} = C^6_2 = 15.
        \end{equation*}
        Now, choose an arbitrary element from $S$, say $s_k$, and consider the set $S' = S - \{s_k\}$. Let $T'$ be the set of all subsets of $S'$ with $2$ elements, then for any $P \in T'$, $L_{P \cup \{s_k\}} = \varnothing$, i.e., $s_k$ must have the keys to open all locks which the pair $P$ cannot open with their own keys. Let this set of keys be $K_P$. Note that $L_P \cap L_Q = \varnothing$ whenever $P \neq Q$, so~$K_P \cap K_Q = \varnothing$ whenever $P \neq Q$. Let the set of all keys held by $s_k$ be $H_k$, then
        \begin{equation*}
            \abs{H_k} \geq \abs{S'} = C^5_2 = 10.
        \end{equation*}
    \end{solution}
\end{genbox}

\section{Distribution Problems}
Another problem in which we are interested is to \textbf{distribute} the objects from a collection into a finite number of sub-collections. Specifically, we consider the number of ways to
\begin{enumerate}
    \item distribute $r$ distinct objects into $n$ identical collections;
    \item distribute $r$ distinct objects into $n$ distinct collections;
    \item distribute $r$ identical objects into $n$ identical collections;
    \item distribute $r$ identical objects into $n$ distinct collections.
\end{enumerate}
These $4$ basic models of distribution problems give rise to many variations. One important tool to solve these problems is the \textit{Stirling Numbers}.
\subsection{Stirling Numbers}
\begin{dfnbox}{Stirling Numbers of the First Kind}{stirling1}
    Let $r, n \in \N$ such that $0 \leq n \leq r$, then the {\color{red} \textbf{Stirling Number of the First Kind}}, $s(r, n)$, is the number of ways to arrange $r$ {\color{red} \textbf{distinct}} onjects around $n$ {\color{red} \textbf{identical}} circles such that no circle is empty.
\end{dfnbox}
\begin{notebox}
    \begin{remark}
        Some obvious results:
        \begin{enumerate}
            \item $s(r, 0) = 0$ if $r \geq 1$.
            \item $s(r, r) = 1$ if $r \geq 0$.
            \item $s(r, 1) = (r - 1)!$ if $r \geq 2$.
            \item $s(r, r - 1) = C^r_2$ if $r \geq 2$.
        \end{enumerate}
    \end{remark}
\end{notebox}
One thing to take note of is that there is no general algebraic formula for $s(r, n)$. Instead, all Stirling Numbers of the First Kind follow a recurrence relation in $2$ parameters.
\begin{thmbox}{A Recurrence Relation for $s(r, n)$}{recurrence1}
    For $r, n \in \N^+$ and $r \leq n$, we have
    \begin{equation*}
        s(r, n) = s(r - 1, n - 1) + (r - 1)s(r - 1, n).
    \end{equation*}
    \tcblower   
    \begin{proof}
        Consider the set
        \begin{displaymath}
            \left\{x_1, x_2, \cdots, x_r\right\}
        \end{displaymath}
        to be the $r$ distinct objects. We consider two cases.
        \\\\
        If $x_r$ is distributed to a circle such that it is the only objects around that circle, it suffices to find the number of ways to arrange the rest $(r - 1)$ distinct objects around the rest $(n - 1)$ identical circles, which there are $s(r - 1, n - 1)$ ways to do.
        \\\\
        If $x_r$ is adjacent to some other object, we can first arrange the rest $(r - 1)$ distinct objects around the $n$ identical circles, which there are $s(r - 1, n)$ ways to do. After that, we can choose any one of the $(r - 1)$ spaces between two adjacent objects to slot in $x_r$. Therefore, there are $(r - 1)s(r - 1, n)$ ways to distribute the objects.
        \\\\
        By Theorem \ref{thm:AP}, the total number of distributions is 
        \begin{equation*}
            s(r, n) = s(r - 1, n - 1) + (r - 1)s(r - 1, n).
        \end{equation*}
    \end{proof}
\end{thmbox}
\end{document}