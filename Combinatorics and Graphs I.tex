\documentclass[math]{amznotes}
\usepackage[utf8]{inputenc}
\usepackage{amsmath}
\usepackage{amsfonts}
\usepackage{graphicx}
\usepackage{tikz}
\usepackage{etoolbox}

\graphicspath{ {./images/} }
\geometry{
    a4paper,
    headheight = 1.5cm
}

\patchcmd{\chapter}{\thispagestyle{plain}}
{\thispagestyle{fancy}}{}{}

\theoremstyle{remark}
\newtheorem*{claim}{Claim}
\newtheorem*{remark}{Remark}
\newtheorem{case}{Case}

\newcommand{\map}[3]{#1: #2 \rightarrow #3} % Mapping
\newcommand{\image}[2]{#2\left[#1\right]} % Image
\newcommand{\preimage}[2]{#2\left[#1\right]^{-1}} % Pre-image
\newcommand{\eval}[3]{\left. #1\right\rvert_{#2 = #3}} % Evaluated at

\begin{document}
\fancyhead[L]{
    Combinatorics and Graphs I 
}
\fancyhead[R]{
    Lecture Notes
}
\tableofcontents

\chapter{Permutations and Combinations}
\section{Basic Counting Principles}
\begin{thmbox}{Addition Priciple}{AP}
   Let $A_1, A_2, \cdots, A_k$ be $k$ pairwise disjoint finite sets, then
   \begin{equation*}
        \left|\bigcup_{i = 1}^k A_i\right| = \sum_{i = 1}^{k}\left|{A_i}\right|.
   \end{equation*} 
\end{thmbox}
\begin{thmbox}{Multiplication Priciple}{MP}
    Let $A_1, A_2, \cdots, A_k$ be $k$ pairwise disjoint finite sets, then
    \begin{equation*}
        \left|\prod_{i = 1}^{k}A_i\right| = \prod_{i = 1}^{k}\left|A_i\right|.
    \end{equation*}
\end{thmbox}
\chapter{Permutations}
\begin{dfnbox}{Permutations}{permutations}
    Let $A$ be a finite set such that $\left|A\right| = n$, an $r$-permutation of $A$ is a way to arrange $r$ elements of $A$, denoted as
    \begin{equation*}
        P^n_r = \prod_{i = 1}^{r}(n - r + i) = \frac{n!}{(n - r)!}.
    \end{equation*}
\end{dfnbox}
\end{document}