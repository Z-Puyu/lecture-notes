\documentclass[math]{amznotes}
\usepackage[utf8]{inputenc}
\usepackage{amsmath}
\usepackage{amsfonts}
\usepackage{graphicx}
\usepackage{tikz}
\usepackage{etoolbox}

\graphicspath{ {./images/} }
\geometry{
    a4paper,
    headheight = 1.5cm
}

\patchcmd{\chapter}{\thispagestyle{plain}}
{\thispagestyle{fancy}}{}{}

\theoremstyle{remark}
\newtheorem*{claim}{Claim}
\newtheorem*{remark}{Remark}
\newtheorem{case}{Case}
\begin{document}
\fancyhead[L]{
    MA2116 Probability
}
\fancyhead[R]{
    Lecture Notes
}
\tableofcontents

\chapter{How to Count}
\section{Basic Counting Principles}
An important motivation to study combinatorics is to count the \textbf{number of ways} in which an event may occur. Intuitively, we have two approaches to count.

The first approach is to categorise the event into \textbf{non-overlapping cases}. This means that we break an event into mutually exclusive sub-events, after which we can count the number of ways for each sub-event to occur. The agregate of these counts is the total number of ways for the original event to occur.

Those familiar with basic set theory may consider $E$ to be the set containing all distinct ways for an event to occur. By breaking up the event, we essentially establish a \textbf{partition} of~$E$, so that the sum of cardinalities of all the elements in that partition equals the cardinality of $E$.

This motivates us to write the following principle using set notations.
\begin{thmbox}{Addition Principle (AP)}{AP}
    Let $k \in \N^+$ and let $A_1, A_2, \cdots, A_k$ be $k$ finite sets which are pairwise disjoint, i.e. for all~$i, j$ such that $1 \leq i, j \leq k$, $A_i \cap A_j = \varnothing$ whenever~$i \neq j$, then
   \begin{equation*}
        \abs{\bigcup_{i = 1}^k A_i} = \sum_{i = 1}^{k}\abs{A_i}.
   \end{equation*} 
   \tcblower
   \begin{proof}
        The case where $k = 1$ is trivial.

        Suppose that when $k = n$, we have
        \begin{equation*}
            \abs{\bigcup_{i = 1}^n A_i} = \sum_{i = 1}^{n}\abs{A_i}
        \end{equation*} 
        for any $n$ finite sets which are pairwise disjoint. Let $A_{n + 1}$ be an arbitrary finite set which is disjoint with any of the $A_i$'s from the $n$ sets. So we have:
        \begin{align*}
            \abs{\bigcup_{i = 1}^{n + 1} A_i} & = \abs{\left(\bigcup_{i = 1}^n A_i\right) \cup A_{n + 1}} \\
            & = \abs{\bigcup_{i = 1}^n A_i} + \abs{A_{n + 1}} - \abs{\left(\bigcup_{i = 1}^n A_i\right) \cap A_{n + 1}} \\
            & = \left(\sum_{i = 1}^{n}\abs{A_i}\right) + \abs{A_{n + 1}} - \abs{\varnothing} \\
            & = \sum_{i = 1}^{n + 1}\abs{A_i}.
        \end{align*}
        Therefore, the original statement holds for all $k \in \N^+$.
   \end{proof}
\end{thmbox}
\begin{notebox}
    In more casual language, this means that if an event $E_k$ has $n_k$ distinct ways to occur, then there is $\sum_{i = 1}^{k}n_k$ ways for at least one of the events $E_1, E_2, \cdots, E_k$ to occur, provided that $E_i$ and $E_j$ can never occur concurrently whenever $i \neq j$.
\end{notebox}
Given an event $E$, the other approach to count the number of ways for it to occur is to break~$E$ up internally into \textbf{non-overlapping stages}.

With set notations, we can write the $i$-th stage for $E$ to occur as $e_i$, and so a way for $E$ to occur can be represented by an ordered tuple $(e_1, e_2, \cdots, e_k)$, where $k$ is the total number of stages to undergo for $E$ to occur.

Let $E_i$ denote the set of all distinct ways to undergo the $i$-th stage of $E$, then it is easy to see that $E$ is just the \textbf{Cartesian product} of all the $E_i$'s. Hence, we derive the following principle:
\begin{thmbox}{Multiplication Principle (MP)}{MP}
    Let $k \in \N^+$ and let $A_1, A_2, \cdots, A_k$ be $k$ pairwise disjoint finite sets, then
    \begin{equation*}
        \abs{\prod_{i = 1}^{k}A_i} = \prod_{i = 1}^{k}\abs{A_i}.
    \end{equation*}
    \tcblower
    \begin{proof}
        The case where $k = 1$ is trivial.

        Suppose that when $k = n$, we have
        \begin{equation*}
            \abs{\prod_{i = 1}^{n}A_i} = \prod_{i = 1}^{n}\abs{A_i}
        \end{equation*} 
        for any $n$ finite sets which are pairwise disjoint. Let $A_{n + 1}$ be an arbitrary finite set which is disjoint with any of the $A_i$'s from the $n$ sets. Take $a_i, a_j \in A_{n + 1}$. Note that for all $\mathbfit{a} \in \prod_{i = 1}^{n}A_i$, $(\mathbfit{a}, a_i) \neq (\mathbfit{a}, a_j)$ whenever $a_i \neq a_j$. This means that
        \begin{align*}
            \abs{\prod_{i = 1}^{n + 1}A_i} & = \abs{\prod_{i = 1}^{n}A_i \times A_{n + 1}} \\
            & = \abs{\prod_{i = 1}^{n}A_i}\abs{A_{n + 1}} \\
            & = \left(\prod_{i = 1}^{n}\abs{A_i}\right)\abs{A_{n + 1}} \\
            & = \prod_{i = 1}^{n + 1}\abs{A_i}
        \end{align*}
        Therefore, the original statement holds for all $k \in \N^+$.
    \end{proof}
\end{thmbox}
\begin{notebox}
    In more casual language, this means that if an event $E$ requires $k$ stages to be undergone before it occurs and the $i$-th stage has $n_i$ ways to complete, then there is $\prod_{i = 1}^{k}n_k$ ways for $E$ to occur, provided that no two different stages complete concurrently.
\end{notebox}
\section{Permutations}
A fundamental problem in combinatorics is described as follows: given a set $S$, how many ways are there to arrange $r$ elements in $S$, i.e. how many \textbf{distinct sequences} can be formed using the elements in $S$ without repetition? The process of selecting elements from $S$ and arranging them as a sequence is known as \textit{permutation}.

Note that forming a sequence using $r$ elements from a set $S$ is an event consisting of $r$ stages, as we need to select an element for each of the $r$ terms of the sequence. Suppose $S$ has $n$ elements. For the first term of the sequence, we can choose any of the elements in $S$, so there is $n$ ways to do it. For the second term, since we cannot repeat the elements, we are left with $(n - 1)$ choices. 

Continue choosing elements in this way, we realise that if we choose the terms sequentially, when we reach the $k$-th term we will be left with $n - k + 1$ options as the previous~$(k - 1)$ terms have taken away $(k - 1)$ elements. By Theorem \ref{thm:MP}, we know that the number of sequences which can be formed is given by $\prod_{i = 1}^{r}(n - r + i)$.
\begin{dfnbox}{Permutations}{permutations}
    Let $A$ be a finite set such that $\abs{A} = n$, an $r$-permutation of $A$ is a way to arrange $r$ elements of $A$, denoted as $P^n_r$ and given by
    \begin{equation*}
        P^n_r = \prod_{i = 1}^{r}(n - r + i) = \frac{n!}{(n - r)!}.
    \end{equation*}
\end{dfnbox}
\subsection{Permutations with Idential Objects}
\begin{thmbox}{Generalised Formula for Permutations}{}
    Let $k \in \N^+$ and let $A_1, A_2, \cdots, A_k$ be $k$ distinct objects, where $A_i$ occurs $n_i > 0$ times for~$i = 1, 2, \cdots, k$, then the number of permutations for these $k$ objects are given by
    \begin{equation*}
        \frac{\left(\sum_{i = 1}^{k}n_i\right)!}{\prod_{i = 1}^{k}\left(n_i!\right)}.
    \end{equation*}
\end{thmbox}
\section{Combinations}
\begin{dfnbox}{Combinations}{combinations}
    Let $A$ be a finite set such that $\abs{A} = n$, an $r$-combination of $A$ is a way to choose $r$ elements from $A$ regardless of the order of selection, denoted as $C^n_r$ and given by
    \begin{equation*}
        C^n_r = \frac{P^n_r}{P^r_r} = \frac{n!}{r!(n - r)!} = \begin{pmatrix}
            n \\
            r
        \end{pmatrix}.
    \end{equation*}
\end{dfnbox}
\begin{notebox}
    \begin{remark}
        Two obvious results:
        \begin{enumerate}
            \item If $r > n$ or $r < 0$, $C^n_r = 0$;
            \item $C^n_r = C^n_{n - r}$.
        \end{enumerate}
    \end{remark}
\end{notebox}
\begin{thmbox}{Pascal's Triangle}{pascalTri}
    Let $n$ be an integer with $n \geq 2$ and let $r$ be an integer with $0 \leq r \leq n$, then
    \begin{equation*}
        C^n_r = C^{n - 1}_{r - 1} + C^{n - 1}_r.
    \end{equation*}
\end{thmbox}

\section{Binomial and Multinomial Coefficients}
Consider the expansion of $(x + y)^n$ where $n \in \N$. Note that this expansion is a linear combination of terms in the form of $x^ky^{n - k}$ where $k = 0, 1, 2, \cdots, n$. 

Thus, fix any $k$, to determine how many copies of $x^ky^{n - k}$ there are, it suffices to compute~$C^n_k$. Therefore, in the expanded form of $(x + y)^n$, the coefficient is exactly $C^n_r$.
\begin{thmbox}{Binomial Expansion}{binomialEx}
    Let $n \in \N$, then
    \begin{equation*}
        (x + y)^n = \sum_{k = 0}^n \left[\begin{pmatrix}
            n \\
            k
        \end{pmatrix}x^ky^{n - k}\right].
    \end{equation*}
\end{thmbox}
We can extend the idea of binomial coefficients onto multinomial expansions, i.e. expressions in the form of $\left(\sum_{i = 1}^{r}x_i\right)^n$.

Note that the binomial coefficient $C^n_r$ is essentially equivalent to dividing $n$ distinct elements into two groups with $r$ and $(n - r)$ members respectively. Now we consider dividing~$n$ distinct elements into $r$ groups with $n_1, n_2, \cdots, n_r$ members respectively for each group.

Notice that we can simply permute the $n$ distinct elements and assign them sequentially into the $r$ groups, i.e. the first $n_1$ elements will go into the first group and so on.

Since the order of elements within each group does not matter, we need to remove repeated selections by dividing by $\prod_{i = 1}^{r}(n_i!)$. So we have the following definition:
\begin{dfnbox}{Multinomial Coefficients}{multinomialCoeff}
    The {\color{red} \textbf{multinomial coefficient}} is defined by
    \begin{equation*}
        \begin{pmatrix}
            n \\
            n_1, n_2, \cdots, n_k
        \end{pmatrix} = \frac{n!}{\prod_{i = 1}^{k}\left(n_i!\right)}
    \end{equation*}
\end{dfnbox}
\begin{thmbox}{Multinomial Expansion}{multinomialEx}
    Let $n \in \N$, then 
    \begin{equation*}
        \left(\sum_{i = 1}^{r}x_i\right)^n = \sum_{
            \substack{
                n_1, n_2, \cdots, n_r \in \N \\
                \sum_{j = 1}^rn_j = n
            }
        } \left[\begin{pmatrix}
            n \\
            n_1, n_2, \cdots, n_r
        \end{pmatrix}\prod_{i = 1}^{r}x_i^{n_i}\right]
    \end{equation*}
\end{thmbox}

\chapter{Axioms of Probability}
\section{Sample Space and Events}
\begin{dfnbox}{Sample Space}{sampleSpace}
    Consider an experiment whose outcome is {\color{red} \textbf{not}} predictable, then the set of all possible outcomes of the experiment is called the {\color{red} \textbf{sample space}} of the experiment, denoted by $S$.
\end{dfnbox}
\begin{notebox}
    \begin{remark}
        Note that $S \neq \varnothing$.
    \end{remark}
\end{notebox}
\begin{dfnbox}{Events}{event}
    Let $S$ be a sample space, a set $E \subseteq S$ is known as an {\color{red} \textbf{event}}.
\end{dfnbox}
\begin{notebox}
    \begin{remark}
        $S$ itself is known as the {\color{red} \textbf{sure event}} and $\varnothing$ is known as the {\color{red} \textbf{null event}}.
    \end{remark}
\end{notebox}
Note that since sample spaces and events are sets, we can apply operations onto events precisely in the same way for sets.

By convention, the intersection of two events $E$ and $F$ is preferably written as $EF$. Two events which are disjoint are called \textit{mutually exclusive}.

\section{Probability}
\begin{dfnbox}{Probability}{probability}
    Let $E$ be any event of an experiment and let $n(E)$ be the number of occurrences of $E$ in the first $n$ repetitions of the experiment, then the {\color{red} \textbf{probability}} of $E$ is
    \begin{equation*}
        P(E) = \lim_{n \to \infty}\frac{n(E)}{n},
    \end{equation*}
    if the limit exists.
\end{dfnbox}
However, notice that from the above, the notion of probability may not be well-defined as~$n(E)$ is not a function, which means that the limit is not defined.

To avoid this problem, we shall use an axiomatic definition instead, i.e., we define probability to be such that if it exists and is well-defined, then it satisfies a series of axioms. 
\begin{dfnbox}{Axioms of Probability}{probAxi}
    Let $S$ be a sample space and let $P(E)$ be a real number defined for every $E \subseteq S$. If
    \begin{itemize}
        \item $0 \leq P(E) \leq 1$,
        \item $P(S) = 1$, and
        \item for all mutually exclusive $E$ and $F$, $P(E \cup F) = P(E) + P(F)$,
    \end{itemize}
    then $P(E)$ is the {\color{red} \textbf{probability}} of $E$.
\end{dfnbox}
With induction, one can easily show that if $E_1, E_2, \cdots$ to be any sequence of events in a sample space $S$, then
\begin{equation*}
    P\left(\bigcup_{i = 1}^\infty E_i\right) = \sum_{i = 1}^{\infty}P(E_i).
\end{equation*}
We now follow up with proofs for two seemingly intuitive results.
\begin{thmbox}{The Null Event}{null}
    Consider the null event $\varnothing$, we have
    \begin{equation*}
        P(\varnothing) = 0.
    \end{equation*}
    \tcblower   
    \begin{proof}
        Let $S$ be a sample space and let $E_1, E_2, \cdots$ be a countably infinite sequence of events such that $E_i = \varnothing$ for all $i \in \N^+$. We can write
        \begin{equation*}
            P\left(\bigcup_{i = 1}^\infty E_i\right) = \sum_{i = 1}^{\infty}P(E_i).
        \end{equation*}
        Note that the countable union of empty sets is empty, so the above is equivalent to
        \begin{equation*}
            P\left(\bigcup_{i = 1}^\infty \varnothing\right) = P(\varnothing) = \sum_{i = 1}^{\infty}P(\varnothing).
        \end{equation*}
        This means that $P(\varnothing)$ equals the sum of a countably infinite sequence of itself, so 
        \begin{equation*}
            P(\varnothing) = 0.
        \end{equation*}
    \end{proof} 
\end{thmbox}
\begin{thmbox}{Monotonity of Probability}{monotone}
    Let $E$ and $F$ be events such that $E \subseteq F$, then 
    \begin{equation*}
        P(F) \geq P(E).
    \end{equation*}
    \tcblower   
    \begin{proof}
        Note that $E$ and $F - E$ are mutually exclusive, so 
        \begin{equation*}
            P(F) = P(E \cup (F - E)) = P(E) + P(F - E).
        \end{equation*}
        Note that $P(F - E) \geq 0$, so $P(E) + P(F - E) \geq P(E)$, which means
        \begin{equation*}
            P(F) \geq P(E).
        \end{equation*}
    \end{proof}
\end{thmbox}

\section{Inclusion-Exclusion Principle}
It is easy to compute the probability of a countable union of mutually exclusive events. However, it may get tricky when an event is the union of events which are not mutually exclusive. Intuitively, we can sum up the probabilities of all individual events and subtract the portions which are double-counted. This approach is rigorously summarised as follows:
\begin{thmbox}{Inclusion-Exclusion Principle}{in-exPrinciple}
    Let $S$ be a sample space and let $E_1, E_2, \cdots, E_n$ be a sequence of events. In general, we have
    \begin{equation*}
        P\left(\bigcup_{i = 1}^n E_i\right) = \sum_{j = 1}^{n} \left[(-1)^{j + 1}\left(\sum_{k_1 \leq k_2 \leq \cdots \leq k_j}P\left(\bigcap_{h = 1}^{j}E_{k_h}\right)\right)\right].
    \end{equation*}
    \tcblower
    \begin{proof}
        Define a function $f_S \colon S \to \{0, 1\}$ by
        \begin{equation*}
            f_S(x) = \begin{cases}
                1 \quad \textrm{if } x \in S \\
                0 \quad \textrm{if } x \notin S
            \end{cases}.
        \end{equation*}
        Let $E = \bigcup_{i = 1}^n E_i$. Consider the function $g \colon S \to \{0, 1\}$ given by
        \begin{displaymath}
            g(x) = \prod_{i = 1}^{n}\left(f_E(x) - f_{E_i}(x)\right).
        \end{displaymath}
        For any $x \in S$, if $x \in E$, then $x \in E_k$ for some $k \in \left\{x \in \N \colon x \leq n\right\}$, which means that $f_E(x) - f_{E_k}(x) = 0$; if $x \notin E$, then $f_E(x) = f_{E_i}(x) = 0$ for all $i \in \left\{x \in \N \colon x \leq n\right\}$. In either case, $g(x) = 0$.
    \end{proof}
\end{thmbox}
\begin{thmbox}{Boole's Inequality}{boole}
    Let $E_1, E_2, \cdots, E_n, \cdots$ be a countable sequence of events, then
    \begin{equation*}
        P\left(\bigcup_{i = 1}^\infty E_i\right) \leq \sum_{i = 1}^{\infty}P(E_i).
    \end{equation*}
    In particular, equality is achieved if and only if the $E_i$'s are mutually exclusive.
\end{thmbox}
\begin{thmbox}{Probability in a Finite Sample Space}
    Let $S$ be a sample space which is finite and let $E \subseteq S$ be an event, then 
    \begin{equation*}
        P(E) = \frac{\abs{E}}{\abs{S}}.
    \end{equation*}
\end{thmbox}

\chapter{Conditional Probability}
\section{Conditional Probability}
Given a sample space $S$, we may wish to find the probability of two events $E$ and $F$ both occurring, $P(EF)$. However, suppose that we already know that event $F$ \textbf{has occurred}, then necessarily, the sample space we consider would no longer be $S$. Essentially, this condition of $F$ having occurred has restricted our sample space to $F$. Thus, we give the following definition:
\begin{dfnbox}{Conditional Probability}{condProb}
    Let $S$ be a sample space and $E, F \subseteq S$ be two events. If $P(F) \geq 0$, then the {\color{red} \textbf{conditional probability}} is the probability that $E$ occurs given that $F$ has occurred, denoted by
    \begin{equation*}
        P(E|F) = \frac{P(EF)}{P(F)}.
    \end{equation*}
    In particular, if $E \subseteq F$, we have $P(E|F) = \frac{P(E)}{P(F)}$.
\end{dfnbox}
\begin{notebox}
    \begin{remark}
        Note that $P(E|F) = P(EF|F)$.
    \end{remark}
\end{notebox}
It is easy to see that $P(EF) = P(E|F)P(F)$, i.e., the probability of $E$ and $F$ both occurring is the product of the probability of $F$ occurring and the probability of $E$ occurring given the occurrence of $F$. This complies with our intuition. We can generalise this for a countable number of events:
\begin{probox}{Multiplication Rule}{multRule}
    Let $S$ be a sample space and let $E_i \subseteq S$ for $i = 1, 2, \cdots, n$ be $n$ events, where $n \geq 2$. Suppose that $P\left(\bigcap_{i = 1}^{n - 1}E_i\right) > 0$, then 
    \begin{equation*}
        P\left(\bigcap_{i = 1}^{n}E_i\right) = P(E_1)\prod_{i = 2}^{n}P\left(E_i\left\lvert\bigcap_{j = 1}^{i - 1}E_j\right.\right).
    \end{equation*}
    \tcblower
    \begin{proof}
        The case where $n = 2$ is immediate from Definition \ref{dfn:condProb}.
        \\\\
        Suppose that there is some $k \in \N$ and $k \geq 2$ such that
        \begin{equation*}
            P\left(\bigcap_{i = 1}^{k}E_i\right) = P(E_1)\prod_{i = 2}^{k}P\left(E_i\left\lvert\bigcap_{j = 1}^{i - 1}E_j\right.\right),
        \end{equation*}
        then we consider
        \begin{align*}
            P\left(E_{k + 1}\left\lvert\bigcap_{i = 1}^{k}E_i\right.\right) & = \frac{P\left(\bigcap_{i = 1}^{k + 1}E_i\right)}{P\left(\bigcap_{i = 1}^{k}E_i\right)} \\
            & = \frac{P\left(\bigcap_{i = 1}^{k + 1}E_i\right)}{P(E_1)\prod_{i = 2}^{k}P\left(E_i\left\lvert\bigcap_{j = 1}^{i - 1}E_j\right.\right)}.
        \end{align*}
        Therefore,
        \begin{align*}
            P\left(\bigcap_{i = 1}^{k + 1}E_i\right) & = \left[P(E_1)\prod_{i = 2}^{k}P\left(E_i\left\lvert\bigcap_{j = 1}^{i - 1}E_j\right.\right)\right]P\left(E_{k + 1}\left\lvert\bigcap_{i = 1}^{k}E_i\right.\right) \\
            & = P(E_1)\prod_{i = 2}^{k + 1}P\left(E_i\left\lvert\bigcap_{j = 1}^{i - 1}E_j\right.\right)
        \end{align*}
    \end{proof}
\end{probox}
\section{Bayes's Formula}
Consider a sample space $S$ and two events $E, F \subseteq S$. Suppose that $E$ occurs, then either $F$ has occurred or $F$ has never occurred (i.e. $F^{\mathrm{c}}$ occurred). Therefore, it is easy to see that
\begin{equation*}
    P(E) = P(EF) + P(EF^{\mathrm{c}}) = P(E|F) + P(E|F^{\mathrm{c}}).
\end{equation*}
We can extend the above argument for more than two events. Suppose that $F_1, F_2, \cdots, F_n$ are $n$ mutually exclusive events such that $\bigcup_{i = 1}^n F_i = S$, then obviously $\left\{F_1, F_2, \cdots, F_n\right\}$ is a \textit{partition} of $S$.

Consider any event $E$ and let $e \in E$. Clearly, $e$ must be in one and only one of $F_1, F_2, \cdots, F_n$. It then follows that $\left\{E \cap F_1, E \cap F_2, \cdots, E \cap F_n\right\}$ is a partition of $E$. Generalising this further to a countably infinite number of mutually exclusive events $F_1, F_2, \cdots$ such that $\bigcup_{i = 1}^\infty F_i = S$, we arrive at the following formula:
\begin{equation*}
    P(E) = \sum_{i = 1}^{\infty}P(E|F_i)P(F_i).
\end{equation*}
This leads to the \textit{Bayes's Formula}:
\begin{thmbox}{Bayes's Formula}{BayesFormula}
    Let $F_1, F_2, \cdots$ be a countably infinite sequence of events over a sample space $S$ such that $\bigcup_{i = 1}^\infty F_i = S$. For any event $E \subseteq S$, we have
    \begin{equation*}
        P(F_j|E) = \frac{P(E|F_j)P(F_j)}{\sum_{i = 1}^{\infty}P(E|F_i)P(F_i)}.
    \end{equation*}
\end{thmbox}
\section{Indepedent Events}
Note that in general, for two events $E$ and $F$, $P(E|F) \neq P(E)$, i.e., the occurrence of $F$ may affect the occurrence of $E$. However, in some cases, we notice that the occurrence of $E$ is \textit{independent} of $F$, and so we introduce the following definition:
\begin{dfnbox}{Independent Events}{indEvents}
    Let $S$ be a sample space and let $E, F \subseteq S$ be two events. We say that $E$ and $F$ are {\color{red} \textbf{independent}} if $P(EF) = P(E)P(F)$, and {\color{red} \textbf{dependent}} otherwise.
\end{dfnbox}
\begin{notebox}
    \begin{remark}
        The following results are immediate:
        \begin{enumerate}
            \item If $P(E) = 0$ or $P(F) = 0$, then $E$ and $F$ are independent.
            \item If $P(E) > 0$ (respectively, $P(F) > 0$), then $E$ and $F$ are independent if and only if $P(F|E) = P(F)$ (respectively, $P(E|F) = P(E)$).
        \end{enumerate}
    \end{remark}
\end{notebox}
Intuitively, given independent events $E$ and $F$, we may believe that if the occurrence of $E$ does not affect the occurrence of $F$, then naturally the occurrence of $E$ should also not affect the ``not-occurring'' of $F$, i.e., the following is true:
\begin{probox}{}{independentComp}
    $E$ and $F$ are independent events if and only if $E$ and $F^{\mathrm{c}}$ are independent events.
    \tcblower
    \begin{proof}
        Since $F = \left(F^{\mathrm{c}}\right)^{\mathrm{c}}$, it suffices to prove for one direction. 
        \\\\
        Notice that $EF \cup EF^{\mathrm{c}} = E(F \cup F^{\mathrm{c}}) = E$, so
        \begin{equation*}
            P(E) = P(EF) + P(EF^{\mathrm{c}}) = P(E)P(F) + P(EF^{\mathrm{c}}).
        \end{equation*}
        Therefore,
        \begin{equation*}
            P(EF^{\mathrm{c}}) = P(E) - P(E)P(F) = P(E)(1 - P(F)) = P(E)P(F^{\mathrm{c}}),
        \end{equation*}
        and so $E$ and $F^{\mathrm{c}}$ are independent.
    \end{proof}
\end{probox} 

\chapter{Random Variables}

\section{Random Variables}
In many contexts, we might wish to generalise a formula to compute the probability of the occurrence of a certain event. However, in cases where the events are abstract or unquantifiable (e.g. the event ``tomorrow is rainy''), it becomes hard to formulate a well-defined mapping from a sample space to $[0, 1]$. Thus, to model all events easily using functions and mappings, we introduce the notion of \textit{random variables}.
\begin{dfnbox}{Random Variable}{randVar}
    Let $\Omega$ be a sample space, the {\color{red} \textbf{random variable}}
    \begin{displaymath}
        X \colon \Omega \to \R
    \end{displaymath}
    is a real-valued function such that for any event $E \subseteq \Omega$, 
    \begin{equation*}
        P(E) = P(X[E]) = P(X \in X[E]),
    \end{equation*}
    where
    \begin{displaymath}
        X[E] \coloneqq \left\{X(\omega) \colon \omega \in \Omega\right\}
    \end{displaymath}
    is the image of the event $E$ under $X$.
\end{dfnbox}
\section{Discrete Random Variables}
Intuitively, there are certain events whose outcomes are finite or can be enumerated. In such cases, we may associate these events with a \textit{discrete random variable}.
\begin{dfnbox}{Discrete Random Variable (DRV)}{DRV}
    Let $X$ be a random variable over a sample space $S$, if $\mathrm{ran}(X)$ is countable, then $X$ is called a {\color{red} \textbf{discrete random variable}}.
\end{dfnbox}
\begin{dfnbox}{Probability Mass Function (PMF)}{PMF}
    Let $X$ be a discrete random variable over a sample space $S$, the function
    \begin{displaymath}
        p_X \colon X[S] \to [0, 1]
    \end{displaymath}
    where $p_X(a) = P(X = a)$ is known as the {\color{red} \textbf{probability mass function}} of $X$.
\end{dfnbox}
\begin{notebox}
    \begin{remark}
        $\sum_{a \in X[S]} p_X(a) = 1$.
    \end{remark}
\end{notebox}
For any discrete random variable $X$, $p_X(a) = 0$ if and only if $\left\{s \in S \colon X(s) = a\right\} = \varnothing$. This essentially means that if $p_X$ evaluates to $0$, then the corresponding event is an impossible event.

Note that $p_X$ essentially gives the probability of \textbf{singleton} events in a sample space. Naturally, we can represent the probability of a union of singleton events as a linear combination of the values of $p_X$.
\begin{dfnbox}{Cumulative Distribution Function (CDF)}{CDF}
    Let $X$ be a discrete random variable over a sample space $S$ with PMF $p_X$, the function
    \begin{displaymath}
        F_X \colon X[S] \to [0, 1]
    \end{displaymath}
    where $F_X(a) = \sum_{x \leq a}p_X(x)$ is known as the {\color{red} \textbf{cumulative distribution function}} of $X$.
\end{dfnbox}
\begin{notebox}
    \begin{remark}
        Suppose that $X(s_i) = a_i$ for all $s_i \in S$ such that $a_i < a_j$ whenever $i < j$, then $F_X$ is a non-decreasing {\color{red} \textbf{step function}}, i.e., for all $a$ such that $a_i \leq a < a_{i + 1}$,
        \begin{equation*}
            F_X(a) = \sum_{x \leq a}p_X(x) = \sum_{k = 1}^{i}p_X(a_k).
        \end{equation*}
    \end{remark}
\end{notebox}
\subsection{Expectation of Discrete Random Variables}
Suppose $X$ is a discrete random variable with range $\left\{x_1, x_2, \cdots, x_m\right\}$ and PMF $p_X$. By repeating an experiment of $X$ for $n$ times, we can approximate the total number of occurrences of $X = x_i$ by $np_X(x_i)$. Therefore, the average value of $X$ can be approximated by
\begin{equation*}
    \frac{\sum_{i = 1}^{m}nx_ip_X(x_i)}{n}.
\end{equation*}
For large $n$, we have
\begin{equation*}
    \lim_{n \to \infty}\frac{\sum_{i = 1}^{m}nx_ip_X(x_i)x_i}{n} = \lim_{n \to \infty}\sum_{i = 1}^{m}x_ip_X(x_i) = \sum_{i = 1}^{m}x_ip_X(x_i).
\end{equation*}
Similarly, if the range of $X$ is countably infinite, replacing from the above $\sum_{i = 1}^{m}nx_ip_X(x_i)x_i$ with $\sum_{i = 1}^{\infty}nx_ip_X(x_i)x_i$ will yield the same limit.

Intuitively, the above limit represents the \textit{expected value} of $X$ when a large number of experiments are conducted, which leads to the following definition:
\begin{dfnbox}{Expectation of Discrete Random Variables}{DRVExp}
    Let $X$ be a discrete random variable. The {\color{red} \textbf{expectation}} (or {\color{red} \textbf{mean}}, {\color{red} \textbf{expected value}}) of~$X$ is defined to be
    \begin{equation*}
        E[X] = \sum_{i = 1}^{m}\left[x_ip_X(x_i)\right] = \sum_{i = 1}^{m}\left[x_iP(X = x_i)\right]
    \end{equation*}
    if $\abs{\mathrm{ran}(X)} = m$, and
    \begin{equation*}
        E[X] = \sum_{i = 1}^{\infty}\left[x_ip_X(x_i)\right] = \sum_{i = 1}^{\infty}\left[x_iP(X = x_i)\right]
    \end{equation*}
    if $\mathrm{ran}(X)$ is countably infinite.
\end{dfnbox}
By convention, we use $\mu$ to denote expectation, so $X[E]$ can be written as $\mu_X$.

For a discrete random variable $X$, we can define a function $g \colon \mathrm{ran}(X) \to \R$. It is easy to see that $g(X)$ is also a discrete random variable. Therefore, we may have the following result:
\begin{thmbox}{Expectation of Functions}{funcExp}
    Let $X$ be a discrete random variable and define $Y = g(X)$, then
    \begin{equation*}
        E[Y] = E[g(X)] = \sum_{x}\left[g(x)P(X = x)\right].
    \end{equation*}
    \tcblower
    \begin{proof}
        Note that for each $y \in \mathrm{ran}(Y)$, $g(x) = y$ for some $x \in \mathrm{ran}(X)$, and so
        \begin{equation*}
            P(Y = y) = \sum_{g(x) = y}P(X = x).
        \end{equation*}
        Therefore,
        \begin{align*}
            E[Y] & = \sum_{y}\left[yP(Y = y)\right] \\
            & = \sum_{y}\left[y\!\sum_{g(x) = y}\!P(X = x)\right] \\
            & = \sum_{y}\left[\sum_{g(x) = y}\!g(x)P(X = x)\right] \\
            & = \sum_{x}\left[g(x)P(X = x)\right].
        \end{align*}
    \end{proof}
\end{thmbox}
Two simple corollaries to the above theorem are:
\begin{align*}
    E[aX + b] & = aE[X] + b \\
    E[X + Y] & = E[X] + E[Y].
\end{align*}
In later sections, we will prove that the same rule applies to continuous random variables as well. The above theorem gives rise to the following notion of \textit{moments}:
\begin{dfnbox}{Moment}{moment}
    Let $X$ be a random variable. $E[X^n]$ is called the $n$-th {\color{red} \textbf{moment}} of $X$.
\end{dfnbox}
Following Theorem \ref{thm:funcExp}, it is easy to see that if $X$ is discrete, then
\begin{equation*}
    E[X^n] = \sum_{i = 1}^{\infty}x_ip_X(x_i).
\end{equation*}

\subsection{Variance}
Note that given two different discrete random variables $X$ and $Y$, their probability mass functions can be different but they can still have identical expectations. For example, consider $p_X$ to be identically $0$ and $p_Y$ to be such that $p_Y(0) = 1$ and $p_Y(y) = 0$ for all $y \neq 0$.

This motivates us to find other properties to classify and characterise random variables. One of these properties is the \textbf{spread} of the possible values taken by a random variable with respect to its mean, i.e., consider the random variable $X$ with $E[X] = \mu$, we wish to determine $E[\abs{X - \mu}]$ or equivalently $E[(X - \mu)^2]$. This spread is known as the \textit{variance} of a random variable.
\begin{dfnbox}{Variance}{var}
    Let $X$ be a random variable with $E[X] = \mu$, the {\color{red} \textbf{variance}} of $X$ is defined to be
    \begin{equation*}
        \mathrm{Var}(X) = E[(X - \mu)^2] = E[X^2] - (E[X])^2.
    \end{equation*}
\end{dfnbox}
By convention, we use $\sigma^2$ to denote variance, so $\mathrm{Var}(X)$ can be written as $\sigma_X^2$.

The formula for $\mathrm{Var}(X)$ can be derived via Theorem \ref{thm:funcExp}:
\begin{align*}
    \mathrm{Var}(X) & = E[(X - \mu)^2] \\
    & = E[X^2 - 2\mu X + \mu^2] \\
    & = E[X^2] - 2\mu E[X] + \mu^2 \\
    & = E[X^2] - 2(E[X])^2 + (E[X])^2 \\
    & = E[X^2] - (E[X])^2.
\end{align*}
Another term we hear often is \textit{standard deviation}, which is defined as follows:
\begin{dfnbox}{Standard Deviation}{sd}
    Let $X$ be a random variable, the {\color{red} \textbf{standard deviation}} of $X$ is defined to be
    \begin{equation*}
        \mathrm{SD}(X) = \sqrt{\mathrm{Var}(X)} = \sigma_X.
    \end{equation*}
\end{dfnbox}
Note that we have computed the general formula for any linear combination of discrete random variables. We shall do the same for variance.
\begin{probox}{Variance of Linear Combinations of Random Variables}{linearCombiVar}
    Let $X$ be a random variable, then
    \begin{align*}
        \mathrm{Var}(aX + b) & = a^2\mathrm{Var}(X) \\
        \mathrm{SD(aX + b)} & =\abs{a}\mathrm{SD}(X)
    \end{align*}
    for all $a, b \in \R$.
    \tcblower
    \begin{proof}
        By using Theorem \ref{thm:funcExp}, we have
        \begin{align*}
            \mathrm{Var}(aX + b) & = E[(aX + b)^2] - (E[aX + b])^2 \\
            & = a^2 E[X^2] + 2ab E[X] + b^2 - \left[a(E[X])^2 + 2abE[X] + b^2\right] \\
            & = a^2\left[E[X^2] - (E[X])^2\right] \\
            & = a^2\mathrm{Var}(X).
        \end{align*}
        Therefore,
        \begin{equation*}
            \mathrm{SD}(aX + b) = \sqrt{\mathrm{Var}(aX + b)} = \abs{a}\mathrm{SD}(X).
        \end{equation*}
    \end{proof}
\end{probox}

\subsection{Bernoulli and Binomial Random Variables}
Suppose we conduct an experiment. In the most simplistic view, only two outcomes are considered, namely \textbf{success} and \textbf{failure}. We can model such experiments using a discrete random variable whose range has a cardinality of $2$.
\begin{dfnbox}{Bernoulli Random Variable}{bernoulliRV}
    A random variable $X$ is a {\color{red} \textbf{Bernoulli random variable}} if
    \begin{align*}
        p_X(x) = \begin{cases}
            p, \quad & \textrm{if } x = 1 \\
            1 - p, \quad & \textrm{if } x = 0 \\
            0, \quad & \textrm{otherwise}
        \end{cases}
    \end{align*}
    for some $p \in [0, 1]$.
\end{dfnbox}
Now, consider $n$ \textbf{independent} trials of an experiment with a probability for success of $p$. Let $X$ be the number of successes among these $n$ trials, then clearly,
\begin{equation*}
    p_X(x) = P(X = x) = \begin{pmatrix}
        n \\
        x
    \end{pmatrix}p^x(1 - p)^{n - x}.
\end{equation*}
\begin{dfnbox}{Binomial Random Variable}{binRV}
    A random variable $X$ is a {\color{red} \textbf{binomial random variable}} if
    \begin{equation*}
        p_X(x) = P(X = x) = \begin{pmatrix}
            n \\
            x
        \end{pmatrix}p^x(1 - p)^{n - x}
    \end{equation*}
    for some $p \in [0, 1]$. $X$ is said to have a {\color{red} \textbf{binomial distribution}} with parameters $(n, p)$, denoted by $X \sim \mathrm{B}(n, p)$.
\end{dfnbox}
\begin{notebox}
    \begin{remark}
        In particular, if $X$ is a Bernoulli random variable, then $X \sim \mathrm{B}(1, p)$.
    \end{remark}
\end{notebox}
Suppose $X \sim \mathrm{B}(n, p)$. Let $N$ be the average number of successes in the $n$ trials, it is expected that $p \approx \frac{N}{n}$. Therefore, we may conjure that $N \approx np$.
\begin{thmbox}{Expectation and Variance of Binomial Distribution}{BExpVar}
    Let $X \sim B(n, p)$, then $E[X] = np$ and $\mathrm{Var}(X) = np(1 - p)$.
    \tcblower
    \begin{proof}
        Note that $iC^n_i = nC^{n - 1}_{i - 1}$, so
        \begin{align*}
            E[X] & = \sum_{i = 0}^{n}\left[i \begin{pmatrix}
                n \\
                i
            \end{pmatrix}p^i(1 - p)^{n - i}\right] \\
            & = \sum_{i = 1}^{n}\left[n \begin{pmatrix}
                n - 1 \\
                i - 1
            \end{pmatrix}p^i(1 - p)^{n - i}\right] \\
            & = n\sum_{j = 0}^{n - 1}\left[\begin{pmatrix}
                n - 1 \\
                j
            \end{pmatrix}p^{j + 1}(1 - p)^{n - 1 - j}\right] \\
            & = np\sum_{j = 0}^{n - 1}\left[\begin{pmatrix}
                n - 1 \\
                j
            \end{pmatrix}p^j(1 - p)^{n - 1 - j}\right] \\
            & = np, \\
            \mathrm{Var}(X) & = E[X^2] - (E[X])^2 \\
            & = \sum_{i = 0}^{n}\left[i^2\begin{pmatrix}
                n \\
                i
            \end{pmatrix}p^i(1 - p)^{n - i}\right] - n^2p^2 \\
            & = n\sum_{i = 1}^{n}\left[i\begin{pmatrix}
                n - 1 \\
                i - 1
            \end{pmatrix}p^i(1 - p)^{n - i}\right] - n^2p^2 \\
            & = n\sum_{j = 0}^{n - 1}\left[(j + 1)\begin{pmatrix}
                n - 1 \\
                j
            \end{pmatrix}p^{j + 1}(1 - p)^{n - 1 - j}\right] - n^2p^2 \\
            & = np\left\{\sum_{j = 0}^{n - 1}\left[j\begin{pmatrix}
                n - 1 \\
                j
            \end{pmatrix}p^j(1 - p)^{n - 1 - j}\right] + 1\right\} - n^2p^2 \\
            & = np\left[(n - 1)p + 1\right] - n^2p^2 \\
            & = np - np^2 \\
            & = np(1 - p).
        \end{align*}
    \end{proof}
\end{thmbox}
Let $X \sim \mathrm{B}(n, p)$, consider
\begin{align*}
    \frac{p_X(i + 1)}{p_X(i)} & = \frac{\frac{n!}{(i + 1)!(n - i - 1)!}p^{i + 1}(1 - p)^{n - i - 1}}{\frac{n!}{i!(n - i)!}p^i(1 - p)^{n - i}} \\
    & = \frac{\frac{1}{i + 1}p}{\frac{1}{n - i}(1 - p)} \\
    & = \frac{(n - i)p}{(i + 1)(1 - p)}.
\end{align*}
Suppose $p_X(i + 1) < p_X(i)$, then $(n - i)p < (i + 1)(1 - p)$. This implies that $i > (n + 1)p - 1$, which means that
\begin{itemize}
    \item $p_X(i)$ is monotonically increasing on $\left[0, (n + 1)p - 1\right]$.
    \item $p_X(i)$ maximises when $i = \left\lceil (n + 1)p - 1 \right\rceil = \left\lfloor (n + 1)p \right\rfloor$.
    \item $p_X(i)$ is monotincally decreasing on $\left((n + 1)p - 1, n\right]$.
\end{itemize}

\subsection{Poisson Random Variable}
Suppose that $X \sim \mathrm{B}(n, p)$ such that $n$ is large and $p$ is small. Let $\lambda = np$, then
\begin{align*}
    p_X(i) & = \begin{pmatrix}
        n \\
        i
    \end{pmatrix}p^i(1 - p)^{n - i} \\
    & = \frac{\prod_{j = 0}^{i - 1}(n - j)}{i!}\left(\frac{\lambda}{n}\right)^i\left(1 - \frac{\lambda}{n}\right)^{n - i} \\
    & = \frac{\prod_{j = 0}^{i - 1}(n - j)}{n^i}\cdot\frac{\lambda^i}{i!}\cdot\frac{\left(1 - \frac{\lambda}{n}\right)^n}{\left(1 - \frac{\lambda}{n}\right)^i}
\end{align*}
Therefore,
\begin{equation*}
    \lim_{n \to \infty}p_X(i) = e^{-\lambda}\frac{\lambda^i}{i!}.
\end{equation*}
Note that this means that we can use $e^{-\lambda}\frac{\lambda^i}{i!}$ as a good approximation for $p_X(i)$ when $n$ is large and $p$ is small! In this case, $\lambda$ is the expected frequency of occurrences of the event corresponding to $X = 1$ within a unit interval.
\begin{dfnbox}{Poisson Random Variable}{poisson}
    A random variable $X$ is a {\color{red} \textbf{Poisson random variable}} if 
    \begin{equation*}
        p_X(x) = P(X = x) = e^{-\lambda}\frac{\lambda^x}{x!}
    \end{equation*}
    for some $\lambda > 0$, denoted as $X \sim \mathrm{Po}(\lambda)$.
\end{dfnbox}
Note that for $X \sim \mathrm{Po}(\lambda)$, we can find some $Y \sim \mathrm{B}(n, p)$ where $n$ is large and $p$ is small such that $np = \lambda$. Therefore, it is expected that 
\begin{align*}
    E[X] & \approx E[Y] = \lambda, \\
    \mathrm{Var}(X) & \approx \mathrm{Var}(Y) = np(1 - p) \approx \lambda.
\end{align*}
\begin{thmbox}{Expectation and Variance of Poisson Random Variables}{poExpVar}
    If $X \sim \mathrm{Po}(\lambda)$ where $\lambda > 0$, then $E[X] = \mathrm{Var}(X) = \lambda$.
\end{thmbox}
\end{document}