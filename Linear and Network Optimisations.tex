\documentclass[math, code]{amznotes}
\usepackage[utf8]{inputenc}
\usepackage{amsmath}
\usepackage{amsfonts}
\usepackage{graphicx}
\usepackage{tikz}
\usepackage{etoolbox}

\graphicspath{ {./images/} }
\geometry{
    a4paper,
    headheight = 1.5cm
}

\patchcmd{\chapter}{\thispagestyle{plain}}
{\thispagestyle{fancy}}{}{}

\theoremstyle{remark}
\newtheorem*{claim}{Claim}
\newtheorem*{remark}{Remark}
\newtheorem{case}{Case}

\newcommand{\map}[3]{#1: #2 \rightarrow #3} % Mapping
\newcommand{\image}[2]{#2\left[#1\right]} % Image
\newcommand{\preimage}[2]{#2\left[#1\right]^{-1}} % Pre-image
\newcommand{\eval}[3]{\left. #1\right\rvert_{#2 = #3}} % Evaluated at
%\newcommand\bigO[1]{\mathcal{O}\left(#1\right)}

\DeclareMathOperator*{\argmax}{argmax}
\DeclareMathOperator*{\argmin}{argmin}

\begin{document}
\fancyhead[L]{
    Linear and Network Optimisations
}
\fancyhead[R]{
    Lecture Notes
}
\tableofcontents

\chapter{Linear Programming}
\section{Linear Programming}
Recall that in general, an optimisation problem can be formulated as
\begin{align*}
    \min_{\mathbfit{x} \in \R^n} & f(\mathbfit{x}) \\
    \textrm{s.t. } & \mathbfit{x} \in P,
\end{align*}
where $P \subseteq \R^n$ is called the \textit{feasible set} (or \textit{feasible region}).
\begin{dfnbox}{Linear Programming Problem}{LP}
    A {\color{red} \textbf{linear programming}} (LP) problem is an optimisation problem where the objective function $f$ is linear and the feasible set $P$ is a polyhedron.
\end{dfnbox}
We can therefore formulate a linear programming problem as
\begin{align*}
    \min_{\mathbfit{x} \in \R^n} & \mathbfit{c}^{\mathrm{T}}\mathbfit{x} \\
    \textrm{s.t. } & \mathbfit{a}_i^{\mathrm{T}}\mathbfit{x} \leq b_i \quad \textrm{for } i = 1, 2, \cdots, p \\
    & \mathbfit{a}_j^{\mathrm{T}}\mathbfit{x} = b_j \quad \textrm{for } i = 1, 2, \cdots, m,
\end{align*}
where $\mathbfit{c} \in \R^n$ is called the \textit{cost} or \textit{profit} vector, $\mathbfit{a}_i^{\mathrm{T}}\mathbfit{x}$ and $\mathbfit{a}_j^{\mathrm{T}}\mathbfit{x}$ are called the \textit{constraints} and~$\mathbfit{x}$ is known as \textit{decision variables}.
\end{document}