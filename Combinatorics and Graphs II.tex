\documentclass[math, code]{amznotes}
\usepackage[utf8]{inputenc}
\usepackage{amsmath}
\usepackage{amsfonts}
\usepackage{graphicx}
\usepackage{tikz}
\usepackage{etoolbox}

\graphicspath{ {./images/} }
\geometry{
    a4paper,
    headheight = 1.5cm
}

\patchcmd{\chapter}{\thispagestyle{plain}}
{\thispagestyle{fancy}}{}{}

\theoremstyle{remark}
\newtheorem*{claim}{Claim}
\newtheorem*{remark}{Remark}
\newtheorem{case}{Case}

\newcommand{\map}[3]{#1: #2 \rightarrow #3} % Mapping
\newcommand{\image}[2]{#2\left[#1\right]} % Image
\newcommand{\preimage}[2]{#2\left[#1\right]^{-1}} % Pre-image
\newcommand{\eval}[3]{\left. #1\right\rvert_{#2 = #3}} % Evaluated at
%\newcommand\bigO[1]{\mathcal{O}\left(#1\right)}

\DeclareMathOperator*{\argmax}{argmax}
\DeclareMathOperator*{\argmin}{argmin}
\DeclareMathAlphabet{\mathcal}{OMS}{cmsy}{m}{n}

\begin{document}
\fancyhead[L]{
    Combinatorics and Graphs II
}
\fancyhead[R]{
    Lecture Notes
}
\tableofcontents

\chapter{Graph Structures}
\section{Multigraphs}
\subsection{Multigraphs}
Intuitively, we would describe a \textit{graph} as a collection of nodes (or vertices) plus some lines (or edges) joining some nodes together. This can be rigorously defined as follows:
\begin{dfnbox}{Multigraph}{multigraph}
    A {\color{red} \textbf{multigraph}} $G$ consists of a non-empty finite set of vertices denoted by $V(G)$ and a finite set of edges denoted by $E(G)$. $\abs{V(G)}$ is known as the {\color{red} \textbf{order}} of $G$, denoted by~$v(G)$ and $\abs{E(G)}$ is known as the {\color{red} \textbf{size}} of $G$, denoted by $e(G)$.
    \\\\
    In particular, if $v(G) = m$ and $e(G) = n$, we say that $G$ is an {\color{red} \textbf{$(m, n)$-graph}}.
    \\\\
    $G$ is said to be {\color{red} \textbf{trivial}} if $v(G) = 1$ and {\color{red} \textbf{non-trivial}} otherwise.
\end{dfnbox}
Note that in a multigraph, by default there can be a plural number of edges between any two vertices. We would define the notion of \textit{simple graph} as a special multigraph.
\begin{dfnbox}{Simple Graph}{simpleGraph}
    A multigraph $G$ is said to be {\color{red} \textbf{simple}} if there is at most one edge between any two distinct vertices.
\end{dfnbox}
\begin{notebox}
    \begin{remark}
        Note that if $G$ is simple and undirected, then
        \begin{displaymath}
            E(G) \subseteq \left\{(v_j, v_i) \colon v_j, v_i \in V(G), j \geq i\right\}.
        \end{displaymath}
    \end{remark}
\end{notebox}
Notice that if $(v_i, v_i)$ is an edge, it connects a vertex to itself. This is known as a \textit{loop}.

In a graph, it is important to know the layout of the vertices and edges. For this purpose, we define the notion of \textit{adjacency} in a multigraph.
\begin{dfnbox}{Adjacency and Neighbourhood}{adjacent}
    Let $v_i, v_j \in V(G)$, we say that they are {\color{red} \textbf{adjacent}} if $v_iv_j \in E(G)$. Alternatively, we say that $v_i$ and $v_j$ are {\color{red} \textbf{neighbours}} to each other. The edge $v_iv_j$ is said to be {\color{red} \textbf{incident}} with~$v_i$ and $v_j$. Two edges $e$ and $f$ are said to be {\color{red} \textbf{adjacent}} if there exists some~$v \in V(G)$ such that both $e$ and $f$ are incident with $v$. 
    \\\\
    The set of all neighbours to some $v_i \in V(G)$ is called the {\color{red} \textbf{neighbourhood set}} of $v_i$, denoted by $N_G(v_i)$. In particular, the set $N_G[v_i] \coloneqq N_G(v_i)\cup\{v_i\}$ is known as the {\color{red} \textbf{closed neighbourhood set}} of $v_i$.
\end{dfnbox}
Alternatively, one may write $v \sim u$ if $v$ and $u$ are adjacent vertices. We can further add on to the definition by discussing the size of the neighbourhood of a vertex.
\begin{dfnbox}{Degree}{deg}
    The {\color{red} \textbf{degree}} of $v$, denoted by $d_G(v)$ is defined as the number of edges incident to $v$. If~$d_G(v)$ is even (repectively, odd), then we say that $v$ is an even (respectively, odd) vertex. If $d_G(v) = 0$, we say that $v$ is {\color{red} \textbf{isolated}}; if $d_G(v) = 1$, we say that $v$ is an {\color{red} \textbf{end}} vertex. 
    \\\\
    In particular, we define
    \begin{equation*}
        \Delta(G) = \max_{v \in G}d_G(v), \qquad \delta(G) = \min_{v \in G}d_G(v).
    \end{equation*}
\end{dfnbox}
In particular, we can denote the average degree of a graph $G$ by
\begin{equation*}
    \bar{d}(G) = \frac{\sum_{v \in V(G)}d_G(v)}{v(G)}.
\end{equation*}
\begin{notebox}
    \begin{remark}
        Note that a high maximal degree does not imply a high average degree. A classic counter example is a complete graph plus an isolated vertex.
    \end{remark}
\end{notebox}
With the notion of degree established, we can now define a \textit{regular graph}.
\begin{dfnbox}{Regular Graph}{regGraph}
    A {\color{red} \textbf{regular graph}} is a graph in which every vertex has the same degree. In particular, if $d_G(v) = k$ for all $v \in V(G)$, $G$ is known as a {\color{red} \textbf{$k$-regular graph}}.
\end{dfnbox}
\begin{notebox}
    \begin{remark}
        A graph $G$ is regular if and only if $\Delta(G) = \delta(G)$.
    \end{remark}
\end{notebox}
Since a graph essentially consists of two sets, it is natural to consider the notion of graph complementation. We now proceed to introducing the notion of \textit{complement}.
\begin{dfnbox}{Complement}{comp}
    Let $G$ be a graph of order $n$, the {\color{red} \textbf{complement}} of $G$, denoted by $\overline{G}$, is the graph of order $n$ where
    \begin{equation*}
        V\left(\overline{G}\right) = V(G), \qquad E\left(\overline{G}\right) = \left\{(u, v) \colon (u, v) \notin E(G)\right\}.
    \end{equation*}
\end{dfnbox}
We discuss the following interesting question:
\begin{quote}
    $15$ students went to a party. During the party some of them shook hands with each other. At the end of the party, the number of handshakes made by each student was recorded and it was reported that the sum was $39$. Was this possible?
\end{quote}
Note that if we represent each student as a vertex, then we can use $V(G) = \left\{v_1, v_2, \cdots, v_{15}\right\}$ to construct a graph, in which an edge $v_iv_j \in E(G)$ if and only if students $i$ and $j$ shook hands. As such, $d_G(v_i)$ is the number of persons student $i$ shook hands with.

Note that if we sum up $d_G(v_i)$ for all the vertices, every edge will be counted exactly twice! This means that
\begin{equation*}
    \sum_{i = 1}^{15}d_G(v_i) = 39
\end{equation*}
is impossible since the left-hand side must be even. In fact, by the above reasoning, we see that in any graph, the sum of degrees of its vertices must be even.
\begin{lembox}{Handshaking Lemma}{handshake}
    If $G$ is a graph of order $n$ and size $m$, then
    \begin{equation*}
        \sum_{i = 1}^{n}d_G(v_i) = 2m.
    \end{equation*}
\end{lembox}
\begin{notebox}
    \begin{remark}
        It can be easily deduced from the above lemma that in any graph, the number of vertices with odd degrees must be even.
    \end{remark}
\end{notebox}
Relating to Definition \ref{dfn:deg}, we see that the average degree of a graph $G$ can be computed as
\begin{equation*}
    \bar{d}(G) = \frac{2e(G)}{v(G)}.
\end{equation*}
As an extension of Lemma \ref{lem:handshake}, the minimum size of any graph is obviously $0$, and the maximum size of a simple graph occurs when there is an edge between any two vertices.
\begin{dfnbox}{Empty and Complete Graphs}{empCom}
    Let $G$ be a simple graph of order $n$. $G$ is said to be an {\color{red} \textbf{empty graph}} or {\color{red} \textbf{null graph}}, denoted by $0_n$, if $e(G) = 0$, and a {\color{red} \textbf{complete graph}}, denoted by $K_n$ if for all $u, v \in V(G)$, we have $(u, v) \in E(G)$. 
\end{dfnbox}
\subsection{Subgraphs}
Since a multigraph is just two sets, we can define a ``subset'' relation between graphs.
\begin{dfnbox}{Subgraph}{subgraph}
    Let $G, H$ be graphs, then $H$ is a {\color{red} \textbf{subgraph}} of $G$ if $V(H) \subseteq V(G)$ and $E(H) \subseteq E(G)$. In particular, a subgraph $H$ is a {\color{red} \textbf{proper subgraph}} of $G$ if $V(H) \neq V(G)$ or $E(H) \neq E(G)$.
\end{dfnbox}
Note that we a graph can possibly be reproduced by connecting its vertices correctly, so a subgraph containing all vertices of a graph ``spans'' the original graph.
\begin{dfnbox}{Spanning Subgraph}{spanSubgraph}
    Let $H$ be a subgraph of $G$. $H$ is called a {\color{red} \textbf{spanning subgraph}} of $G$ if $V(H) = V(G)$.
\end{dfnbox}
Note that by definition, a spanning subgraph retains all vertices of the original graph. Therefore, a way to quickly generate a subgraph of a given graph $G$ is to keep the vertex set and delete some edges from the edge set. We denote such a graph by $H = G - F$ for some $F \subset E(G)$. Observe that 
\begin{equation*}
    V(H) = V(G), \qquad E(H) = E(G) - F.
\end{equation*}
On the other hand, by deleting some vertices together with edges incident to them, we can produce a subgraph from any given graph.
\begin{dfnbox}{Induced Subgraph}{induceSubgraph}
    Let $H$ be a subgraph of $G$. $H$ is called a {\color{red} \textbf{induced subgraph}} of $G$ if
    \begin{equation*}
        E(H) = \left\{uv \in E(G) \colon u, v \in V(H)\right\}.
    \end{equation*}
    Let $S \subseteq V(G)$, the subgraph of $G$ induced by $S$ is denoted by $[S]$.
    \\\\
    Alternatively, let $F \subseteq E(G)$. Define
    \begin{equation*}
        V' \coloneqq \bigcup \bigl\{\left\{u, v\right\} \colon uv \in F\bigr\},
    \end{equation*}
    then $(V', F)$ is the subgraph of $G$ induced by $F$, denoted by $G[F]$.
\end{dfnbox}
Intuitively, an induced subgraph pf $G$ consists of a selected subset of the vertices of $G$ together with all edges in $G$ connecting any vertices in this subset.

Using the idea of deletion, we can quickly generate an induced subgraph $H$ of $G$ as follows: first, set $V(H) = V(G) - A$ for some $A \subseteq V(G)$, i.e., remove some vertices; then, we will remove all edges from $E(G)$ which are incident to some vertices in $A$, i.e., set
\begin{equation*}
    E(H) = \left\{e \in E(G) \colon e \textrm{ is not incident to any } v \in A\right\}.
\end{equation*}
Then, $H = \bigl(V(H), E(H)\bigr)$ is an induced subgraph of $G$. In fact, we can denote $H$ as
\begin{equation*}
    H = G - A = [V(G) - A].
\end{equation*}
This leads to the following proposition:
\begin{probox}{}{}
    Let $G$ be a graph. If $A \subseteq V(G)$, then $G - A = [V(G) - A]$. Suppose $H$ is a subgraph of $G$, then $H$ is an induced subgraph of $G$ if and only if $H = G - \bigl(V(G) - V(H)\bigr)$.
\end{probox}
We would like to consider the relations between the properties of a graph to those of its subgraphs. In particular, it follows from intuition that if a graph has a high average degree, then we can naturally find a subgraph with a high minimal degree.

This seems trivial from intuition, as we can always keep deleting vertices with the minimal degree from a graph until we cannot find any vertex whose degree is less than $\frac{1}{2}\bar{d}(G)$, but there is more to that --- for instance, how do we know if we would not end up deleting all vertices from the original graph? Thus, to prove our claim is essentially asking to prove the correctness of this greedy deletion, which we shall do in the following proposition.
\begin{probox}{Subgraphs with High Minimum Degree}{highAvgDeg}
    Every non-empty graph $G$ has a subgraph $H$ such that $\delta(H) \geq \frac{1}{2} \bar{d}(G)$.
    \tcblower
    \begin{proof}
        Define $H_0 = G$ and $H_{i + 1} = H_i - v_i$ where $v_i$ is a vertex in $H_i$ whose degree is the smallest. Note that we can always find this $v_i$ for any $H_i$ with $V(H_i) \neq \varnothing$ because $V(H_i)$ is a finite well-ordered set. We will repeatedly perform the deletion until we obtain some $H_k$ with $d_{H_k}(v_k) \geq \frac{1}{2}\bar{d}(G)$. We shall proceed to proving that this algorithm always terminates with $V(H_k) \neq \varnothing$.
        \\\\
        Suppose on contrary that $V(H_k) = \varnothing$, then we have performed the deletion for $v(G)$ times. Suppose we have deleted $N$ edges in total, then clearly,
        \begin{equation*}
            N < v(G)\frac{1}{2}\bar{d}(G) = v(G)\frac{e(G)}{v(G)} = e(G),
        \end{equation*}
        which means $e(H_k) = e(G) - N > 0$. However, this is impossible since $v(H_k) = 0$, which is a contradiction.
    \end{proof}
\end{probox}

\subsection{Graph Isomorphism}
Given two graphs $G$ and $H$, are they the same graph? This seemingly innocent question proves to be extremely hard to answer. Two graphs can look drastically different but be structurally identical in reality. For example, we could shift around the vertices of a graph without changing any edge to alter the shape of the graph dramatically. Therefore, to compare the structures of graphs, we require some rigorous definition.
\begin{dfnbox}{Graph Isomorphism}{graphIsomorph}
    Two graphs $G$ and $H$ are said to be {\color{red} \textbf{isomorphic}}, denoted by $G \cong H$, if there exists a bijection $f \colon V(G) \to V(H)$ such that
    \begin{equation*}
        uv \in E(G) \quad\textrm{ if and only if } f(u)f(v) \in E(H).
    \end{equation*}
\end{dfnbox}
\begin{notebox}
    \begin{remark}
        One may check that graph isomorphism is obviously an equivalence relation.
    \end{remark}
\end{notebox}
Such a bijection $f$ is said to \textbf{preserve adjacency}, i.e., if $u$ and $v$ are neighbours in $G$, then their images are also neighbours in $H$. In particular, it is possible to map a graph to its own complement via an isomorphism.
\begin{dfnbox}{Self-Complementary Graph}{selfComp}
    A graph $G$ is said to be {\color{red} \textbf{self-complementary}} if $G \cong \overline{G}$.
\end{dfnbox}
It turns out that a self-complementary graph satisfies some special properties.
\begin{probox}{Order of Self-Complementray Graphs}{selfCompOrder}
    If $G$ is a self-complementary graph of order $n$, then $n = 4k$ or $n = 4k + 1$ for some~$k \in \Z^+$.
    \tcblower
    \begin{proof}
        Since $G$ is self-complementary, $G \cong \overline{G}$ and so $e(G) = e\left(\overline{G}\right)$. Note that
        \begin{equation*}
            e(G) + e\left(\overline{G}\right) = \begin{pmatrix}
                n \\
                2
            \end{pmatrix} = \frac{n(n - 1)}{2},
        \end{equation*}
        so we have $e(G) = \frac{n(n - 1)}{4}$. Since $e(G) \in \Z^+$, either $n$ or $n - 1$ is divisible by $4$, so~$n = 4k$ or $n = 4k + 1$ for some $k \in \Z^+$.
    \end{proof}
\end{probox}
It is very hard to determine whether two specific graphs are isomorphic, but there are some considerations in the general case. For example, some trivial conclusions include:
\begin{itemize}
    \item Two graphs with different orders cannot be isomorphic.
    \item Two graphs with different sizes cannot be isomorphic.
    \item Two graphs with different numbers of components cannot be isomorphic.
    \item If the numbers of vertices with degree $k$ are different in two graphs, they cannot be isomorphic.
\end{itemize}
Notice that by Definition \ref{dfn:graphIsomorph}, we can define a function $g$ such that $uv \notin E(G)$ if and only if $g(u)g(v) \notin E(H)$ and relate this function to the complement graphs of $G$ and $H$. Here we introduce a way to determine isomorphism by considering complement graphs, the proof of which is left to the reader as an exercise.
\begin{thmbox}{Complementation Preserves Isomorphism}{compIso}
    Let $G$ and $H$ be two graphs of the same order. $G \cong H$ if and only if $\overline{G} \cong \overline{H}$.
\end{thmbox}
Now let us think the reverse: if it is not easy to prove isomorphism, can we find a way to quickly determine that two graphs are not isomorphic? Here we present the necessary conditions for isomorphism:
\begin{thmbox}{Necessary Conditions for Isomorphism}{necessaryCondIso}
    If $G \cong H$, then
    \begin{enumerate}
        \item $G$ and $H$ must have the same order and size.
        \item $\delta(G) = \delta(H)$ and $\Delta(G) = \Delta(H)$.
        \item The number of vertices with degree $i$ in $G$ and $H$ is the same for all $i \in \N$.
    \end{enumerate}
\end{thmbox}
The first two conditions are easy to observe. For the third condition, we introduce a tool known as \textit{degree sequences}.
\begin{dfnbox}{Degree Sequence}{degSeq}
    Let $G$ be a graph of order $n$. If we label its vertices by $v_1, v_2, \cdots, v_n$ such that
    \begin{equation*}
        d(v_1) \geq d(v_2) \geq \cdots \geq d(v_n),
    \end{equation*}
    then the non-increasing sequence $\bigl(d(v_n)\bigr)$ is known as the {\color{red} \textbf{degree sequence}} of $G$.
\end{dfnbox}
Now we consider the following question:
\begin{quote}
    Let $(d_n)$ be a non-increasing sequence, is there some graph whose degree sequence is $(d_n)$?
\end{quote}
We will make use of the following definition:
\begin{dfnbox}{Graphic Sequence}{graphicSeq}
    Let $(d_n)$ be a sequence of non-negative integers at most $n - 1$. $(d_n)$ is said to be {\color{red} \textbf{graphic}} if there exists a graph $G$ whose degree sequence is $(d_n)$.
\end{dfnbox}
To determine whether a sequence is graphic by eye power is difficult. Fortunately, we have the following recursive algorithm:
\begin{thmbox}{Havel-Hakimi Algorithm}{HavelHakimi}
    Let $(d_n)$ be a non-increasing sequence of non-negative integers at most $n - 1$. Define
    \begin{equation*}
        d^*_m = \begin{cases}
            d_{m + 1} - 1 & \quad\textrm{if } 1 \leq m \leq d_1 \\
            d_{m + 1} & \quad\textrm{if } d_1 + 1 \leq m \leq n - 1
        \end{cases}.
    \end{equation*}
    $(d_n)$ is graphic if and only if $\left(d^*_m\right)$ is graphic.
    \tcblower
    \begin{proof}
        Suppose $(d_n)$ is graphic,we consider the following lemma:
        \begin{lembox}{}{graphicLemma}
            For any graphic sequence $(d_n)$, there is some graph $G$ of order $n$ with $d_G(v_i) = d_i$ for $i = 1, 2, \cdots, n$ such that $v_1$ is adjacent to $v_j$ for $j = 2, 3, \cdots, d_1 + 1$.
            \tcblower
            \begin{proof}
                Suppose on contrary there is no such a graph $G$, then for any graph $G$ with degree sequence $(d_n)$, $v_1$ is adjacent to at most $d_1 - 1$ vertices in
                \begin{equation*}
                    A \coloneqq \left\{v_j \colon j = 2, 3, \cdots, d_1 + 1\right\}.
                \end{equation*}
                Thus, there exists some $v_j \in A$ such that $v_1v_j \notin E(G)$. However,~$d_G(v_1) = d_1$, so there exists some $v_k \in V(G) - A$ with $v_k \neq v_1$ such that $v_1v_k \in E(G)$.
                \\\\
                Notice that since $v_k \notin A \cup \{v_1\}$, $k > j$ and so $d_G(v_k) = d_k \leq d_j = d_G(v_j)$. Therefore, $v_j$ has at least as many neighbours as $v_k$, which means there must exists some $v_r$ with $v_jv_r \in E(G)$ such that $v_kv_r \notin E(G)$.
                \\\\
                Define a graph $G'$ by
                \begin{equation*}
                    V(G') = V(G), \qquad E(G') = E(G) - v_jv_r - v_1v_k + v_jv_k + v_1v_j.
                \end{equation*}
                Note that $d_{G'}(v_i) = d_G(v_i)$ for all $i = 1, 2, \cdots, n$, so $(d_n)$ is also a degree sequence for $G'$. However, now $v_1$ in $G'$ is adjacent to all vertices in $A$, which is a contradiction.
            \end{proof}
        \end{lembox}
        By Lemma \ref{lem:graphicLemma}, we can choose some graph $G$ whose degree sequence is $(d_n)$ such that $v_1$ is adjacent to $d_1$ vertices $v_2, v_3, \cdots, v_{d_1 + 1}$. Let $H = G - v_1$, then for any $u_i \in H$,
        \begin{equation*}
            d_H(u_i) = \begin{cases}
                d_G(v_{i + 1}) - 1 & \quad\textrm{if } 1 \leq i \leq d_1 \\
                d_G(v_{i + 1}) & \quad\textrm{if } d_1 + 1 \leq i \leq n - 1
            \end{cases}.
        \end{equation*}
        Therefore, $\left(d^*_m\right)$ is graphic.
        \\\\
        Conversely, suppose $\left(d^*_m\right)$ is graphic, then there is some graph $H$ of order $n - 1$ such that $d_H(u_i) = d^*_i$ for $i = 1, 2, \cdots, d_1$. Construct a graph $G = H + v_1$ such that~$v_i = u_{i - 1}$ for $i = 2, 3, \cdots, n$ and $v_1$ is adjacent to $v_2, v_3, \cdots, v_{d_1 + 1}$. Therefore,
        \begin{equation*}
            d_G(v_i) = \begin{cases}
                d_1 & \quad\textrm{if } i = 1 \\
                d^*_{i - 1} + 1 & \quad\textrm{if } 2 \leq i \leq d_1 + 1 \\
                d^*_{i} & \quad\textrm{if } d_1 + 2 \leq i \leq n
            \end{cases}.
        \end{equation*}
        Note that $d_G(v_i) = d_i$, so $(d_n)$ is graphic.
    \end{proof}
\end{thmbox}
By Theorem \ref{thm:HavelHakimi}, we can reduce a sequence recursively until we reach some obvious case. This obvious case is graphic if and only if the original sequence is graphic. We can also build the graph corresponding to the original sequence from a simple graphic sequence.

Alternatively, the following is another algorithmic way to determine whether a sequence is graphic.
\begin{thmbox}{Erdos-Gallai Algorithm}{ErdosGallai}
    A non-increasing sequence $(d_1, d_2, \cdots, d_n)$ is graphic if and only if $\sum_{i = 1}^n d_i$ is even and for all $k = 1, 2, \cdots, n$,
    \begin{equation*}
        \sum_{i = 1}^{k} d_i \leq k(k - 1) + \sum_{i = k + 1}^{n}\min\{d_i, k\}.
    \end{equation*}
\end{thmbox}
We will not give the full proof but only briefly justify the ``only if'' direction. Let $G$ be any graph. Define $d_i = d_G(v_i)$ where $v_i \in V(G)$ for $i = 1, 2, \cdots, n$, then $(d_1, d_2, \cdots, d_n)$ is graphic. Note that $\sum_{i = 1}^{n}d_i$ is obviously even by Lemma \ref{lem:handshake}. 

Take some $k \in \Z^+$ with $k \leq n$ and let $S = \left\{v_1, v_2, \cdots, v_k\right\}$, then by Lemma \ref{lem:handshake} again,
\begin{align*}
    \sum_{i = 1}^{k}d_i & = 2e(G[S]) + e_G\bigl(S, V(G) - S\bigr) \\
    & \leq k(k - 1) + e_G\bigl(S, V(G) - S\bigr),
\end{align*}
where $e_G\bigl(S, V(G) - S\bigr)$ is the number of edges in $G$ which connect a vertex in $S$ to a vertex in $V(G) - S$. Notice that
\begin{align*}
    e_G\bigl(S, V(G) - S\bigr) & = \sum_{i = k + 1}^n\abs{\left\{u \in S \colon uv_i \in E(G)\right\}} \\
    & \sum_{i = k + 1}^n\leq \min\{k, d_i\},
\end{align*}
and so it follows that
\begin{equation*}
    \sum_{i = 1}^{k} d_i \leq k(k - 1) + \sum_{i = k + 1}^{n}\min\{d_i, k\}.
\end{equation*}
\begin{notebox}
    \begin{remark}
        Both Theorems \ref{thm:HavelHakimi} and \ref{thm:ErdosGallai} are $\mathcal{O}(n^2)$ algorithms.
    \end{remark}
\end{notebox}


\section{Graph Traversal}
\subsection{Paths and Cycles}
We are often interested in the ``connectedness'' in a graph. In particular, we would wish to study how to traverse a graph from some vertices to others.
\begin{dfnbox}{Walk, Trail, Path}{wtp}
    Let $G$ be a graph and $x, y \in V(G)$, then an {\color{red} \textbf{$x$-$y$ walk}} is an alternating sequence
    \begin{equation*}
        W \coloneqq v_0e_1v_1e_2v_2\cdots v_{n - 1}e_nv_n
    \end{equation*}
    where $v_i \in V(G)$, $e_i = v_{i - 1}v_i \in E(G)$, $x = v_0$ and $y = v_n$. In particular, $W$ is {\color{red} \textbf{open}} if $v_0 \neq v_n$ and {\color{red} \textbf{closed}} otherwise. The {\color{red} \textbf{length}} of a walk is defined as the number of edges in it.
    \\\\
    If $e_i \neq e_j$ whenever $i \neq j$, then $W$ is called a {\color{red} \textbf{trail}}. A closed trail is called a {\color{red} \textbf{circuit}}. If $v_i \neq v_j$ whenever $i \neq j$, then $W$ is called a {\color{red} \textbf{path}}. A closed path is called a {\color{red} \textbf{cycle}}.
\end{dfnbox}
\begin{notebox}
    \begin{remark}
        All paths are trails and all trails are walks, but the converses are not true.
    \end{remark}
\end{notebox}
In more informal terms, a trail is a walk with no repeated edge and a path is a trail with no repeated vertex.

Note that a path is essentially a graph. We denote a path of order $n$ by $P_n$ and a cycle of order $n$ by $C_n$. It is easy to see that all cycles are circuits and all circuits are all closed walks, but the converses are not true. However, if we have a circuit, we can guarantee that this circuit contains at least one cycle.

Intuitively, all walks can be reduced to a path by removing duplicated edges and vertices repeatedly.
\begin{probox}{Any Walk Contains A Path}{walkAndPath}
    If a graph $G$ contains a $u$-$v$ walk of length $k$, then $G$ contains a $u$-$v$ path of length at most $k$.
    \tcblower
    \begin{proof}
        Let $S$ be the set of all $u$-$v$ walks in $G$, Note that $S \neq \varnothing$ since there is a $u$-$v$ walk of length $k$. Let $P$ be a walk of the shortest length. We claim that $P$ must be a path.
        \\\\
        Suppose $P$ is not a path, then there exists some vertices $w_1 = w_2$ in $P$. Suppose $Q$ is obtained by removing the $w_1$-$w_2$ walk from $P$, then $Q$ is a walk. However,~$Q$ is shorter than $P$ which is not possible. Therefore, $P$ must be a path.
        \\\\
        Note that the length of $P$ is at most $k$, so $G$ contains a $u$-$v$ path of length at most $k$.
    \end{proof}
\end{probox}

We also have the following relevant definitions with regard to cycles:
\begin{dfnbox}{Girth and Circumference}{girthCircumference}
    Let $G$ be a graph. The {\color{red} \textbf{girth}} of $G$ is the size of the shortest non-trivial cycle in $G$ and the {\color{red} \textbf{circumference}} of $G$ is the size of the longest non-trivial cycle in $G$.
\end{dfnbox}
If $G$ is acyclic, we define the girth of $G$ to be $\infty$ and the circumference to be $0$. In some special graphs, we realise that we can traverse through every vertex and return to the starting point in a single traversal. Such graphs are said to be \textit{Hamiltonian}.
\begin{dfnbox}{Hamiltonian Graph}{Hamiltonian}
    A graph $G$ is {\color{red} \textbf{Hamiltonian}} if it contains a cycle of size $v(G)$.
\end{dfnbox}


\subsection{Connected Components}
An important question we are interested in with graphs is the reachability of a vertex, i.e., given two vertices $u$ and $v$, we want to know whether we can reach one vertex from another. Intuitively, we can traverse between two vertices if there is a path between them. From here we define the notion of a \textit{connected graph}.
\begin{dfnbox}{Connected Graph}{connected}
    Let $G$ be a graph. Two vertices $u$ and $v$ are {\color{red} \textbf{connected}} if there is a path between them. $G$ is said to be {\color{red} \textbf{connected}} if for any $u, v \in V(G)$, there exists a path from $u$ to $v$. For any $u \in V(G)$, we denote the set of all vertices connected to $u$ (inclusive of $u$) as $c(u)$.
\end{dfnbox}
Intuitively, we of course always can find a connected subgraph of a connected graph, but we can in fact produce a much stronger result.
\begin{probox}{Labelling of Vertices of Connected Graphs}{labelConnected}
    For every connected graph $G$, $V(G)$ can be labelled as $\left\{v_1, v_2, \cdots, v_n\right\}$ such that the graph~$G_i \coloneqq \bigl[\left\{v_1, v_2, \cdots, v_i\right\}\bigr]$ is connected for every $i = 1, 2, \cdots, n$.
    \tcblower
    \begin{proof}
        For $i = 1$, we can pick any vertex in $V(G)$ as $v_1$ and $G_1$ is always trivially connected.
        \\\\
        Suppose we have fixed up to $k$ vertices for some $k = 1, 2, \cdots, n - 1$, then $G_1, G_2, \cdots, G_k$ are connected subgraphs of $G$. Note that $v(G) = n > k$, so there exists some $v \in V(G) - V(G_k)$ such that $v$ and $v_1$ are connected by a path $P$. Note that there exists some $u \in V(P)$ such that $u \sim v$ for some $v \in V(G_k)$, so taking $v_{k + 1} = u$ guarantees that $G_{k + 1}$ is connected.
    \end{proof}
\end{probox}
Intuitively, if a graph has many edges, we would believe that it is easy to find a long path in it. Note that the notion of ``having many edges'' can be related to the minimal degree of a graph. More formally, the following result is true:
\begin{probox}{Finding Paths and Cycles in Graphs}{pathCycleInGraph}
    Every graph $G$ with $\delta(G) \geq 2$ contains a path of length $\delta(G)$ and a cycle of length at least $\delta(G) + 1$.
    \tcblower
    \begin{proof}
        Let $P = x_0x_1\cdots x_k$ be the longest path in $G$. Note that it suffices to prove that~$k \geq \delta(G)$. Suppose that there exists some $v \in N(x_k)$ such that $v \notin V(P)$, then we can find a longer path which is not possible. Therefore, $N(x_k) \subseteq V(P)$. Note that~$\abs{N(x_k)} \geq \delta(G)$, so $k \geq \delta(G)$ and so $G$ contains a $P_{\delta(G)} \subseteq P$.
        \\\\
        Let $i \in \N$ be such that $x_jx_k \notin E(G)$ for all $0 \leq j < i$. Take $P' = x_ix_{i + 1}\cdots x_k$, then $e(P') \geq \delta(G)$ by the previous argument. However, $x_i \sim x_k$, so $P' + x_ix_k$ is a cycle of length at least $\delta(G) + 1$ in $G$.
    \end{proof}
\end{probox}
A classic problem regarding graph traversal in connected graphs is the K\"{o}nisburg Bridge Problem:
\begin{quote}
    Let $A, B, C, D$ be $4$ distinct vertices such that there are $2$ edges between $A$ and $B$, $2$ edges between $A$ and $C$, and $1$ edge between $A$ and $D$, $B$ and $D$, $C$ and $D$. Can all the edges be visited exactly once in a single traversal of the graph produced?
\end{quote}
This can be generalised as: given a connected multigraph $G$, does there exist a circuit in $G$ which contains all the edges of $G$? Such multigraphs are said to be \textit{Eulerian}.
\begin{dfnbox}{Eulerian Circuit}{eulerianCircuit}
    Let $G$ be a connected multigraph, an {\color{red} \textbf{Eulerian circuit}} in $G$ is a circuit which uses all edges of $G$ exactly once.
\end{dfnbox}
Intuitively, if a graph is Eulerian, then following any Eulerian circuit will visit each of its vertices for an even number of times as there are equally many incoming edges as outgoing edges for every vertex in a circuit. This implies that an Eulerian graph will necessarily force all of its vertices to have even degrees. However, it turns out that actually this is also a sufficient condition for the existence of Eulerian circuits!
\begin{thmbox}{Necessary and Sufficient Condition for Eulerian Graphs}{NSCondEulerian}
    A connected multigraph $G$ is Eulerian if and only if every vertex in $G$ has an even degree.
    \tcblower
    \begin{proof}
        The necessity is trivial, so we will only prove for sufficiency.
        \\\\
        Consider the case where $e(G) = 0$, then the only connected multigraph is the trivial graph consisting of a singleton vertex, which is Eulerian. 
        \\\\
        Suppose that for all $k < m$ where $m \in \Z^+$, a connected multigraph $H$ with $e(H) = k$ is Eulerian if every vertex in $H$ has an even degree. Let $G$ be a connected multigraph with $e(G) = m > 0$ such that every vertex in $G$ has an even degree, then $\delta(G) \geq 2$. By Proposition \ref{pro:pathCycleInGraph}, there exists a cycle $S \subseteq G$ where $e(S) \geq 3$.
        \\\\
        Let $G'$ be the multigraph such that $V(G') = V(G)$ and $E(G') = E(G) - E(S)$, then clearly, $e(G') < m$ and every vertex in $G'$ has an even degree. Therefore, each connected component of $G'$ is Eulerian.
        \\\\
        Let $C_1$ be a connected component of $G'$ and let $v_1 \in V(C_1) \cap V(S)$. Note that there exists an Eulerian circuit in $C_1$ which starts and ends with $v_1$. We can traverse along this circuit and then traverse in the clockwise orientation along $S$ until we reach another vertex in $S$ which is in some connected component of $G'$ and repeat the process. Note that $v_1 \in V(S)$, so we will eventually traverse back to $v_1$, which gives an Eulerian circuit in $G$.
    \end{proof}
\end{thmbox}
Intuitively, even if a graph $G$ is not Eulerian, it is possible to traverse all edges in $G$ with finitely many trails. This is known as a decomposition of $G$ into trails.
\begin{probox}{Minimal Number of Trails to Decompose A Multigraph}{minNumOfTrails}
    Let $G$ be a graph with $2k$ vertices having odd degrees for some $k \in \N$, then the minimal number of trails needed to decompose $G$ is $\max\{k, 1\}$.
    \tcblower
    \begin{proof}
        The case where $k = 0$ is immediate by Theorem \ref{thm:NSCondEulerian}. For $k \geq 1$, suppose on contrary that $G$ can be decomposed with $k - 1$ trails labelled as $T_1, T_2, \cdots, T_{k - 1}$. Since a trail contains at most $2$ vertices with odd degrees, then
        \begin{equation*}
            \abs{\left\{v \in \bigcup_{i = 1}^{k - 1}T_i \colon d_G(v) \textrm{ is odd}\right\}} \leq 2(k - 1) < 2k,
        \end{equation*}
        which is a contradiction. Therefore, we need at least $k$ trails to decompose $G$. 
        \\\\
        Let $v_1, v_2, \cdots, v_{2k}$ be the vertices in $G$ with odd degrees. Let $G'$ be the graph produced by adding the edges $v_1v_2, v_3v_4, \cdots, v_{2k - 1}v_{2k}$ to $G$, then the vertices in $G'$ all have even degrees. Therefore, there exists an Eulerian circuit in $G'$ by Theorem \ref{thm:NSCondEulerian}.
        \\\\
        Note that for all $i = 1, 2, \cdots, k - 1$, $v_{2i - 1}v_{2i}$ and $v_{2i + 1}v_{2i + 2}$ are disjoint, so there exists a $v_{2i}$-$v_{2i + 1}$ trail. Note also that there exists a $v_{2k}$-$v_1$ trail. The union of these trails is exactly $G' - v_1v_2 - v_3v_4 - \cdots - v_{2k - 1}v_{2k} = G$.
    \end{proof}
\end{probox}
Note that connectedness is an equivalence relation. Let $P$ be a partition formed by the equivalence classes of $G$ under the connectedness relation, then for any $u, v \in V(G)$, $u$ and~$v$ are in the same equivalence class if and only if $u$ and $v$ are connected. Any $c(u) \in P$ induces a subgraph $[c(u)]$.
\begin{dfnbox}{Connected Component}{CC}
    Let $G$ be a graph and $R$ be a relation such that for any $u, v \in V(G)$, $uRv$ if and only if~$u$ and $v$ are connected. Let
    \begin{equation*}
        C \coloneqq V(G)/R
    \end{equation*}
    be the quotient set, then any $c(u) \in C$ is known as a (connected) {\color{red} \textbf{component}} of $G$.
    \\\\
    The number of components of $G$ is denoted by $\omega(G)$.
\end{dfnbox}
Alternatively, a connected component is a maximally connected subgraph, that is, a connected component of $G$ is a subgraph $H \subseteq G$ such that for all $e \in E(G)$, $H + e$ is not connected and for all $v \in V(G) - V(H)$, $\left[V(H) \cup \{v\}\right]$ is not connected.

It can be easily seen that $G$ is connected if and only if $\omega(G) = 1$.

We will now establish a relationship between connectedness and complementation.
\begin{thmbox}{Connectedness of Complement}{connectedComp}
    If $G$ is disconnected, then $\overline{G}$ is connected.
    \tcblower
    \begin{proof}
        Let $u, v \in V\left(\overline{G}\right)$ be two arbitrary vertices. If $uv \notin E(G)$, then $uv \in \left(\overline{G}\right)$ and so $u$ and $v$ are connected in $\overline{G}$.
        \\\\
        If $uv \in E(G)$, since $G$ is disconnected, there exists some $w \in V(G)$ such that~$uw, wv \notin E(G)$. Therefore, $uw, wv \in E\left(\overline{G}\right)$. This means that there is a~$u$-$v$ path in $\overline{G}$ and so $u$ and $v$ are connected in $\overline{G}$.
        \\\\
        Therefore, $\overline{G}$ is connected.
    \end{proof}
\end{thmbox}
Intuitively, we can transform a connected graph into a disconnected one by deleting some vertices or edges. It can be easily observed that in certain graphs, deleting one particular vertex or edge will immediately disconnect the graph.
\begin{dfnbox}{Cut-Vertex, Bridge}{cutNBridge}
    Let $G$ be a non-trivial graph. $v \in V(G)$ is called a {\color{red} \textbf{cut-vertex}} of $G$ if $\omega(G - v) > \omega(G)$. $e \in E(G)$ is called a {\color{red} \textbf{bridge}} of $G$ if $\omega(G - e) > \omega(G)$.
\end{dfnbox}
Note that a graph does not have to be connected to have cut vertices and bridges. Essentially, a cut-vertex (or bridge) is just a vertex (or edge) which disconnects a component upon removal.

Now the next question is how do we identify a cut-vertex or a bridge in a graph? Intuitively, a cut-vertex divides a graph into two portions such that traversal between the two portions has to pass through it.
\begin{thmbox}{Cut-Vertex Characterisation}{cutVertex}
    Let $G$ be a graph. $v \in V(G)$ is a cut-vertex if and only if there exists $a, b \in V(G)$ such that $v$ is in every path between $a$ and $b$.
    \tcblower
    \begin{proof}
        Let $v \in V(G)$ be a cut-vertex and consider $H = G - v$, then we can find two non-empty disjoint connected components of $H$, say $H_1$ and $H_2$. Now, take $a \in V(H_1)$ and $b \in V_(H_2)$, then they are disconnected. However, $a$ and $b$ are connected in $G$, which means $v$ is in every $a$-$b$ path.
        \\\\
        We will prove the converse by contrapositive. Suppose $v$ is not a cut-vertex of $G$, then $G - v$ is connected. Therefore, for any $u, w \in V(G)$, there is a path between them which does not contain $v$.
    \end{proof}
\end{thmbox}
A similar argument can be established for bridges. Furthermore, since we have to pass through the bridge when traversing between the two portions, it can be easily seen that we have to reuse the bridge in order to traverse back.
\begin{thmbox}{Bridge Characterisation}{bridge}
    Let $G$ be a graph. $e \in E(G)$ is a bridge if and only if $e$ is not contained by any cycle in $G$.
    \tcblower
    \begin{proof}
        We will prove the forward direction by contrapositive. Suppose $e = xy$ is contained in some cycle in $G$, then there is some $x$-$y$ path in $G - e$. Suppose $a, b \in V(G)$ are connected in $G$ via a path containing $e$, this implies that there is some $a$-$b$ path in $G - e$ and so $G - e$ is connected, which means that $e$ is not a bridge.
        \\\\
        We will prove the converse also by contrapositive. Suppose $e = xy$ is not a bridge, the $G - e$ is connected and so there is some $x$-$y$ path in $G - e$. Let this path be $P$, then $P \cup \{e\}$ is a cycle in $G$.
    \end{proof}
\end{thmbox}
Combining the two characterisations, we can relate cut-vertices to bridges in the following proposition, the proof of which is left to the reader as an exercise:
\begin{probox}{}{cutVertexNBridge}
    Let $G$ be a graph. If $uv \in E(G)$ is a bridge and $u$ is not an end vertex, then $u$ is a cut-vertex.
\end{probox}
A direct consequence of this is the following corollary:
\begin{corbox}{}{}
    If $G$ is a graph with order at least $3$ and $G$ contains a bridge, then $G$ contains a cut-vertex.
\end{corbox}
\subsection{Graphs as Metric Spaces}
Given a graph $G$ with $u$, $v$ being two connected vertices, we are interested in the notion of distance between them. Intuitively, if there are multiple paths between $u$ and $v$, we would take the length of the shortest one to represent their distance.
\begin{dfnbox}{Distance, Eccentricity, Diameter}{distEDia}
    Let $G$ be a connected graph and let $u, v \in V(G)$. The {\color{red} \textbf{distance}} between $u$ and $v$, denoted by $d(u, v)$, is the length of the shortest path between $u$ and $v$. The {\color{red} \textbf{eccentricity}} of $u$ is defined to be
    \begin{equation*}
        e(u) = \max_{v \in V(G)}\{d(u, v)\}.
    \end{equation*}
    The {\color{red} \textbf{diameter}} of $G$ is defined by
    \begin{equation*}
        \mathrm{diam}(G) = \max_{u \in V(G)}\{e(u)\}.
    \end{equation*}
    The {\color{red} \textbf{radius}} of $G$ is defined by
    \begin{equation*}
        \mathrm{rad}(G) = \min_{u \in V(G)}\{e(u)\}.
    \end{equation*}
    A vertex $v$ is called a {\color{red} \textbf{central}} vertex if
    \begin{equation*}
        e(v) = \mathrm{rad}(v).
    \end{equation*}
    The subgraph induced by the set of central vertices of $G$ is known as the {\color{red} \textbf{centre}} of $G$.
\end{dfnbox}
This also justifies the use of words ``diameter'' and ``radius'' in a circle. We can view a circle as a graph consisting of a centre and infinitely many vertices in the circumference, with edges of length $r$ connecting the centre to the circumference. Indeed, in this definition, the farthest vertices will be any two on the circumference with a distance of $2r$, and the nearest vertices will be the centre and any vertex on the circumference with a distance of $r$.

Intuitively, having a large diameter means that the graph is sparse, i.e., one has to take a very long path to traverse between two vertices. Now let us consider a connected but sparse graph, the above intuition means that we may expect to find that the cycles in this graph are very large.
\begin{probox}{Graphs with High Girth Have High Diameter}{highGirthHighDiam}
    Let $G$ be a graph with girth $g(G)$. If $G$ contains a cycle, then $g(G) \leq 2\mathrm{diam}(G) + 1$.
    \tcblower
    \begin{proof}
        Suppose on contrary that $g(G) \geq 2\mathrm{diam}(G) + 2$. Let $C \subseteq G$ be the shortest cycle in $G$, then we can take $x, y \in V(C)$ such that $d_C(x, y)$ is the greatest. Clearly, we have $d_C(x, y) \geq \mathrm{diam}(G) + 1$.
        \\\\
        By Definition \ref{dfn:distEDia}, there exists some path $P$ between $x$ and $y$ with $e(P) \leq \mathrm{diam}(G)$. Let $P' \subseteq C$ be the $x$-$y$ path in $C$, then $e(P) < e(P')$. This means that $P$ and $P'$ are distinct paths and so there is some $q \in V(P') - V(P)$.
        \\\\
        Since $P$ and $P'$ are both $x$-$y$ walks, $\abs{P \cap P'} \geq 2$. Therefore, there are $p_1, p_2 \in P \cap P'$ which are connected to $q$, which implies that there is a $p_1$-$p_2$ path in $P'$. Since $p_1$ and~$p_2$ are connected in $P$, we have found $2$ different $p_1$-$p_2$ paths and so there is some cycle $Q \subseteq P \cup P'$. Therefore,
        \begin{align*}
            e(Q) & \leq e(P\cup P') \\
            & \leq e(P) + e(P') \\
            & \leq \mathrm{diam}(G) + d_C(x, y) \\
            & \leq 2d_C(x, y) - 1 \\
            & = 2\left\lfloor\frac{e(C)}{2}\right\rfloor - 1 \\
            & < e(C),
        \end{align*}
        which is a contradiction.
    \end{proof}
\end{probox}

For readers with knowledge in real analysis or topology, it is easy to see that an undirected unweighted connected graph is a metric space with distance between vertices as its metric. 
\begin{thmbox}{Connected Graph as A Metric Space}{graphMetric}
    If $G$ is connected, then $(V(G), d_G)$ is a metric space.
    \tcblower   
    \begin{proof}
        Clearly, for all $x, y \in V(G)$, $d_G(x, y) \geq 0 \in V(G)$ with $d_G(x, y) = 0$ if and only if $x = y$ and $d_G(x, y) = d_G(y, x)$.

        Let $U$, $V$ be the shortest $u$-$w$ path and shortest $w$-$v$ path respectively, then~$U + V$ is a $u$-$v$ walk with length $d(u, w) + d(w, v)$. Since the shortest $u$-$v$ path has length $d(u, v)$, we have
        \begin{equation*}
            d(u, v) \leq d(u, w) + d(w, v).
        \end{equation*}
        Therefore, $(V(G), d_G)$ is a metric space.
    \end{proof}
\end{thmbox}
An application of Theorem \ref{thm:graphMetric} allows us to establish the following:
\begin{thmbox}{Boundedness of Diameter}{boundedDiam}
    If $G$ is connected, then
    \begin{equation*}
        \mathrm{rad}(G) \leq \mathrm{diam}(G) \leq 2\mathrm{rad}(G).
    \end{equation*}
    \tcblower
    \begin{proof}
        $\mathrm{rad}(G) \leq \mathrm{diam}(G)$ is immediate from Definition \ref{dfn:distEDia}. Take $u, v \in V(G)$ with~$d(u, v) = \mathrm{diam}(G)$ and take $w \in V(G)$ such that $e(w) = \mathrm{rad}(G)$. Notice that this implies
        \begin{align*}
            d(u, w) & \leq e(w) = \mathrm{rad}(G) \\
            d(w, v) & \leq e(w) = \mathrm{rad}(G).
        \end{align*}
        By Theorem \ref{thm:thm:graphMetric}, 
        \begin{equation*}
            \mathrm{diam}(G) = d(u, v) \leq d(u, w) + d(w, v) = 2\mathrm{rad}(G).
        \end{equation*}
    \end{proof}
\end{thmbox}
\section{Graphs and Matrices}
So far we have been using hand-drawn illustrations to aid us in graph visualisation. However, for large graphs with many vertices and edges, drawing out the graph structures becomes nearly impossible. Therefore, we need to use some algebraic tools to help us represent a large graph.
\begin{dfnbox}{Adjacency Matrix}{adMat}
    Let $G$ be a multigraph with $v(G) = n$, then the {\color{red} \textbf{adjacency matrix}} denoted by $\mathbfit{A}(G)$ is an $n \times n$ matrix such that $a_{ij}$ is the number of edges incident to both $v_i, v_j \in V(G)$.
\end{dfnbox}
\begin{notebox}
    \begin{remark}
        For an undirected multigraph, $\mathbfit{A}$ is always symmetric.
    \end{remark}
\end{notebox}
Note that an adjacency matrix does not tell us which edges are incident to a particular vertex, so we may consider the following alternative:
\begin{dfnbox}{Incidence Matrix}{inMat}
    Let $G$ be a multigraph with $v(G) = n$ and $e(G) = m$, then the {\color{red} \textbf{incidence matrix}} denoted by $\mathbfit{M}(G)$ is an $n \times m$ matrix such that
    \begin{equation*}
        m_{ij} = \begin{cases}
            1, &\textrm{if } e_j \textrm{ is incident with } v_i \\
            0, &\textrm{otherwise}
        \end{cases}.
    \end{equation*}
\end{dfnbox}
\begin{notebox}
    \begin{remark}
        For an undirected multigraph, consider an arbitrary $k$-th column of $\mathbfit{M}$. Note that any column will have exactly $2$ non-zero entries. Suppose $m_{ik} = m_{jk} = 1$, it means that $e_k = v_iv_j$.
    \end{remark}
\end{notebox}
For readers who have learnt about basic data structures in computer science, an incidence matrix is just an adjacency list with each list of neighbours converted to a (non-compact) boolean array.
\\\\
Note that the sum of the $i$-th row for both $\mathbfit{A}$ and $\mathbfit{M}$ is the degree of vertex $i$. 
\\\\
Since $\mathbfit{A}$ and $\mathbfit{M}$ carries information about adjacencies between vertices, we can use them to draw some conclusions regarding graph traversal.
\begin{probox}{Counting the Number of Walks between Two Vertices}{numWalks}
    Let $G$ be a simple graph of order $n$ with adjacency matrix $\mathbfit{A}$, then the $(i, j)$ entry of $\mathbfit{A}^k$ is the number of $v_i$-$v_j$ walks of length $k$.
    \tcblower   
    \begin{proof}
        The case where $k = 1$ is trivially true since there is a $v_i$-$v_j$ walk of length $1$ if and only if $v_i$ and $v_j$ are neighbours.
        \\\\
        Let the $(i, j)$ entry of $\mathbfit{A}^m$ be $m_{ij}$. Suppose $m_{ij}$ is the number of $v_i$-$v_j$ walks of length $m$. Then the $(i, j)$ entry of $\mathbfit{A}^{m + 1}$ is given by
        \begin{align*}
            \sum_{r = 1}^{n}m_{ir}a_{rj}.
        \end{align*}
        Notice that $m_{ir}$ is the number of $v_i$-$v_r$ walks of length $m$ and $a_{rj}$ is the number of $v_r$-$v_j$ walks of length $1$, so by Theorem \ref{thm:MP}, $m_{ir}a_{rj}$ is the number of $v_i$-$v_j$ walks of length~$n$ that passes through $v_r$. Therefore, summing up $m_{ir}a_{rj}$ for $r = 1, 2, \cdots, n$ gives the total number of $v_i$-$v_j$ walks of length $n + 1$.
    \end{proof}
\end{probox}
Additionally, we have the following corollary:
\begin{corbox}{Connectedness Test}{}
    Let $G$ be a graph with $v(G) \geq 2$ and adjacency matrix $\mathbfit{A}$. Let
    \begin{equation*}
        \mathbfit{B} = \sum_{i = 1}^{n - 1}\mathbfit{A}^i,
    \end{equation*}
    then $G$ is connected if and only if $b_{ij} \neq 0$ for all $i \neq j$.
\end{corbox}
As it turns out, the incidence and adjacency matrices are in fact related by a simple equation.
\begin{probox}{}{}
    Let $\mathbfit{M}$ and $\mathbfit{A}$ be the incidence matrix and adjacency matrix of a simple graph $G$ respectively, then
    \begin{equation*}
        \mathbfit{M}\mathbfit{M}^{\mathrm{T}} = \begin{bmatrix}
            d_1 & 0 & 0 & \cdots & 0 \\
            0 & d_2 & 0 & \cdots & 0 \\
            0 & 0 & d_3 & \cdots & 0 \\
            \vdots & \vdots & \vdots & \ddots & \vdots \\
            0 & 0 & 0 & \cdots & d_n
        \end{bmatrix} + \mathbfit{A}.
    \end{equation*}
\end{probox}

\section{Tree}
In computer science and data science, it is often useful to use a tree diagram. Colloquially, we view a tree as a graph consisting of nodes, each of which has exactly one parent node and some children nodes. Here, we shall introduce the rigorous mathematical definition for a tree.
\begin{dfnbox}{Tree and Forest}{treeForest}
    A {\color{red} \textbf{forest}} is an acyclic graph. A {\color{red} \textbf{tree}} is a connected forest.
\end{dfnbox}
In other words, a tree is an \textbf{connected acyclic graph}. We also have the following alternative definition for a tree:
\begin{thmbox}{Equivalent Definitions of a Tree}{equivTreeDef}
    The followings are equivalent:
    \begin{enumerate}
        \item $T$ is a tree.
        \item For any $u, v \in V(T)$, there exists a unique $u$-$v$ path in $T$.
        \item $T$ is minimally connected, i.e., $T$ is connected but $T - e$ is disconnected for any $e \in E(T)$.
        \item $T$ is maximally acyclic, i.e., $T$ is acyclic but $T + xy$ for any $x, y \in V(T)$ with $xy \notin E(T)$ is cyclic.
    \end{enumerate}
    \tcblower
    \begin{proof}
        Let $T$ be a tree. For any $u, v \in V(T)$, there exists a $u$-$v$ path $P \subseteq T$. Suppose $P' \subseteq T$ is also a $u$-$v$ path, then there exists a non-trivial closed walk between $u$ and $v$ which contains a cycle, leading to a contradiction.
        \\\\
        Suppose $T$ is such that any $u, v \in V(T)$ are connected by a unique path. Suppose on contrary $T - e$ is connected for some $e = xy \in E(T)$, then there exists some $x$-$y$ path in $T$ which does not contain $xy$. However, $xy$ is a different $x$-$y$ path, which is a contradiction.
        \\\\
        Suppose $T$ is minimally connected. Note that if $T$ contains some cycle $C$, then for any $e \in E(C)$, $T - e$ is connected. So $T$ being minimally connected implies that $T$ is acyclic. Take any $x, y \in V(T)$ with $xy \notin E(T)$, then there exists some $x$-$y$ path $P \subseteq T$ which does not contain $xy$. Therefore, $P + xy$ is a cycle, which implies that $T$ is maximally acyclic.
        \\\\
        Suppose $T$ is maximally acyclic. Suppose $T$ is not connected, then there exists $u, v \in V(T)$ which are disconnected. Note that $T + uv$ is acyclic because $uv$ is the unique $u$-$v$ path. Therefore, $T$ is not maximally acyclic. The contrapositive shows that $T$ being maximally acyclic implies that $T$ is a tree.
    \end{proof}
\end{thmbox}
We first consider the following proposition:
\begin{probox}{Degrees of Vertices in a Tree}{numDeg1}
    Let $T$ be a tree with $\Delta(T) = k$. Let $n_i$ be the number of vertices in $T$ of degree $i$, then
    \begin{equation*}
        n_1 = 2 + \sum_{i = 1}^{k - 2}in_{i + 2}.
    \end{equation*}
    \tcblower
    \begin{proof}
        Let the order of $T$ be $n$, then $n = \sum_{i = 1}^{k}n_i$. By Lemma \ref{lem:handshake}, we have
        \begin{equation*}
            \sum_{v \in V(T)}d(v) = 2(n - 1).
        \end{equation*}
        However, since there are $n_i$ vertices with degree $i$, we have
        \begin{equation*}
            \sum_{v \in V(T)}d(v) = \sum_{i = 1}^{k}in_i.
        \end{equation*}
        Therefore,
        \begin{align*}
            2\sum_{i = 1}^{k}n_i - 2 & = \sum_{i = 1}^{k}in_i \\
            n_1 & = \sum_{i = 3}^{k}in_i \\
            n_1 & = \sum_{i = 1}^{k - 2}in_{i + 2}.
        \end{align*}
    \end{proof}
\end{probox}
The following corollary is an immediate consequence of Proposition \ref{pro:numDeg1}:
\begin{corbox}{Number of End Vertices in a Tree}{numEndVtx}
    A tree of order at least $2$ contains at least $2$ end vertices.
\end{corbox}
In particular, a tree with exactly $2$ end vertices is a doubly linked list. We will now give a name to these vertices in a tree whose degree is $1$.
\begin{dfnbox}{Leaf}{leaf}
    Let $G$ be a tree. A {\color{red} \textbf{leaf}} of $G$ is a vertex $v \in V(G)$ with $d_G(v) = 1$.
\end{dfnbox}
Given any tree, we can fix a vertex as the ``root''. We can carry out depth-first search to traverse the tree's vertices starting from the root. An interesting phenomenon here is that at any point of time, we notice that the vertex which we are currently at is adjacent to exactly one of the vertices which we have visited.
\begin{probox}{Re-labelling of Vertices in a Tree}{relabelTreeVtx}
    Let $T$ be a tree with order $n$, then there exists an ordering of $V(T)$ such that for any $v_k \in V(T)$, $v_k$ is adjacent to exactly one of $v_1, v_2, \cdots, v_{k - 1}$.
    \tcblower
    \begin{proof}
        By Corollary \ref{cor:numEndVtx}, $T$ has a leaf vertex. Let this vertex be $v_n$. Note that $T - v_n \subseteq T$, so it is acyclic. Let $T' = T - v_n$, then $v_n$ has a unique neighbour in $V(T')$. Note that $T'$ is also a tree, so we can repeat this process to take $v_{n - 1}$. Continue doing so until we have taken all vertices in $T$, then for any $k = 1, 2, \cdots, n$, $v_k$ has a unique neighbour in $v_1, v_2, \cdots, v_k$.
    \end{proof}
\end{probox}
In fact, such an ordering of the tree is known as a \textit{topological sort}. 

Intuitively, if a connected graph contains cycles, we can always reduce it to a tree by deleting some edges. This leads to the following result:
\begin{corbox}{Number of Egdes in a Tree}{numEgdesTree}
    A connected graph $T$ is a tree of order $n$ if and only if $e(T) = n - 1$.
    \tcblower
    \begin{proof}
        If $T$ is a tree, then by Proposition \ref{pro:relabelTreeVtx} we can order $V(T)$ such that for all $i = 2, 3, \cdots, n$, $v_i$ is adjacent to exactly one vertex from $v_1, v_2, \cdots, v_{i - 1}$. Therefore, $e(T) = n - 1$.
        \\\\
        If $e(T) = n - 1$, note that $T$ contains a tree $T' \subseteq T$ which has $n$ vertices. Therefore, $e(T') = n - 1 = e(T)$. This implies that $T = T'$, so $T$ is a tree.
    \end{proof}
\end{corbox}
Recall that every graph contains a tree. In fact, this result can be further strengthened.
\begin{probox}{Existence of Subgraph Trees}{existTreeSubgraph}
    Let $T$ be a tree with $k$ vertices. If $G$ is a graph with $\delta(G) \geq k - 1$, then $G$ contains a subgraph which is isomorphic to $T$.
    \tcblower
    \begin{proof}
        Fix an ordering on $V(T)$ such that $v_i$ has a unique neighbour in $v_1, v_2, \cdots, v_{i - 1}$. Take any vertex in $G$ to be $v_1$. Suppose we have chosen $v_1, v_2, \cdots, v_i$ in $G$ for some $i < k$, for any $v_j$ among these vertices, at most $(i - 1)$ of $v_j$'s neighbours have been chosen, but
        \begin{equation*}
            d_G(v_j) \geq \delta(G) = k - 1 > i - 1.
        \end{equation*}
        Therefore, there exists some $u \in N_G(v_j)$ which is not one of $v_1, v_2, \cdots, v_i$. Take $u$ as $v_{i + 1}$ with the edge $uv_j$, then $v_{i + 1}$ is adjacent to exactly one of $v_1, v_2, \cdots, v_i$. This means we can choose all vertices of $T$ from $V(G)$, which means $T \subseteq G$.
    \end{proof}
\end{probox}

\section{Bipartite Graphs}
Consider a tree $T$. Take some $r \in V(T)$ as the root. We define a vertex $v \in V(T)$ to be at depth $k$ from the root if there is a $r$-$v$ path of length $k$. Clearly, if we partition $V(T)$ into vertices of odd depth and even depth, we can easily see that all edges in $T$ joins one vertex from the former to one from the latter.

In fact, there are many graphs which can be bi-partitioned in a similar manner, which are known as \textit{bipartite} graphs.
\begin{dfnbox}{Bipartite Graph}{bipartite}
    A graph $G$ is {\color{red} \textbf{bipartite}} if $V(G)$ can be partitioned into two disjointed subsets $V_1$ and~$V_2$ such that every edge of $G$ joins a vertex in $V_1$ to a vertex in $V_2$. $(V_1, V_2)$ is known as a {\color{red} \textbf{bipartition}} of $G$.
\end{dfnbox}
Note that the size of a bipartite graph $G$ with partite sets $V_1$ and $V_2$ is just
\begin{equation*}
    e(G) = \sum_{x \in V_1}d_G(x) = \sum_{y \in V_2}d_G(y).
\end{equation*}
It can be deduced that the maximal size of a bipartite graph is achieved when every vertex in $V_1$ is connected to every vertex in $V_2$. To define this rigorously, we first introduce the notion of \textit{join}.
\begin{dfnbox}{Join}{join}
    Let $G_1$ and $G_2$ be two disjoint graphs of order $n_1$ and $n_2$ respectively. The {\color{red} \textbf{join}} of $G_1$ and $G_2$, deonoted by $G_1 + G_2$ is the graph such that
    \begin{align*}
        V(G_1 + G_2) & = V(G_1) \cup V(G_2) \\
        E(G_1 + G_2) & = E(G_1) \cup E(G_2) \cup \left\{uv \colon u \in V(G_1), v \in V(G_2)\right\}.
    \end{align*}
\end{dfnbox}
Informally, we can describe the join of $G_1$ and $G_2$ as a graph which contains all the vertices from $G_1$ and $G_2$, all existing edges from~$G_1$ and $G_2$, and new edges connecting every vertex in $G_1$ to every vertex in $G_2$. Thus, we can define a complete bipartite graph using a join.
\begin{dfnbox}{Complete Bipartite Graph}{completeBipartite}
    A {\color{red} \textbf{complete bipartite graph}} is defined as $K_{p, q} = 0_p + 0_q$.
\end{dfnbox}
Consider a $k$-regular bipartite graph. Intuitively, if every vertex in the first partite set must be adjacent to $k > 0$ vertices in the second partite set and vice versa, then it is only natural that the two partite sets have equal cardinalities.
\begin{probox}{Regular Bipartite Graph}{regBip}
    For all $k \in \Z^+$, a $k$-regular bipartite graph with bipartition $(U, V)$ is such that $\abs{U} = \abs{V}$.
    \tcblower
    \begin{proof}
        Note that we have
        \begin{equation*}
            k\abs{U} = e(G) = k\abs{V}.
        \end{equation*}
        Since $k > 0$, this implies $\abs{U} = \abs{V}$.
    \end{proof}
\end{probox}
It is not easy to see whether a graph is bipartite by observation alone, so we consider the following characterisation:
\begin{thmbox}{Bipartite Graph Characterisation}{bipartiteIff}
    A graph is bipartite if and only if it contains no odd cycles.
    \tcblower
    \begin{proof}
        The necessity following from the previous argument, so we only prove the sufficiency. Suppose $G$ contains no odd cycles. Note that there exists some spanning tree $T \subseteq G$ with root $u$. For every $v \in V(T)$, $v$ is coloured red if $d_T(u, v)$ is odd, and blue otherwise.
        \\\\
        Suppose that there exist some $w_1, w_2 \in V(G)$ such that both receive the same colour but $w_1w_2 \in E(G)$. Note that this means that there exist a $w_1$-$u$ path in $T$ and a $w_2$-$u$ path in $T$ with the same parity. This implies that there exists an closed walk of odd length in $G$, and so $G$ contains an odd cycle, which is a contradiction. Therefore, $G$ is bi-colourable and so is bipartite.
    \end{proof}
\end{thmbox}

\chapter{Matchings and Vertex Covers}
\section{Matching}
In set theory, we define the notion of a function
\begin{equation*}
    f \colon X \to Y
\end{equation*}
as a mapping such that every $x \in X$ has a unique image in $Y$. In a graph-theoretical perspective, we can view $X$ and $Y$ as vertex sets and a ``mapping'' in this case can be seen as a set of disjoint edges joining a vertex from $X$ to a vertex from $Y$.
\begin{dfnbox}{Matching}{match}
    A {\color{red} \textbf{matching}} in a graph $G$ is a set $M \subseteq E(G)$ of vertex-disjoint edges.
\end{dfnbox}
The term ``vertex-disjoint edges'' means that for any $e, f \in M$, they are not incident to the same vertex in $G$. Here we define two notions which sound similar:
\begin{dfnbox}{Maximal and Maximum Matchings}{maxMatch}
    A {\color{red} \textbf{maximal matching}} is a matching $M \subseteq E(G)$ such that for all $e \in E(G) - M$, $M \cup \{e\}$ is not a matching.

    A {\color{red} \textbf{maximum matching}} is a matching $M \subseteq E(G)$ such that for any matching $N \subseteq E(G)$, $\abs{N} \leq \abs{M}$.
\end{dfnbox}
It is clear that a maximum matching must be maximal but not the other way round.
\begin{dfnbox}{Matching Number}{matchN}
    The {\color{red} \textbf{matching number}} of a graph $G$ is the size of the maximum matching in $G$, denoted as $\nu(G)$.
\end{dfnbox}
Clearly, for any graph $G$, $\nu(G) \leq \frac{v(G)}{2}$ as the best we can do is to partition $V(G)$ into disjoint pairs of vertices with respect to $E(G)$.
\begin{dfnbox}{Perfect Matching}{perfectMatch}
    Let $G$ be a graph. We say that $G$ has a {\color{red} \textbf{perfect matching}} if $\nu(G) = \frac{v(G)}{2}$.
\end{dfnbox}
\begin{notebox}
    \begin{remark}
        In particular, if $G$ is a bipartite graph with partite sets $A$ and $B$, we say that a matching $M \subseteq G$ is perfect to $A$ if for any $a \in A$, $a \in M$.
    \end{remark}
\end{notebox}
Intuitively, if $G$ has a perfect matching, then we can find a bipartition $(U, V)$ of $V(G)$ such that there exists a bijection between $U$ and $V$.
\begin{dfnbox}{Vertex Covering Number}{vtxCoverN}
    Let $G$ be a graph. A {\color{red} \textbf{vertex cover}} of $G$ is defined to be a set $S \subseteq V(G)$ such that for all $e \in E(G)$, there is some $v \in S$ such that $v \in e$. The {\color{red} \textbf{vertex covering number}} is defined as the size of the smallest vertex cover of $G$, denoted as $\tau(G)$.
\end{dfnbox}
Now let us examine some special graphs:
\begin{itemize}
    \item Complete graph $K_n$: $\nu(K_n) = \left\lfloor\frac{n}{2}\right\rfloor, \tau(K_n) = n - 1$.
    \item Cycle $C_n$: $\nu(C_n) = \left\lfloor\frac{n}{2}\right\rfloor, \tau(C_n) = \left\lceil\frac{n}{2}\right\rceil$.
    \item Path $P_n$: $\nu(P_n) = \left\lfloor\frac{n + 1}{2}\right\rfloor, \tau(P_n) = \left\lceil\frac{n}{2}\right\rceil$.
    \item Complete bipartite graph $K_{m, n}$: $\nu(K_{m, n}) = \tau(K_{m, n}) = \min\{m, n\}$.
    \item Hypercube graph $Q^d$: $\nu(Q^d) = \tau(Q^d) = 2^{d - 1}$.
    \item Petersen graph $P$: $\nu(P) = 5, \tau(P) = 6$.
\end{itemize}
We may have guessed based on the above observation that $\tau(G) \geq \nu(G)$ for all graphs.
\begin{probox}{Vertex Covering Number Is at Most Matching Number}{tauGeqNu}
    For any graph $G$, $\tau(G) \geq \nu(G)$.
    \tcblower
    \begin{proof}
        Fix $M \coloneqq \left\{e_1, e_2, \cdots, e_{\nu(G)}\right\}$ to be a maximum matching in $G$. Let $S \subseteq V(G)$ be any vertex cover of $G$. For any $i = 1, 2, \cdots, \nu(G)$, we have $\abs{e_i \cap S} \geq 1$, otherwise $S$ does not cover $e_i$. Therefore,
        \begin{equation*}
            \tau(G) = \abs{S} = \sum_{i = 1}^{\nu(G)}\abs{e_i \cap S} \geq \nu(G).
        \end{equation*}
    \end{proof}
\end{probox}
\section{Clique and Independent Set}
Given any graph $G$, then theoretically, if we were to find a subgraph of order $n$, the one with the greatest size is a complete graph and that with the smallest size is an empty graph.
\begin{dfnbox}{Clique}{clique}
    A {\color{red} \textbf{clique}} in $G$ is a complete subgraph of $G$. The {\color{red} \textbf{clique number}} of $G$ is the order of the largest clique in $G$, denoted as $\omega(G)$.
\end{dfnbox}
\begin{dfnbox}{Independent Set}{indSet}
    The {\color{red} \textbf{independent set}} of $G$ is a subset $I \subseteq V(G)$ such that for all $u, v \in I$, $uv \notin E(G)$. The {\color{red} \textbf{independence number}} of $G$, denoted as $\alpha(G)$, is the size of the largest independent set in $G$.
\end{dfnbox}
Intuitively, if a set $S \subseteq V(G)$ induces an empty subgraph in $G$, it will induce a complete subgraph in $\overline{G}$. Therefore, $\alpha(G) = \omega(\overline{G})$.

We shall examine the independence numbers of some special graphs.
\begin{itemize}
    \item Complete graph $K_n$: $\alpha(K_n) = 1$.
    \item Cycle $C_n$: $\alpha(C_n) = \left\lfloor\frac{n}{2}\right\rfloor$.
    \item Path $P_n$: $\alpha(P_n) = \left\lfloor\frac{n + 1}{2}\right\rfloor$.
    \item Complete bipartite graph $K_{m, n}$: $\alpha(K_{m, n}) = \max\{m, n\}$.
    \item Hypercube graph $Q^d$: $\alpha(Q^d) = 2^{d - 1}$.
    \item Petersen graph $P$: $\alpha(P) = 4$.
\end{itemize}
The following theorem discusses the numerical relationship between the independence number of vertex covering number:
\begin{probox}{Sum of Independent Number and Vertex Covering Number}{Order=Sum}
    For any graph $G$ of order $n$, $\alpha(G) + \tau(G) = n$.
    \tcblower
    \begin{proof}
        It suffices to prove that $S$ is an independent set in $G$ if and only if $V(G) - S$ is a vertex cover of $G$.
        \\\\
        Let $S \subseteq V(G)$ be an independent set, then for any edge $e = uv \in E(G)$, at least one of $u$ and $v$ is in $V(G) - S$ for otherwise there exists $u, v \in S$ such that $u \sim v$, which is not possible. Therefore, $V(G) - S$ is a vertex cover.
        \\\\
        Suppose conversely that $S \subseteq V(G)$ is such that $V(G) - S$ is a vertex cover.
    \end{proof}
\end{probox}
So far we have been talking a lot about the Petersen graph, but it turns out that the Petersen graph can be generalised into a \textit{Kneser graph}.
\begin{dfnbox}{Kneser Graph}{kneser}
    The {\color{red} \textbf{Kneser graph}} is a graph $KG(n, k)$ such that 
    \begin{align*}
        V\bigl(KG(n, k)\bigr) & = \left\{S \in \mathcal{P}([n]) \colon \abs{S} = k\right\}, \\
        E\bigl(KG(n, k)\bigr) & = \left\{(S, T) \colon S \cap T = \varnothing\right\}.
    \end{align*}
\end{dfnbox}
The vertex set of a Kneser graph is the set of all $k$-element subsets of $\left\{1, 2, \cdots, n\right\}$ and two vertices are adjacent if and only if their corresponding subsets are disjoint.

We introduce a theorem closely related to the Kneser graph:
\begin{thmbox}{Erdos-Ko-Rado Theorem}{ErdosKoRado}
    Let $n \geq 2k$, then any family
    \begin{equation*}
        \mathcal{F} \subseteq \left\{S \in \mathcal{P}([n]) \colon \abs{S} = k\right\}
    \end{equation*}
    with $S \cap T \neq \varnothing$ for all $S, T \in \mathcal{F}$ is such that $\abs{\mathcal{F}} \leq \begin{pmatrix}
        n - 1 \\
        k - 1
    \end{pmatrix}$.
\end{thmbox}
We can see that $\mathcal{F}$ is in fact a subset of $V(KG(n, k))$, so this is equivalent to saying that the biggest subset of $V(KG(n, k))$ such that no vertices are adjacent is $C^{n - 1}_{k - 1}$, which is just to prove that $\alpha(KG(n, k)) = C^{n - 1}_{k - 1}$.

Another property of the independence number is related to the notion of a \textit{line graph}.
\begin{dfnbox}{Line Graph}{lineGraph}
    Let $G$ be a graph. The {\color{red} \textbf{line graph}} of $G$ is defined as $L(G)$ with $V\bigl(L(G)\bigr) = E(G)$ and
    \begin{equation*}
        E\bigl(L(G)\bigr) \coloneqq \left\{(e, f) \in E(G)^2 \colon e \neq f, e \cap f \neq \varnothing\right\}.
    \end{equation*}
\end{dfnbox}
In other words, $L(G)$ uses $E(G)$ as the vertex set and two vertices are adjacent if and only if the respective edges share at least $1$ common vertex in $G$.
\begin{probox}{Independence Number of Line Graphs}{lineGraphAlpha}
    For any graph $G$, $\nu(G) = \alpha\bigl(L(G)\bigr)$.
    \tcblower
    \begin{proof}
        Fix a maximum matching $M \subseteq G$, then for every $e, f \in E(M)$ with $e \neq f$, we have $ef \notin E\bigl(L(G)\bigr)$, therefore, $E(M)$ is an independent set in $L(G)$ and so $\alpha\bigl(L(G)\bigr) \geq \nu(G)$.
        \\\\
        Suppose $\alpha\bigl(L(G)\bigr) > \nu(G)$, then there exists some $e_0 \in V\bigl(L(G)\bigr) = E(G)$ such that $e_0$ is vertex-disjoint with all $e \in E(M)$. However, this means that we can find a matching with bigger size in $G$, which is a contradiction. Therefore, 
        \begin{equation*}
            \nu(G) = \alpha\bigl(L(G)\bigr).
        \end{equation*}
    \end{proof}
\end{probox}
\section{Matchings in Bipartite Graphs}
Let us take a slight detour and unpack the matching number from a different perspective. Note that while forming a matching in a graph $G$, we actually divide $E(G)$ into two subsets, one of which is included in the matching and the other excluded. Now, let us denote an included edge by letting it have a weight of $1$, and assign $0$ weight to all other edges which are unselected.

For every $e \in E(G)$, let $x_e$ denote its weight. This way, to find a maximum matching is in fact equivalent to finding an optimal solution for the \textit{integer program}
\begin{align*}
    \max & \sum_{e \in E(G)}x_e \\
    \textrm{such that} & \quad x_e \in \{0, 1\}, \\
    & \quad \sum_{v \in e}x_e \leq 1 \quad\textrm{for all } v \in V(G),
\end{align*}
where the optimal value is just $\nu(G)$. Similarly, we can assign $0$-$1$ weights, $y_v$, to every $v \in V(G)$, then a vertex cover is equivalent to an optimal solution for the integer program 
\begin{align*}
    \min & \sum_{v \in V(G)}y_v \\
    \textrm{such that} & \quad y_v \in \{0, 1\}, \\
    & \quad \sum_{v \in e}y_v \geq 1 \quad \textrm{for all } e \in E(G),
\end{align*}
where the optimal value is just $\tau(G)$. We can relax both programs into linear programs and it is clear that the $2$ programs form a pair of primal and dual problems. Let the optimal values for the $2$ linear programs be $\nu^*(G)$ and $\tau^*(G)$ respectively. By considering strong duality of linear programs, we can check that
\begin{equation*}
    \nu(G) \leq \nu^*(G) = \tau^*(G) \leq \tau(G).
\end{equation*}
Here $\nu^*(G)$ and $\tau^*(G)$ are known as \textit{fractional matching number} and \textit{fractional vertex covering number} respectively.

It turns out that for bipartite graphs, the above inequality can be further strengthened to strict equality. We shall first discuss a few useful concepts.

Fix some matching $M$ in a bipartite graph $G$. Intuitively, we can find some paths in $G$ which consisting of alternating edges from $M$ and $G - M$.
\begin{dfnbox}{Alternating Path}{altPath}
    Let $G$ be a graph and fix $M \subseteq G$ to be a matching. An {\color{red} \textbf{alternating path}} in $G$ with respect to $M$ is a path $v_1v_2\cdots v_n$ such that $v_1v_2 \in G - M$, $v_kv_{k + 1} \in M$ for all even $k$ and $v_kv_{k + 1} \in G - M$ for all odd $k$.
\end{dfnbox}
Intuitively, if an alternating path contains more edges from $G - M$ than from $M$, we can utilise it to extend the matching $M$ by doing a switch of edges.
\begin{dfnbox}{Augmenting Path}{augPath}
    An {\color{red} \textbf{augmenting path}} with respect to a matching $M$ in $G$ is an alternating path which ends at some $v \in V(G) - V(M)$
\end{dfnbox}
In general, we can characterise a maximum matching by using augmenting paths.
\begin{thmbox}{Maximum Matching Characterisation}{maxMatchChar}
    A matching $M \subseteq G$ is maximum if and only if there is no augmenting path in $G$ with respect to $M$.
    \tcblower
    \begin{proof}
        Suppose $M$ is a maximum matching in $G$. We will prove that there is no augmenting path in $G$ with respect to $M$ by considering the contrapositive statement. Suppose we can find some augmenting path $P \coloneqq v_1v_2\cdots v_n$ in $G$ with respect to $M$, then $P$ must contain an odd number of edges. Consider the subgraph
        \begin{equation*}
            M' \coloneqq M - \left\{v_iv_{i + 1} \in P \colon i \textrm{ is even }\right\} + \left\{v_jv_{j + 1} \in P 
            \colon j \textrm{ is odd }\right\}.
        \end{equation*}
        Note that $M'$ is clearly a matching and $e(M') = e(M) + 1$, so $M$ is not a maximum matching.
        \\\\
        Suppose conversely that there is no augmenting path in $G$ with respect to $M$. We will prove that $M$ is maximum by considering the contrapositive statement. Suppose that $M$ is not maximum, then there exists some matching $M'$ with $e(M') > e(M)$. Consider the subgraph $H \coloneqq M \triangle M'$. Note that $\Delta(H) = 2$ and that every connected component of $H$ is either a path or a cycle whose edges alternate between $E(M)$ and $E(M')$. Note that $H$ is bipartite, so every cycle in $H$ must be even and so contain equal number of edges from $M$ and $M'$. However, since $e(M') > e(M)$, we can find a path in $H$ which begins and ends with edges in $M' - M$ and whose edges alternate between $M$ and $M'$. This path is an augmenting path with respect to $M$.
    \end{proof}
\end{thmbox}
With these preliminary results established, we are now ready to prove \textit{K\"{o}nig's Theorem}.
\begin{thmbox}{K\"{o}nig's Theorem}{Konig}
    If $G$ is bipartite, then $\nu(G) = \tau(G)$.
    \tcblower
    \begin{proof}
        By Proposition \ref{pro:tauGeqNu}, we already know that $\nu(G) \leq \tau(G)$. Therefore, it suffices to prove that $\nu(G) \geq \tau(G)$. Fix $M \subseteq G$ to be a maximum matching. Let $(A, B)$ be a bipartition of $G$, then for every $e \in E(M)$, we have $e = v_Av_B$ for some $v_A \in A$ and $v_B \in B$.
        \\\\
        We select a subset of vertices from $V(M)$ with the following algorithm: for each $v_Av_B \in E(M)$, we take $v_B$ if there exists some alternating path starting at a vertex in $A$ ends at $v_B$. Otherwise, we take $v_A$. Let the vertex set produced this way be $U$. We claim that $U$ is a vertex cover for $G$.
        \\\\
        Take any edge $ab \in E(G)$. If $ab \in E(M)$, then $a \in U$ or $b \in U$. Otherwise, since $M$ is a maximum matching in $G$ which is bipartite, at least one of $a$ and $b$ is matched.
        \\\\
        If $b \in V(M)$, consider $a$. If $a \notin V(M)$, then $ab$ is an alternating path and so $b \in U$. Otherwise, $a \in V(M)$ which reduces the case to the other case.
        \\\\
        If $a \in V(M)$ and $a \in U$ then we are done. Otherwise, $a \notin U$. Suppose $ab' \in E(M)$, then $b' \in U$. This means that we can find some alternating path $P$ with respect to $M$ which ends at $b'$. If $b \in P$, then there is some alternating path with respect to $M$ which ends at $b$. Otherwise, note that $a \notin P$ because $a$ is matched, so $P + ab$ is an alternating path which ends at $b$ with respect to $M$. Since $M$ is maximum, by Theorem \ref{thm:maxMatchChar}, this alternating path is not augmenting, which means that $b \in V(M)$. Therefore, $b \in U$. 
        \\\\
        Therefore, $U$ is a vertex cover for $G$. Note that $\abs{U} = \nu(G)$, so $\tau(G) \leq \nu(G)$. This means that $\nu(G) = \tau(G)$.
    \end{proof}
\end{thmbox}
We can generalise Theorem \ref{thm:Konig} to \textit{fractional vertex covers}.
\begin{dfnbox}{Fractional Vertex Cover}{fracVtxCover}
    The {\color{red} \textbf{fractional vertex cover}} of a non-negatively weighted graph $G$ is defined as the function
    \begin{equation*}
        c \colon V(G) \to \R^+_0
    \end{equation*}
    such that $c(v) + c(u) \geq w(uv)$ for all $u, v \in V(G)$, where $w \colon E(G) \to \R^+_0$ is the weight function.
\end{dfnbox}
We shall state the generalised theorem without proof.
\begin{thmbox}{Egerv\'{a}ry's Theorem}{Egervary}
    For every bipartite $G$, the minimum weight of a fractional vertex cover equals the weight of the matching with the maximum weight.
\end{thmbox}
Suppose $G$ is a bipartite graph with bipartition $(A, B)$. In the case where $\abs{A} \neq \abs{B}$ we cannot get any matching with size greater than $\min\{\abs{A}, \abs{B}\}$ for the obvious reason. Therefore, the best we can hope is to construct a matching \textbf{perfect to one of the partite sets}.

Suppose $S \subseteq A$ is a set of vertices in $G$. We denote the union of the neighbourhoods of all vertices in $S$ by $N_G(S)$. Apparently, if we can find some $S$ such that $\abs{N_G(S)} < \abs{S}$, it means that we cannot find a matching perfect to $A$ because we cannot match every vertex in $S$.

This means that for $G$ to contain a matching perfect to $A$, it necessarily needs to satisfy $\abs{S} \leq \abs{N_G(S)}$ for every $S \subseteq A$. However, it turns out that amazingly, this necessary condition alone is also sufficient to ascertain the existence of a matching perfect to $A$.
\begin{thmbox}{Hall's Theorem}{Hall}
    Let $G$ be a bipartite graph $G$ with bipartition $(A, B)$. $G$ contains a matching perfect to $A$ if and only if for every $S \subseteq A$,
    \begin{equation*}
        \abs{N_G(S)} \geq \abs{S}.
    \end{equation*}
    \tcblower
    \begin{proof}
        Suppose $M \subseteq G$ is a matching perfect to $A$. Let $S \subseteq A$ be any subset. Clearly, for each $a \in S$, there exists a unique $b \in B$ such that $ab \in E(M)$ and if $ab, a'b' \in E(M)$ for $a \neq a'$, we have $b \neq b'$. Therefore, $\abs{N_G(S)} \geq \abs{S}$.
        \\\\
        Suppose that for every $S \subseteq A$, $\abs{N_G(S)} \geq \abs{S}$, we take any matching $M \subseteq G$ such that $A - V(M) \neq \varnothing$. Take $(a_0, b_1, a_1, \cdots)$ to be a finite sequence of pair-wise distinct vertices with greatest length such that $a_0$ is unmatched, $b_i$ is adjacent to some $a_{f(i)} \in \left\{a_0, a_1, \cdots, a_{i - 1}\right\}$ in $G$ and $b_ia_i \in M$ for all $i \geq 1$.
        \\\\
        We claim that this sequence does not end in $A$. Suppose on contrary that the sequence ends at some $a_{n - 1} \in A$. Take $T = \left\{a_0, a_1, \cdots, a_{n - 1}\right\}$, then $\abs{N_G(T)} \geq n$. Therefore, there exists some $b_n \in N_G(T) - \left\{b_1, b_2, \cdots, b_{n - 1}\right\}$, which is a contradiction because we can now extend the length of the sequence. 
        \\\\
        Let $b_k$ be the end of the sequence, then the path $P \coloneqq a_0b_1a_{f(1)}b_2a_{f(2)}\cdots a_{f(k - 1)}b_k$ is an alternating path. If $b_k$ is matched to some $a \in A \cap V(M)$, we know that $a \notin \left\{a_1, a_2,\cdots, a_{k - 1}\right\}$ because that would imply $b_k = b_i$ for some $i \in \left\{1, 2, \cdots, k - 1\right\}$, which is a contradiction. Since $a_0$ is unmatched, we have find a longer sequence satisfying our construction, which is not possible. Therefore, $b_k$ must be unmatched. This means that $P$ is an augmenting path and so by Theorem \ref{thm:maxMatchChar} $M$ is not a maximum matching. This means that the maximum matching in $G$ must be perfect to $A$.
    \end{proof}
\end{thmbox}
Beside the above proof, there are several alternative ways to prove Theorem \ref{thm:Hall}. Here we first present a proof by contradiction using Theorem \ref{thm:Konig}.
\begin{genbox}{Proof for Hall's Theorem by K\"{o}nig's Theorem}
    \begin{proof}
        Observe that it suffices to show that $\nu(G) = \abs{A}$. First, note that since $G$ is bipartite, it is not possible to have $\nu(G) > \abs{A}$ because otherwise there exists an edge between $2$ distinct vertices in $B$. Suppose on contrary that $\nu(G) < \abs{A}$, then by Theorem \ref{thm:Konig}, $\tau(G) = \nu(G) < \abs{A}$. This means that there exists a vertex cover with strictly less vertices than $A$.
        \\\\
        Suppose $U = A' \cup B'$ is a vertex cover for $G$ with $A' \subseteq A$ and $B' \subseteq B$, then $\abs{A'} + \abs{B'} < \abs{A}$. Clearly, there is no edge between $A - A'$ and $B - B'$. Take $S \coloneqq A - A'$, then $N_G(S) \subseteq B'$. Therefore,
        \begin{equation*}
            \abs{N_G(S)} \leq \abs{B'} < \abs{A} - \abs{A'} = \abs{S},
        \end{equation*}
        which is a contradiction.
    \end{proof}
\end{genbox}
Another proof uses an argument by strong induction:
\begin{genbox}{Proof for Hall's Theorem by Strong Induction}
    \begin{proof}
        We shall perform strong induction on $\abs{A}$. The case where $\abs{A} = 1$ is trivial. Suppose that for all $\abs{A} < n$, the theorem holds. Let $G$ be a bipartite graph with bipartition $(A, B)$ such that $\abs{A} = n$. We consider $2$ cases.
        \\\\
        Suppose that for any proper subset $S \subset A$, we have $\abs{N_G(S)} \geq \abs{S} + 1$. Consider an edge $ab \in E(G)$ and define $G' \coloneqq G - \{a, b\}$ be the graph with the vertices~$a$ and $b$ removed. Let $(A', B')$ be the bipartition of $G'$. Notice that for any $S' \subseteq A'$, we have $\abs{N_{G'}(S')} \geq \abs{S'}$. By the inductive hypothesis, this means that $G'$ contains a matching $M'$ perfect to $A'$. Clearly, $M' + ab$ is a matching perfect to $A$ in $G$.
        \\\\
        Suppose otherwise that there exists some proper subset $A' \subset A$ with $\abs{N_G(A')} = \abs{A'}$. Let $B' \coloneqq N_G(A')$ and define
        \begin{equation*}
            G' \coloneqq G[A' \cup B'], \qquad G'' \coloneqq G[(A - A') \cup (B - B')].
        \end{equation*}
        Note that for any $S \subseteq A'$, $N_{G}(S) \subseteq B'$, and so $\abs{N_{G'}(S)} = \abs{N_G(S)} \geq \abs{S}$. By the inductive hypothesis, this implies that $G'$ contains a matching $M'$ perfect to $A'$. For any $T \subseteq A - A'$, we claim that $\abs{N_{G''}(T)} \geq \abs{T}$.
        \\\\
        Suppose on contrary that $\abs{N_{G''}(T)} < \abs{T}$. Note that $N_G(T \cup A') \subseteq N_{G''}(T) \cup B'$, so
        \begin{equation*}
            \abs{N_G(T \cup A')} \leq \abs{N_{G''}(T)} + \abs{B'} < \abs{T} + \abs{A'} = \abs{T \cup A},
        \end{equation*}
        which is a contradiction. Therefore, this implies that $G''$ contains a matching $M''$ perfect to $A - A'$. Note that $M' \cup M''$ is a matching in $G$ perfect to $A$.
    \end{proof}
\end{genbox}
The last proof we are going to present stems from the following observation: suppose we have a graph $G$ satisfying the conditions in Hall's Theorem, we wish to delete ``redundant'' edges until we cannot remove any more edge without violating the condition. We claim that the remaining graph is a matching perfect to $A$.
\begin{genbox}{Proof for Hall's Theorem by Edge-Minimal Subgraph}
    \begin{proof}
        Let $H \subseteq G$ be the subgraph with the smallest size such that for every $S \subseteq A$, $\abs{N_H(S)} \geq \abs{S}$. We shall prove that $H$ is a matching perfect to $A$.
        \\\\
        First, observe that $d_H(a) \geq 1$ for all $a \in A$. Suppose on contrary that $H$ is not a matching perfect to $A$, then there exists some $a \in A$ which has two distinct neighbours $b_1, b_2 \in B$ in $H$. Define $H_1 \coloneqq H - ab_1$ and $H_2 \coloneqq H - ab_2$, then by the minimality of $H$, there exist subsets $A_1, A_2 \subseteq A$ both containing $a$ such that~$\abs{N_{H_1}(A_1)} < \abs{A_1}$ and $\abs{N_{H_2}(A_2)} < \abs{A_2}$. By our construction of $H_1$ and $H_2$, we know that $b_1 \in N_{H_2}(A_2) - N_{H_1}(A_1)$ and $b_2 \in N_{H_1}(A_1) - N_{H_2}(A_2)$. Consider 
        \begin{equation*}
            S \coloneqq A_1 \cap A_2 - \{a\},
        \end{equation*}
        then $N_H(S) \subseteq N_{H_1}(A_1) \cap N_{H_2}(A_2)$. Therefore,
        \begin{align*}
            \abs{N_H(S)} & \leq \abs{N_{H_1}(A_1) \cap N_{H_2}(A_2)} \\
            & = \abs{N_{H_1}(A_1)} + \abs{N_{H_2}(A_2)} - \abs{N_{H_1}(A_1) \cup N_{H_2}(A_2)} \\
            & = \abs{N_{H_1}(A_1)} + \abs{N_{H_2}(A_2)} - \abs{N_{H}(A_1 \cup A_2)} \\
            & \leq \abs{A_1} - 1 + \abs{A_2} - 1 - \abs{N_{H}(A_1 \cup A_2)} \\
            & = \abs{A_1 \cap A_2 - \{a\}} - 1,
        \end{align*}
        which is a contradiction.
    \end{proof}
\end{genbox}
Here we introduce a famous corollary of Hall's Theorem:
\begin{corbox}{Regular Bipartite Graphs Have A Perfect Matching}{regBipMatch}
    If $G$ is a $d$-regular bipartite graph with bipartition $(A, B)$, then $G$ has a perfect matching.
    \tcblower
    \begin{proof}
        It suffices to prove that $G$ contains a matching perfect to $A$. Since $e(G) = d\abs{A} = d\abs{B}$, we have $\abs{A} = \abs{B}$. Take any $X \subseteq A$ and consider the induced subgraph 
        \begin{equation*}
            H \coloneqq G[X \cup N_G(X)].
        \end{equation*}
        Note that for every $v \in N_G(X)$, $d_H(v) \leq d_G(v) = d$, so $d\abs{N_G(X)} \geq e(H) = d\abs{X}$. Therefore, $\abs{X} \leq \abs{N_G(X)}$ and by Theorem \ref{thm:Hall}, $G$ contains a matching perfect to~$A$.
    \end{proof}
\end{corbox}
The above result immediately leads to the following corollary:
\begin{corbox}{Decomposition of Regular Bipartite Graphs}{regBipDecomposition}
    Let $G$ be a $d$-regular bipartite graph, then $E(G)$ can be decomposed into perfect matchings.
    \tcblower
    \begin{proof}
        By Corollary \ref{cor:regBipMatch}, we can find some perfect matching $M_d$ in $G$. Consider the graph $G'$ with $V(G') = V(G)$ and $E(G') = E(G) - E(M_d)$, then $G'$ is a $(d - 1)$-regular bipartite graph and so it contains a perfect matching $M_{d - 1}$. Note that $E(M_{d - 1}) \cap E(M_d) = \varnothing$ and that $M_{d - 1}$ is also a perfect matching in $G$. Repeat this process until we have used all edges in $G$. The set of all perfect matchings obtained this way forms a partition of $E(G)$.
    \end{proof}
\end{corbox}
Can we derive a similar result for any kind of graphs in general? It turns out that this curious question is closely related to the number of connected components with odd orders in a graph. We denote this quantity by $o(G)$ for any graph $G$.
\begin{thmbox}{Tutte's Theorem}{tutte}
    A graph $G$ has a perfect matching if and only if for every $U \subseteq V(G)$, $o(G - U) \leq \abs{U}$.
    \tcblower
    \begin{proof}
        Let $M \subseteq G$ be a perfect matching. Take any $U \subseteq V(G)$ and consider the induced subgraph $H \coloneqq G - U$. Let $C \subseteq H$ be a connected component of odd order. Since $M$ is a perfect matching, then clearly there is at least one vertex $v \in C$ which is matched to some vertex $u \in U$ in $M$. Since every $u \in U$ is matched to exactly one vertex in $M$, this implies that $o(H) \leq \abs{U}$.
        \\\\
        Suppose conversely that for every $U \subseteq V(G)$, $o\bigl(G[V(G) - U]\bigr) \leq \abs{U}$. We shall prove that $G$ contains a perfect matching by considering the contrapositive statement. Let $v(G) = n$. First, we consider the following lemma:
        \begin{lembox}{}{badSet}
            If graphs $G \subseteq G'$ are such that $V(G) = V(G')$ but $E(G) \subseteq E(G')$, then if $U \subseteq V(G')$ is such that $o(G' - U) > \abs{U}$, we have $o(G - U) > \abs{U}$.
            \tcblower
            \begin{proof}
                Note that $G$ is produced by deleting some edges in $G'$. For each edge $e \in E(G') - E(G)$, if $e$ is in an odd component $H_1 \subseteq G'$, then $H_1 - e$ contains at least one odd component in $G$. Otherwise, if $e$ is in an even component $H_2 \subseteq G'$, then $H_2 - e$ may or may not contain any odd component. Therefore, we have
                \begin{equation*}
                    o(G - U) \geq o(G' - U) > \abs{U}.
                \end{equation*}  
            \end{proof}
        \end{lembox}
        By Lemma \ref{lem:badSet}, without loss of generality we can take $G$ to be the edge-maximal graph of order $n$ such that $G$ does not contain any perfect matching. 
        \\\\
        Consider $U \coloneqq \left\{v \colon v \sim u \textrm{ for all } u \in V(G) - v\right\}$, i.e., $U$ consists of vertices in $G$ which are adjacent to every other vertex in $G$, then $G[U]$ is a clique. We claim that every connected component of $G - U$ is complete. Suppose on contrary that there is some connected component $C_i \subseteq G - U$ which is not complete, then there exists distinct vertices $a, a' \in V(C_i)$ such that $a \not\sim a'$. Therefore, there exists an $a$-$a'$ path in $C_i$ of length at least $2$, and so we can find vertices $a, b, c \in V(C_i)$ such that $a \sim b$, $b \sim c$ but $a \not\sim c$. Note that $b \notin U$, so there exists some $d \in V(G)$ such that $b \not\sim d$.
        \\\\
        Define graphs $G_1 \coloneqq G + ac$ and $G_2 \coloneqq G + bd$. By the edge-maximality of $G$, $G_1$ contains a perfect matching $M_1$ and $G_2$ contains a perfect matching $M_2$. Note that $ac \in E(M_1)$, because otherwise $M_1$ is a perfect matching with vertex set $V(G)$ and edge set $E(G)$, which is not possible. Similarly, $bd \in E(M_2)$ Let $P \subseteq G$ be a maximal $d$-$v$ path for some $v \in V(G)$ which is alternating between edges in $M_1$ and $M_2$.
        \\\\
        Suppose $P$ ends with an $M_1$-edge, then $v$ must be matched to $d$ in $M_2$ because otherwise we can find a longer $M_1$-$M_2$ alternating path than $P$, which is not possible. Since $bd \in E(M_2)$, this implies that $v = b$. Note that $M_1 \cap P \subseteq G$ is a perfect matching in $P$ and $M_2 - P - bd$ is a perfect matching in $G - P$, then $(M_1 \cap P) \cup (M_2 - P - bd)$ is a perfect matching in $G$.
        \\\\
        Similarly, suppose $P$ ends with an $M_2$-edge, then $v$ must be either $a$ or $c$, because otherwise there exists an edge $uv \in E(M_1) \cap E(G)$ that can further extend $P$, which is not possible. Without loss of generality, suppose $v = c$, then it is clear that $a \notin P$. We claim that $b \notin P$. Suppose on contrary that $b \in P$, then there must be an $M_1$ edge in $P$ containing $b$ because there is no $M_2$ edge in $P$ containing $b$. However, this means that $P$ ends with an $M_2$ edge, which is a contradiction. Note that $M_2 \cap P \subseteq G$ is a perfect matching in $P$ and $M_1 - P - ac + ab$ is a perfect matching in $G - P$. Therefore, $(M_2 \cap P) \cup (M_1 - P - ac + ab)$ is a perfect matching in $M$.
        \\\\
        Either way, we have reached a contradiction because $M$ does not contain any perfect matching. Therefore, every connected component of $G - U$ is complete. If $U = \varnothing$, then $G - U = G$ is complete. Since $G$ has no perfect matching, $v(G)$ must be odd and so $o(G - U) = 1 > 0 = \abs{U}$. Otherwise, for each connected component $C_i \subseteq G$, if $C_i$ has an even order, it contains a perfect matching. Otherwise, $C_i$ has an odd order. Take some $v_i \in V(C_i)$ and match it to some $u_i \in U$. Note that $C_i - v_i$ is a complete graph of even order and so it contains a perfect matching. Ho
    \end{proof}
\end{thmbox}
Tutte's Theorem has many interesting applications, one of which is related to the concept of an \textit{cut edge} or \textit{cut vertex}.
\begin{thmbox}{Petersen's Theorem}{petersen}
    Every connected $3$-regular graph with no cut edge contains a perfect matching.
    \tcblower
    \begin{proof}
        Let $U \subseteq V(G)$ be non-empty and consider $H \coloneqq G - u$. If $H$ contains no odd-order component, then we are done. Otherwise, fix some odd-order component $C \subseteq H$. Consider
        \begin{equation*}
            \sum_{v \in V(C)}d_G(v) = 3\abs{C} = 2e(G[V(C)]) + e_G(V(C), U).
        \end{equation*}
        Note that the above sum is odd, so $e_G(V(C), U)$ is odd. Since $G$ contains no cut-edge, $e_G(V(C), U) \geq 3$. Let $p$ be the number of odd components in $G - u$ and $M$ be the number of edges between $U$ and $V(G) - U$, then $M \geq 3p$. However, $M \leq 3\abs{U}$ because $G$ is $3$-regular, so $p \leq \abs{U}$. By Theorem \ref{thm:tutte}, $G$ contains a perfect matching.
    \end{proof}
\end{thmbox}
Note that a graph $G = (V, E)$ can actually be viewed as a collection of $2$-element subsets of $V$. Therefore, a natural generalisation is to consider a graph defined as the collection of subsets of any size of $V$.
\begin{dfnbox}{Hypergraph}{hypergraph}
    Let $V$ be a vertex set, a {\color{red} \textbf{hypergraph}} on $V$ is a graph whose edge set is a family
    \begin{equation*}
        E \coloneqq \left\{X \colon X \in \mathcal{P}(V)\right\}.
    \end{equation*}
    If $\abs{X} = k$ for all $X \in E$, we say that the hypergraph is $k$-uniform.
\end{dfnbox}
\chapter{Connectivity}
\section{Vertex Connectivity}
Recall that in previous chapters, we have already defined what a \textit{connected graph} is (Definition \ref{dfn:connected}). However, intuitively, we may feel that graphs may have varying degrees of ``connectedness''. For example, a complete graph seems more connected than a path, since one could easily disconnect a path by deleting $\mathcal{O}(1)$ vertices but needs to delete $\mathcal{O}(n)$ vertices to disconnect $K_n$.

In this chapter, we probe further into the notion of connectivity. In particular, we would like to discuss the strength of connectedness in a graph.
\begin{dfnbox}{Separating Set}{separateSet}
    Let $G$ be a connected graph. $S \subseteq V(G)$ is a {\color{red} \textbf{separating set}} (or {\color{red} \textbf{vertex cut}}) of $G$ if $G - S$ is disconnected. If $S = \{s\}$ is a singleton, then $s$ is known as a {\color{red} \textbf{cut vertex}}.
\end{dfnbox}
Separating set provides an intuitive measure for how connected a graph really is. Obviously, any finite graph must have only finitely many different separating set. Naturally, we could consider the size of the smallest separating set as an indicator for the amount of deletion needed to disconnect the graph.
\begin{dfnbox}{Vertex Connectivity}{vtxConnect}
    The {\color{red} \textbf{vertex connectivity}} of a graph $G$, denoted by $\kappa(G)$ is the cardinality of the smallest subset $S \subseteq V(G)$ such that $G - S$ is either disconnected or a trivial graph. $G$ is said to be {\color{red} \textbf{$k$-connected}} if $\kappa(G) \geq k$.
\end{dfnbox}
Obviously, $\kappa(G) \leq v(G) - 1$ for any graph $G$, and a $k$-connected graph must be $(k - 1)$-connected. Our ``old-fashioned'' notion of a connected graph is simply a $1$-connected graph. As usual, let us examine some special graphs as examples.
\begin{enumerate}
    \item Complete graph $K_n$: $\kappa(K_n) = n - 1$.
    \item Path $P_n$: $\kappa(P_n) = 1$.
    \item Non-trivial cycle $C_n$: $\kappa(C_n) = 2$.
    \item Complete bipartite graph $K_{m, n}$: $\kappa(K_{m, n}) = \min\{m, n\}$.
\end{enumerate}
We may also wish to consider the hypercube graph $Q^d$. Intuitively, $Q^d$ is $d$-regular, so it suffices to fix any $v \in V\left(Q^d\right)$ and delete all of its $d$ neighbours to isolate the vertex. Therefore, $\kappa\left(Q^d\right) \leq d$. However, can we force the inequality to an equality?
\begin{probox}{The Hypercube Graph $Q^d$ is $d$-connected}{hypercubeKappa}
    Every hypercube graph $Q^d$ is $d$-connected.
    \tcblower
    \begin{proof}
        The case where $d = 1$ is trivial. Suppose that for some $d \in \N^+$, $Q^d$ is $d$-connected. Consider $Q^{d + 1}$. Notice that $Q^{d + 1} = Q^+ \cup Q^- \cup M$, where $Q^+$ and $Q^-$ are both $Q^d$ and $M$ is a perfect matching joining vertices in $Q^+$ and those in $Q^-$ with exactly $1$ different digit. Let $S$ be the smallest vertex cut in $Q^{d + 1}$. We consider $2$ cases.
        \\\\
        If $Q^+ - S$ and $Q^- - S$ are both connected, then $V(M) \subseteq S$ because otherwise there is an edge between $Q^+ - S$ and $Q^- - S$ in $Q^{d - 1} - S$ and so $Q^{d - 1} - S$ is connected, which is a contradiction. Therefore,
        \begin{equation*}
            \kappa\left(Q^{d + 1}\right) = \abs{S} \geq \abs{M} = 2^{d} \geq d + 1.
        \end{equation*}
        Otherwise, without loss of generality assume that $Q^+$ is disconnected, then $S \cap Q^+$ is a vertex cut in $Q^+$. By the inductive hypothesis, $\abs{S \cap Q^+} \geq d$. We claim that $\abs{S \cap Q^-} \geq 1$. Suppose on contrary that $\abs{S \cap Q^-} = 0$, then $S \cap V\left(Q^-\right) = \varnothing$. This means that $Q^- - S$ is connected. Note that for every $v \in V\left(Q^+ - S\right)$, there exists some $u \in V\left(Q^- - S\right)$ such that $uv \in E\left(Q^{d + 1} - S\right)$, because otherwise $S \cap V\left(Q^-\right) \neq \varnothing$. Therefore, for any $v_1, v_2 \in V\left(Q^+ - S\right)$, we can find $u_1, u_2 \in V\left(Q^- - S\right)$ such that there is a $v_1$-$v_2$ path in $Q^{d + 1}$ containing $v_1u_1$ and $v_2u_2$. Therefore, $Q^{d + 1} - S$ is connected, which is a contradiction. Therefore, 
        \begin{equation*}
            \kappa\left(Q^{d + 1}\right) = \abs{S} = \abs{S \cap Q^+} + \abs{S \cap Q^-} \geq d + 1.
        \end{equation*}
        This means that $Q^{d + 1}$ is $(d + 1)$-connected.
    \end{proof}
\end{probox}
Next, we will focus on $2$-connected graph, i.e., graphs which we need to delete at least $2$ vertices to disconnect. We will start with a straight-forward proposition.
\begin{probox}{$2$-connected Graphs Contain A Cycle}{cycleIn2Connected}
    Every $2$-connected graph contains a cycle.
    \tcblower
    \begin{proof}
        Suppose on contrary that there exists a $2$-connected graph $T$ which is acyclic. Note that $T$ is connected, so $T$ must be a tree. By Theorem \ref{thm:equivTreeDef}, $T - uv$ is disconnected for any $uv \in E(T)$. If both $u$ and $v$ are leaves, then $T$ is an edge, which implies that deleting any vertex reduces $T$ to a trivial graph. Otherwise, without loss of generality, assume $u$ is not a leaf, and so $T - u$ is disconnected. Either way, this contradicts with the fact that $\kappa(T) \geq 2$.
    \end{proof}
\end{probox}
It should be taken note of that the converse of Proposition \ref{pro:cycleIn2Connected} is not true in general. Let $C_n$ be a cycle, then adding a vertex $v$ into $C_n$ by joining it with any $u \in V(C_n)$ produces a graph $H$ containing a cycle. However, $H - u$ is disconnected so $\kappa(H) = 1$.

An intuition about connectivity is that, if we were to increase the connectivity of a graph, we should try to make each vertex to have as many incident edges as possible.
\begin{probox}{High Connectivity Implies High Minimum Degree}{highKappaThenHighDelta}
    For any graph $G$, $\kappa(G) \leq \delta(G)$.
    \tcblower
    \begin{proof}
        Take some $v \in V(G)$ with $d_G(v) = \delta(G)$, then clearly $N_G(v)$ is a vertex cut because $v$ is an isolated vertex in $G - N_G(v)$, and so $\kappa(G) \leq \abs{N_G(v)} = \delta(G)$.
    \end{proof}
\end{probox}
The above trivial proposition shows that a graph with high connectivity has high minimum degree. However, the converse is clearly not true. Consider the graph $G$ formed by $2$ disjoint copies of $K_n$, then clearly $\delta(G) = n - 1$ but $\kappa(G) = 0$, i.e., $G$ is not even connected.

Recall that in Proposition \ref{pro:highAvgDeg}, while we cannot ensure a graph with a high average degree to have a high minimum degree, we have proved that the graph will contain a subgraph of a high minimum degree. In the same spirit, given a graph with a high minimum degree, we can try to find a subgraph with a high connectivity.
\begin{thmbox}{Mader's Theorem}{Mader}
    For every graph $G$, if $\bar{d}(G) \geq 4k$, then $G$ contains a $k$-connected subgraph.
    \tcblower
    \begin{proof}
        For $k = 1$, we have $\bar{d}(G) \geq 4$. This means that $G$ is non-empty and so it contains an edge which is a $1$-connected subgraph.
        \\\\
        For $k \geq 2$, we consider the following claims: first, suppose $v(G) < 2k - 1$, then $\bar{d}(G) \leq v(G) - 1 < 2k - 2$. Therefore, if $\bar{d}(G) \geq 4k > 2k - 2$, we have $v(G) \geq 2k - 1$. On the other hand, notice that
        \begin{equation*}
            e(G) = \frac{v(G)\bar{d}(G)}{2} \geq (2k - 3)\bigl(v(G) - k + 1\bigr) + 1.
        \end{equation*}
        Therefore, it suffices to prove that whenever $k > 2$, if $v(G) \geq 2k - 1$ and $e(G) \geq (2k - 3)\bigl(v(G) - k + 1\bigr) + 1$, then $G$ contains a $k$-connected subgraph.
        \\\\
        For $v(G) = 2k - 1$, we have 
        \begin{equation*}
            e(G) \geq k(2k - 3) + 1 = (2k - 1)(k - 1) = \begin{pmatrix}
                2k - 1 \\
                2
            \end{pmatrix}.
        \end{equation*}
        However, we know that $e(G) \leq \left(\begin{smallmatrix}
            v(G) \\
            2
        \end{smallmatrix}\right)$, so $G \cong K_{2k - 1}$. Therefore, $\kappa(G) = 2k - 2 \geq k$ and so $G$ is $k$-connected.
        \\\\
        For $v(G) = n \geq 2k$, suppose that for all $v(G) = m \in [2k - 1, n - 1]$, $G$ contains a $k$-connected subgraph. We consider $2$ cases. If $\delta(G) \leq 2k - 3$, then we can take some $v \in V(G)$ with $d_G(v) = \delta(G)$. Consider
        \begin{align*}
            e(G - v) & = e(G) - \delta(G) \\
            & \geq (2k - 3)(n - k + 1) + 1 - (2k - 3) \\
            & = (2k - 3)\bigl((n - 1) - k + 1\bigr) + 1.
        \end{align*}
        By the inductive hypothesis, $G - v$ contains a $k$-connected subgraph $H$. Since $G - v \subseteq G$, $H \subseteq G$.
        \\\\
        If $\delta(G) \geq 2k - 2 > 2k - 3$, we again consider $2$ cases. If $G$ is $k$-connected, then we are done. Otherwise, $G$ contains a vertex cut with size at most $k - 1$. Therefore, we can consider $G_1, G_2 \subseteq G$ such that $G_1 \cup G_2 = G$ and $V(G_1) \cap V(G_2)$ is a vertex cut such that $\abs{V(G_1) \cap V(G_2)} \leq k - 1$. Notice that this implies that there is no edge between $V(G_1) - V(G_2)$ and $V(G_2) - V(G_1)$. Consider any $v \in V(G_1)$. Note that $N_G(v) \subseteq V(G_1)$ and $d_G(v) \geq \delta(G) \geq 2k - 2$, so $\abs{V(G_1)} \geq d_G(v) + 1 = 2k - 1$. Similarly, $\abs{V(G_2)} \geq 2k - 1$. Suppose that 
        \begin{align*}
            e(G_1) \leq (2k - 3)\bigl(v(G_1) - k + 1\bigr), \\
            e(G_2) \leq (2k - 3)\bigl(v(G_2) - k + 1\bigr),
        \end{align*}
        then 
        \begin{align*}
            e(G) & \leq e(G_1) + e(G_2) \\
            & \leq (2k - 3)\bigl(v(G_1) + v(G_2) - 2k + 2\bigr) \\
            & = (2k - 3)\bigl(n + \abs{V(G_1) \cap V(G_2)} - 2k + 2\bigr) \\
            & \leq (2k - 3)\bigl(n + k - 1 - 2k + 2\bigr) \\
            & = (2k - 3)(n - k + 1),
        \end{align*}
        which is a contradiction. Therefore, without loss of generality, we have 
        \begin{equation*}
            e(G_1) \geq (2k - 3)\bigl(v(G_1) - k + 1\bigr) + 1.
        \end{equation*}
        Therefore, by the inductive hypothesis, $G_1$ contains a $k$-connected subgraph $H$. Since $G_1 \subseteq G$, $H \subseteq G$. Therefore, any graph $G$ with $\bar{d}(G) \geq 4k$ contains a $k$-connected subgraph.
    \end{proof}
\end{thmbox}
Since a high minimum degree implies a high average average degree, Theorem \ref{thm:Mader} in fact guarantees that a graph with a high minimum degree will have a high connectivity.
\section{Edge Connectivity}
Since we can measure connectivity in terms of the minimum number of vertices which need to be deleted to disconnected a graph, it is only natural to consider from the perspective of edges.
\begin{dfnbox}{Edge Connectivity}{edgeConnectivity}
    A graph $G$ is said to be {\color{red} \textbf{$\ell$-edge-connected}} if for any $F \subseteq E(G)$ with $\abs{F} < \ell$, $G - F$ is connected. The {\color{red} \textbf{edge connectivity}} of $G$, denoted by $\kappa'(G)$, is defined as the size of the smallest subset $F \subseteq E(G)$ such that $G - F$ is disconnected.
\end{dfnbox}
Similar to separating sets, we can analogously define a set of edges which will disconnect the graph when removed.
\begin{dfnbox}{Disconnecting Set}{disconnectingSet}
    Let $G$ be a graph, a set $F \subseteq E(G)$ is called a {\color{red} \textbf{disconnecting set}} if $G - F$ is disconnected.
\end{dfnbox}
However, as opposed to vertex cuts, the notion of an \textit{edge cut} is slightly different.
\begin{dfnbox}{Edge Cut}{edgeCut}
    Let $G$ be a graph and $\left(S, \bar{S}\right)$ be a partition for $V(G)$, an {\color{red} \textbf{edge cut}} in $G$ with respect to $\left(S, \bar{S}\right)$ is defined as 
    \begin{equation*}
        E_G\left(S, \bar{S}\right) \coloneqq \left\{xy \in E(G) \colon x \in S, y \in \bar{S}\right\}.
    \end{equation*}
    If an edge cut contains only a single edge, that edge is known as a {\color{red} \textbf{cut edge}} or {\color{red} \textbf{bridge}}.
\end{dfnbox}
Intuitively, any edge set is a disconnecting set, but not every disconnecting set is an edge cut. However, it turns out that every \textbf{minimal disconnecting set} is an edge cut! Let us do the routine examination on the special graphs.
\begin{enumerate}
    \item Complete graph $K_n$: $\kappa'(K_n) = n - 1$.
    \item Path $P_n$: $\kappa'(P_n) = 1$.
    \item Non-trivial cycle $C_n$: $\kappa'(C_n) = 2$.
    \item Complete bipartite graph $K_{m, n}$: $\kappa'(K_{m, n}) = \min\{m, n\}$.
    \item Hypercube graph $Q^d$: $\kappa'\left(Q^d\right) = d$.
\end{enumerate}
Notice that in all of these examples, $\kappa = \kappa'$. If for any graph $G$, $\kappa(G) = \kappa'(G)$, things will not be very interesting. However, this is luckily not the case. Consider a graph formed by $2$ complete graphs of order $n$ with exactly one common vertex. One may check that we only need to remove the shared vertex to disconnect the graph but need to delete at least $(n - 1)$ edges to do the same.

In fact, we can derive the following relation:
\begin{probox}{Numerical Relationship between Connectivities}{relateConnectivity}
    For every graph $G$, $\kappa(G) \leq \kappa'(G) \leq \delta(G)$.
    \tcblower
    \begin{proof}
        If $G$ is disconnected, the result is trivial. 
        \\\\
        Otherwise, let $v \in V(G)$ be such that $d_G(v) = \delta(G)$. Note that deleting all edges incident to $v$ will isolate the vertex, so $\kappa'(G) \leq \delta(G)$.
        \\\\
        Let $E\left(S, \bar{S}\right)$ be an edge cut with $\abs{E\left(S, \bar{S}\right)} = \kappa'(G)$. We consider $2$ cases. If $E\left(S, \bar{S}\right)$ induces a complete graph, then 
        \begin{align*}
            \kappa'(G) & = \abs{S}\abs{\bar{S}} \\
            & = \abs{S}\left(v(G) - \abs{\bar{S}}\right) \\
            & \geq v(G) - 1 \\
            & \geq \kappa(G).
        \end{align*}
        Otherwise, there exist some $x \in S$ and $y \in \bar{S}$ such that $xy \notin E\left(S, \bar{S}\right)$. Define
        \begin{equation*}
            U \coloneqq N_G(x) \cap \bar{S} \cup \left\{v \in S - \{x\} \colon N_G(v) \cap \bar{S} \neq \varnothing\right\}.
        \end{equation*}
        Consider $H \coloneqq G - U$. Notice that there is no $x$-$y$ path in $H$ because $E_H\left(S - U, \bar{S}\right) = \varnothing$. Therefore, $H$ is disconnected and so $U$ is a vertex cut. Therefore,
        \begin{equation*}
            \kappa(G) \leq \abs{U} \leq \abs{S} \leq \abs{E\left(S, \bar{S}\right)} = \kappa'(G).
        \end{equation*}
        Therefore, $\kappa(G) \leq \kappa'(G) \leq \delta(G)$.
    \end{proof}
\end{probox}
\begin{notebox}
    \begin{remark}
        Note that there exists some graph $G$ with $\kappa(G) < \kappa'(G) < \delta(G)$.
    \end{remark}
\end{notebox}
Lastly, we will state the following theorem without proof:
\begin{thmbox}{Chartrad-Haray Theorem}{ChartradHaray}
    For all positive integers $a, b, c$ with $a \leq b \leq c$, there exists a graph $G$ with $\kappa(G) = a$, $\kappa'(G) = b$ and $\delta(G) = c$.
\end{thmbox}
\section{Blocks}
We know that every graph can be naturally decomposed into connected components. However, can we actually present a stronger definition? Here, we consider components of a graph that are $2$-connected.
\begin{dfnbox}{Block}{block}
    For any graph $G$, a {\color{red} \textbf{block}} in $G$ is defined as a maximal connected subgraph $B \subseteq G$ without any cut vertex.
\end{dfnbox}
\begin{notebox}
    \begin{remark}
        We define an edge $K_2$ and a trivial graph both as blocks.
    \end{remark}
\end{notebox}
Let us make some observations regarding the properties of a block.

First, observe that a block is either an isolated vertex, a cut edge or a $2$-connected graph. Next, we shall characterise the vertices of blocks.
\begin{probox}{Blocks Share at Most One Common Vertex}{blockIntersectOne}
    Let $B_i$ and $B_j$ be distinct blocks of $G$, then $\abs{V(B_i) \cap V(B_j)} \leq 1$.
    \tcblower
    \begin{proof}
        Suppose on contrary that $\abs{V(B_i) \cap V(B_j)} \geq 2$. Consider $B \coloneqq B_i \cup B_j$. Note that for any vertex $v \in V(B_i) \triangle V(B_j)$, $v$ is not a cut vertex of $B$. Take any $u_1, u_2 \in V(B_i) \cap V(B_j)$ and consider $B - u_1$. Let $w_1, w_2 \in V(B - u_1)$ be any vertices. If $w_1, w_2 \in V(B_i)$ or $w_1, w_2 \in V(B_j)$, then they are connected in $B - u_1$. Otherwise, without loss of generality, assume $w_1 \in V(B_i) - V(B_j)$ and $w_2 \in V(B_j) - V(B_i)$. Note that there exist a $u_2$-$w_1$ path and a $u_2$-$w_2$ path in $B - u_1$, so $w_1$ and $w_2$ are connected in $B - u_1$. Therefore, $B$ does not have a cut vertex and so is a block, but $B_i, B_j \subset B$, which is a contradiction.
    \end{proof}
\end{probox}
The above result allows us to show the following corollary about the edges in a graph with respect to its blocks:
\begin{corbox}{Every Edge Is in A Unique Block}{edgeInBlock}
    Let $G$ be a graph. Every $e \in E(G)$ is in a unique block of $G$.
    \tcblower
    \begin{proof}
        First, notice that every edge is a connected subgraph of $G$ without any cut vertex, so any edge $e \in E(G)$ must be in some block of $G$. Suppose on contrary that there exist distinct blocks $B_i, B_j \subseteq G$ such that $e \in E(B_i) \cap E(B_j)$, then $\abs{V(B_i) \cap V(B_j)} \geq 2$, which contradicts Proposition \ref{pro:blockIntersectOne}.
    \end{proof}
\end{corbox}
By the above corollary, we know that no two different blocks can share common edges.

It is clear that every graph can be decomposed into blocks. We would like to further study block decomposition and hopefully derive a relation between the structure of blocks and that of the graph itself.
\begin{dfnbox}{Block Graph}{blockGraph}
    Let $G$ be a graph. Define $\mathcal{A}$ to be the set of cut vertices of $G$ and $\mathcal{B}$ to be the set of blocks in $G$. A {\color{red} \textbf{block graph}} on $G$ is a bipartite graph with bipartition $(\mathcal{A}, \mathcal{B})$ such that a cut vertex $a \in \mathcal{A}$ is adjacent to $B \in \mathcal{B}$ if $a \in V(B)$.
\end{dfnbox}
Block graph might be a bit abstract to visualise. To better understand its structure, let us consider the following result:
\begin{probox}{A Block Graph on A Connected Graph Is A Tree}{blockGraphTree}
    Let $T$ be a block graph of a connected graph $G$, then $T$ is a tree.
    \tcblower
    \begin{proof}
        Take any $B_1, B_n \in \mathcal{B}$. Note that there exists a sequence $(B_1, B_{2}, \cdots, B_{n - 1}, B_n)$ of blocks in $G$ such that $B_i$ and $B_{i + 1}$ share a common vertex. Denote this vertex as $v_i$, then clearly, $v_i$ is adjacent to $B_i$ and $B_{i + 1}$ in $T$. Therefore, $B_1v_1B_2v_2\cdots B_{n - 1}v_{n - 1}B_n$ is a $B_1$-$B_n$ walk in $T$, and so $B_1$ and $B_n$ are connected. Take any $a \in \mathcal{A}$ and any $B \in \mathcal{B}$. Note that there exists some $B' \in \mathcal{B}$ such that $a \sim B'$. Since $B$ and $B'$ are connected, $a$ and $B$ are connected. Similarly, one can check that any $a, a' \in \mathcal{A}$ are connected in~$T$. Therefore, $T$ is connected.
        \\\\
        We will proceed to proving that $T$ is acyclic by contradiction. Suppose on contrary that $T$ contains a cycle. Without loss of generality, we can re-label the vertices such that the cycle is in the form of  $a_1B_1a_2B_2\cdots a_nB_na_{n + 1}$ with $a_1 = a_{n + 1}$. This means that for any integer $k \in [1, n]$, $a_k, a_{k + 1} \in V(B_k)$ and that $V(B_{k - 1}) \cap V(B_k) = a_k$ whenever $k \geq 2$. Consider $B = \bigcup_{i = 1}^{n}B_i$. Notice that $B$ is connected and does not have any cut vertex, so $B$ is a block in $G$. However, for any of the $B_i$'s, $B_i \subset B$, which contradicts the maximality of $B_i$. Therefore, $T$ is acyclic, which implies that $T$ is a tree.
    \end{proof}
\end{probox}
To summarise, we see that for any connected graph $G$, its block graph is a tree such that every $B \in \mathcal{B}$ is either an edge or a $2$-connected graph.

We introduce a useful definition which will help us investigate the structure of $2$-connected graphs later.
\begin{dfnbox}{$H$-path}{HPath}
    Let $G$ be a graph with a subgraph $H$. An {\color{red} \textbf{$H$-path}} is a path $P$ in $G$ such that its $2$ ends are the only vertices in $P \cap H$.
\end{dfnbox}
Recall that in Proposition \ref{pro:cycleIn2Connected}, we have shown that every $2$-connected graph contains a cycle, but that the converse is not true. However, can we try to extend from this necessary condition somehow to give a more precise characterisation of a $2$-connected graph?
\begin{thmbox}{Ear-Decomposition}{2ConnectedGraph}
    A graph $G$ is $2$-connected if and only if it can be constructed from a cycle by successively augmenting the partially constructed graph $H$ with $H$-paths.
    \tcblower
    \begin{proof}
        Suppose that $G$ is constructed from a cycle by successively augmenting $H$-paths to a partially constructed graph $H$. Note that a cycle is $2$-connected. Assume that there exists some $k \in \N$ such that after $k$ iterations of such augmentation, the resulted graph $H$ is $2$-connected. Consider $H' \coloneqq H \cup P_H$ where $P_H$ is any $H$-path, the clearly $H'$ is also $2$-connected. Therefore, $G$ must be $2$-connected.
        \\\\
        Suppose conversely that $G$ is $2$-connected, then by Proposition \ref{pro:cycleIn2Connected} it contains a cycle $C_n$. Therefore, $G$ contains a maximal subgraph $H$ which can be constructed by augmenting $H$-paths successively to a partially constructed graph starting from $C_n$. Note that it suffices to show that $G = H$.
        \\\\
        We claim that $H$ must be an induced subgraph of $G$. Suppose on contrary that there exists $u, v \in V(H)$ such that $uv \in E(G)$ but $uv \notin E(H)$, then clearly $uv$ itself is an $H$-path in $G$. Therefore, $H + uv$ is a larger graph constructible from $C_n$ by successively augmenting $H$-paths to the partially constructed graph, which is a contradiction because $H$ is assumed to be maximal.
        \\\\
        Suppose on contrary that $G \neq H$, then $V(G) - V(H) \neq \varnothing$. Note that $G$ is connected, so there exist $v \in V(G) - V(H)$ and $w \in V(H)$ such that $vw \in E(G)$. Note that $C_n \subseteq H$, so there exists some $u \in V(H)$ with $u \neq w$. Since $G$ is $2$-connected, we know that $G - w$ is connected. This implies that there is a $v$-$u$ path in $G$ which does not contain $w$. Without loss of generality, let $u'$ be the first vertex in the path which is in $V(H)$, the we have found a $v$-$u'$ path which is an $H$-path. This means we can construct a larger graph $H'$ by augmenting the $H$-path to $H$, which contradicts the maximality of $H$.
    \end{proof}
\end{thmbox}
An $H$-path is vividly termed as an ``ear'' to the graph. Clearly, Theorem \ref{thm:2ConnectedGraph} shows that any $2$-connected graph can be decomposed into a cycle and a series of $H$-paths. This is thus also known as the ``ear-decomposition'' of $G$.

We would like to take one step further and probe into $3$-connected graphs. We will state the following theorem by Tutte without proof:
\begin{thmbox}{Tutte's Characterisation of $3$-connected Graphs}{Tutte3Connected}
    A graph $G$ is $3$-connected if and only if there is a sequence $\left(G_0, G_1, \cdots, G_n\right)$ with $G_n \cong G$ such that $G_0 = K_4$ and $G_{i + 1}$ contains an edge $xy$ with $d_{G_{i + 1}}(x), d_{G_{i + 1}}(y) \geq 3$ and $G_i = G_{i + 1} \circ xy$. 
\end{thmbox}
Ear-decomposition is a fairly nice construction for a $2$-connected graph. However, as a characterisation it is not very intuitive, in a sense that one still needs to work towards the graph from a cycle to actually verify $2$-connectivity. Can we do better and propose a more direct characterisation?
\begin{thmbox}{Whitney's Characterisation of $2$-connected Graphs}{Whitney}
    A graph $G$ is $2$-connected if and only if for every $u, v \in V(G)$, there exist $2$ internally disjoint $u$-$v$ paths in $G$.
\end{thmbox}
Here, ``internally disjoint'' paths means that the paths share no other vertices except $u$ and $v$. More generally, we define the following:
\begin{dfnbox}{$A$-$B$ Path}{ABPath}
    Let $G$ be a graph with $A, B \subseteq V(G)$. An {\color{red} \textbf{$A$-$B$ path}} is a path in $G$ which contains exactly one vertex in $A$ and one vertex in $B$.
\end{dfnbox} 
Note that in the above definition, we do not require $A$ and $B$ to be disjoint. Clearly, if $A \cap B \neq \varnothing$, then every vertex in $A \cap B$ is a trivial $A$-$B$ path. Intuitively, to disconnect $G$, we necessarily need to disconnect $A$ and $B$, i.e., we need to make sure that there is no $A$-$B$ path in $G$.
\begin{dfnbox}{Separating Set for $(A, B)$}{ABSeparate}
    Let $G$ be a graph with $A, B \subseteq V(G)$. A {\color{red} \textbf{separating set}} for $(A, B)$ is a set $X$ consisting of both edges and vertices such that there is no $A$-$B$ path in $G - X$.
\end{dfnbox}
Notice that we have not given the proof for Theorem \ref{thm:Whitney} yet. One may choose to use ear-decomposition to prove the theorem, but we can also see it as a corollary of another theorem. However, before proving the actual theorem, we first consider the following lemma:
\begin{lembox}{$A$-$B$ Paths Extension}{moreABPaths}
    Let $G$ be a graph with $A, B \subseteq V(G)$ such that the minimum separating set for $(A, B)$ has size $k$, then for any set $P$ of less than $k$ vertex-disjoint $A$-$B$ paths in $G$, there is a set $Q$ of vertex-disjoint $A$-$B$ paths in $G$ whose set of end vertices contains the set of end vertices in $P$ as a proper subset.
    \tcblower
    \begin{proof}
        Fix any $A \subseteq V(G)$. We will perform induction on $\abs{V(G)} - \abs{B}$. Note that when $\abs{V(G)} - \abs{B} = 0$, we have $B = V(G)$ and so $k = \abs{A}$. If $\abs{P} < k$, then there is at least one vertex $a \in A$ which is not contained by any path in $P$. Take $Q \coloneqq P \cup \{a\}$ then $Q$ is a set of $\abs{P} + 1$ vertex disjoint $A$-$B$ paths in $G$ whose set of end vertices contains the end vertices in $P$ as a proper subset.
        \\\\
        Take any $A, B \subseteq V(G)$ and fix $P \coloneqq \left\{P_1, P_2, \cdots, P_n\right\}$ to be a set of $n < k$ vertex-disjoint $A$-$B$ paths. We denote the set of end vertices in $P$ by $P_A \coloneqq \left\{a_1, a_2, \cdots, a_n\right\}$ and $P_B \coloneqq \left\{b_1, b_2, \cdots, b_n\right\}$. Clearly, $P_B$ is not a separating set for $(A, B)$, so there is some $A$-$B$ path which does not contain any vertex from $P_B$. Let this path be $R$ with end vertex $b_R \in B$. If $R$ is vertex-disjoint to every $P_i \in P$, then we are done.
        \\\\
        Otherwise, $R \cap P_j \neq \varnothing$ for some $P_j \in P$. Without loss of generality, let $x \in V(R)$ be the vertex such that the $x$-$b_R$ path in $R$ contains no vertex from $P$, i.e., $x$ is the last intersection between $R$ and the paths in $P$. Without loss of generality, we can re-label the paths in $P$ such that $x \in P_n$. Denote by $xR$ the $x$-$b_R$ path in $R$ and by $xP_n$ the $x$-$b_n$ path in $P_n$. Define $B' \coloneqq B \cup V(xR) \cup V(xP_n)$, then $B \subsetneqq B'$ and so $\abs{V(G)} - \abs{B'} < \abs{V(G)} - \abs{B}$. Let $P' \coloneqq \left\{P_1, P_2, \cdots, P_{n - 1}, P_nx\right\}$, then $P'$ is a set of $n$ vertex-disjoint $A$-$B'$ paths in $G$. Now, take any $X$ to be a minimum separating set for $(A, B')$, then clearly $X$ separates $(A, B)$, so $\abs{X} \geq k > n$. By the inductive hypothesis, we can find a set $Q'$ of vertex-disjoint $A$-$B'$ paths in $G$ such that the set of end vertices of $Q'$ contains the end vertices of $P$ as a proper subset.
        \\\\
        Let $P'_{B'} \coloneqq \left\{b_1, b_2, \cdots, b_{n - 1}, x\right\}$ be the set of end vertices in $B'$ for paths in $P'$. Let $Q' \coloneqq \left\{Q'_1, Q'_2, \cdots, Q'_{n + 1}\right\}$ and denote the end vertex of $Q'_i$ in $B$ by $B(Q'_i)$, then without loss of generality, we can re-label the vertices such that 
        \begin{equation*}
            B(Q'_i) = \begin{cases}
                b_i & \textrm{if } i = 1, 2, \cdots, n - 1 \\
                x & \textrm{if } i = n \\
                y & \textrm{if } i = n + 1
            \end{cases},
        \end{equation*}
        where $y \notin P'_{B'}$. Note that $Q'_1, Q'_2, \cdots, Q'_{n - 1}$ are already vertex-disjoint $A$-$B$ paths. We will try to extend $Q'$ to obtain $Q$. We consider $2$ cases on $y$.
        \\\\
        If $y \in xP_n$, then every vertex in $V(xR) - \{x\}$ is unused by all paths in $Q'$, and so we can extend $Q'_n$ by $xR$ to obtain an $A$-$B$ path, and extend $Q'_{n + 1}$ by $yP_n$ to obtain another $A$-$B$ path. Note that $xR$ and $yP_n$ are vertex-disjoint, so we have obtained a set $Q$ of $(n + 1)$ vertex-disjoint $A$-$B$ paths as desired.
        \\\\
        If $y \notin xP_n$, then we can extend $Q'_n$ by $xP_n$ to connect $b_n$. Note that either $y \in yR$ or $y \in B$, but either way we can extend $Q'_{n + 1}$ to a new vertex in $B$, so we still can obtain $Q$ as desired.
    \end{proof}
\end{lembox}
Now we are ready to state the following theorem:
\begin{thmbox}{Menger's Theorem}{Menger}
    Let $G$ be a graph with $A, B \subseteq V(G)$, then the minimum size of separating sets for $(A, B)$ equals the maximum number of vertex-disjoint $A$-$B$ paths in $G$.
    \tcblower
    \begin{proof}
        Let $X$ to be a minimum separating set for $(A, B)$ and $k$ to be the number of vertex-disjoint $A$-$B$ paths in $G$. Clearly, to disconnect $A$ and $B$, we need to at least delete either one edge or one vertex from each of the $k$ $A$-$B$ paths. Therefore, $\abs{X} \geq k$.
        \\\\
        Take any $n < \abs{X}$ disjoint $A$-$B$ paths, then by Lemma \ref{lem:moreABPaths}, we can extend this set of paths by another vertex-disjoint $A$-$B$ paths until $n = \abs{X}$. Therefore, $k \geq \abs{X}$, which implies that $k = \abs{X}$ as desired.
    \end{proof}
\end{thmbox}
By taking both sets of vertices as singletons, we obtain a corollary of Theorem \ref{thm:Menger} at a ``local'' level.
\begin{corbox}{Local Version of Menger's Theorem}{localMenger}
    Let $G$ be a graph. If $a, b \in V(G)$ are not adjacent, then 
    \begin{enumerate}
        \item the minimum number of vertices which separate $a$ and $b$ when removed equals the maximum number of internally vertex-disjoint $a$-$b$ paths in $G$;
        \item the minimum number of edges which separate $a$ and $b$ when removed equals the maximum number of edge-disjoint $a$-$b$ paths in $G$.
    \end{enumerate}
    \tcblower
    \begin{proof}
        Note that the minimum number of vertices that separate $a$ and $b$ when removed is just the minimum number of vertices that separate $N_G(a)$ and $N_G(b)$ when removed. By Theorem \ref{thm:Menger}, this quantity is just the maximum number of vertex-disjoint $N_G(a)$-$N_G(b)$ paths in $G$, which are internally vertex-disjoint $a$-$b$ paths.
        \\\\
        Let $L(G)$ be the line graph of $G$ and let $A$ and $B$ be the sets of edges containing $a$ and $b$ respectively. Then, the minimum number of edges to separate $a$ and $b$ when removed is equal to the minimum number of vertices to separate $A$ and $B$ when removed in $L(G)$. By Theorem \ref{thm:Menger}, this quantity is just the number of vertex-disjoint $A$-$B$ paths in $L(G)$, which correspond to edge-disjoint $a$-$b$ paths in $G$.
    \end{proof}
\end{corbox}
Correspondingly, the following ``global'' version of Menger's Theorem characterises a $k$-connected graph, to which Theorem \ref{thm:Whitney} is an immediate corollary.
\begin{corbox}{Global Version of Menger's Theorem}{globalMenger}
    A graph $G$ is $k$-connected if and only if for any $u, v \in V(G)$, there are $k$ internally vertex-disjoint $u$-$v$ paths in $G$. A graph $G$ is $k$-edge-connected if and only if for any $u, v \in V(G)$, there are $k$ edge-disjoint $u$-$v$ paths in $G$. 
    \tcblower
    \begin{proof}
        Suppose the for any $u, v \in V(G)$, there exist $k$ internally vertex-disjoint $u$-$v$ paths in $G$. Let $S$ be a set of less than $k$ vertices in $G$ and consider $G - S$. For any $u, v \in V(G) - S$, we know that there is at least one $u$-$v$ path in $G - S$, so $G - S$ is connected and so $G$ is $k$-connected.
        \\\\
        Suppose conversely that $G$ is $k$-connected and suppose on contrary that there exist $a, b \in V(G)$ such that $a, b$ are connected by fewer than $k$ internally vertex-disjoint $a$-$b$ paths in $G$. By Corollary \ref{cor:localMenger}, $ab \in E(G)$ because otherwise we can disconnected $a$ and $b$ by deleting fewer than $k$ vertices, which is impossible because $G$ is $k$-connected. Let $H \coloneqq G - ab$, then there are at most $(k - 2)$ internally vertex-disjoint $a$-$b$ paths in $H$. By Corollary \ref{cor:localMenger}, there exists a set $X \subseteq V(H)$ with $\abs{X} \leq k - 2$ which separates $a$ and $b$ in $H$. Note that there exists some $w \in V(H) - X - \{a, b\}$ such that $X$ separates $w$ from either $a$ or $b$ in $H$, because otherwise $a$ and $b$ are connected in $H - X$. Without loss of generality, let $X$ separates $w$ from $a$. 
        \\\\
        Note that $a$ and $w$ are connected in $G - X$ because $G$ is $k$-connected, so there exists at least one $a$-$w$ path in $G$ which does not contain any vertex in $X$. Notice that every such path must contain the edge $ab$ because otherwise $a$ and $w$ are connected in $H - X$. Therefore, $w$ and $a$ are disconnected in $G - X - \{b\}$, which means $G$ is $(k - 1)$-connected, which is a contradiction.
    \end{proof}
\end{corbox}
With Menger's Theorem, we can prove Theorem \ref{thm:Konig} as a corollary. Consider a bipartite graph $G$ with bipartition $(A, B)$. Add vertices $a, b$ to $G$ to form a new graph $H$ such that $N_H(a) = A$ and $N_H(b) = B$.

Clearly, $\nu(G)$ is the maximum number of internally vertex-disjoint $a$-$b$ paths in $H$. By Corollary \ref{cor:localMenger}, this is just the minimum number of vertices which separate $a$ and $b$ in $H$ when removed. Let $X \subseteq V(G)$ be the minimal set of vertices to disconnect $a$ and $b$ when removed, then every $a$-$b$ path in $H$ must contain a vertex in $X$. Suppose there is an edge $uv \in E(G)$ such that $u \notin X$ and $v \notin X$, then $auvb$ is an $a$-$b$ path in $H - X$. Therefore, $X$ is a vertex cover for $G$ and so $\nu(G) = \abs{X} = \tau(G)$.
\chapter{Graph Colouring}
Graph colouring stems from the following simple motivation:
\begin{quote}
    Any map of countries on a plane can be coloured by $4$ distinct colours such that no two countries sharing at least one border have the same colour.
\end{quote}
The above is summarised as the $4$ Colour Theorem. To study such problems, we introduce to notion of graph colouring.
\section{Vertex Colouring}
\begin{dfnbox}{Proper Colouring}{colouring}
    A {\color{red} \textbf{proper $k$-colouring}} of a graph $G$ is a mapping 
    \begin{equation*}
        c \colon V(G) \to \left\{1, 2, \cdots, k\right\}
    \end{equation*}
    such that $c(v) \neq c(u)$ whenever $uv \in E(G)$.
\end{dfnbox}
Given a graph $G$ with a proper $k$-colouring $c$, denote with $V_i \coloneqq \left\{v \colon c(v) = i\right\}$ the set of all vertices with colour $i$. Clearly, every $V_i$ is an independent set in $G$.

Suppose $G$ is bipartite with bipartition $(U, V)$, then we can colour the vertices in $G$ such that all vertices in $U$ receives one colour and all those in $V$ receives another.

Clearly, if a graph has a proper $k$-colouring, then it must also have a proper $(k + 1)$-colouring. Therefore, we are interested to know the least number of colours to properly colour a graph.
\begin{dfnbox}{Chromatic Number}{chromaticN}
    The {\color{red} \textbf{chromatic number}} of a graph $G$, denoted as $\chi(G)$, is the smallest positive integer $k$ such that $G$ has a proper $k$-colouring.
\end{dfnbox}
Observe that if $\chi(G) \leq 2$, it is equivalent to saying that we can find a bipartition of $V(G)$, so a bipartite graph can be alternatively defined as a graph $G$ with $\chi(G) \leq 2$. 

Furthermore, observe that if $H \subseteq G$, then $\chi(H) \leq \chi(G)$. Consider a cycle $C_{2n + 1}$ for some $n \in \N$. It is clear that $\chi(C_{2n + 1}) = 3$, so if $C_{2n + 1} \subseteq G$, then $G$ cannot be bipartite.

Similar to bipartite graphs, we can define the notion of an $r$-partite graph with chromatic numbers.
\begin{dfnbox}{$r$-partite Graph}{rpartite}
    A graph $G$ is an {\color{red} \textbf{$r$-partite graph}} if $\chi(G) \leq r$.
\end{dfnbox}
Let us examine some special graphs to see their chromatic numbers:
\begin{itemize}
    \item Complete graph $K_n$: $\chi(K_n) = n$.
    \item Cycle $C_n$: 
    \begin{equation*}
        \chi(C_n) = \begin{cases}
            2 & \textrm{if } 2 \mid n \\
            3 & \textrm{if } 2 \not\mid n
        \end{cases}.
    \end{equation*}
    \item Empty graph $\overline{K_n}$: $\chi\left(\overline{K_n}\right) = 1$.
    \item Petersen graph $P$: $\chi(P) = 3$.
\end{itemize}
Recall the Kneser graph $KG(n, k)$ from Definition \ref{dfn:kneser}. For $i = 2k, 2k + 1, \cdots, n$, let $V_i$ be all $k$-element subsets whose greatest element is $i$. Note that for any $S, T \in V_i$, we have $i \in S \cap T$ and so $ST \notin E\bigl(KG(n, k)\bigr)$. Therefore, we can colour each $V_i$ using a distinct colour $c_i$. Notice that in doing so, $(n - 2k + 1)$ colours are used. For any $U$ not in any of the $V_i$'s, its elements must be from $\left\{1, 2, \cdots, k - 1\right\}$. By Pigeonhole Principle, this means that the remaining subsets are pair-wise intersecting, and so we can assign an extra colour for all of them. Therefore, $\chi\bigl(KG(n, k)\bigr) \leq n - 2k + 2$. However, the other direction is not trivial and requires some knowledge of the Borsuk-Ulam Theorem in Topology.

Intuitively, one only needs to construct a proper colouring to establish an upper bound for $\chi(G)$, but to establish the lower bound is not so easy and we often need to do some estimation using other parameters.
\begin{probox}{Lower Bound of Chromatic Number}{lowerBoundChi}
    For any graph $G$, $\chi(G) \geq \omega(G)$ and $\chi(G) \geq \frac{\abs{V(G)}}{\alpha(G)}$.
    \tcblower
    \begin{proof}
        $\chi(G) \geq \omega(G)$ is trivial and is left to the reader as an exercise.
        \\\\
        Fix any proper $\chi(G)$-colouring for $G$. For every colour $i$, the set of vertices coloured in $i$, $V_i$, is an independent set, so $\abs{V_i} \leq \alpha(G)$. Therefore,
        \begin{equation*}
            \abs{V(G)} = \sum_{i = 1}^{\chi(G)}\abs{V_i} \leq \chi(G)\alpha(G),
        \end{equation*}
        and so $\chi(G) \geq \frac{\abs{V(G)}}{\alpha(G)}$.
    \end{proof}
\end{probox}
An interesting question to ponder is that: do there exist infinitely many pairs of positive integers $(k, \ell)$ such that $k = \omega(G) < \chi(G) = \ell$? In fact, Lov\'{a}sz proposed a theorem stating that by considering $KG(3k - 1, k)$, one can find a triangle-free graph with arbitrarily high chromatic number.

The chromatic number of a graph can be upper-bounded in multiple ways.
\begin{probox}{Upper Bound of Chromatic Number}{upperBoundChi}
    For every graph $G$, $\chi(G) \leq \frac{1}{2} + \sqrt{2\abs{E(G)} + \frac{1}{4}}$.
    \tcblower
    \begin{proof}
        Fix a proper $\chi(G)$-colouring in $G$. Let $V_i$ and $V_j$ be $2$ sets of vertices with colours $i$ and $j$ respectively. Note that there is at least one edge between $V_i$ and $V_j$, otherwise we can obtain a $(\chi(G) - 1)$-colouring in $G$, which is not possible. Therefore, 
        \begin{equation*}
            \abs{E(G)} \geq \begin{pmatrix}
                \chi(G) \\
                2
            \end{pmatrix}.
        \end{equation*}
        Solving the inequality gives $\chi(G) \leq \frac{1}{2} + \sqrt{2\abs{E(G)} + \frac{1}{4}}$.
    \end{proof}
\end{probox}
This bound is sharp for complete graphs. However, we can use a greedy algorithm to obtain a better bound for many other graphs.
\begin{probox}{Upper Bound of Chromatic Number by Greedy Algorithm}{greedyUpperBoundChi}
    For any graph $G$, $\chi(G) \leq \Delta(G) - 1$.
    \tcblower
    \begin{proof}
        We can label the vertices in $G$ as $v_1, v_2, \cdots, v_n$ and colour them in this order. Fix any colour for $v_1$. For any $v_i$, colour each of its neighbours with the first available colours. If all existing colours are unavailable, we introduce a new colour. Note that for every vertex $v \in V(G)$, its neighbours use at most $d_G(v) \leq \Delta(G)$ distinct colours, so $\chi(G) \leq \Delta(G) - 1$.
    \end{proof}
\end{probox}
We can further improve this bound by considering the following definition:
\begin{dfnbox}{$k$-critical Graph}{kCrit}
    A graph $G$ is said to be {\color{red} \textbf{$k$-critical}} if $\chi(G) = k$ and $\chi(G - e) < k$, $\chi(G - v) < k$ for any $e \in E(G)$ and any $v \in V(G)$.
\end{dfnbox}
An easy example here is that any odd cycle is $3$-critical. In fact, every $3$-critical graph must be an odd cycle, which can be proven by a later theorem.

We first examine some properties of $k$-critical graphs.
\begin{probox}{Lower Bound of Degrees of $k$-critical Graphs}{lowerBoundKCritDeg}
    For any $k$-critical graph $G$, $\delta(G) \geq k - 1$.
    \tcblower
    \begin{proof}
        It suffices to prove that $d_G(x) \geq k - 1$ for every $x \in V(G)$. Suppose on contrary that there exists some $v \in V(G)$ such that $d_G(v) < k - 1$. Since $G$ is $k$-critical, $\chi(G - v) \leq k - 1$ and so there exists a proper $(k - 1)$-colouring of $G - v$. Note that this means that we can find some colour among these $(k - 1)$ distinct colours to colour $v$ in $G$ such that $v$ does not share the colour with any of its neighbours, which is a contradiction because $\chi(G) = k$.
    \end{proof}
\end{probox}
We can establish a better upper bound for chromatic number by using a $k$-critical graph as follows:
\begin{probox}{Upper Bound of Chromatic Number By $k$-critical Subgraph}{upperBoundChiKCrit}
    For every graph $G$, $\chi(G) \leq 1 + \max_{H \subseteq G}\delta(H)$.
    \tcblower
    \begin{proof}
        Note that we can delete vertices and edges from $G$ to obtain a graph $G'$ such that $\chi(G') < \chi(G)$. Therefore, there exists $H \subseteq G$ such that $H$ is $\chi(G)$-critical. By Proposition \ref{pro:lowerBoundKCritDeg}, this implies that
        \begin{equation*}
            \chi(G) = \chi(H) \leq \delta(H) + 1 \leq 1 + \max_{H \subseteq G}\delta(H).
        \end{equation*}
    \end{proof}
\end{probox}
\begin{notebox}
    \begin{remark}
        Note that $1 + \max_{H \subseteq G}\delta(H) \leq 1 + \max_{H \subseteq G}\Delta(H) = 1 + \Delta(G)$, so this upper bound is stronger than Proposition \ref{pro:greedyUpperBoundChi}.
    \end{remark}
\end{notebox}
Consider any tree $T$. Note that $T$ is bipartite and $2$-colourable. By Proposition \ref{pro:upperBoundChiKCrit}, since any subgraph of $T$ is a forest with a minimum degree of $1$, $\chi(G) \leq 2$ is a sharp bound. However, $\Delta(T)$ can be arbitrarily large, which means the bound given by \ref{pro:greedyUpperBoundChi} can be very loose. In fact, we can prove that unless $G$ is a complete graph or an odd cycle, the bound given by \ref{pro:greedyUpperBoundChi} is always loose.
\begin{thmbox}{Brooks' Theorem}{Brooks}
    Let $G$ be a connected graph not isomorphic to a complete graph or an odd cycle, then $\chi(G) \leq \Delta(G)$.
    \tcblower
    \begin{proof}
        We shall perform strong induction on $\abs{V(G)}$. If $\abs{V(G)} = 1$, then clearly $\chi(G) = 1 = \Delta(G)$. Suppose there is some $n \in \Z^+$ with $n > 1$ such that for any integer $m \in [1, n - 1]$, any graph $G$ with order $m$ is such that $\chi(G) \leq \Delta(G)$.
        \\\\
        Suppose on contrary that there exists a graph $G'$ of order $n$ such that $\chi(G') > \Delta(G')$. Take any vertex $v \in V(G')$ and consider $H \coloneqq G' - v$. Let $H'$ be any connected component of $H$, then $\Delta(H') \leq \Delta(G')$. If $H'$ is not a complete graph nor an odd cycle, by the inductive hypothesis, $\chi(H') \leq \Delta(H') \leq \Delta(G')$. Otherwise, note that $H'$ will be $\Delta(H')$-regular. Since at least one vertex in $H'$ is adjacent to $v$ in $G'$, we have $\chi(H') = \Delta(H') + 1 \leq \Delta(G')$. Therefore, $\chi(H) \leq \Delta(G')$.
        \\\\
        Fix a $\Delta(G')$-colouring on $H$. Note that $N_{G'}(v) \subseteq V(H)$. We claim that $N_{G'}(v)$ used all $\Delta(G')$ colours, because otherwise we can use one of the $\Delta(G')$ colours to colour $v$ in $G'$ and obtain a $\Delta(G')$-colouring in $G'$, which is not possible. Therefore, $d_{G'}(v) = \Delta(G')$, and so every vertex in $N_{G'}(v)$ has a colour distinct to the other neighbours. Let $v_i$ be the neighbour of $v$ coloured in $i$, and let $H_{ij}$ be the subgraph spanned by all $i$- and $j$-vertices in $H$. We claim that whenever $i \neq j$, $v_i$ and $v_j$ must be in the same connected component $C_{ij}$ of $H_{ij}$, otherwise, we can interchange all $i$- and $j$-vertices in the connected component containing $v_i$ and colour $v$ in $i$.
        \\\\
        Let $P$ be a $v_i$-$v_j$ path in $C_{ij}$. Note that $d_H(v_i) = d_{G'}(v_i) - 1 \leq \Delta(G') - 1$. We claim that the neighbours of $v_i$ use exactly $(\Delta(G') - 1)$ distinct colours, otherwise we can re-colour $v_i$ in a different colour and colour $v$ with one of the $\Delta(G')$ colours, which is not possible. This implies that $v_i$ has only one neighbour with colour $j$, and so the neighbour of $v_i$ in $P$ is its only neighbour in $C_{ij}$. 
        \\\\
        We claim that $P = C_{ij}$. Otherwise, take $u \in V(C_{ij})$ to be the vertex closest to $v_i$ such that $u$ is adjacent to at least $3$ distinct vertices. Notice that this means that there is a unique $v_i$-$u$ path in $C_{ij}$. Note that $u$ has at most $\Delta(G')$ neighbours in $H$ and at least $3$ of them are of the same colour. Therefore, the neighbours of $u$ in $H$ are coloured by at most $(\Delta(G') - 2)$ colours. Therefore, we can change $u$ to some other colour such that $v_i$ and $v_j$ belong to different connected components in $H_{ij}$, which is not possible.
        \\\\
        Let $i, j, k$ be distinct colours. We claim that $V(C_{ij}) \cap V(C_{ik}) = \{v_i\}$. Otherwise, let $w \neq v_i$ and $w \in V(C_{ij}) \cap V(C_{ik})$. Note that $d_{C_{ij}}(w) = d_{C_{ik}} = 2$, so the neighbours of $w$ uses at most $(\Delta(G') - 2)$ colours. Similar to the previous claim, this means we can re-colour $w$ such that $v_i$ and $v_j$ are in different connected components, which is not possible.
        \\\\
        If $v, v_1, v_2, \cdots, v_{\Delta(G')}$ are pair-wise adjacent, then $G'$ must be complete because each $C_{ij}$ is just an edge, which is not possible. Without loss of generality, let $v_1 \not\sim v_2$ in $G'$. Consider $C_{12}$ and $C_{13}$. We first switch all $1$-vertices and $3$-vertices in $C_{13}$, and switch all $1$-vertices and $2$-vertices in $C_{12} - v_1$. Note that the resultant colouring is still proper in $H$, and now $v_2$ and $v_3$ both have colour $1$. Therefore, we can colour $v$ in colour $2$ to obtain a proper $\Delta(G')$-colouring of $G'$, which is a contradiction.
    \end{proof}
\end{thmbox}
Previously in Proposition \ref{pro:greedyUpperBoundChi}, we have used a greedy approach to establish that a graph with large chromatic number must have high maximum degree. Intuitively, if a graph contains a clique, its chromatic number is naturally high. However, this is not necessary. In fact, there exists graphs with high girth but also high chromatic number. Moreover, notice that $K_{m, n}$ is $2$-colourable, so in fact none of the other graph parameters we have learnt so far can successfully ``force'' a high chromatic number in a graph.

Here, we devise a construction for a graph with high chromatic number.
\begin{dfnbox}{Edge Contraction}{edgeContraction}
    Let $G$ be a graph. For any $e = xy \in E(G)$, the {\color{red} \textbf{edge contraction}} of $G$ at $e$, denoted by $G/e$, is a graph produced by deleting $e$ and merging $x$ and $y$ such that $N_{G/e}(xy) = N_G(x) \cup N_G(y)$.
\end{dfnbox}
We define the notion of a $k$-\textit{constructible graph} recursively.
\begin{dfnbox}{$k$-constructible Graph}{kConstruct}
    A {\color{red} \textbf{$k$-constructible graph}} $G$ for some $k \in \Z^+$ is such that 
    \begin{enumerate}
        \item $G \cong K_k$;
        \item there is some $k$-constructible graph $H$ with $x, y \in V(H)$ and $xy \notin E(H)$ such that $G \cong (H + xy)/xy$;
        \item there are $k$-constructible graphs $H_1$, $H_2$ with $V(H_1) \cap V(H_2) = \{x\}$ and $y_1x \in E(H_1), y_2x \in E(H_2)$ such that $G \cong H_1 \cup H_2 - y_1x - y_2x + y_1y_2$.
    \end{enumerate}
\end{dfnbox}
We can easily derive a lower bound for the chromatic number of $k$-constructible graphs.
\begin{probox}{Chromatic Number of $k$-constructible Graphs}{lowerBoundChiKCons}
    If $G$ is a $k$-constructible graph, then $\chi(G) \geq k$.
    \tcblower
    \begin{proof}
        If $G \cong K_k$, then $\chi(G) = k \geq k$ is trivial. 
        \\\\
        Suppose $G \cong (H + xy)/xy$ for some $k$-constructible graph $H$ with $\chi(H) \geq k$, we shall prove that $\chi(G) \geq k$ by considering the contrapositive statement. Suppose that $\chi(G) \leq k - 1$, then we can fix a $(k - 1)$-colouring in $G$. Suppose that $x$ and $y$ are contracted into $z$ in $G$, then we can just colour $x$ and $y$ in $H$ both using the colour of $z$ in $G$ while keep all other vertices the same colour. Since $N_H(x) \cup N_H(y) = N_G(z)$, this colouring is proper in $H$ and so $\chi(H) \leq k - 1$. 
        \\\\
        Suppose $G \cong H_1 \cup H_2 - y_1x - y_2x + y_1y_2$ for some $k$-constructible graphs $H_1$ and $H_2$ with $V(H_1) \cap V(H_2) = \{x\}$ and $y_1x \in E(H_1), y_2x \in E(H_2)$. We shall prove that if $\chi(H_1) \geq k$ and $\chi(H_2) \geq k$, then $\chi(G) \geq k$ by considering the contrapositive. Suppose that $\chi(G) \leq k - 1$, then we can fix a $(k - 1)$-colouring in $G$. Note that $y_1y_2 \in E(G)$, so $y_1$ and $y_2$ must have different colours. This means at least one of $y_1$ and $y_2$ has a colour distinct from the colour of $x$. Without loss of generality, assume that $x$ and $y_1$ have different colours. This means that the colouring of $G$ is still proper in $H_1$, and so $\chi(H_1) \leq k - 1$. Therefore, we have proved that any $k$-constructible graph $G$ is such that $\chi(G) \geq k$.
    \end{proof}
\end{probox}
Remarkably, we can use $k$-constructible graphs to characterise all graphs with a certain chromatic number.
\begin{thmbox}{Haj\'{o}s's Theorem}{Hajos}
    For any graph $G$, $\chi(G) \geq k$ if and only if $G$ has a $k$-constructible subgraph.
    \tcblower
    \begin{proof}
        Suppose $G$ has $k$-constructible subgraph $H$. By Proposition \ref{pro:lowerBoundChiKCons}, $\chi(H) \geq k$, and so $\chi(G) \geq \chi(H) \geq k$.
        \\\\
        Suppose $\chi(G) \geq k$. The case where $k = 1$ is trivial. If $k = 2$, then $\chi(G) \geq 2$ implies that $G$ contains a $K_2$ which is $2$-constructible. For $k \geq 3$, take $G$ to be the edge-maximal graph with $\chi(G) \geq k$ such that $G$ contains not $k$-constructible subgraph. If $G$ is a complete multi-partite graph, then it contains at least $k$ partite sets. Fix $k$ partite sets in $G$ and take one vertex from each set, then clearly these vertices induce a $K_k \subseteq G$ which is $k$-constructible.
        \\\\
        If $G$ is not complete multi-partite, consider the following lemma:
        \begin{lembox}{Complete Multi-partite Graph Characterisation}{completeMultPartite}
            Define a graph $H$ with $V(H) = \{x, y, z\}$ and $E(H) = \{yz\}$. A graph $G$ is complete multi-partite if it contains no induced subgraph isomorphic to $H$.
            \tcblower
            \begin{proof}
                If $G$ contains no induced subgraph isomorphic to $H$, then $\overline{G}$ contains no induced subgraph isomorphic to $\overline{H}$. Let $G' \subseteq \overline{G}$ be a connected component. Notice that if there are vertices $u, v, w \in V(G')$ such that $u \sim v$ and $v \sim w$, then $u \sim w$ because otherwise $\{u, v, w\}$ induces a $P_2 \cong \overline{H}$ in $G'$. Therefore, adjacency is a transitive relation on $V(G')$. For any $x, y \in V(G')$, there exists an $x$-$y$ path in $G'$. By transitivity, this means that $x \sim y$ in $G'$. Therefore, $G'$ is a clique. This means that $\overline{G}$ is the union of disjoint cliques, and so $G$ must be complete multi-partite.
            \end{proof}
        \end{lembox}
        By Lemma \ref{lem:completeMultPartite}, we know that $G$ must contain some subgraph of order $3$ with vertex set $\{x, y_1, y_2\}$ and edge set $\{y_1y_2\}$. For $i = 1, 2$, let $G_i \coloneqq G + xy_i$. Since $G$ is edge-maximal with respect to not having $k$-constructible subgraphs, $G_i$ must contain a $k$-constructible subgraph $H_i$. We claim that $xy_i \in E(H_i)$, because otherwise $H_i \subseteq G$, which is not possible. Let $H_1'$ and $H_2'$ be copies of $H_1$ and $H_2$ and define a graph $H \coloneqq H_1' \cup H_2'$. For every vertex $z \in V(H_1) \cap V(H_2)$ with $z \neq x$, let $z_1$, $z_2$ be its copy in $H_1'$ and $H_2'$ respectively, then $z_1z_2 \notin E(H)$. Therefore, $V(H_1') \cap V(H_2') = \{x\}$ and $xy_1 \in E(H_1')$, $xy_2 \in E(H_2')$. Therefore, $H' \coloneqq H - xy_1 - xy_2 + y_1y_2$ is $k$-constructible. Note that for every pair $(z_1, z_2) \in V(H')^2$, $z_1z_2 \notin E(H')$, and so $(H + z_1z_2)/z_1z_2$ is $k$-constructible. After contracting all $(z_1, z_2)$'s, we have restored the graph $H_1 \cup H_2 \subseteq G$, which is $k$-constructible.
    \end{proof}
\end{thmbox}
Theorem \ref{thm:Hajos} leads to an important property of $k$-critical graphs.
\begin{corbox}{$k$-critical Graphs Are $k$-constructible}{characteriseKCrit}
    A graph $G$ is $k$-constructible if it is $k$-critical. 
    \tcblower
    \begin{proof}
        Let $H \subseteq G$ be the maximal $k$-constructible subgraph of $G$. It suffices to prove that $H = G$. Suppose on contrary that $H \neq G$, then there exists either an edge $e$ or a vertex $v$ in $G - H$. Therefore, $H \subseteq G - e$ and $H \subseteq G - v$, which implies that $\chi(G - e) = \chi(G - v) = k$ by Theorem \ref{thm:Hajos}, which is a contradiction because $G$ is $k$-critical.
    \end{proof}
\end{corbox}
\section{Edge Colouring}
Just like vertices, we can also colour the edges of a graph. 
\begin{dfnbox}{Edge Colouring}{edgeColour}
    Let $G$ be a graph. A {\color{red} \textbf{proper $k$-edge-colouring}} on $G$ is a map $c \colon E(G) \to \left\{1, 2, \cdots, k\right\}$ such that if $\abs{e \cap f} \geq 1$, then $c(e) \neq c(f)$. The minimum $k$ such that $G$ has a proper $k$-edge-colouring is known as the {\color{red} \textbf{edge chromatic number}} of $G$, denoted by $\chi'(G)$.
\end{dfnbox}
Intuitively, colouring the edges in a graph is equivalent to colouring the vertices in its line graph, so the following proposition is intuitive:
\begin{probox}{Chromatic Number of Line Graphs}{lineGraphChi}
    For any graph $G$, $\chi'(G) = \chi\bigl(L(G)\bigr)$.
\end{probox}
Fix any proper edge-colouring in a graph $G$, then intuitively, the set of all edges in the same colour forms a matching in $G$, so one may choose to view $\chi'(G)$ as the minimum number of matchings to decompose $G$. For example, for a complete graph $K_n$, we know that $\nu(K_n) = \left\lfloor\frac{n}{2}\right\rfloor$, so 
\begin{equation*}
    \chi'(K_n) \geq \frac{C^n_2}{\left\lfloor\frac{n}{2}\right\rfloor} = \frac{n(n - 1)}{2\left\lfloor\frac{n}{2}\right\rfloor} = \begin{cases}
        n - 1 & \textrm{if } n \textrm{ is even} \\
        n & \textrm{otherwise}
    \end{cases}.
\end{equation*}
Obviously, if $d_G(v) = \Delta(G)$ for some vertex in a graph $G$, then all edges incident to $v$ in $G$ induce a clique of order $\Delta(G)$ in $L(G)$. By Proposition \ref{pro:lineGraphChi}, this means that 
\begin{equation*}
    \chi'(G) = \chi\bigl(L(G)\bigr) \geq \Delta(G).
\end{equation*}
Recall that according to Corollary \ref{cor:regBipDecomposition}, every $d$-regular bipartite graph $G$ can be decomposed into $d$ perfect matchings. If we assign each of these perfect matchings a colour, we essentially obtain a proper edge colouring in $G$. Clearly, in this case $\chi'(G) = d = \Delta(G)$. In fact, this equality can be extended to any bipartite graph.
\begin{thmbox}{K\"{o}nig's Edge Colouring Theorem}{konig2}
    If $G$ is a bipartite graph, then $\chi'(G) = \Delta(G)$.
    \tcblower
    \begin{proof}
        It suffices to prove that $\chi'(G) \leq \Delta(G)$. We shall perform induction on $\abs{E(G)}$. The case where $\abs{E(G)} = 1$ is trivial. For $\abs{E(G)} = m$, take any $xy \in E(G)$ and consider $H \coloneqq G - xy$. Note that $\abs{E(H)} = m - 1$, so by the inductive hypothesis we have 
        \begin{equation*}
            \chi'(H) = \Delta(H) \leq \Delta(G).
        \end{equation*}
        Therefore, there exists a $\Delta(G)$-edge-colouring on $H$. Notice that $d_H(x) \leq \Delta(G) - 1$ and $d_H(y) \leq \Delta(G) - 1$, but there are $\Delta(G)$ colours. Without loss of generality, assume that $x$ is not incident to any $\alpha$-edge and $y$ is not incident to any $\beta$-edge. If $x$ is not incident to any $\beta$-edge, we can colour $xy$ with $\beta$ in $G$ and obtain a $\Delta(G)$-edge-colouring in $G$. Otherwise, take a maximal $\alpha$-$\beta$ alternating path starting from $x$ and ending at $x'$. Note that there is no $\alpha$-edge between $x$ and $x'$, so switching the $\alpha$- and $\beta$-edges along the path retains the proper colouring in $H$. However, notice that now $x$ is no longer incident to any $\beta$-edge in $H$, and so we can colour $xy$ as $\beta$ in $G$ to obtain a proper $\Delta(G)$-edge-colouring.
    \end{proof}
\end{thmbox}
It turns out that for general graphs, we need at most one more colour to obtain a proper edge-colouring as compared to a bipartite graph.
\begin{thmbox}{Vizing's Theorem}{Vizing}
    For every graph $G$, we have $\Delta(G) \leq \chi'(G) \leq \Delta(G) + 1$.
\end{thmbox}
We can generalise Vizing's result to multigraphs.
\begin{corbox}{Vizing's Theorem for Multigraphs}{VizingMult}
    For every loopless multigraph $G$, we have $\Delta(G) \leq \chi'(G) \leq \Delta(G) + \mu(G)$, where $\mu(G)$ is the greatest multiplicity of edges in $G$.
\end{corbox}
\section{Perfect Graphs}
In Proposition \ref{pro:lowerBoundChi}, we see that $\chi(G) \geq \omega(G)$. This motivates us to define a special type of graphs, the colouring of which is closely tied to the largest clique in them.
\begin{dfnbox}{Perfect Graph}{perfectGraph}
    A graph $G$ is {\color{red} \textbf{perfect}} if for every induced subgraph $H \subseteq G$, we have $\chi(H) = \omega(H)$.
\end{dfnbox}
Clearly, $K_n$ is a simple example for a perfect graph as every induced subgraph of $K_n$ is also complete. Correspondingly, one can easily check that the empty graph $\overline{K_n}$ is also perfect. A non-empty bipartite graph $G$ contains an edge as its maximum clique and so $\chi(G) = \omega(G) = 2$, which means $G$ is also perfect. We can also prove that the complement of any bipartite graph must also be perfect.
\begin{probox}{Complement of a Bipartite Graph Is Perfect}{bipCompPerfect}
    Let $G$ be a bipartite graph, then $\overline{G}$ is perfect.
    \tcblower
    \begin{proof}
        Let $\overline{H} \subseteq \overline{G}$ be an induced subgraph, then $\abs{V\left(\overline{H}\right)}$ induces a subgraph $H \subseteq G$ which is the complement of $\overline{H}$. Note that every clique in $\overline{H}$ is induced by an independent set of $H$, so $\omega\left(\overline{H}\right) = \alpha(H) = \abs{V(H)} - \tau(H)$ by Proposition \ref{pro:Order=Sum}. Observe that $\chi\left(\overline{H}\right)$ is the smallest number of independent sets to partition $V\left(\overline{H}\right)$, but since $H$ is bipartite, each of these independent sets induces either a clique, i.e., an edge, or an empty graph in $H$. Therefore, $\chi\left(\overline{H}\right) + \nu(H) = \abs{V(H)}$. By Theorem \ref{thm:Konig}, we have 
        \begin{equation*}
            \chi\left(\overline{H}\right) = \abs{V(H)} - \nu(H) = \abs{V(H)} - \tau(H) = \omega\left(\overline{H}\right),
        \end{equation*}
        and so $\overline{G}$ is perfect.
    \end{proof}
\end{probox}
We can prove that the line graph of bipartite graphs are also perfect.
\begin{probox}{Line Graphs of Bipartite Graphs Are Perfect}{lineGraphPerfect}
    Let $G$ be a bipartite graph, then $L(G)$ is perfect.
    \tcblower
    \begin{proof}
        By Theorem \ref{thm:konig2}, it suffices to prove that $\omega\bigl(L(G)\bigr) = \Delta(G)$. Let $H \subseteq L(G)$ be a clique, then $H$ consists of edges of $G$ which pair-wise share an incident vertex, so clearly $\omega\bigl(L(G)\bigr) = \Delta(G)$.
    \end{proof}
\end{probox}
There are many other families of perfect graphs, such as \textit{comparability graphs}
(undirected graph of a poset), \textit{interval graphs} (intersection graph of intervals on the real
line), \textit{chordal graphs} (each cycle of length at least $4$ has a chord), etc.

Perfect graphs have many nice properties, one of which is that given an $n$-vertex perfect graph, many graph parameters such as $\alpha$, $\omega$ and $\chi$ can be computed with an $\mathcal{O}(n)$ algorithm. 

Now, let us devise a way to characterise perfect graphs.
\begin{thmbox}{Weak Perfect Graph Theorem}{weakPerfect}
    A graph is perfect if and only if its complement is perfect.
    \tcblower
    \begin{proof}
        It suffices to prove that if $G$ is perfect, then $\overline{G}$ is perfect. We shall apply strong induction on $\abs{V(G)}$. If $\abs{V(G)} = 1$, then clearly both $G$ and $\overline{G}$ are perfect. Suppose that there is some integer $n > 1$ such that whenever $\abs{V(G)} < n$, $G$ being perfect implies that $\overline{G}$ is perfect. Let $\abs{V(G)} = n$ and take $\overline{H} \subseteq \overline{G}$ to be an induced subgraph. If $\abs{V\left(\overline{H}\right)} < n$, then by the inductive hypothesis $\overline{H}$ is perfect, and so $\chi\left(\overline{H}\right) = \omega\left(\overline{H}\right)$. Otherwise, $\overline{H} = \overline{G}$, so it suffices to prove $\chi\left(\overline{G}\right) = \omega\left(\overline{G}\right)$. Consider the following lemma:
        \begin{lembox}{Vertex Expansion Preserves Perfectness}{vtxExpand}
            Let $G$ be a perfect graph and define $G'$ by 
            \begin{equation*}
                V(G') \coloneqq V(G) \cup \{x'\}, \qquad E(G') \coloneqq E(G) \cup \left\{x'y \colon y \sim x\right\} \cup \{xx'\},
            \end{equation*}
            then $G'$ is perfect.
            \tcblower
            \begin{proof}
                We shall apply strong induction on $\abs{V(G)}$. If $\abs{V(G)} = 1$, then $G' \cong K_2$ which is perfect. Suppose that there exists $n > 1$ such that whenever $\abs{V(G)} < n$, $G'$ is perfect. Let $\abs{V(G)} = n$ and $H \subseteq G'$ be an induced subgraph with $H \neq G'$. If at most one of $x'$ and $x$ is contained in $H$, then $H$ is isomorphic to an induced subgraph in $G$. Otherwise, $H$ can be obtained by extending $x$ on some induced subgraph $H'$ of $G$.  Either case, by our inductive hypothesis we have $\omega(H) = \chi(H)$. Therefore, it suffices to prove that $\omega(G') \geq \chi(G')$.
                \\\\
                Note that $\omega(G') \in \{\omega(G), \omega(G) + 1\}$, so we consider $2$ cases. If $\omega(G') = \omega(G) + 1$, since $G$ is perfect, we have $\omega(G') = \chi(G) + 1$, and so $\chi(G') \leq \chi(G) + 1 = \omega(G')$. Otherwise, $\omega(G') = \omega(G) = \chi(G)$. Fix a proper $\omega(G)$-colouring on $G$ and take any maximal clique $K \cong K_{\omega(G)}$ in $G$. Notice that $x \notin K$ because otherwise $V(K) \cup \{x'\}$ induces a clique of size $\omega(G) + 1$ in $G'$, which is not possible. Suppose $x \in U$ where $U$ is some colour class in $G$, then clearly $K$ contains some vertex in $U$. Therefore, $G - (U - \{x\})$ contains no $K_{\omega(G)}$. Since $G$ is perfect,
                \begin{equation*}
                    \chi\bigl(G - (U - \{x\})\bigr) = \omega\bigl(G - (U - \{x\})\bigr) < \omega(G).
                \end{equation*}
                Note that in $U - \{x\}$, there is no vertex adjacent to $x$, and so $(U - \{x\}) \cup \{x'\}$ is an independent set in $G'$. Therefore, we can assign a new colour to $(U - \{x\}) \cup \{x'\}$. Together with an $\bigl(\omega(G) - 1\bigr)$-colouring of $\bigl(G - (U - \{x\})\bigr)$, this gives an $\omega(G)$-colouring in $G'$, and so 
                \begin{equation*}
                    \omega(G') = \omega(G) \geq \chi(G').
                \end{equation*}
            \end{proof}
        \end{lembox}
        Let $\mathcal{K}$ be the family of cliques and $\mathcal{A}$ be the family of maximal independent sets in $G$. We claim that there exists some $K \in \mathcal{K}$ such that $K \cap A \neq \varnothing$ for all $A \in \mathcal{A}$. Suppose on contrary that for every $K \in \mathcal{K}$, there is some $A_K \in \mathcal{A}$ such that $K \cap A = \varnothing$. For every vertex $x \in V(G)$, define 
        \begin{equation*}
            k(x) \coloneqq \abs{\left\{K \in \mathcal{K} \colon x \in A_K\right\}}.
        \end{equation*}
        Let $G_x \cong K_{k(x)}$ be a clique. We construct a graph $G'$ by 
        \begin{align*}
            V(G') & \coloneqq \left\{G_x \colon x \in V(G)\right\}, \\
            E(G') & \coloneqq \bigcup_{xy \in E(G)}\bigl(V(G_x) \times V(G_y)\bigr),
        \end{align*}
        i.e., $V(G_x) \cup V(G_y)$ induces a clique in $G'$ if and only if $x \sim y$ in $G$. Notice that every $G_x \subseteq G'$ corresponds to some unique $x \in V(G)$, so we can construct $G'$ by repeatedly expanding on vertices of $G$. Since $G$ is perfect, by Lemma \ref{lem:vtxExpand} we know that $G'$ is perfect and so $\chi(G') \leq \omega(G')$. Notice that if $\bigcup_{i = 1}^{m}V\left(G_{x_i}\right)$ induces a maximal clique $K'$ in $G'$, then $\left\{x_1, x_2, \cdots, x_m\right\}$ must also induce a maximal clique $X \in \mathcal{K}$ in $G$. Note that 
        \begin{align*}
            \abs{V(K')} & = \sum_{x \in V(X)}k(x) \\
            & = \sum_{x \in V(X)}\abs{\left\{K \in \mathcal{K}\colon x \in A_K\right\}} \\
            & = \sum_{K \in \mathcal{K}}\abs{\left\{X \cap A_K\right\}}
        \end{align*}
        Since $X$ is a clique, $\abs{X \cap A_K} \leq 1$. Notice that there exists $A_X \in \mathcal{A}$ with $X \cap A_X = \varnothing$, so $\omega(G') = \abs{V(K')} \leq \abs{\mathcal{K}} - 1$. However,
        \begin{align*}
            \abs{V(G')} & = \sum_{x \in V(G)}k(x) \\
            & = \sum_{K \in \mathcal{K}}\abs{A_K} \\
            & = \alpha(G)\abs{\mathcal{K}}.
        \end{align*}
        Take any independent set $A \subseteq V(G')$, then for any $a, b \in A$, $a, b$ are in different cliques. Suppose $a \in G_x$ and $b \in G_y$, then $xy \notin E(G)$ because otherwise $V(G_x) \cup V(G_y)$ induces a clique in $G'$, which is not possible. Therefore, there exists an independent set in $G$ which has the same size as $A$, and so $\alpha(G') \leq \alpha(G)$. Therefore,
        \begin{equation*}
            \chi(G') \geq \frac{\abs{V(G')}}{\alpha(G')} \geq \frac{\alpha(G)\abs{\mathcal{K}}}{\alpha(G)} = \abs{\mathcal{K}},
        \end{equation*}
        which is a contradiction. Therefore, we can take some $K \in \mathcal{K}$ such that $K \cap A \neq \varnothing$ for all $A \in \mathcal{A}$, and so 
        \begin{align*}
            \omega\left(\overline{G} - K\right) & = \alpha(G - K) < \alpha(G) = \omega\left(\overline{G}\right). 
        \end{align*}
        By the inductive hypothesis, $\overline{G} - K$ is perfect, and so $\chi\left(\overline{G} - K\right) = \omega\left(\overline{G} - K\right)$. Fixing a $\chi\left(\overline{G} - K\right)$-colouring in $\overline{G} - K$ and colouring $V(K)$ with a new colour yields a proper colouring in $\overline{G}$ because $V(K)$ is an independent set in $\overline{G}$. Therefore,
        \begin{align*}
            \chi\left(\overline{G}\right) & \leq \chi\left(\overline{G} - K\right) + 1 \\
            & = \omega\left(\overline{G} - K\right) + 1 \\
            & \leq \omega\left(\overline{G}\right).
        \end{align*}
    \end{proof}
\end{thmbox}
There is a stronger characterisation of perfect graphs, which has remained an unsolved conjecture until 2006. We first define the following structure:
\begin{dfnbox}{Odd Hole}{oddHole}
    An induced odd cycle of length at least $5$ is called an {\color{red} \textbf{odd hole}} in a graph. The complement of an odd hole is said to be an {\color{red} \textbf{odd anti-hole}}.
\end{dfnbox}
The characterisation states the following:
\begin{thmbox}{Strong Perfect Graph Theorem}{strongPerfect}
    A graph $G$ is perfect if and only if it contains no odd hole or odd anti-hole.
\end{thmbox}
\end{document}