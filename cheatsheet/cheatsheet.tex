\documentclass[10pt]{article}
\usepackage[a4paper,margin=0.25cm,landscape]{geometry}
\usepackage{multicol}
\setlength{\columnseprule}{1pt}
\usepackage[utf8]{inputenc}
\usepackage{amsmath}
\usepackage{amsfonts}
\usepackage{amssymb}
\usepackage{amsthm}
\usepackage{mathtools}
\usepackage{bm}

\newcommand{\norm}[1]{\left\lVert#1\right\rVert}
\newcommand{\abs}[1]{\left\lvert#1\right\rvert}
\newcommand{\im}{\mathrm{i}}
\newcommand{\R}{\mathbb{R}}

\begin{document}
    \begin{multicols*}{3}
        \begin{itemize}
            \item \textbf{Cayley-Hamilton Thm:} $\chi_T(T) = 0$ for all linear operator $T$.
            \item $T^k = -\frac{1}{c_k}\sum_{i = 0}^{k - 1}c_iT^i$.
            \item $T^{-1} = T^{-1} \circ \mathrm{id}_V = T^{-1} \circ \left(-\frac{1}{c_0}\sum_{i = 1}^{k}c_iT^i\right)$.
            \item Rotation matrix: 
            \begin{equation*}
                \begin{bmatrix}
                    \cos\theta_i & -\sin\theta_i \\
                    \sin\theta_i & \cos\theta_i
                \end{bmatrix} = \begin{bmatrix}
                    1 & -1 \\
                    -\im & -\im
                \end{bmatrix}\begin{bmatrix}
                    \mathrm{e}^{\im\theta} & 0 \\
                    0 & \mathrm{e}^{-\im\theta}
                \end{bmatrix}\begin{bmatrix}
                    1 & -1 \\
                    -\im & -\im
                \end{bmatrix}^{-1}.
            \end{equation*}
            \item Euclidean operators:
            \begin{enumerate}
                \item a series of stretching and/or crushing transformations if it has $n$ distinct real eigenvalues;
                \item a series of rotations followed by a series of stretching and/or crushing transformations if it has $n$ distinct eigenvalues not all real;
                \item a series of shearing transformations followed by a series of stretching and/or crushing transformations if it has less than $n$ distinct real eigenvalues;
                \item a series of shearing transformations, followed by a series of rotations, and then followed by another series of stretching and/or crushing transformations if it has less than $n$ distinct eigenvalues not all real.
            \end{enumerate}
            \item Bilinear forms: $b = \sum_{i = 1}^{n}\sum_{j = 1}^{n}b\left(\bm{z}_i, \bm{z}_j\right)\zeta^i\otimes\zeta^j$.
            \item For every bilinear form $b$ on $V$, there is $b^{\#} \colon V \to \widehat{V}$ such that $b^{\#}(\bm{v})(\bm{u}) = b(\bm{u}, \bm{v})$.
            \item $i$-th component of $b^{\#}(\bm{v})$: $b^{\#}(\bm{v})(\bm{z}_i) = b(\bm{z}_i, \bm{v}) = \sum_{j = 1}^{n}b_{ij}v_j$.
            \item Change-of-basis:
            \begin{enumerate}
                \item In general, $\bm{P}$ is the matrix for $z^{-1} \circ y$, and $M_y(T) = \bm{P}^{-1}M_z(T)\bm{P}$.
                \item Bilinear form: $M_y(b) = \bm{P}^{\mathrm{T}}M_z(b)\bm{P}$.
                \item Change between orthonormal bases: $\bm{P}^{\mathrm{T}}\bm{P} = \bm{I}$. Bilinear form has $\bm{I}$ under orthonormal basis.
                \item Sesquilinear form: $M_y(b) = \bm{P}^{\mathrm{T}}M_z(b)\overline{\bm{P}}$.
                \item Unitary matrix: $\overline{\bm{P}^{\mathrm{T}}}\bm{P} = \bm{I}$. \textbf{Any unitary matrix forms an orthonormal basis!}
            \end{enumerate}
            \item \textbf{Orthogonal decomposition:} $\bm{u} = \frac{g(\bm{u}, \bm{v})}{\abs{\bm{v}}^2}\bm{v} + \bm{u} - \frac{g(\bm{u}, \bm{v})}{\abs{\bm{v}}^2}\bm{v}$.
            \item \textbf{Cauchy-Schwarz Inequality:} $g(\bm{u}, \bm{v}) \leq \abs{\bm{u}}\abs{\bm{v}}$.
            \item \textbf{Angle:} $\cos\theta = \frac{g(\bm{u}, \bm{v})}{\abs{\bm{u}}\abs{\bm{v}}}$.
            \item \textbf{Orthonormal basis:} $g(\bm{u}, \bm{v}) = z^{-1}(\bm{u}) \cdot z^{-1}(\bm{v})$.
            \item \textbf{Gram-Schmidt:}
            \begin{equation*}
                \bm{z}_1^+ = \bm{z}_1, \qquad \bm{z}_k^+ \coloneqq \frac{\bm{z}_k - \sum_{i = 1}^{k - 1}g\left(\bm{z}_k, \bm{z}_i^+\right)\bm{z}_i^+}{\abs{\bm{z}_k - \sum_{i = 1}^{k - 1}g\left(\bm{z}_k, \bm{z}_i^+\right)\bm{z}_i^+}}.
            \end{equation*}
            Matrix representation is upper-triangular.
            \item \textbf{Riesz-Representation:} For any inner product space $(V, g)$, let the mapping $\Gamma \colon V \to \widehat{V}$ be such that
            \begin{equation*}
                \Gamma(\bm{u})(\bm{v}) = g(\bm{v}, \bm{u}),
            \end{equation*}
            then for every $\alpha \in \widehat{V}$, there is a unique $\bm{u}_{\alpha} \in V$ such that $\alpha = \Gamma\left(\bm{u}_{\alpha}\right)$.
            \item $\zeta^i = \Gamma(\bm{z}P_i)$, $i$-th component of $\bm{v}$ is $\Gamma(\bm{z}_i)(\bm{v})$, $i$-th component of $\Gamma(\bm{v})$ is $\Gamma(\bm{v})(\bm{z}_i) = \sum_{j = 1}^{n}g_{ij}v_j$.
            \item $\Gamma(\bm{v}) = \sum_{i = 1}^{n}\sum_{j = 1}^{n}g_{ij}v_j\zeta^i$.
            \item If $T$ is an operator over $V$, then $b_T \coloneqq \Gamma \circ T$ is a bilinear form over $V$, $M_z(b_T) = M_z(g)M_z(T)$. If $b$ is a bilinear form over $V$, then $\Gamma^{-1} \circ b$ is an operator over $V$.
            \item Riesz-equivalent: $b(\bm{u}, \bm{v}) = (\Gamma \circ T)(\bm{v})(\bm{u})$ or $b(\bm{u}, \bm{v}) = g\bigl(\bm{u}, T(\bm{v})\bigr)$, have the (conjugate-)same matrix representation under orthonormal basis.
            \item Riesz-equivalent operator and sesquilinear form: $M_z(s_T) = M_z(g)\overline{M_z(T)}$.
            \item \textbf{Schur's Triangularisation Thm:} let $T$ be a complex operator with basis $y$, then there is an orthonormal basis $z$ such that $M_z(T)$ is upper-triangular, where $M_z(T) = \bm{G}^{-1}\bm{M}^{-1}\bm{G}^{-1}M_y(T)\bm{G}\bm{M}\bm{G}$. $\bm{G}$ is the Gram-Schmidt upper-triangular matrix and $\bm{M}$ changes to an upper-triangular matrix. Here, $\bm{M}\bm{G}$ changes between orthonormal bases and so is unitary.
            \item \textbf{Hermitian:} $s(\bm{u}, \bm{v}) = \overline{s(\bm{v}, \bm{u})}$.
            \item If $\tau$ is a Riesz-equivalent sesquilinear form to $T$, then $\tau$ is Hermitian iff $g\bigl(\bm{u}, T(\bm{v})\bigr) = g\bigl(T(\bm{u}), \bm{v}\bigr)$.
            \item \textbf{Spectral Thm:} Every Hermitian sesquilinear form over a complex inner product space has a real diagonal matrix representation under some orthonormal basis.
            \item Let $T$ be a Riesz-equivalent operator to a Hermitian sesquilinear form $s$, then all eigenvalues of $T$ are real.
            \item Every symmetric bilinear form $b$ over a real inner product space $V$ has a real diagonal matrix representation under some orthonormal basis such that there is a real eigenvector associated to each eigenvalue.
            \item \textbf{Wedge product:} $\alpha \wedge \beta \coloneqq \alpha \otimes \beta - \beta \otimes \alpha$ is a two-form.
            \item Dimension of $m$-form space in $n$-dimensional space: $\binom{n}{m}$.
            \item $n$-form space: $\left\{\lambda\bigwedge_{i = 1}^n\zeta^i \colon \lambda \in \mathbb{F}\right\}$.
            \item \textbf{Determinant:} For any $n$-form $\Omega$, $\widehat{T}(\Omega) = \Delta(T)\Omega$. Hermitian operators have real determinant. Non-zero determinant iff bijective.
            \item \textbf{Volume:} $\Theta \colon V^n \to \mathbb{F}$ such that 
            \begin{enumerate}
                \item $\Theta\left(\bm{u}_1, \bm{u}_2, \cdots, \bm{u}_n\right) \geq 0$;
                \item $\Theta\left(\bm{u}_1, \bm{u}_2, \cdots, \bm{u}_n\right) \neq 0$ if and only if $\bm{u}_1, \bm{u}_2, \cdots, \bm{u}_n$ are linearly independent;
                \item $\Theta\left(\bm{u}_1, \bm{u}_2, \cdots, c\bm{u}_i, \cdots \bm{u}_n\right) = c\Theta\left(\bm{u}_1, \bm{u}_2, \cdots, \bm{u}_n\right)$ for any $c \in \mathbb{F}$;
                \item $\Theta\bigl(\bm{u}_1, \bm{u}_2 \cdots, \bm{u}_i, \cdots, \bm{u}_j + c\bm{u}_i, \cdots \bm{u}_n\bigr) = \Theta\left(\bm{u}_1, \bm{u}_2, \cdots, \bm{u}_n\right)$ for any $c \in \mathbb{F}$;
                \item $\Theta\left(\bm{u}_1, \bm{u}_2, \cdots, \bm{u}_n\right) = 1$ whenever $\left\{\bm{u}_1, \bm{u}_2, \cdots, \bm{u}_n\right\}$ is an orthonormal basis.
            \end{enumerate}
            \item \textbf{Volume form:} Let $V$ be an $n$-dimensional inner product space over a well-ordered field $\mathbb{F}$. Take any orthonormal basis $\left\{\bm{z}_1, \bm{z}_2, \cdots, \bm{z}_n\right\}$ with dual basis $\left\{\zeta^1, \zeta^2, \cdots, \zeta^n\right\}$. 
            \begin{equation*}
                \Theta \coloneqq \abs{\omega_z} = \abs{\bigwedge_{i = 1}^n\zeta^i}.
            \end{equation*}
            $\omega_z$ is called the \textbf{Orientation}.
            \item $\Theta\left(\bm{u}, \bm{v}, \bm{w}\right) = \abs{\bm{u} \cdot \bm{v} \times \bm{w}}$
            \item For general basis: $\Theta = \sqrt{\Delta(g)}\abs{\bigwedge_{i = 1}^n\eta^i}$.
        \end{itemize}
    \end{multicols*}
\end{document}
