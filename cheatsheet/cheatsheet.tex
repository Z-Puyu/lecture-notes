\documentclass[10pt]{article}
\usepackage[a4paper,margin=0.25cm,landscape]{geometry}
\usepackage{multicol}
\setlength{\columnseprule}{1pt}
\usepackage[utf8]{inputenc}
\usepackage{amsmath}
\usepackage{amsfonts}
\usepackage{amssymb}
\usepackage{amsthm}
\usepackage{mathtools}
\usepackage{bm}

\newcommand{\C}{\mathbb{C}} % Complex numbers
\newcommand{\R}{\mathbb{R}} % Real numbers
\newcommand{\N}{\mathbb{N}} % Natural numbers
\newcommand{\Q}{\mathbb{Q}} % Rational Numbers
\newcommand{\Z}{\mathbb{Z}} % Integers
\newcommand{\e}{\mathrm{e}} % Natural Constant
\newcommand{\Id}{\bm{I}} % Identity
\newcommand{\zero}{\bm{0}} % Zero vector
\newcommand{\powerset}[1]{\mathcal{P}\left(#1\right)} % Power set
\newcommand{\pbrac}[1]{\left(#1\right)} % Parenthesis
\newcommand{\sbrac}[1]{\left[#1\right]} % Square brackets
\newcommand{\cbrac}[1]{\left\{#1\right\}} % Curly brackets
\newcommand{\abs}[1]{\left\lvert#1\right\rvert} % Modulus
\newcommand{\norm}[1]{\left\lVert#1\right\rVert} % Norm
\newcommand{\func}[3]{#1: #2 \to #3} % Function
\newcommand{\map}[3]{#1: #2 \mapsto #3} % Mapping
\newcommand{\image}[2]{#2\left[#1\right]} % Image
\newcommand{\preimage}[2]{#2\left[#1\right]^{-1}} % Pre-image
\newcommand{\eval}[3]{\left. #1\right\rvert_{#2 = #3}} % Evaluated at
\newcommand{\dif}[3][]{\frac{\mathrm{d}^{#1}#2}{\mathrm{d}#3^{#1}}} % Derivative (dy/dx)
\newcommand{\diff}[3][]{\frac{\mathrm{d}^{#1}}{\mathrm{d}#3^{#1}}#2} % Alternate derivative (d/dx f(x))
\newcommand{\pardif}[3][]{\frac{\partial^{#1} #2}{\partial #3^{#1}}} % Partial derivative
\newcommand{\pardiff}[3][]{\frac{\partial^{#1}}{\partial #3^{#1}}#2} % Alternate partial derivative
\newcommand{\indefint}[2]{\int #1\,\mathrm{d}#2} % Indefinite integral
\newcommand{\defint}[4]{\int_{#1}^{#2}#3\,\mathrm{d}#4} % Definite integral
\newcommand{\tspose}[1]{#1^{\mathrm{T}}} % Transpose
\newcommand{\refto}{\xrightarrow[\mathrm{Elimination}]{\mathrm{Gaussian}}} % Row-echelon form
\newcommand{\rrefto}{\xrightarrow[\mathrm{Elimination}]{\mathrm{Gauss-Jordan}}} % Reduced row-echelon form
\newcommand{\ero}[2][]{\xrightarrow[#1]{#2}} % Elementary row operation

\begin{document}
    \begin{multicols*}{3}
        \begin{itemize}
            \item \textbf{Topology}: $\mathcal{T} \subseteq \mathcal{P}\left(X\right)$ s.t. 
            \begin{itemize}
                \item $\varnothing, X \in \mathcal{T}$;
                \item closed under \textbf{arbitrary union} and \textbf{finite intersection}.
            \end{itemize}
            \item \textbf{Co-finite topology}: set of \textbf{complements of finite subsets}.
            \item \textbf{Basis}: 
            \begin{itemize}
                \item $\forall x \in X$, $\exists B \in \mathcal{B}$ s.t. $x \in B \iff X \subseteq \bigcup_{B \in \mathcal{B}}B$.
                \item $\forall x \in X$ and $B_1, B_2 \in \mathcal{B}$ with $x \in B_1 \cap B_2$, $\exists B \in \mathcal{B}$ s.t. $x \in B \subseteq B_1 \cap B_2$.
            \end{itemize}
            \item \textbf{Topology generated by basis}: set of all unions of sets in $\mathcal{B}$.
            \item $\mathcal{T}_1 \subseteq \mathcal{T}_2 \iff \mathcal{T}_2$ is finer.
            \item Topology generated by $\mathcal{B}$ is the coarsest topology containing $\mathcal{B}$.
            \item $\mathcal{T}_1 \subseteq \mathcal{T}_2 \iff \forall B \in \mathcal{B}_1$ and $\forall x \in B$, $\exists B_x \in \mathcal{B}_2$ s.t. $x \in B_x \subseteq B$.
            \item \textbf{sub-basis}: $\bigcup_{S \in \mathcal{S}}S = X$ and every basis is a sub-basis.
            \item All finite intersections of sets in a sub-basis is a basis.
            \item \textbf{Metric}: \textbf{positive, definite, symmetric, $\triangle$-inequality}.
            \item \textbf{Pseudo-metric}: $d\left(x, x\right) = 0$ but can be not definite.
            \item \textbf{Quasi-metric}: can be not symmetric.
            \item \textbf{Norm}: \textbf{positive, definite, $\triangle$-inequality}, $\norm{\lambda x} = \abs{\lambda}\norm{x}$.
            \item \textbf{Distance between sets}: smallest pointwise distance.
            \item \textbf{Diameter of set}: greatest pointwise distance.
            \item \textbf{Metrisable} topology: induced with open balls.
            \item $L^p$\textbf{-metric}: generates the standard topology on $\R$.
            \begin{equation*}
                \max\norm{y_i - x_i} \leq \left[\sum_{i = 1}^{n}\norm{y_i - x_i}^p\right]^{\frac{1}{p}} \leq n^{\frac{1}{p}}\max\norm{y_i - x_i}.
            \end{equation*}
            \item Metrics are equivalent iff $c_1d \leq d' \leq c_2d$.
            \item \textbf{Subspace topology}: $\left\{U \cap X \colon U \textrm{ is open}\right\}$. Basis is analogous.
            \item Open sets in open subspace is open in superspace.
            \item \textbf{Subspace metric}: restriction to subspace. Induces subspace topology with respect to metrisable topology.
            \item If $Y \subseteq X$, then $A$ is closed in $Y \iff \exists G$ closed in $X$ s.t. $A = G \cap Y$.
            \item Closed sets in closed subspace is closed in superspace.
            \item \textbf{Interior} $\mathring{A}$: union of all open subsets of $A$.
            \item \textbf{Closure} $\overline{A}$: intersection of all closed supersets of $A$. \textbf{Smallest closed superset of $A$}.
            \item \textbf{Boundary} $\partial A = \overline{A} \setminus \mathring{A}$.
            \item \textbf{Limit point}: $\left(A \setminus \left\{x\right\}\right) \cap U \neq \varnothing$ for any open $U$.
            \item $x \in \overline{A} \iff \forall$ open neighbourhood $U$ of $x$, $U \cap A \neq \varnothing$;
            \item $\overline{A} = A \cup A'$, i.e., closure is the set plus all its limit point.
            \item Limit may not be a limit point.
            \item \textbf{Continuity}: $U$ open $\implies f^{-1}\left(U\right)$ open, equivalent to $f^{-1}\left(S\right)$ is open $\forall S$ in sub-basis.
            \item TFAE:
            \begin{enumerate}
                \item $f$ is continuous;
                \item for all $A \subseteq X$, $f\left(\overline{A}\right) \subseteq \overline{f\left(A\right)}$;
                \item for any closed set $B \subseteq Y$, $f^{-1}\left(B\right)$ is closed in $X$;
                \item $\forall x \in X$ and any open $V \subseteq Y$ with $f\left(x\right) \in V$, there is an open set $U \subseteq X$ s.t. $x \in U$ and $f\left(U\right) \subseteq V$.
            \end{enumerate}
            \item \textbf{Pasting lemma}: if $X = A \cup B$ for closed $A, B$ and $f\left(x\right) = g\left(x\right)$ for all $x \in A \cap B$, then 
            \begin{equation*}
                h\left(x\right) = \begin{cases}
                    f\left(x\right) & \quad\textrm{if } x \in A \\
                    g\left(x\right) & \quad\textrm{if } x \in B
                \end{cases}
            \end{equation*}
            is continuous if $f \colon A \to Y$ and $g \colon B \to Y$ are.
            \item \textbf{Pull-back} topology: $\left\{f^{-1}\left(U\right) \colon U \textrm{ is open}\right\}$ is the coarsest topology ensuring continuous $f$.
            \item \textbf{Uniform continuity}: $\forall \epsilon > 0$, there exists some $\delta > 0$ s.t. $d_X\left(x, y\right) < \delta \implies d_Y\bigl(f\left(x\right), f\left(y\right)\bigr) < \epsilon$.
            \item $f$ is uniformly continuous $\iff \forall \left\{x_i\right\}_i^{\infty}, \left\{y_i\right\}_{i}^{\infty}$ s.t. $\lim_{i \to \infty}d_X\left(x_i, y_i\right) = 0$, $\lim_{i \to \infty}d_Y\bigl(f\left(x_i\right), f\left(y_i\right)\bigr) = 0$.
            \item $\left\{f_n\right\}$ \textbf{converges pointwisely}: $\forall x$, $f_n\left(x\right) to f\left(x\right)$.
            \item $\left\{f_n\right\}$ \textbf{converges uniformly}: $\forall \epsilon > 0$, $\exists N \in \N^+$ s.t. $\forall n \geq N, \forall x \in X$, $d\bigl(f_n\left(x\right), f\left(x\right)\bigr) < \epsilon$.
            \item Limit of uniformly convergent sequence is continuous.
            \item \textbf{Projection} $\pi_{X_{\beta}} \coloneqq \bm{x} \mapsto x_{\beta}$, $\pi_{X_{\beta}}^{-1}\left(U\right)$ is all vectors whose $\beta$-th component is in $U$.
            \item \textbf{Product} topology is generated by the sub-basis of all pre-images of all projections.
            \item \textbf{Box} topology is generated by the basis of all products of open sets.
            \item Box topology and product topology are equal only for finite product.
            \item Product topology is the coarsest topology to ensure continuous projection.
            \item $f\left(y\right) = \bigl(f_{\alpha}\left(y\right)\bigr)_{\alpha \in \Lambda}$ is continuous iff $f_{\alpha}$'s are continuous.
            \item Subspace topology of product topology equals product topology of subspace topologies.
            \item Standard topology on $\R^n$ is the product topology by standard topologies on $\R^{m_i}$'s.
            \item Product of basis is the basis for product topology.
            \item $d_1\left(\bm{x}, \bm{y}\right) = \sum_{i = 1}^{n}d_{X_i}\left(x_i, y_i\right)$ and $d_{\infty}\left(\bm{x}, \bm{y}\right) = \max d_{X_i}\left(x_i, y_i\right)$ both induce the product topology.
            \item $\rho\left(x, y\right) \coloneqq \frac{d\left(x, y\right)}{1 + d\left(x, y\right)}$ is a metric with $\mathrm{diam}\left(X\right) < 1$. $\rho$ and $d$ generate the same topology.
            \item $d\left(\bm{x}, \bm{y}\right) \coloneqq \sup \left\{\frac{\rho_{i}\left(x_i, y_i\right)}{i} \colon i \in \Z^+\right\}$ is a metric inducing the infinite product topology.
            \item \textbf{Quotient map}: surjective and $U$ is open $\iff f^{-1}\left(U\right)$ is open.
            \item \textbf{Open map}: continuous map from open set to open set.
            \item Surjective continuous $+$ open or closed $=$ quotient map.
            \item Quotient, open and closed maps are preserved under $\circ$.
            \item \textbf{Saturated}: pre-image under a surjective continuous map. $A = f^{-1}\bigl(f\left(A\right)\bigr)$.
            \item Surjective continuous $f$ is a quotient map iff $f\left(A\right)$ is open (closed) in $Y$ whenever $A$ is saturated open (closed).
            \item Restriction of quotient map to a saturated set is a quotient map.
            \item \textbf{Quotient topology}: unique topology on co-domain to ensure quotient map.
            \item $T_1$: $\forall x \neq y$, $\exists$ open $U$ containing only $x$.
            \item $T_2$ (\textbf{Hausdorff}): $\forall x \neq y$, $\exists$ disjoint neighbourhoods.
            \item Co-finite topology is $T_1$ and $T_2$ if $X$ is finite.
            \item $T_1 \iff \left\{x\right\}$ is closed $\forall x \in X$.
            \item Metric spaces are $T_2$ so all finite subsets are closed.
            \item \textbf{Countable basis}: countable $\mathcal{B}$ s.t. $\forall$ open $Y$ containing $x$, $\exists B \in \mathcal{B}, B \subseteq Y$. \textbf{First countable} if every $x$ has a countable basis.
            \item Uncountable co-finite have no countable basis.
            \item $\exists$ nested countable basis $B_1 \subseteq B_2 \subseteq \cdots$.
            \item Limit $x \in \overline{A}$, if $X$ is first countable, then $x \in \overline{A}$ is a limit.
            \item If $f$ is continuous, then for any sequence $f\left(x_i\right) \to f\left(x\right)$. The converse is true if $X$ is first countable.
            \item Closed subspace of compact space is compact.
            \item Subset of co-finite space is compact but closed iff it's finite.
            \item Compact subspace of Hausdorff space is closed.
            \item Continuous $f$ maps compact set to compact set.
            \item \textbf{Tube lemma}: If $Y$ is compact and $N \subseteq X \times Y$ is open and contains $\left\{x_0\right\} \times Y$, then $\exists W \supseteq \left\{x_0\right\}$ open s.t. $W \times Y \subseteq N$.
            \item Cartesian product of compact spaces is compact.
            \item \textbf{Finite intersection property}: $\mathcal{G} \subseteq \mathcal{P}\left(X\right)$ s.t. finite intersections of sets in $\mathcal{G}$ are non-empty.
            \item $X$ is compact iff for any collection of closed sets $\mathcal{G}$ with the finite intersection property, we have $\bigcap_{G \in \mathcal{G}}G \neq \varnothing$.
            \item $x$ is \textbf{isolated} $\iff \left\{x\right\}$ is open.
            \item If $U \neq \varnothing$ is open in a Hausdorff space and $x \in X$ is not isolated, then $\exists$ non-empty open $V \subseteq U$ s.t. $x \notin \overline{V}$.
            \item Non-empty Hausdorff space is uncountable if it has no isolated point.
            \item \textbf{Limit point compact}: every infinite $Y \subseteq X$ has a limit point in $X$. Limit point compact $\not\implies$ compact but compact $\implies$ limit point compact.
            \item \textbf{Sequentially compact}: every sequence has a convergent subsequence. Sequentially compact $\implies$ limit point compact but limit point compact $\not\implies$ sequentially compact.
            \item $\mathcal{U}$: open cover for a metric space $X$. $\delta > 0$ is a \textbf{Lebesgue number} for $\mathcal{U}$ if $\forall S \subseteq X$ with $\mathrm{diam}\left(S\right) < \delta$, $\exists U \in \mathcal{U}$ s.t. $S \subseteq U$. 
            \item Every open cover of sequentially compact metric space has a Lebesgue number.
            \item \textbf{Totally bounded}: $\forall \epsilon > 0$, $\exists$ finite cover of $X$ by $B_{\epsilon}\left(x_i\right)$.
            \item Every sequentially compact metrisable topological space is totally bounded.
            \item If $X$ is a metrisable topological space, TFAE:
            \begin{enumerate}
                \item $X$ is compact;
                \item $X$ is limit point compact;
                \item $X$ is sequentially compact.
            \end{enumerate}
            \item Let $\left(X, d_X\right)$ and $\left(Y, d_Y\right)$ be metric spaces and $f \colon X \to Y$ be continuous. If $X$ is compact, then $f$ is uniformly continuous.
        \end{itemize}
    \end{multicols*}
\end{document}