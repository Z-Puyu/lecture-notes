\documentclass[11pt]{article}
\usepackage{amsthm}
\usepackage{amsmath}
\usepackage{amssymb}
\usepackage{amsfonts}
\usepackage{bm}
\usepackage{mathtools}
\usepackage{multicol}
\setlength{\columnseprule}{1pt}
\usepackage[utf8]{inputenc}
\usepackage[a4paper,margin=1cm,landscape]{geometry}

\begin{document}
    \begin{multicols*}{3}
        \textbf{P \& C}: $P^{n + 1}_r = P^n_r + rP^n_{r - 1}$.

        \textbf{Circular permutation}: $Q^n_r = \frac{P^n_r}{r}$.
        
        $C^{n + 1}_r = C^n_{r - 1} + C^n_r$.
        
        $H^n_r = C^{r + n - 1}_r$.
        
        Arrange $r$ distinct objects around $n$ identical circles such that no circle is empty: $s(r, n) = s(r - 1, n - 1) + (r - 1)s(r - 1, n)$.
        \\\\
        $s(r, r - 1) = C^r_2$.
        \\\\
        \textbf{Binomial \& Multinomial}:
        \\
        $\begin{pmatrix}
            n \\
            r
        \end{pmatrix} = \frac{n}{r}\begin{pmatrix}
            n - 1 \\
            r - 1
        \end{pmatrix}$.
        \\\\
        $\begin{pmatrix}
            n \\
            r
        \end{pmatrix} = \frac{n - r + 1}{r}\begin{pmatrix}
            n \\
            r - 1
        \end{pmatrix}$.
        \\\\
        $\begin{pmatrix}
            n \\
            m
        \end{pmatrix}\begin{pmatrix}
            m \\
            r
        \end{pmatrix} = \begin{pmatrix}
            n \\
            r
        \end{pmatrix}\begin{pmatrix}
            n - r \\
            m - r
        \end{pmatrix}$.
        \\\\
        \textbf{Vandermonde's Identity}: 
        \begin{equation*}
            \sum_{i = 0}^{r}\left[\begin{pmatrix}
                m \\
                i
            \end{pmatrix}\begin{pmatrix}
                n \\
                r - i
            \end{pmatrix}\right] = \begin{pmatrix}
                m + n \\
                r
            \end{pmatrix}.
        \end{equation*}
        \textbf{Chu Shih-Chieh Identity}:
        \begin{align*}
            \sum_{i = 0}^{n - r}\begin{pmatrix}
                r + i \\
                r
            \end{pmatrix} & = \begin{pmatrix}
                n + 1 \\
                r + 1
            \end{pmatrix} \\
            \sum_{i = 0}^{k}\begin{pmatrix}
                r + i \\
                i
            \end{pmatrix} & = \begin{pmatrix}
                r + k + 1 \\
                k
            \end{pmatrix}.
        \end{align*}
        \textbf{Multinomial coefficient}:
        \begin{equation*}
            \begin{pmatrix}
                n \\
                n_1, n_2, \cdots, n_m
            \end{pmatrix} = \frac{n!}{\prod_{i = 1}^{m}n_i!}.
        \end{equation*}
        \textbf{Pigeonhole Principle}: If at least $kn + 1$ objects are distributed into $n$ distinct sets, then there exists a set with at least $k + 1$ objects.
        \\\\
        \textbf{Generalised PP}: If at least $\sum_{i = 1}^{n}k_i + (n - 1)$ distinct objects are distributed into $n$ distinct sets, then there exists at least one set ($i$-th) with at least $k_i$ objects.
        \\\\
        \textbf{Ramsey Numbers}: $R(p, q) \leq R(p - 1, q) + R(p, q - 1)$. (Bound lowered by $1$ if both on RHS are even.)
        \\\\
        $R(2, q) = q$, $R(1, q) = 1$.
        \\\\
        \textbf{Distribution Problems}:
        \\
        \textbf{Distinct into distinct}:
        \begin{itemize}
            \item Each box at most 1: $P^n_r$.
            \item Each box any number of objects: $n^r$.
            \item Each box any number of objects with internal ordering: $\frac{(n - 1 + r)!}{(n - 1)!}$.
        \end{itemize}
        \textbf{Identical into distinct}: 
        \begin{itemize}
            \item Each box any number of objects: $H^n_r = C^{n + r - 1}_r$.
            \item No box empty: $H^n_{r - n} = C^{r - 1}_{r - n}$.
        \end{itemize}
        \textbf{Distinct into identical}:
        \begin{itemize}
            \item No box empty: $S(r, n) = S(r - 1, n - 1) + nS(r - 1, n)$
        \end{itemize}
        Number of partitions of $A$ with $\left\lvert A \right\rvert = n$: $\sum_{i = 1}^{n}S(n, i)$.
        \\\\
        Number of surjective mapping from $[1, r] \cap \mathbb{N}$ to $[1, n] \cap \mathbb{N}$: $F(r, n) = \sum_{k = 0}^{n}(-1)^kC^n_k(n - k)^r$.
        \\\\
        $S(r, n) = \frac{1}{n!}F(r, n)$.
        \\\\
        $D(n, r, k) = \frac{C^r_k}{(n - r)!}\sum_{i = 0}^{r - k}(-1)^iC^{r - k}_i(n - k - i)!$.
        \\\\
        $D_n = D(n, n, 0) = n!\sum_{i = 0}^{n}\frac{(-1)^i}{i!}$.
        \\\\
        $\varphi(n) = n\prod_{i = 1}^{k}\left(1 - \frac{1}{p_i}\right)$.
        \\\\
        $\varphi(2n) = \varphi(n)$ ($n$ odd) or $2\varphi(n)$ ($n$ even)

        \textbf{Identicial into Identical}
        \begin{itemize}
            \item Number of partitions of $n$ into $k$ parts equals the number of partition of $k$ where the largest part has size $k$.
            \item $r$-partition of $n$ of size at most $k$: $\frac{1}{\prod_{i = 1}^{k}(1 - x^i)}$.
            \item Number of partitions of $r$ into distinct parts is the number of partitions of $r$ into odd parts.
            \item Number of partitions of $r$ into at most $k$ parts is the number of partitions of $r$ into parts whose size does not excced $k$
        \end{itemize}
        $\sum_{k = 1}^{m}p(n, k) = p(n + m, m)$.
        \\\\
        \textbf{GPIE}: 
        \begin{equation*}
            E(m) = \sum_{k = m}^{q}(-1)^{k - m}C^k_m\omega(k).
        \end{equation*}
        \textbf{OGF (Identical into distinct \& identical into identical)}:
        \begin{itemize}
            \item $\begin{pmatrix}
                \alpha \\
                r
            \end{pmatrix} = \frac{\prod_{i = 0}^{r - 1}(\alpha - i)}{r!}$.
            \item $(1 \pm x)^\alpha = \sum_{r = 0}^{\infty}\begin{pmatrix}
                \alpha \\
                r
            \end{pmatrix}(\pm x)^r$.
            \item $\frac{1}{1 - kx}$ generates $(1, k, k^2, \cdots)$.
            \item $(1 - x)^{-n} = \sum_{i = 0}^{\infty}\begin{pmatrix}
                n - 1 + i \\
                i
            \end{pmatrix}x^i$.
            \item $\alpha A(x) + \beta B(x)$ generates $\alpha a_r + \beta b_r$.
            \item $A(x)B(x)$ generates $\sum_{i = 0}^{r}a_ib_{r - i}$.
            \item $x^mA(x)$ translates $a_i$ to $a_{i + m}$.
            \item $A(kx)$ generates $k^ra_r$.
            \item $(1 - x)A(x)$ generates $c_r = a_r - a_{r - 1}$.
            \item $\frac{A(x)}{1 - x}$ generates $c_r = \sum_{i = 0}^{r}a_i$.
            \item $A'(x)$ generates $(r + 1)a_{r + 1}$.
            \item $xA'(x)$ generates $ra_r$.
            \item $r$-combination of multi-set: $\prod_{i = 1}^{k}\left(\sum_{j = 0}^{n_i}x^j\right)$.
        \end{itemize}
        \textbf{EGF (Distinct into distinct)}:
        \begin{itemize}
            \item $\mathrm{e}^x = \sum_{i = 0}^{\infty}\frac{x^i}{i!}$ generates $a_r = 1$.
            \item $\frac{1}{1 - x} = \sum_{i = 0}^{\infty}x_i$ generates $a_r = r!$.
            \item $\mathrm{e}^{kx} = \sum_{i = 0}^{\infty}\frac{(kx)^i}{i!}$ generates $a_r = k^r$.
            \item $(1 + x)^n = \sum_{i = 0}^{n}i!C^n_r\frac{x^i}{i!}$ generates $P^n_r$.
            \item $r$-permutation of multi-set: $\prod_{i = 1}^{k}\left(\sum_{j = 0}^{n_i}\frac{x^j}{j!}\right)$
            \item $\frac{\mathrm{e}^x + \mathrm{e}^{-x}}{2}$: even number of elements.
            \item $\frac{\mathrm{e}^x - \mathrm{e}^{-x}}{2}$: odd number of elements.
        \end{itemize}
        \textbf{Particular solutions:}
        \begin{center}
            \begin{tabular}{|c|l|}
                \hline
                $f(n)$ &  $a_n^{(p)}$ \\
                \hline
                $Ak^n$ & $\begin{cases}
                    Bk^n & k \textrm{ not a root} \\
                    Bn^mk^n & k \textrm{ has multiplicity } m
                \end{cases}$ \\
                \hline
                $\sum_{i = 0}^{t}p_in^i$ & $\begin{cases}
                    \sum_{i = 0}^{t}q_in^i & 1 \textrm{ not a root} \\
                    n^m\sum_{i = 0}^{t}p_in^i & 1 \textrm{ multiplicity } m
                \end{cases}$ \\
                \hline
                $An^tk^n$ & $\begin{cases}
                    \left(\sum_{i = 0}^{t}q_in^i\right)k^n \\
                    n^m\left(\sum_{i = 0}^{t}p_in^i\right)k^n
                \end{cases}$ \\
                \hline
            \end{tabular}
        \end{center}
        \textbf{Graph}:
        \begin{itemize}
            \item \textbf{Handshaking Lemma}: $\sum d_G(v) = 2e(G)$. The number of vertices with odd degrees is even.
            \item A subgraph $H$ of $G$ is induced iff $H = G - (V(G) - V(H))$.
            \item $G \cong H$ iff $\overline{G} \cong \overline{H}$.
            \item If $G \cong H$, then
            \begin{itemize}
                \item $G$ and $H$ have same order and size.
                \item $\delta(G) = \delta(H)$ and $\Delta(G) = \delta(H)$.
                \item Number of vertices with degree $i$ is the same.
            \end{itemize}
            \item $(d_1, d_2, \cdots, d_n)$ is graphic iff $(d_2 - 1, \cdots, d_{d_1 + 1} - 1, d_{d_1 + 2}, \cdots, d_n)$ is graphic.
            \item Order of self-complementary $G$ is either $4k$ or $4k + 1$.
            \item $\exists u$-$v$ walk of length $k \implies \exists u$-$v$ path of length at most $k$.
            \item $\omega(G)$: number of components of $G$.
            \item Complement of connected graph is connected.
            \item $v$ is a cut-vertex iff $\exists a, b$ such that $v$ is in every $a$-$b$ path.
            \item $e$ is a bridge iff it is not part of any cycle.
            \item $uv$ is a bridge and $u$ is not end vertex $\implies u$ is a cut-vertex.
            \item A graph with order at least $3$ which contains a bridge contains a cut-vertex.
            \item $\mathrm{rad}(G) \leq \mathrm{diam}(G) \leq 2\mathrm{rad}(G)$.
            \item Incidence matrix is $v(G)$ rows $e(G)$ columns.
            \item $(i, j)$ entry of $\bm{A}^k$ is the number of $v_i$-$v_j$ walks of length $k$.
            \item $G$ is connected if and only if the $(i, j)$ entry of $\sum_{i = 1}^{n - 1}\bm{A}^i$ is nonzero for all $i \neq j$.
            \item Size of bipartite is sum of degrees of any partite set.
            \item Join: $V(G + H) = V(G) \cup V(H)$, $E(G, H) = E(G) \cup E(H) \cup \left\{uv \colon u \in V(G), v \in V(H)\right\}$.
            \item Bipartite iff no odd cycles.
            \item $T$ is a tree iff every two vertices are joined by a unique path iff $e(T) = v(T) - 1$.
            \item If $\Delta(T) = k$, then $n_1 = 2 + \sum_{i = 1}^{k - 2}in_{i + 2}$.
            \item A tree of order at least $2$ contains at least $2$ end vertices.
            \item The centre of a tree is either $K_1$ or $K_2$.
            \item A graph is connected iff it contains a spanning tree.
            \item $\tau(G) = \tau(G - e) + \tau(G \circ e)$.
            \begin{itemize}
                \item $\tau(C_n) = n$.
                \item Connected with a cut-vertex or a bridge: $\tau(G) = \tau(G_1)\tau(G_2)$.
                \item $C_p$ and $C_q$ sharing a commen edge: $\tau(G) = p + q - 2 + (p - 1)(q - 1)$.
                \item $C_p$ with a duplicated edge: $\tau(G) = 2p - 1$.
                \item $C_p$ and $C_q$ sharing a pair of duplicated edges: $\tau(G) = p + q - 2 + 2(p - 1)(q - 1)$.
                \item $\tau(K_n) = n^{n - 2}$.
                \item $\tau(K_{2, r}) = 2^{r - 1}r$.
            \end{itemize}
            \item \textbf{Matrix tree theorem:} $\tau(G)$ is the cofactor of any entry of $\bm{C - A}$ where $\bm{C}$ is a diagonal matrix containing the degrees of vertices.
        \end{itemize}
        
        
    \end{multicols*}
\end{document}