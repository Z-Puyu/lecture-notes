\documentclass[math, code]{amznotes}
\usepackage[utf8]{inputenc}
\usepackage{amsmath}
\usepackage{amsfonts}
%\usepackage{yhmath}
\usepackage{graphicx}
\usepackage{tikz}
\usepackage{etoolbox}
\DeclareSymbolFont{yhlargesymbols}{OMX}{yhex}{m}{n} \DeclareMathAccent{\yhwidehat}{\mathord}{yhlargesymbols}{"62}

\usepackage{scalerel}[2014/03/10]
\usepackage{stackengine}

\renewcommand\widetilde[1]{\ThisStyle{%
  \setbox0=\hbox{$\SavedStyle#1$}%
  \stackengine{1pt-\LMpt}{$\SavedStyle#1$}{%
    \stretchto{\scaleto{\SavedStyle\mkern.2mu\sim}{.5467\wd0}}{.5\ht0}%
%    .2mu is the kern imbalance when clipping white space
%    .5467++++ is \ht/[kerned \wd] aspect ratio for \sim glyph
  }{O}{c}{F}{T}{S}%
}}
\makeatletter
\let\save@mathaccent\mathaccent
\newcommand*\if@single[3]{%
  \setbox0\hbox{${\mathaccent"0362{#1}}^H$}%
  \setbox2\hbox{${\mathaccent"0362{\kern0pt#1}}^H$}%
  \ifdim\ht0=\ht2 #3\else #2\fi
  }
%The bar will be moved to the right by a half of \macc@kerna, which is computed by amsmath:
\newcommand*\rel@kern[1]{\kern#1\dimexpr\macc@kerna}
%If there's a superscript following the bar, then no negative kern may follow the bar;
%an additional {} makes sure that the superscript is high enough in this case:
\newcommand*\widebar[1]{\@ifnextchar^{{\wide@bar{#1}{0}}}{\wide@bar{#1}{1}}}
%Use a separate algorithm for single symbols:
\newcommand*\wide@bar[2]{\if@single{#1}{\wide@bar@{#1}{#2}{1}}{\wide@bar@{#1}{#2}{2}}}
\newcommand*\wide@bar@[3]{%
  \begingroup
  \def\mathaccent##1##2{%
%Enable nesting of accents:
    \let\mathaccent\save@mathaccent
%If there's more than a single symbol, use the first character instead (see below):
    \if#32 \let\macc@nucleus\first@char \fi
%Determine the italic correction:
    \setbox\z@\hbox{$\macc@style{\macc@nucleus}_{}$}%
    \setbox\tw@\hbox{$\macc@style{\macc@nucleus}{}_{}$}%
    \dimen@\wd\tw@
    \advance\dimen@-\wd\z@
%Now \dimen@ is the italic correction of the symbol.
    \divide\dimen@ 3
    \@tempdima\wd\tw@
    \advance\@tempdima-\scriptspace
%Now \@tempdima is the width of the symbol.
    \divide\@tempdima 10
    \advance\dimen@-\@tempdima
%Now \dimen@ = (italic correction / 3) - (Breite / 10)
    \ifdim\dimen@>\z@ \dimen@0pt\fi
%The bar will be shortened in the case \dimen@<0 !
    \rel@kern{0.6}\kern-\dimen@
    \if#31
      \overline{\rel@kern{-0.6}\kern\dimen@\macc@nucleus\rel@kern{0.4}\kern\dimen@}%
      \advance\dimen@0.4\dimexpr\macc@kerna
%Place the combined final kern (-\dimen@) if it is >0 or if a superscript follows:
      \let\final@kern#2%
      \ifdim\dimen@<\z@ \let\final@kern1\fi
      \if\final@kern1 \kern-\dimen@\fi
    \else
      \overline{\rel@kern{-0.6}\kern\dimen@#1}%
    \fi
  }%
  \macc@depth\@ne
  \let\math@bgroup\@empty \let\math@egroup\macc@set@skewchar
  \mathsurround\z@ \frozen@everymath{\mathgroup\macc@group\relax}%
  \macc@set@skewchar\relax
  \let\mathaccentV\macc@nested@a
%The following initialises \macc@kerna and calls \mathaccent:
  \if#31
    \macc@nested@a\relax111{#1}%
  \else
%If the argument consists of more than one symbol, and if the first token is
%a letter, use that letter for the computations:
    \def\gobble@till@marker##1\endmarker{}%
    \futurelet\first@char\gobble@till@marker#1\endmarker
    \ifcat\noexpand\first@char A\else
      \def\first@char{}%
    \fi
    \macc@nested@a\relax111{\first@char}%
  \fi
  \endgroup
}
\makeatother

\graphicspath{ {./images/} }
\geometry{
    a4paper,
    headheight = 1.5cm
}

\patchcmd{\chapter}{\thispagestyle{plain}}
{\thispagestyle{fancy}}{}{}

\theoremstyle{remark}
\newtheorem*{claim}{Claim}
\newtheorem*{remark}{Remark}
\newtheorem{case}{Case}

\newcommand{\zero}{\mathbf{0}}
\newcommand{\one}{\mathbf{1}}
\newcommand{\I}{\mathbfit{I}}
\newcommand{\e}{\mathrm{e}}
\renewcommand{\d}{\mathrm{d}}
\newcommand{\im}{\mathrm{i}}
\newcommand{\map}[3]{#1: #2 \rightarrow #3} % Mapping
\newcommand{\image}[2]{#2\left[#1\right]} % Image
\newcommand{\preimage}[2]{#2\left[#1\right]^{-1}} % Pre-image
\newcommand{\eval}[3]{\left. #1\right\rvert_{#2 = #3}} % Evaluated at
%\newcommand\bigO[1]{\mathcal{O}\left(#1\right)}

\DeclareMathOperator*{\argmax}{argmax}
\DeclareMathOperator*{\argmin}{argmin}
\DeclareMathAlphabet{\mathcal}{OMS}{cmsy}{m}{n}

\begin{document}
\fancyhead[L]{
    Metric and Topological Spaces
}
\fancyhead[R]{
    Lecture Notes
}
\tableofcontents
\chapter{Topology}
\section{Topological Spaces}
\begin{dfnbox}{Topology}{topology}
    A {\color{red} \textbf{topology}} on a set $X$ is a collection $\mathcal{T} \subseteq \mathcal{P}(X)$ such that 
    \begin{itemize}
        \item $\varnothing, X \in \mathcal{T}$;
        \item for any index set $I$, if $\{X_i \colon i \in I\} \subseteq \mathcal{P}(\mathcal{T})$, then $\bigcup_{i \in I}X_i \in \mathcal{T}$;
        \item for any $X_1, X_2, \cdots, X_n \in \mathcal{T}$, $\bigcap_{i = 1}^nX_i \in \mathcal{T}$.
    \end{itemize}
    The pair $(X, \mathcal{T})$ is said to be a {\color{red} \textbf{topological space}}. A subset $Y \subseteq X$ is {\color{red} \textbf{open}} if $Y \in \mathcal{T}$.
\end{dfnbox}
\begin{notebox}
    \begin{remark}
        For any set $X$, we define $\{\varnothing, X\}$ as the \textit{trivial topology} on $X$, $\mathcal{P}(X)$ as the \textit{discrete topology}, and $\left\{X \backslash U \colon U \subseteq X \textrm{ is finite}\right\}$ as the \textit{co-finite topology}.
    \end{remark}
\end{notebox}
The set $\left\{(-\alpha, \alpha) \colon \alpha > 0\right\} \cup \{\R, \varnothing\}$ defines a topology on $\R$. This example also demonstrates why it is crucial to only consider closure under finite intersections when defining a topology, because
\begin{equation*}
    \bigcap_{n = 1}^{\infty}\left(-1 - \frac{1}{n}, 1 + \frac{1}{n}\right) = [-1, 1] \notin \mathcal{T}.
\end{equation*}
We now seek a systematic method to generate a topology given any set. The idea here is to make use of a \textit{cover}.
\begin{dfnbox}{Basis}{basis}
    A {\color{red} \textbf{basis}} for a topology on $X$ is a collection $\mathcal{B} \subseteq \mathcal{P}(X)$ such that 
    \begin{itemize}
        \item for any $x \in X$, there exists some $B \in \mathcal{B}$ such that $x \in B$;
        \item for any $x \in X$ and $B_1, B_2 \in \mathcal{B}$ with $x \in B_1 \cap B_2$, there exists some $B \in \mathcal{B}$ such that $x \in B \subseteq B_1 \cap B_2$.
    \end{itemize}
\end{dfnbox}
It may be useful to see a basis as a cover of a set with the second additional property as stated in the above definition. Notice that the first property of the basis $\mathcal{B}$ is basically saying that 
\begin{equation*}
    X \subseteq \bigcup \mathcal{B},
\end{equation*}
i.e., $\mathcal{B}$ is a cover of $X$.

Given any basis $\mathcal{B}$ for some topology on $X$, a set generated by $\mathcal{B}$ can be defined as 
\begin{equation*}
    \mathcal{T} \coloneqq \left\{U \subseteq X \colon \textrm{for any } u \in U, \textrm{ there exists some } B \in \mathcal{B} \textrm{ such that } u \in B \subseteq U\right\}
\end{equation*}
We will show that $\mathcal{T}$ is a topology on $X$. First, it is clear that $\varnothing, X \in \mathcal{T}$. 

Let $I$ be an index set and $\{X_i \colon i \in I\}\subseteq \mathcal{P}(\mathcal{T})$ be any collection of subsets of $X$. Notice that for any $x \in \bigcup_{i \in I}X_i$, there exists some $j \in I$ such that $x \in X_j \subseteq \mathcal{T}$. According to our construction, this means that there exists some $B \in \mathcal{B}$ such that $x \in B \subseteq X_j \subseteq \mathcal{T}$. Therefore, $\bigcup_{i \in I}X_i \subseteq \mathcal{T}$ as desired.

To prove that $\mathcal{T}$ is closed under finite intersection, we consider the following lemma:
\begin{lembox}{Finite Intersection of Elements in Basis Is Covered}{coverFiniteIntersectionOfBasis}
    Let $\mathcal{B}$ be a basis for a topology on $X$ and $B_1, B_2, \cdots, B_n \in \mathcal{B}$, then for any $x \in \bigcap_{i = 1}^nB_i$, there exists some $B \in \mathcal{B}$ such that $x \in B \subseteq \bigcap_{i = 1}^nB_i$.
    \tcblower
    \begin{proof}
        The case where $n = 1$ is trivial by taking $B = B_1$. Suppose that there is some integer $k \geq 1$ such that for any $B_1, B_2, \cdots, B_k \in \mathcal{B}$ and any $x \in \bigcap_{i = 1}^kB_i$, there exists some $B \in \mathcal{B}$ such that $x \in B \subseteq \bigcap_{i = 1}^kB_i$. Take any $B_{k + 1} \in \mathcal{B}$. It is clear that for any~$x \in \bigcap_{i = 1}^{k + 1}B_i$, there exists some $B \in \mathcal{B}$ such that 
        \begin{equation*}
            x \in B \subseteq \bigcap_{i = 1}^kB_i.
        \end{equation*}
        Notice that $x \in B_{k + 1} \in \mathcal{B}$, so we know that $x \in B \cap B_{k + 1}$. By Definition \ref{dfn:basis}, this means that there exists some $B' \in \mathcal{B}$ such that 
        \begin{equation*}
            x \in B' \subseteq B \cap B_{k + 1} \subseteq \bigcap_{i = 1}^{k + 1}B_i.
        \end{equation*}
    \end{proof}
\end{lembox}
Now, suppose $X_1, X_2, \cdots, X_n \in \mathcal{T}$ are finitely many subsets of $X$. Take any $x \in \bigcap_{i = 1}^nX_i$. It is clear that $x \in X_i$ for each $i = 1, 2, \cdots, n$. Therefore, for each $i = 1, 2, \cdots, n$, there exists some $B_i \in \mathcal{B}$ such that $x \in B_i \subseteq X_i$. By Lemma \ref{lem:coverFiniteIntersectionOfBasis}, this means that there exists some set~$B \in \mathcal{B}$ such that 
\begin{equation*}
    x \in B \subseteq \bigcap_{i = 1}^nB_i \subseteq \bigcap_{i = 1}^nX_i.    
\end{equation*}
Therefore, $\bigcap_{i = 1}^nX_i \in \mathcal{T}$. So this set $\mathcal{T}$ generated by $\mathcal{B}$ is indeed a topology on $X$. 

The following proposition further shows that the topology generated by a basis $\mathcal{B}$ is the set of all possible unions of elements in $\mathcal{B}$:
\begin{probox}{\small Equivalent Construction of Topologies Generated from Bases}{equivalentConstruction}
    Let $X$ be any set. If $\mathcal{B}$ is a basis for a topology $\mathcal{T}$ on $X$, then 
    \begin{equation*}
        \mathcal{T} = \left\{\bigcup_{A \in \mathcal{V}} A  \colon \mathcal{V}\in \mathcal{P}(\mathcal{B})\right\}.
    \end{equation*}
    \tcblower
    \begin{proof}
        Denote 
        \begin{equation*}
            \mathcal{T}_{\mathcal{B}} \coloneqq \left\{U \subseteq X \colon \textrm{for any } u \in U, \textrm{ there exists some } B \in \mathcal{B} \textrm{ such that } u \in B \subseteq U\right\}.
        \end{equation*}
        It suffices to prove that $\mathcal{T} = \mathcal{T}_{\mathcal{B}}$. Take any $T \in \mathcal{T}$, then there exists some $V \in \mathcal{P}(\mathcal{B})$ such that $T = \bigcup_{A \in \mathcal{V}}A$. This means that for every $t \in T$, there exists some $B_t \in \mathcal{V}$ such that $t \in B_t \subseteq T$. Therefore, $T \in \mathcal{T}_{\mathcal{B}}$. Conversely, for any $S \in \mathcal{T}_{\mathcal{B}}$, there exists some $B_s \in \mathcal{B}$ for each $s \in S$ such that $s \in B_s$. Denote $U \coloneqq \left\{B_s \colon s \in S\right\} \in \mathcal{P}(\mathcal{B})$, then it is clear that $S \subseteq \bigcup_{B \in U}B$. Since $B_s \subseteq S$ for each $s \in S$, we have $\bigcup_{B \in U}B \subseteq S$, which implies that $S = \bigcup_{B \in U}B$. This means that $S \in \mathcal{T}$. Therefore, $\mathcal{T} \subseteq \mathcal{T}_{\mathcal{B}}$ and $\mathcal{T}_{\mathcal{B}} \subseteq \mathcal{T}$, which means that~$\mathcal{T} = \mathcal{T}_{\mathcal{B}}$.
    \end{proof}
\end{probox}
Next, we define a special topology in Euclidean spaces using open balls.
\begin{dfnbox}{Standard Topology}{standardTopology}
    For any $\mathbfit{x} = (x_1, x_2, \cdots, x_n) \in \R^n$ and any $r > 0$. Denote the Euclidean open ball centred at $\mathbfit{x}$ with radius $r$ by
    \begin{equation*}
        B_{r}(\mathbfit{x}) \coloneqq \left\{\mathbfit{y} = (y_1, y_2, \cdots, y_n) \in \R^n \colon \sqrt{\sum_{i = 1}^{n}(x_i - y_i)^2} < r\right\}
    \end{equation*} 
    The {\color{red} \textbf{standard topology}} on $\R^n$ is the set generated by the basis
    \begin{equation*}
        \mathcal{B} \coloneqq \left\{B_{r}(\mathbfit{x}) \colon \mathbfit{x} \in \R^n, r \in \R^+\right\}.
    \end{equation*}
\end{dfnbox}
It may be helpful to actually show that this set $\mathcal{B}$ is indeed a basis of a topology on $\R^n$. The fact that $\mathcal{B}$ is a cover for $\R^n$ is trivial enough. Take any $\mathbfit{x} \in \R^n$ and balls $B_{\alpha}(\mathbfit{x}_1), B_{\beta}(\mathbfit{x}_2) \in \mathcal{B}$ such that $\mathbfit{x} \in B_{\alpha}(\mathbfit{x}_1) \cap B_{\beta}(\mathbfit{x}_2)$ (the existence of these $2$ balls is again trivial enough). Take 
\begin{equation*}
    r = \min\left\{\alpha - \norm{\mathbfit{x} - \mathbfit{x}_1}, \beta - \norm{\mathbfit{x} - \mathbfit{x}_2}\right\}.
\end{equation*}
Clearly, $r > 0$ and $\mathbfit{x} \in B_r(\mathbfit{x})$, so we are done.

Now, we discuss the analogue of the subset relation in topologies.
\begin{dfnbox}{Fineness and Coarseness}{fine}
    Let $\mathcal{T}$ and $\mathcal{T}'$ be topologies on some set $X$. We say that $\mathcal{T}$ is {\color{red} \textbf{finer}} than $\mathcal{T}'$, or equivalently, that $\mathcal{T}'$ is {\color{red} \textbf{coarser}} than $\mathcal{T}$, if $\mathcal{T}' \subseteq \mathcal{T}$.
\end{dfnbox}
Observe that any topology of $X$ must be a subset of $\mathcal{P}(X)$, which is the discrete topology on~$X$, so the discrete topology is the finest topology on a set.
\begin{notebox}
    \begin{remark}
        For any basis $\mathcal{B}$ for a topology on $X$, the topology generated by $\mathcal{B}$ is the coarsest topology containing $\mathcal{B}$.
    \end{remark}
\end{notebox}
The above remark is easy to verify. Let $\mathcal{T}$ be any topology on $X$ with $\mathcal{B} \subseteq \mathcal{T}$ and $\mathcal{T}_{\mathcal{B}}$ be the topology generated by $\mathcal{B}$. For any $T \in \mathcal{T}_{\mathcal{B}}$, by Proposition \ref{pro:equivalentConstruction}, there exists some $V \subseteq \mathcal{B}$ such that $T = \bigcup_{A \in \mathcal{V}}A$. Note that $A \in \mathcal{T}$ for all $A \in \mathcal{V}$, so by Definition \ref{dfn:topology}, $T \in \mathcal{T}$ and so $\mathcal{T}_{\mathcal{B}} \subseteq \mathcal{T}$ as desired.

This motivates us to consider fineness in terms of bases.
\begin{probox}{Fineness in Terms of Bases}{fineBasis}
    Let $\mathcal{B}$ and $\mathcal{B}'$ generate topologies $\mathcal{T}$ and $\mathcal{T}'$ respectively on $X$, then $\mathcal{T}'$ is finer than $\mathcal{T}$ if and only if for every $B \in \mathcal{B}$, there exists some $B' \in \mathcal{B}'$ such that for any $x \in B$, we have~$x \in B' \subseteq B$.
    \tcblower
    \begin{proof}
        Suppose that $\mathcal{T}'$ is finer than $\mathcal{T}$, then $\mathcal{T} \subseteq \mathcal{T}'$. Take any $B \in \mathcal{B}$, then by Proposition \ref{pro:equivalentConstruction}, $B \in \mathcal{T}$, which means that $B \in \mathcal{T}'$. Since $\mathcal{B}'$ is a basis for $\mathcal{T}'$, by Definition \ref{dfn:basis} for any $x \in B$, there exists some $B' \in \mathcal{B}'$ such that $x \in B' \subseteq B$.
        \\\\
        Suppose conversely that for every $B \in \mathcal{B}$, there exists some $B' \in \mathcal{B}'$ such that for any $x \in B$, we have $x \in B' \subseteq B$. Take any $T \in \mathcal{T}$, for each $x \in T$, by Definition \ref{dfn:basis} there exists some $B \in \mathcal{B}$ such that $x \in B \subseteq T$. Notice that there exists some~$B' \in \mathcal{B}'$ such that $x \in B' \subseteq B \subseteq T$, so $T \in \mathcal{T}'$. Therefore, $\mathcal{T} \subseteq \mathcal{T}'$ and so $\mathcal{T}'$ is finer than~$\mathcal{T}$.
    \end{proof}
\end{probox}
Recall that every basis of a topology on $X$ is an open cover of $X$ consisting only of subsets of $X$. Therefore, the union of the elements in the basis is essentially $X$ itself. This motivates us to propose another way to generate a topology on a set.
\begin{dfnbox}{Sub-basis}{subbasis}
    A {\color{red} \textbf{sub-basis}} of $X$ is a collection $\mathcal{S} \subseteq \mathcal{P}(X)$ such that $\bigcup_{A \in \mathcal{S}}A = X$.
\end{dfnbox}
\begin{notebox}
    \begin{remark}
        Every basis is a sub-basis.
    \end{remark}
\end{notebox}
For an arbitrary set $X$, let $\mathcal{S}$ be a sub-basis and denote the collection of all finite subsets of $\mathcal{S}$ as $\mathcal{F}_{\mathcal{S}}$. Define 
\begin{equation*}
    \mathcal{U}_{\mathcal{S}} \coloneqq \left\{\bigcap_{A \in F}A \colon F \in \mathcal{F}_{\mathcal{S}}\right\}
\end{equation*}
to be the collection of all finite intersections of sets in $\mathcal{S}$. The topology generated by a sub-basis of $X$ is given by 
\begin{equation*}
    \mathcal{T} \coloneqq \left\{\bigcup_{A \in \mathcal{V}}A  \colon \mathcal{V}\subseteq \mathcal{U}_{\mathcal{S}}\right\}.
\end{equation*}
We shall show that $\mathcal{T}$ is indeed a topology on $X$ by considering the following proposition:
\begin{probox}{Finite Intersections of Sets in a Sub-basis Form a Basis}{finiteIntersectionBasis}
    Let $\mathcal{S}$ be a sub-basis for a set $X$ and let $\mathcal{U}_{\mathcal{S}}$ be the set of all finite intersections of sets in $\mathcal{S}$, then $\mathcal{U}_{\mathcal{S}}$ is a basis of a topology on $X$.
    \tcblower
    \begin{proof}
        Take any $x \in X$. By Definition \ref{dfn:subbasis}, we have $x \in \bigcup_{A \in \mathcal{S}}A$. Therefore, there exists some $A \in \mathcal{S} \subseteq \mathcal{P}(X)$ such that $x \in A$. For any $x \in X$ and $B_1, B_2 \in \mathcal{U}_{\mathcal{S}}$ such that~$x \in B_1 \cap B_2$, notice that $B_1 \cap B_2$ is a finite intersection of sets in $\mathcal{S}$, so $B_1 \cap B_2 \in \mathcal{U}_{\mathcal{S}}$. Therefore, by Definition \ref{dfn:basis}, $\mathcal{U}_{\mathcal{S}}$ is a basis. 
    \end{proof}
\end{probox}
With Propositions \ref{pro:finiteIntersectionBasis} and \ref{pro:equivalentConstruction}, it is clear that $\mathcal{T}$ as constructed above is a topology on $X$.
\end{document}