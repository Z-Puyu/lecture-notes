\documentclass[math, code]{amznotes}
\usepackage[utf8]{inputenc}
\usepackage{amsmath}
\usepackage{amsfonts}
%\usepackage{yhmath}
\usepackage{graphicx}
\usepackage{tikz}
\usepackage{etoolbox}
\DeclareSymbolFont{yhlargesymbols}{OMX}{yhex}{m}{n} \DeclareMathAccent{\yhwidehat}{\mathord}{yhlargesymbols}{"62}

\usepackage{scalerel}[2014/03/10]
\usepackage{stackengine}

\renewcommand\widetilde[1]{\ThisStyle{%
  \setbox0=\hbox{$\SavedStyle#1$}%
  \stackengine{1pt-\LMpt}{$\SavedStyle#1$}{%
    \stretchto{\scaleto{\SavedStyle\mkern.2mu\sim}{.5467\wd0}}{.5\ht0}%
%    .2mu is the kern imbalance when clipping white space
%    .5467++++ is \ht/[kerned \wd] aspect ratio for \sim glyph
  }{O}{c}{F}{T}{S}%
}}
\makeatletter
\let\save@mathaccent\mathaccent
\newcommand*\if@single[3]{%
  \setbox0\hbox{${\mathaccent"0362{#1}}^H$}%
  \setbox2\hbox{${\mathaccent"0362{\kern0pt#1}}^H$}%
  \ifdim\ht0=\ht2 #3\else #2\fi
  }
%The bar will be moved to the right by a half of \macc@kerna, which is computed by amsmath:
\newcommand*\rel@kern[1]{\kern#1\dimexpr\macc@kerna}
%If there's a superscript following the bar, then no negative kern may follow the bar;
%an additional {} makes sure that the superscript is high enough in this case:
\newcommand*\widebar[1]{\@ifnextchar^{{\wide@bar{#1}{0}}}{\wide@bar{#1}{1}}}
%Use a separate algorithm for single symbols:
\newcommand*\wide@bar[2]{\if@single{#1}{\wide@bar@{#1}{#2}{1}}{\wide@bar@{#1}{#2}{2}}}
\newcommand*\wide@bar@[3]{%
  \begingroup
  \def\mathaccent##1##2{%
%Enable nesting of accents:
    \let\mathaccent\save@mathaccent
%If there's more than a single symbol, use the first character instead \left(see below\right):
    \if#32 \let\macc@nucleus\first@char \fi
%Determine the italic correction:
    \setbox\z@\hbox{$\macc@style{\macc@nucleus}_{}$}%
    \setbox\tw@\hbox{$\macc@style{\macc@nucleus}{}_{}$}%
    \dimen@\wd\tw@
    \advance\dimen@-\wd\z@
%Now \dimen@ is the italic correction of the symbol.
    \divide\dimen@ 3
    \@tempdima\wd\tw@
    \advance\@tempdima-\scriptspace
%Now \@tempdima is the width of the symbol.
    \divide\@tempdima 10
    \advance\dimen@-\@tempdima
%Now \dimen@ = \left(italic correction / 3\right) - \left(Breite / 10\right)
    \ifdim\dimen@>\z@ \dimen@0pt\fi
%The bar will be shortened in the case \dimen@<0 !
    \rel@kern{0.6}\kern-\dimen@
    \if#31
      \overline{\rel@kern{-0.6}\kern\dimen@\macc@nucleus\rel@kern{0.4}\kern\dimen@}%
      \advance\dimen@0.4\dimexpr\macc@kerna
%Place the combined final kern \left(-\dimen@\right) if it is >0 or if a superscript follows:
      \let\final@kern#2%
      \ifdim\dimen@<\z@ \let\final@kern1\fi
      \if\final@kern1 \kern-\dimen@\fi
    \else
      \overline{\rel@kern{-0.6}\kern\dimen@#1}%
    \fi
  }%
  \macc@depth\@ne
  \let\math@bgroup\@empty \let\math@egroup\macc@set@skewchar
  \mathsurround\z@ \frozen@everymath{\mathgroup\macc@group\relax}%
  \macc@set@skewchar\relax
  \let\mathaccentV\macc@nested@a
%The following initialises \macc@kerna and calls \mathaccent:
  \if#31
    \macc@nested@a\relax111{#1}%
  \else
%If the argument consists of more than one symbol, and if the first token is
%a letter, use that letter for the computations:
    \def\gobble@till@marker##1\endmarker{}%
    \futurelet\first@char\gobble@till@marker#1\endmarker
    \ifcat\noexpand\first@char A\else
      \def\first@char{}%
    \fi
    \macc@nested@a\relax111{\first@char}%
  \fi
  \endgroup
}
\makeatother

\graphicspath{ {./images/} }
\geometry{
    a4paper,
    headheight = 1.5cm
}

\patchcmd{\chapter}{\thispagestyle{plain}}
{\thispagestyle{fancy}}{}{}

\theoremstyle{remark}
\newtheorem*{claim}{Claim}
\newtheorem*{remark}{Remark}
\newtheorem{case}{Case}

\newcommand{\zero}{\mathbf{0}}
\newcommand{\one}{\mathbf{1}}
\newcommand{\I}{\mathbfit{I}}
\newcommand{\e}{\mathrm{e}}
\renewcommand{\d}{\mathrm{d}}
\newcommand{\im}{\mathrm{i}}
\newcommand{\map}[3]{#1: #2 \rightarrow #3} % Mapping
\newcommand{\image}[2]{#2\left[#1\right]} % Image
\newcommand{\preimage}[2]{#2\left[#1\right]^{-1}} % Pre-image
\newcommand{\eval}[3]{\left. #1\right\rvert_{#2 = #3}} % Evaluated at
%\newcommand\bigO[1]{\mathcal{O}\left(#1\right)}

\DeclareMathOperator*{\argmax}{argmax}
\DeclareMathOperator*{\argmin}{argmin}
\DeclareMathAlphabet{\mathcal}{OMS}{cmsy}{m}{n}

\begin{document}
\fancyhead[L]{
    Metric and Topological Spaces
}
\fancyhead[R]{
    Lecture Notes
}
\tableofcontents
\chapter{Topology}
\section{Topological Spaces}
\begin{dfnbox}{Topology}{topology}
    A {\color{red} \textbf{topology}} on a set $X$ is a collection $\mathcal{T} \subseteq \mathcal{P}\left(X\right)$ such that 
    \begin{itemize}
        \item $\varnothing, X \in \mathcal{T}$;
        \item for any index set $I$, if $\left\{X_i \colon i \in I\right\} \subseteq \mathcal{P}\left(\mathcal{T}\right)$, then $\bigcup_{i \in I}X_i \in \mathcal{T}$;
        \item for any $X_1, X_2, \cdots, X_n \in \mathcal{T}$, $\bigcap_{i = 1}^nX_i \in \mathcal{T}$.
    \end{itemize}
    The pair $\left(X, \mathcal{T}\right)$ is said to be a {\color{red} \textbf{topological space}}. A subset $Y \subseteq X$ is {\color{red} \textbf{open}} if $Y \in \mathcal{T}$.
\end{dfnbox}
\begin{notebox}
    \begin{remark}
        For any set $X$, we define $\left\{\varnothing, X\right\}$ as the \textit{trivial topology} on $X$, $\mathcal{P}\left(X\right)$ as the \textit{discrete topology}, and $\left\{X \backslash U \colon U \subseteq X \textrm{ is finite}\right\} \cup \left\{\varnothing\right\}$ as the \textit{co-finite topology}.
    \end{remark}
\end{notebox}
The set $\left\{\left(-\alpha, \alpha\right) \colon \alpha > 0\right\} \cup \left\{\R, \varnothing\right\}$ defines a topology on $\R$. This example also demonstrates why it is crucial to only consider closure under finite intersections when defining a topology, because
\begin{equation*}
    \bigcap_{n = 1}^{\infty}\left(-1 - \frac{1}{n}, 1 + \frac{1}{n}\right) = [-1, 1] \notin \mathcal{T}.
\end{equation*}
We now seek a systematic method to generate a topology given any set. The idea here is to make use of a \textit{cover}.
\begin{dfnbox}{Basis}{basis}
    A {\color{red} \textbf{basis}} for a topology on $X$ is a collection $\mathcal{B} \subseteq \mathcal{P}\left(X\right)$ such that 
    \begin{itemize}
        \item for any $x \in X$, there exists some $B \in \mathcal{B}$ such that $x \in B$;
        \item for any $x \in X$ and $B_1, B_2 \in \mathcal{B}$ with $x \in B_1 \cap B_2$, there exists some $B \in \mathcal{B}$ such that $x \in B \subseteq B_1 \cap B_2$.
    \end{itemize}
\end{dfnbox}
It may be useful to see a basis as a cover of a set with the second additional property as stated in the above definition. Notice that the first property of the basis $\mathcal{B}$ is basically saying that 
\begin{equation*}
    X \subseteq \bigcup \mathcal{B},
\end{equation*}
i.e., $\mathcal{B}$ is a cover of $X$.

Given any basis $\mathcal{B}$ for some topology on $X$, a set generated by $\mathcal{B}$ can be defined as 
\begin{equation*}
    \mathcal{T} \coloneqq \left\{U \subseteq X \colon \textrm{for any } u \in U, \textrm{ there exists some } B \in \mathcal{B} \textrm{ such that } u \in B \subseteq U\right\}
\end{equation*}
We will show that $\mathcal{T}$ is a topology on $X$. First, it is clear that $\varnothing, X \in \mathcal{T}$. 

Let $I$ be an index set and $\left\{X_i \colon i \in I\right\}\subseteq \mathcal{P}\left(\mathcal{T}\right)$ be any collection of subsets of $X$. Notice that for any $x \in \bigcup_{i \in I}X_i$, there exists some $j \in I$ such that $x \in X_j \subseteq \mathcal{T}$. According to our construction, this means that there exists some $B \in \mathcal{B}$ such that $x \in B \subseteq X_j \subseteq \mathcal{T}$. Therefore, $\bigcup_{i \in I}X_i \subseteq \mathcal{T}$ as desired.

To prove that $\mathcal{T}$ is closed under finite intersection, we consider the following lemma:
\begin{lembox}{Finite Intersection of Elements in Basis Is Covered}{coverFiniteIntersectionOfBasis}
    Let $\mathcal{B}$ be a basis for a topology on $X$ and $B_1, B_2, \cdots, B_n \in \mathcal{B}$, then for any $x \in \bigcap_{i = 1}^nB_i$, there exists some $B \in \mathcal{B}$ such that $x \in B \subseteq \bigcap_{i = 1}^nB_i$.
    \tcblower
    \begin{proof}
        The case where $n = 1$ is trivial by taking $B = B_1$. Suppose that there is some integer $k \geq 1$ such that for any $B_1, B_2, \cdots, B_k \in \mathcal{B}$ and any $x \in \bigcap_{i = 1}^kB_i$, there exists some $B \in \mathcal{B}$ such that $x \in B \subseteq \bigcap_{i = 1}^kB_i$. Take any $B_{k + 1} \in \mathcal{B}$. It is clear that for any~$x \in \bigcap_{i = 1}^{k + 1}B_i$, there exists some $B \in \mathcal{B}$ such that 
        \begin{equation*}
            x \in B \subseteq \bigcap_{i = 1}^kB_i.
        \end{equation*}
        Notice that $x \in B_{k + 1} \in \mathcal{B}$, so we know that $x \in B \cap B_{k + 1}$. By Definition \ref{dfn:basis}, this means that there exists some $B' \in \mathcal{B}$ such that 
        \begin{equation*}
            x \in B' \subseteq B \cap B_{k + 1} \subseteq \bigcap_{i = 1}^{k + 1}B_i.
        \end{equation*}
    \end{proof}
\end{lembox}
Now, suppose $X_1, X_2, \cdots, X_n \in \mathcal{T}$ are finitely many subsets of $X$. Take any $x \in \bigcap_{i = 1}^nX_i$. It is clear that $x \in X_i$ for each $i = 1, 2, \cdots, n$. Therefore, for each $i = 1, 2, \cdots, n$, there exists some $B_i \in \mathcal{B}$ such that $x \in B_i \subseteq X_i$. By Lemma \ref{lem:coverFiniteIntersectionOfBasis}, this means that there exists some set~$B \in \mathcal{B}$ such that 
\begin{equation*}
    x \in B \subseteq \bigcap_{i = 1}^nB_i \subseteq \bigcap_{i = 1}^nX_i.    
\end{equation*}
Therefore, $\bigcap_{i = 1}^nX_i \in \mathcal{T}$. So this set $\mathcal{T}$ generated by $\mathcal{B}$ is indeed a topology on $X$. 

The following proposition further shows that the topology generated by a basis $\mathcal{B}$ is the set of all possible unions of elements in $\mathcal{B}$:
\begin{probox}{\small Equivalent Construction of Topologies Generated from Bases}{equivalentConstruction}
    Let $X$ be any set. If $\mathcal{B}$ is a basis for a topology $\mathcal{T}$ on $X$, then 
    \begin{equation*}
        \mathcal{T} = \left\{\bigcup_{A \in \mathcal{V}} A  \colon \mathcal{V}\in \mathcal{P}\left(\mathcal{B}\right)\right\}.
    \end{equation*}
    \tcblower
    \begin{proof}
        Denote 
        \begin{equation*}
            \mathcal{T}_{\mathcal{B}} \coloneqq \left\{U \subseteq X \colon \textrm{for any } u \in U, \textrm{ there exists some } B \in \mathcal{B} \textrm{ such that } u \in B \subseteq U\right\}.
        \end{equation*}
        It suffices to prove that $\mathcal{T} = \mathcal{T}_{\mathcal{B}}$. Take any $T \in \mathcal{T}$, then there exists some $V \in \mathcal{P}\left(\mathcal{B}\right)$ such that $T = \bigcup_{A \in \mathcal{V}}A$. This means that for every $t \in T$, there exists some $B_t \in \mathcal{V}$ such that $t \in B_t \subseteq T$. Therefore, $T \in \mathcal{T}_{\mathcal{B}}$. Conversely, for any $S \in \mathcal{T}_{\mathcal{B}}$, there exists some $B_s \in \mathcal{B}$ for each $s \in S$ such that $s \in B_s$. Denote $U \coloneqq \left\{B_s \colon s \in S\right\} \in \mathcal{P}\left(\mathcal{B}\right)$, then it is clear that $S \subseteq \bigcup_{B \in U}B$. Since $B_s \subseteq S$ for each $s \in S$, we have $\bigcup_{B \in U}B \subseteq S$, which implies that $S = \bigcup_{B \in U}B$. This means that $S \in \mathcal{T}$. Therefore, $\mathcal{T} \subseteq \mathcal{T}_{\mathcal{B}}$ and $\mathcal{T}_{\mathcal{B}} \subseteq \mathcal{T}$, which means that~$\mathcal{T} = \mathcal{T}_{\mathcal{B}}$.
    \end{proof}
\end{probox}
Next, we define a special topology in Euclidean spaces using open balls.
\begin{dfnbox}{Standard Topology}{standardTopology}
    For any $\mathbfit{x} = \left(x_1, x_2, \cdots, x_n\right) \in \R^n$ and any $r > 0$. Denote the Euclidean open ball centred at $\mathbfit{x}$ with radius $r$ by
    \begin{equation*}
        B_{r}\left(\mathbfit{x}\right) \coloneqq \left\{\mathbfit{y} = \left(y_1, y_2, \cdots, y_n\right) \in \R^n \colon \sqrt{\sum_{i = 1}^{n}\left(x_i - y_i\right)^2} < r\right\}
    \end{equation*} 
    The {\color{red} \textbf{standard topology}} on $\R^n$ is the set generated by the basis
    \begin{equation*}
        \mathcal{B} \coloneqq \left\{B_{r}\left(\mathbfit{x}\right) \colon \mathbfit{x} \in \R^n, r \in \R^+\right\}.
    \end{equation*}
\end{dfnbox}
It may be helpful to actually show that this set $\mathcal{B}$ is indeed a basis of a topology on $\R^n$. The fact that $\mathcal{B}$ is a cover for $\R^n$ is trivial enough. Take any $\mathbfit{x} \in \R^n$ and balls $B_{\alpha}\left(\mathbfit{x}_1\right), B_{\beta}\left(\mathbfit{x}_2\right) \in \mathcal{B}$ such that $\mathbfit{x} \in B_{\alpha}\left(\mathbfit{x}_1\right) \cap B_{\beta}\left(\mathbfit{x}_2\right)$ (the existence of these $2$ balls is again trivial enough). Take 
\begin{equation*}
    r = \min\left\{\alpha - \norm{\mathbfit{x} - \mathbfit{x}_1}, \beta - \norm{\mathbfit{x} - \mathbfit{x}_2}\right\}.
\end{equation*}
Clearly, $r > 0$ and $\mathbfit{x} \in B_r\left(\mathbfit{x}\right)$, so we are done.

Now, we discuss the analogue of the subset relation in topologies.
\begin{dfnbox}{Fineness and Coarseness}{fine}
    Let $\mathcal{T}$ and $\mathcal{T}'$ be topologies on some set $X$. We say that $\mathcal{T}$ is {\color{red} \textbf{finer}} than $\mathcal{T}'$, or equivalently, that $\mathcal{T}'$ is {\color{red} \textbf{coarser}} than $\mathcal{T}$, if $\mathcal{T}' \subseteq \mathcal{T}$.
\end{dfnbox}
Observe that any topology of $X$ must be a subset of $\mathcal{P}\left(X\right)$, which is the discrete topology on~$X$, so the discrete topology is the finest topology on a set.
\begin{notebox}
    \begin{remark}
        For any basis $\mathcal{B}$ for a topology on $X$, the topology generated by $\mathcal{B}$ is the coarsest topology containing $\mathcal{B}$.
    \end{remark}
\end{notebox}
The above remark is easy to verify. Let $\mathcal{T}$ be any topology on $X$ with $\mathcal{B} \subseteq \mathcal{T}$ and $\mathcal{T}_{\mathcal{B}}$ be the topology generated by $\mathcal{B}$. For any $T \in \mathcal{T}_{\mathcal{B}}$, by Proposition \ref{pro:equivalentConstruction}, there exists some $V \subseteq \mathcal{B}$ such that $T = \bigcup_{A \in \mathcal{V}}A$. Note that $A \in \mathcal{T}$ for all $A \in \mathcal{V}$, so by Definition \ref{dfn:topology}, $T \in \mathcal{T}$ and so $\mathcal{T}_{\mathcal{B}} \subseteq \mathcal{T}$ as desired.

This motivates us to consider fineness in terms of bases.
\begin{probox}{Fineness in Terms of Bases}{fineBasis}
    Let $\mathcal{B}$ and $\mathcal{B}'$ generate topologies $\mathcal{T}$ and $\mathcal{T}'$ respectively on $X$. $\mathcal{T}'$ is finer than $\mathcal{T}$ if and only if for every $B \in \mathcal{B}$ and any $x \in B$, there exists some $B_x \in \mathcal{B}'$ such that~$x \in B_x \subseteq B$.
    \tcblower
    \begin{proof}
        Suppose that $\mathcal{T}'$ is finer than $\mathcal{T}$, then $\mathcal{T} \subseteq \mathcal{T}'$. Take any $B \in \mathcal{B}$, then by Proposition \ref{pro:equivalentConstruction}, $B \in \mathcal{T}$, which means that $B \in \mathcal{T}'$. Since $\mathcal{B}'$ is a basis for $\mathcal{T}'$, by Definition \ref{dfn:basis} for any $x \in B$, there exists some $B_x \in \mathcal{B}'$ such that $x \in B_x \subseteq B$.
        \\\\
        Suppose conversely that for every $B \in \mathcal{B}$ and any $x \in B$, there is some $B_x \in \mathcal{B}'$ such that $x \in B' \subseteq B$. Take any $T \in \mathcal{T}$, for each $x \in T$, by Definition \ref{dfn:basis} there exists some $B \in \mathcal{B}$ such that $x \in B \subseteq T$, and so we can find some~$B_x \in \mathcal{B}'$ such that~$x \in B_x \subseteq B \subseteq T$, so $T \in \mathcal{T}'$. Therefore, $\mathcal{T} \subseteq \mathcal{T}'$ and so $\mathcal{T}'$ is finer than~$\mathcal{T}$.
    \end{proof}
\end{probox}
Recall that every basis of a topology on $X$ is an open cover of $X$ consisting only of subsets of $X$. Therefore, the union of the elements in the basis is essentially $X$ itself. This motivates us to propose another way to generate a topology on a set.
\begin{dfnbox}{Sub-basis}{subbasis}
    A {\color{red} \textbf{sub-basis}} of $X$ is a collection $\mathcal{S} \subseteq \mathcal{P}\left(X\right)$ such that $\bigcup_{A \in \mathcal{S}}A = X$.
\end{dfnbox}
\begin{notebox}
    \begin{remark}
        Every basis is a sub-basis.
    \end{remark}
\end{notebox}
For an arbitrary set $X$, let $\mathcal{S}$ be a sub-basis and denote the collection of all finite subsets of $\mathcal{S}$ as $\mathcal{F}_{\mathcal{S}}$. Define 
\begin{equation*}
    \mathcal{U}_{\mathcal{S}} \coloneqq \left\{\bigcap_{A \in F}A \colon F \in \mathcal{F}_{\mathcal{S}}\right\}
\end{equation*}
to be the collection of all finite intersections of sets in $\mathcal{S}$. The topology generated by a sub-basis of $X$ is given by 
\begin{equation*}
    \mathcal{T} \coloneqq \left\{\bigcup_{A \in \mathcal{V}}A  \colon \mathcal{V}\subseteq \mathcal{U}_{\mathcal{S}}\right\}.
\end{equation*}
We shall show that $\mathcal{T}$ is indeed a topology on $X$ by considering the following proposition:
\begin{probox}{Finite Intersections of Sets in a Sub-basis Form a Basis}{finiteIntersectionBasis}
    Let $\mathcal{S}$ be a sub-basis for a set $X$ and let $\mathcal{U}_{\mathcal{S}}$ be the set of all finite intersections of sets in $\mathcal{S}$, then $\mathcal{U}_{\mathcal{S}}$ is a basis of a topology on $X$.
    \tcblower
    \begin{proof}
        Take any $x \in X$. By Definition \ref{dfn:subbasis}, we have $x \in \bigcup_{A \in \mathcal{S}}A$. Therefore, there exists some $A \in \mathcal{S} \subseteq \mathcal{P}\left(X\right)$ such that $x \in A$. For any $x \in X$ and $B_1, B_2 \in \mathcal{U}_{\mathcal{S}}$ such that~$x \in B_1 \cap B_2$, notice that $B_1 \cap B_2$ is a finite intersection of sets in $\mathcal{S}$, so $B_1 \cap B_2 \in \mathcal{U}_{\mathcal{S}}$. Therefore, by Definition \ref{dfn:basis}, $\mathcal{U}_{\mathcal{S}}$ is a basis. 
    \end{proof}
\end{probox}
With Propositions \ref{pro:finiteIntersectionBasis} and \ref{pro:equivalentConstruction}, it is clear that $\mathcal{T}$ as constructed above is a topology on $X$.
\section{Metric Spaces}
\begin{dfnbox}{Metric}{metric}
    A {\color{red} \textbf{metric}} on a set $S$ is a function $d \colon S \times S \to \R$ such that:
    \begin{itemize}
        \item $d\left(x, y\right) \geq 0$ for all $x, y \in S$ (positivity);
        \item $d\left(x, y\right) = 0$ if and only if $x = y$ (definiteness);
        \item $d\left(x, y\right) = d\left(x, y\right)$ for all $x, y \in S$ (symmetry);
        \item $d\left(x, y\right) \leq d\left(x, z\right) + d\left(y, z\right)$ for all $x, y, z \in S$ (triangular inequality).
    \end{itemize}
\end{dfnbox}
\begin{notebox}
    \begin{remark}
        A metric is sometimes also called a \textit{distance function}.
    \end{remark}
\end{notebox}
A metric generalises the notion of distance in Euclidean spaces. We can weaken the above axioms to arrive at the following definition:
\begin{dfnbox}{Pseudo-metric}{pseudometric}
    A {\color{red} \textbf{pseudo-metric}} on a set $S$ is a function $d \colon S \times S \to \R$ such that:
    \begin{itemize}
        \item $d\left(x, y\right) \geq 0$ for all $x, y \in S$ (positivity);
        \item $d\left(x, x\right) = 0$ for all $x \in S$;
        \item $d\left(x, y\right) = d\left(x, y\right)$ for all $x, y \in S$ (symmetry);
        \item $d\left(x, y\right) \leq d\left(x, z\right) + d\left(y, z\right)$ for all $x, y, z \in S$ (triangular inequality).
    \end{itemize}
\end{dfnbox}
The key difference between a pseudo-metric and a metric is that a pseudo-metric only requires that every element is at $0$ distance away from itself, whereas a metric requires that every element is \textbf{the only element} that is at $0$ distance away from itself.

By dropping the requirement on symmetry, we obtain the following definition:
\begin{dfnbox}{Quasi-metric}{quasimetric}
    A {\color{red} \textbf{quasi-metric}} on a set $S$ is a function $d \colon S \times S \to \R$ such that:
    \begin{itemize}
        \item $d\left(x, y\right) \geq 0$ for all $x, y \in S$ (positivity);
        \item $d\left(x, y\right) = 0$ if and only if $x = y$ (definiteness);
        \item $d\left(x, y\right) \leq d\left(x, z\right) + d\left(y, z\right)$ for all $x, y, z \in S$ (triangular inequality).
    \end{itemize}
\end{dfnbox}
We equip a set with a metric to generalise the Euclidean spaces.
\begin{dfnbox}{Metirc Space}{metricSpace}
    A {\color{red} \textbf{metric space}} $\left(S, d\right)$ is a set $S$ together with a metric $d$ on $S$.    
\end{dfnbox}
The most basic example of a metric is the \textit{discrete metric} defined by 
\begin{equation*}
    d\left(x, y\right) = \begin{cases}
        1 & \quad \textrm{if } x \neq y \\
        0 & \quad \textrm{if } x = y
    \end{cases}
\end{equation*}
over any set $X$, which essentially is just a characteristic function.

Recall that in an inner product space $\left(V, g\right)$ over some field $\F$, we can define the length of any~$\mathbfit{v} \in V$ as 
\begin{equation*}
    \norm{\mathbfit{v}} = \sqrt{g\left(\mathbfit{v}, \mathbfit{v}\right)}.
\end{equation*}
This length function induces a metric over $V$ given by 
\begin{equation*}
    d\left(\mathbfit{x}, \mathbfit{y}\right) = \norm{\mathbfit{x - y}}.
\end{equation*}
In the Euclidean space $\R^n$, a usual definition for distance is
\begin{equation*}
    d_2\left(\mathbfit{x}, \mathbfit{y}\right) = \left[\sum_{i = 1}^{n}\left(y_i - x_i\right)^2\right]^{\frac{1}{2}}.
\end{equation*}
Note that $\left(\R^n, d_2\right)$ is a metric space, where $d_2$ is known as the \textit{Euclidean distance}. In general, we can prove that for any $p \in \N^+$,
\begin{equation*}
    d_p\left(\mathbfit{x}, \mathbfit{y}\right) = \left[\sum_{i = 1}^{n}\norm{y_i - x_i}^p\right]^{\frac{1}{p}}
\end{equation*}
is a metric over $\F^n$ for any inner product space $\left(\F^n, g\right)$ where $\F$ is a field, known as the $L^p$\textit{-norm}. Furthermore, notice that
\begin{equation*}
    \max_{i \in \N^+, i \leq n}\norm{y_i - x_i}^p \leq \sum_{i = 1}^{n}\norm{y_i - x_i}^p \leq n\max_{i \in \N^+, i \leq n}\norm{y_i - x_i}^p.
\end{equation*}
Taking the $p$-th root on all three parts, we have
\begin{equation*}
    \max_{i \in \N^+, i \leq n}\norm{y_i - x_i} \leq \left[\sum_{i = 1}^{n}\norm{y_i - x_i}^p\right]^{\frac{1}{p}} \leq n^{\frac{1}{p}}\max_{i \in \N^+, i \leq n}\norm{y_i - x_i}.
\end{equation*}
By Squeeze Theorem, this allows us to define
\begin{equation*}
    d_\infty\left(\mathbfit{x}, \mathbfit{y}\right) = \lim_{p \to \infty}d_p(\mathbfit{x}, \mathbfit{y}) = \max_{i \in \N^+, i \leq n}\norm{y_i - x_i}.
\end{equation*}
$d_\infty\left(\mathbfit{x}, \mathbfit{y}\right)$ can be alternatively written as $\norm{\mathbfit{x - y}}_\infty$, which is known as the \textit{infinite norm}.

The $p$-adic numbers can be defined from the following lemma:
\begin{lembox}{$p$-adic Numbers}{padic}
    Let $p$ be any prime number. For all $x \in \Q \backslash \left\{0\right\}$, there exists a unique $k \in \Z$ such that
    \begin{equation*}
        x = \frac{p^kr}{s}, \qquad r, s \in \Z
    \end{equation*}
    with $p \not\mid r, s$ and $s \neq 0$.
\end{lembox}
The $p$-\textit{adic norm} is defined as 
\begin{equation*}
    \abs{x}_p = \begin{cases}
        p^{-k} & \quad\textrm{if } x = \frac{p^kr}{s} \\
        0 & \quad\textrm{if } x = 0
    \end{cases},
\end{equation*}
which induces a metric over $\Q$ defined by 
\begin{equation*}
    d\left(x, y\right) =\abs{x - y}_p.
\end{equation*}
We can show that the $p$-adic metric satisfies 
\begin{equation*}
    d\left(x, z\right) \leq \max\left\{d\left(x, y\right), d\left(y, z\right)\right\}
\end{equation*}
for all $x, y, z \in \Q$. Such a metric is known as an \textit{ultra-metric}.

Given any metric space, the metric will induce a distance between subsets of the space.
\begin{dfnbox}{Distance between Subsets}{distSubset}
    Let $\left(X, d\right)$ be a metric space and $A, B \subseteq X$ be non-empty. The {\color{red} \textbf{distance}} between $A$ and $B$ is defined as 
    \begin{equation*}
        d\left(A, B\right) \coloneqq \inf \left\{d\left(x, y\right) \colon \left(x, y\right) \in A \times B\right\}.
    \end{equation*}
\end{dfnbox}
Additionally, we may wish to define a measure for the size of a subset in a metric space.
\begin{dfnbox}{Diameter}{diam}
    Let $\left(X, d\right)$ be a metric space. The {\color{red} \textbf{diameter}} of a set $A \subseteq X$ is defined as 
    \begin{equation*}
        \mathrm{diam}\left(A\right) \coloneqq \sup\left\{d\left(x, y\right) \colon \left(x, y\right) \in A \times A\right\}.
    \end{equation*}
    The set $A$ is {\color{red} \textbf{bounded}} if $\mathrm{diam}\left(A\right)$ is finite.
\end{dfnbox}
The name ``diameter'' is not a coincidence with the diameter of a graph. Specifically, if we consider a graph $G = \left(V, E\right)$, the pair $\left(V, d\right)$ forms a metric space with $d\left(u, v\right)$ being the usual distance between two vertices in $G$ defined as the size of the shortest $u$-$v$ path in $G$. It is clear that $d$ is indeed a metric.

Now, let us consider the subgraph $H \subseteq G$ induced by any $U \subseteq V$ and check the eccentricity for $H$, i.e.,
\begin{align*}
    \epsilon\left(u\right) = \max\left\{d_H\left(u, u'\right) \colon u' \in U\right\} & \qquad\textrm{for all } u \in U.
\end{align*}
Now, the diameters for $H$ can be computed as 
\begin{align*}
    \mathrm{diam}\left(H\right) & = \max\left\{\epsilon\left(u\right) \colon u \in U\right\} \\
    & = \sup\left\{d_H\left(u, u'\right) \colon \left(u, u'\right) \in U \times U\right\},
\end{align*}
and this obviously agrees with Definition \ref{dfn:diam}!

Recall that in Definition \ref{dfn:standardTopology}, we use Euclidean open balls to construct a basis for a topology on $\R^n$. We can generalise this idea in any metric space.
\begin{probox}{Metric Induces a Basis}{metricBasis}
    Let $\left(X, d\right)$ be a metric space. Define 
    \begin{equation*}
        B_r\left(x\right) \coloneqq \left\{y \in X \colon d\left(x, y\right) < r\right\},
    \end{equation*}
    then collection
    \begin{equation*}
        \mathcal{B}_d \coloneqq \left\{B_r\left(x\right) \colon x \in X, r \in \R^+\right\}
    \end{equation*}
    is a basis for a topology on $X$.
    \tcblower
    \begin{proof}
        Notice that for any $x \in X$, we have $x \in B_1\left(x\right) \in \mathcal{B}_d$. Let $B_p\left(x_1\right), B_q\left(x_2\right) \in \mathcal{B}_d$ be such that $x \in B_p\left(x\right) \cap B_q\left(x\right)$. Take $k = \min\left\{p - d\left(x, x_1\right), q - d\left(x, x_2\right)\right\}$, then clearly $k > 0$ and we can find $B_k\left(x\right) \subseteq B_p\left(x\right) \cap B_q\left(x\right)$ such that $x \in B_k\left(x\right) \in \mathcal{B}_d$. Therefore, $\mathcal{B}_d$ is a basis for a topology on $X$.
    \end{proof}
\end{probox}
Since we can obtain a basis from a metric, it follows naturally that we can generate a topology using this induced basis.
\begin{dfnbox}{Metrisable Topology}{metrisable}
    Let $\left(X, d\right)$ be a metric space. A topology $\mathcal{T}$ on $X$ is {\color{red} \textbf{metrisable}}, or {\color{red} \textbf{induced}} by $d$, if it is generated by $\mathcal{B}_d$
\end{dfnbox}
We can verify that the discrete topology $\mathcal{P}\left(X\right)$ is induced by the discrete metric. Let the discrete metric on $X$ be $\chi$, then it is easy to see that 
\begin{equation*}
    B_r\left(x\right) = \begin{cases}
        \left\{x\right\} & \quad\textrm{if } 0 < r \leq 1 \\
        X & \quad\textrm{if } r > 1
    \end{cases}.
\end{equation*}
Therefore, 
\begin{equation*}
    \mathcal{B}_{\chi} = \left\{X\right\} \cup \bigl\{\left\{x\right\} \colon x \in X\bigr\}.
\end{equation*}
Let $\mathcal{T}_{\chi}$ be the topology on $X$ generated by $\mathcal{B}_{\chi}$, then it suffices to prove that $\mathcal{P}\left(X\right) \subseteq \mathcal{T}_{\chi}$. Take any $U \in \mathcal{P}\left(X\right)$, then for any $u \in U$, we have $u \in \left\{u\right\} \subseteq U$. Clearly $\left\{u\right\} \in \mathcal{B}_{\chi}$, so $\mathcal{T}_{\chi} = \mathcal{P}\left(X\right)$ is the discrete topology indeed.

In particular, for Euclidean spaces, the following result extends Definition \ref{dfn:standardTopology}:
\begin{probox}{Every $L^p$-metric Generates the Standard Topology}{LpMetricGenerateStandardTopology}
    Let $\mathcal{T}$ be the standard topology on $\R^n$, then $\mathcal{T}$ is induced by any $L^p$-metric $d_p$.
    \tcblower
    \begin{proof}
        For any $p \in \N^+$, notice that 
        \begin{equation*}
            \max_{i \in \N^+, i \leq n}\norm{y_i - x_i}^p \leq \sum_{i = 1}^{n}\norm{y_i - x_i}^p \leq n\max_{i \in \N^+, i \leq n}\norm{y_i - x_i}^p.
        \end{equation*}
        Taking the $p$-th root yields
        \begin{equation*}
            \max_{i \in \N^+, i \leq n}\norm{y_i - x_i} \leq \left[\sum_{i = 1}^{n}\norm{y_i - x_i}^p\right]^{\frac{1}{p}} \leq n^{\frac{1}{p}}\max_{i \in \N^+, i \leq n}\norm{y_i - x_i}.
        \end{equation*}
        This means that 
        \begin{equation*}
            d_{\infty}\left(\mathbfit{x}, \mathbfit{y}\right) \leq d_p\left(\mathbfit{x}, \mathbfit{y}\right) \leq n^{\frac{1}{p}}d_{\infty}\left(\mathbfit{x}, \mathbfit{y}\right).
        \end{equation*}
        Let $\mathcal{T}_0$ and $\mathcal{T}_p$ be topologies on $\R^n$ generated by $\mathcal{B}_{d_{\infty}}$ and $\mathcal{B}_{d_p}$ respectively. Take any~$T \in \mathcal{T}_p$, then for any $\mathbfit{t} \in T$, there is some $B_{r}\left(\mathbfit{t}'\right) \in \mathcal{B}_{d_{p}}$ such that~$\mathbfit{t} \in B_{r}\left(\mathbfit{t}'\right) \subseteq T$. Take some $\ell = \frac{1}{2}\abs{r - d_p\left(\mathbfit{t}, \mathbfit{t}'\right)}$, then we have found $B_{\ell}\left(\mathbfit{t}\right) \in \mathcal{B}_{d_p}$ such that 
        \begin{equation*}
            \mathbfit{t} \in B_{\ell}\left(\mathbfit{t}\right) \subseteq B_{r}\left(\mathbfit{t}'\right) \subseteq T.
        \end{equation*}
        Take $k = \frac{1}{2}\ell n^{-\frac{1}{p}}$ and consider 
        \begin{equation*}
            B_{k}\left(\mathbfit{t}\right) \coloneqq \left\{\mathbfit{y} \in \R^n \colon d_{\infty}\left(\mathbfit{t}, \mathbfit{y}\right) < k\right\} \in \mathcal{B}_{d_{\infty}}.
        \end{equation*}
        Notice that for each $\mathbfit{y} \in B_{k}\left(\mathbfit{t}\right)$, we have 
        \begin{equation*}
            d_p\left(\mathbfit{t}, \mathbfit{y}\right) \leq n^{\frac{1}{p}}d_{\infty}\left(\mathbfit{t}, \mathbfit{y}\right) < \ell,
        \end{equation*}
        so $\mathbfit{t} \in B_{k}\left(\mathbfit{t}\right) \subseteq B_{\ell}\left(\mathbfit{t}\right) \subseteq T$. This implies that $T \in \mathcal{T}_0$ and so $\mathcal{T}_p \subseteq \mathcal{T}_0$. By a similar argument, one may check that $\mathcal{T}_0 \subseteq \mathcal{T}_p$. Therefore, $\mathcal{T}_0 = \mathcal{T}_p$ for any $p \in \N^+$. Note that by Definition \ref{dfn:standardTopology}, $\mathcal{T}$ is generated by $\mathcal{B}_{d_2}$, which means that $\mathcal{T} = \mathcal{T}_2 = \mathcal{T}_0 = \mathcal{T}_p$ for any $p \in \N^+$. Therefore, $\mathcal{T}$ is induce by any $L^p$-metric $d_p$.
    \end{proof}
\end{probox}
The fact that
\begin{equation*}
    d_{\infty}\left(\mathbfit{x}, \mathbfit{y}\right) \leq d_p\left(\mathbfit{x}, \mathbfit{y}\right) \leq n^{\frac{1}{p}}d_{\infty}\left(\mathbfit{x}, \mathbfit{y}\right)
\end{equation*}
means that all $L^p$-metrics are equivalent over the same space.
\section{Subspace Topologies}
\begin{dfnbox}{Subspace Topology}{subTopology}
    Let $\left(Y, \mathcal{T}_Y\right)$ be a topological space and $X \subseteq Y$ be some subset. The collection
    \begin{equation*}
        \mathcal{T}_X \coloneqq \left\{U \cap X \colon U \in \mathcal{T}_Y\right\}
    \end{equation*}
    is the {\color{red} \textbf{subspace topology}} on $X$.
\end{dfnbox}
We may check that $\mathcal{T}_X$ defined as such is indeed a topology on $X$. First, by taking $U = \varnothing$ and~$U = Y$ respectively, we know that $\varnothing, X \in \mathcal{T}_X$. For any $U \in \mathcal{T}_Y$, we have $Y \backslash U \in \mathcal{T}_Y$ and so 
\begin{equation*}
    X \backslash \left(U \cap X\right) = \left(Y \backslash U\right) \cap X \in \mathcal{T}_X.
\end{equation*}
For any $\mathcal{V} \subseteq \mathcal{T}_X$, we define a subset $\mathcal{U_{\mathcal{V}}} \subseteq \mathcal{T}_Y$ such that for each $V \in \mathcal{V}$ there is a unique~$U_V \in \mathcal{U}_{\mathcal{V}}$ such that $V = U_V \cap X$. Then, 
\begin{align*}
    \bigcup_{A \in \mathcal{V}}A & = \bigcup_{B \in \mathcal{U}_{\mathcal{V}}}\left(B \cap X\right) \\
    & = \left(\bigcup_{B \in \mathcal{U}_{\mathcal{V}}}B\right) \cap X \\
    & \in \mathcal{T}_X.
\end{align*}
Let $X_1, X_2, \cdots, X_n \in \mathcal{T}_X$ and define $X_i = U_i \cap X$ where $U_i \in \mathcal{T}_Y$ for $i = 1, 2, \cdots, n$, then 
\begin{align*}
    \bigcap_{i = 1}^nX_i & = \bigcap_{i = 1}^n\left(U_i \cap X\right) \\
    & = \left(\bigcap_{i = 1}^nU_i\right) \cap X \\
    & \in \mathcal{T}_X.
\end{align*}
So $\mathcal{T}_X$ is really a topology on $X$. Intuitively, the following holds:
\begin{probox}{Basis for a Subspace}{subspaceBasis}
    Let $\left(Y, \mathcal{T}_Y\right)$ be a topological space and $\mathcal{T}_X$ be the subspace topology on some $X \subseteq Y$. If~$\mathcal{B}_Y$ is a basis of $\mathcal{T}_Y$, then 
    \begin{equation*}
        \mathcal{B}_X \coloneqq \left\{B \cap X \colon B \in \mathcal{B}_Y\right\}
    \end{equation*}
    is a basis of $\mathcal{T}_X$.
    \tcblower
    \begin{proof}
        We first prove that $\mathcal{B}_X$ is a basis. Take any $x \in X \subseteq Y$. Note that there exists some $B \in \mathcal{B}_Y$ such that $x \in B$. Take $B \cap X \in \mathcal{B}_X$, then $x \in B \cap X$. For any $B_1, B_2 \in \mathcal{B}_X$ with~$x \in B_1 \cap B_2$, we write $B_1 \coloneqq B_1' \cap X$ and $B_2 \coloneqq B_2' \cap X$ where $B_1', B_2' \in \mathcal{B}_Y$, then we have $x \in B_1' \cap B_2'$. This means that there is some $B \in \mathcal{B}_Y$ such that~$x \in B \subseteq B_1' \cap B_2'$. Write $B' \coloneqq B \cap X \in \mathcal{B}_X$, then for each $b \in B'$, we know that $b \in B_1' \cap B_2'$ and $b \in X$, which implies that $b \in B_1 \cap B_2$. Therefore, $x \in B' \subseteq B_1 \cap B_2$. This means that $\mathcal{B}_X$ is a basis of a topology on $X$. 
        \\\\
        We then prove that $\mathcal{T}_X$ is generated by $\mathcal{B}_X$. Let $\mathcal{T}$ be the topology generated by~$\mathcal{B}_X$. By Proposition \ref{pro:equivalentConstruction}, we have 
        \begin{equation*}
            \mathcal{T} = \left\{\bigcup_{A \in \mathcal{V}}A \colon \mathcal{V} \subseteq \mathcal{B}_X\right\}.
        \end{equation*}
        Similarly, we can write 
        \begin{equation*}
            \mathcal{T}_Y = \left\{\bigcup_{A \in \mathcal{V}}A \colon \mathcal{V} \subseteq \mathcal{B}_Y\right\}.
        \end{equation*}
        Take any $T \in \mathcal{T}_X$, then there exists some $\mathcal{V} \subseteq \mathcal{B}_Y$ such that
        \begin{align*}
            T & = \left(\bigcup_{A \in \mathcal{V}}A\right) \cap X \\
            & = \bigcup_{A \in \mathcal{V}}A \cap X \\
            & \in \mathcal{T}.
        \end{align*}
        Therefore, $\mathcal{T}_X \subseteq \mathcal{T}$. Conversely, take any $T' \in \mathcal{T}$, there exists some $\mathcal{U} \subseteq \mathcal{B}_Y$ such that
        \begin{align*}
            T' & = \bigcup_{B \in \mathcal{U}}\left(B \cap X\right) \\
            & = \left(\bigcup_{B \in \mathcal{U}}B\right) \cap X \\
            & \in \mathcal{T}_X.
        \end{align*}
        Therefore, $\mathcal{T} \subseteq \mathcal{T}_X$ and so $\mathcal{T}_X = \mathcal{T}$.
    \end{proof}
\end{probox}
The following result shows that open sets in subspaces remain open in the superspace:
\begin{probox}{Superspace Preserve Open Sets}{preserveOpen}
    Let $\left(Y, \mathcal{T}_Y\right)$ be a topological space. If $X \subseteq Y$ is open in $Y$ and $U \subseteq X$ is open in $X$, then~$U$ is open in $Y$.
    \tcblower
    \begin{proof}
        Let $\mathcal{T}_X$ be the subspace topology on $X$. Since $U$ is open in $X$, we have $U \in \mathcal{T}_X$. By Definition \ref{dfn:subTopology}, there exists some $V \in \mathcal{T}_Y$ such that $U = V \cap X$. However, $U \subseteq X$, so $U = V \in \mathcal{T}_Y$, which means that $U$ is open in $Y$.
    \end{proof}
\end{probox}
We can do a similar manipulation with metric spaces and induce a metric on a subspace.
\begin{dfnbox}{Subspace Metric}{subspaceMetric}
    Let $\left(X, d\right)$ be a metric space. The {\color{red} \textbf{subspace metric}} of some $A \subseteq X$ is the restriction of $d$ to $A$, denoted as 
    \begin{equation*}
        d_A\left(x, y\right) = d\left(x, y\right), \qquad \textrm{for all } x, y \in A.
    \end{equation*}
\end{dfnbox}
Naturally, the following result is true:
\begin{probox}{Subspace Metric Induces Subspace Topology}{subMetricSubTopology}
    Let $\left(X, d\right)$ be a metric space. The topology induced by the subspace metric $d_A$ on some subspace $A \subseteq X$ is the subspace topology on $A$.
    \tcblower
    \begin{proof}
        Let $\mathcal{T}_d$ and $\mathcal{T}_{d_A}$ be topologies induced by $d$ on $X$ with basis $\mathcal{B}_d$ and by $d_A$ on $A$ with basis $\mathcal{B}_{d_A}$ respectively. Let $\mathcal{T}_A$ be the subspace topology on $A$ with basis $\mathcal{B}_A$. Take any $B \in \mathcal{B}_{d_A} \subseteq \mathcal{B}_d$, then clearly any $x \in B$ is such that $x \in B \cap A \in \mathcal{B}_A$. Therefore, by Proposition \ref{pro:fineBasis}, $\mathcal{T}_{d_A} \subseteq \mathcal{T}_A$. Conversely, take any $B_A \mathcal{B}_A$. For any $x \in B_A$, there exists some $B_1 \in \mathcal{B}_d$ such that $x \in B_1$ and $B_A = B_1 \cap A$. Notice that this implies that~$x \in A \in \mathcal{T}_{d_A}$, so we can find some $B_2 \in \mathcal{B}_{d_A} \subseteq \mathcal{B}_d$ such that $x \in B_2$. By Definition \ref{dfn:basis}, there exists some $B \subseteq B_1 \cap B_2 \in \mathcal{B}_{d_A}$ such that $x \in B$. By Proposition \ref{pro:fineBasis}, this means that $\mathcal{T}_A \subseteq \mathcal{T}_{d_A}$. Therefore, $\mathcal{T}_A = \mathcal{T}_{d_A}$. 
    \end{proof}
\end{probox}
\end{document}