\documentclass[math]{amznotes}
\usepackage[utf8]{inputenc}
\usepackage{amsmath}
\usepackage{amsfonts}
%\usepackage{yhmath}
\usepackage{graphicx}
\usepackage{tikz}
\usepackage{etoolbox}
\usepackage{accents}
\DeclareSymbolFont{yhlargesymbols}{OMX}{yhex}{m}{n} \DeclareMathAccent{\yhwidehat}{\mathord}{yhlargesymbols}{"62}
\renewcommand{\mathring}[1]{\accentset{\circ}{#1}}

\usepackage{scalerel}[2014/03/10]
\usepackage{stackengine}

\renewcommand\widetilde[1]{\ThisStyle{%
  \setbox0=\hbox{$\SavedStyle#1$}%
  \stackengine{1pt-\LMpt}{$\SavedStyle#1$}{%
    \stretchto{\scaleto{\SavedStyle\mkern.2mu\sim}{.5467\wd0}}{.5\ht0}%
%    .2mu is the kern imbalance when clipping white space
%    .5467++++ is \ht/[kerned \wd] aspect ratio for \sim glyph
  }{O}{c}{F}{T}{S}%
}}
\makeatletter
\let\save@mathaccent\mathaccent
\newcommand*\if@single[3]{%
  \setbox0\hbox{${\mathaccent"0362{#1}}^H$}%
  \setbox2\hbox{${\mathaccent"0362{\kern0pt#1}}^H$}%
  \ifdim\ht0=\ht2 #3\else #2\fi
  }
%The bar will be moved to the right by a half of \macc@kerna, which is computed by amsmath:
\newcommand*\rel@kern[1]{\kern#1\dimexpr\macc@kerna}
%If there's a superscript following the bar, then no negative kern may follow the bar;
%an additional {} makes sure that the superscript is high enough in this case:
\newcommand*\widebar[1]{\@ifnextchar^{{\wide@bar{#1}{0}}}{\wide@bar{#1}{1}}}
%Use a separate algorithm for single symbols:
\newcommand*\wide@bar[2]{\if@single{#1}{\wide@bar@{#1}{#2}{1}}{\wide@bar@{#1}{#2}{2}}}
\newcommand*\wide@bar@[3]{%
  \begingroup
  \def\mathaccent##1##2{%
%Enable nesting of accents:
    \let\mathaccent\save@mathaccent
%If there's more than a single symbol, use the first character instead \left(see below\right):
    \if#32 \let\macc@nucleus\first@char \fi
%Determine the italic correction:
    \setbox\z@\hbox{$\macc@style{\macc@nucleus}_{}$}%
    \setbox\tw@\hbox{$\macc@style{\macc@nucleus}{}_{}$}%
    \dimen@\wd\tw@
    \advance\dimen@-\wd\z@
%Now \dimen@ is the italic correction of the symbol.
    \divide\dimen@ 3
    \@tempdima\wd\tw@
    \advance\@tempdima-\scriptspace
%Now \@tempdima is the width of the symbol.
    \divide\@tempdima 10
    \advance\dimen@-\@tempdima
%Now \dimen@ = \left(italic correction / 3\right) - \left(Breite / 10\right)
    \ifdim\dimen@>\z@ \dimen@0pt\fi
%The bar will be shortened in the case \dimen@<0 !
    \rel@kern{0.6}\kern-\dimen@
    \if#31
      \overline{\rel@kern{-0.6}\kern\dimen@\macc@nucleus\rel@kern{0.4}\kern\dimen@}%
      \advance\dimen@0.4\dimexpr\macc@kerna
%Place the combined final kern \left(-\dimen@\right) if it is >0 or if a superscript follows:
      \let\final@kern#2%
      \ifdim\dimen@<\z@ \let\final@kern1\fi
      \if\final@kern1 \kern-\dimen@\fi
    \else
      \overline{\rel@kern{-0.6}\kern\dimen@#1}%
    \fi
  }%
  \macc@depth\@ne
  \let\math@bgroup\@empty \let\math@egroup\macc@set@skewchar
  \mathsurround\z@ \frozen@everymath{\mathgroup\macc@group\relax}%
  \macc@set@skewchar\relax
  \let\mathaccentV\macc@nested@a
%The following initialises \macc@kerna and calls \mathaccent:
  \if#31
    \macc@nested@a\relax111{#1}%
  \else
%If the argument consists of more than one symbol, and if the first token is
%a letter, use that letter for the computations:
    \def\gobble@till@marker##1\endmarker{}%
    \futurelet\first@char\gobble@till@marker#1\endmarker
    \ifcat\noexpand\first@char A\else
      \def\first@char{}%
    \fi
    \macc@nested@a\relax111{\first@char}%
  \fi
  \endgroup
}
\makeatother

\graphicspath{ {./images/} }
\geometry{
    a4paper,
    headheight = 1.5cm
}

\patchcmd{\chapter}{\thispagestyle{plain}}
{\thispagestyle{fancy}}{}{}

\theoremstyle{remark}
\newtheorem*{claim}{Claim}
\newtheorem*{remark}{Remark}
\newtheorem{case}{Case}

\newcommand{\zero}{\mathbf{0}}
\newcommand{\one}{\mathbf{1}}
\newcommand{\I}{\mathbfit{I}}
\newcommand{\e}{\mathrm{e}}
\renewcommand{\d}{\mathrm{d}}
\newcommand{\im}{\mathrm{i}}
\newcommand{\map}[3]{#1: #2 \rightarrow #3} % Mapping
\newcommand{\image}[2]{#2\left[#1\right]} % Image
\newcommand{\preimage}[2]{#2\left[#1\right]^{-1}} % Pre-image
\newcommand{\eval}[3]{\left. #1\right\rvert_{#2 = #3}} % Evaluated at
%\newcommand\bigO[1]{\mathcal{O}\left(#1\right)}

\DeclareMathOperator*{\argmax}{argmax}
\DeclareMathOperator*{\argmin}{argmin}
\DeclareMathAlphabet{\mathcal}{OMS}{cmsy}{m}{n}

\begin{document}
\fancyhead[L]{
    Metric and Topological Spaces
}
\fancyhead[R]{
    Lecture Notes
}
\tableofcontents
\chapter{Topology}
\section{Topological Spaces}
\begin{dfnbox}{Topology}{topology}
    A {\color{red} \textbf{topology}} on a set $X$ is a collection $\mathcal{T} \subseteq \mathcal{P}\left(X\right)$ such that 
    \begin{itemize}
        \item $\varnothing, X \in \mathcal{T}$;
        \item for any index set $I$, if $\left\{X_i \colon i \in I\right\} \subseteq \mathcal{P}\left(\mathcal{T}\right)$, then $\bigcup_{i \in I}X_i \in \mathcal{T}$;
        \item for any $X_1, X_2, \cdots, X_n \in \mathcal{T}$, $\bigcap_{i = 1}^nX_i \in \mathcal{T}$.
    \end{itemize}
    The pair $\left(X, \mathcal{T}\right)$ is said to be a {\color{red} \textbf{topological space}}. A subset $Y \subseteq X$ is {\color{red} \textbf{open}} if $Y \in \mathcal{T}$.
\end{dfnbox}
\begin{notebox}
    \begin{remark}
        For any set $X$, we define $\left\{\varnothing, X\right\}$ as the \textit{trivial topology} on $X$, $\mathcal{P}\left(X\right)$ as the \textit{discrete topology}, and $\left\{X \setminus U \colon U \subseteq X \textrm{ is finite}\right\} \cup \left\{\varnothing\right\}$ as the \textit{co-finite topology}.
    \end{remark}
\end{notebox}
The set $\left\{\left(-\alpha, \alpha\right) \colon \alpha > 0\right\} \cup \left\{\R, \varnothing\right\}$ defines a topology on $\R$. This example also demonstrates why it is crucial to only consider closure under finite intersections when defining a topology, because
\begin{equation*}
    \bigcap_{n = 1}^{\infty}\left(-1 - \frac{1}{n}, 1 + \frac{1}{n}\right) = [-1, 1] \notin \mathcal{T}.
\end{equation*}
We now seek a systematic method to generate a topology given any set. The idea here is to make use of a \textit{cover}.
\begin{dfnbox}{Basis}{basis}
    A {\color{red} \textbf{basis}} for a topology on $X$ is a collection $\mathcal{B} \subseteq \mathcal{P}\left(X\right)$ such that 
    \begin{itemize}
        \item for any $x \in X$, there exists some $B \in \mathcal{B}$ such that $x \in B$;
        \item for any $x \in X$ and $B_1, B_2 \in \mathcal{B}$ with $x \in B_1 \cap B_2$, there exists some $B \in \mathcal{B}$ such that $x \in B \subseteq B_1 \cap B_2$.
    \end{itemize}
\end{dfnbox}
It may be useful to see a basis as a cover of a set with the second additional property as stated in the above definition. Notice that the first property of the basis $\mathcal{B}$ is basically saying that 
\begin{equation*}
    X \subseteq \bigcup \mathcal{B},
\end{equation*}
i.e., $\mathcal{B}$ is a cover of $X$.

Given any basis $\mathcal{B}$ for some topology on $X$, a set generated by $\mathcal{B}$ can be defined as 
\begin{equation*}
    \mathcal{T} \coloneqq \left\{U \subseteq X \colon \textrm{for any } u \in U, \textrm{ there exists some } B \in \mathcal{B} \textrm{ such that } u \in B \subseteq U\right\}
\end{equation*}
We will show that $\mathcal{T}$ is a topology on $X$. First, it is clear that $\varnothing, X \in \mathcal{T}$. 

Let $I$ be an index set and $\left\{X_i \colon i \in I\right\}\subseteq \mathcal{P}\left(\mathcal{T}\right)$ be any collection of subsets of $X$. Notice that for any $x \in \bigcup_{i \in I}X_i$, there exists some $j \in I$ such that $x \in X_j \subseteq \mathcal{T}$. According to our construction, this means that there exists some $B \in \mathcal{B}$ such that $x \in B \subseteq X_j \subseteq \mathcal{T}$. Therefore, $\bigcup_{i \in I}X_i \subseteq \mathcal{T}$ as desired.

To prove that $\mathcal{T}$ is closed under finite intersection, we consider the following lemma:
\begin{lembox}{Finite Intersection of Elements in Basis Is Covered}{coverFiniteIntersectionOfBasis}
    Let $\mathcal{B}$ be a basis for a topology on $X$ and $B_1, B_2, \cdots, B_n \in \mathcal{B}$, then for any $x \in \bigcap_{i = 1}^nB_i$, there exists some $B \in \mathcal{B}$ such that $x \in B \subseteq \bigcap_{i = 1}^nB_i$.
    \tcblower
    \begin{proof}
        The case where $n = 1$ is trivial by taking $B = B_1$. Suppose that there is some integer $k \geq 1$ such that for any $B_1, B_2, \cdots, B_k \in \mathcal{B}$ and any $x \in \bigcap_{i = 1}^kB_i$, there exists some $B \in \mathcal{B}$ such that $x \in B \subseteq \bigcap_{i = 1}^kB_i$. Take any $B_{k + 1} \in \mathcal{B}$. It is clear that for any~$x \in \bigcap_{i = 1}^{k + 1}B_i$, there exists some $B \in \mathcal{B}$ such that 
        \begin{equation*}
            x \in B \subseteq \bigcap_{i = 1}^kB_i.
        \end{equation*}
        Notice that $x \in B_{k + 1} \in \mathcal{B}$, so we know that $x \in B \cap B_{k + 1}$. By Definition \ref{dfn:basis}, this means that there exists some $B' \in \mathcal{B}$ such that 
        \begin{equation*}
            x \in B' \subseteq B \cap B_{k + 1} \subseteq \bigcap_{i = 1}^{k + 1}B_i.
        \end{equation*}
    \end{proof}
\end{lembox}
Now, suppose $X_1, X_2, \cdots, X_n \in \mathcal{T}$ are finitely many subsets of $X$. Take any $x \in \bigcap_{i = 1}^nX_i$. It is clear that $x \in X_i$ for each $i = 1, 2, \cdots, n$. Therefore, for each $i = 1, 2, \cdots, n$, there exists some $B_i \in \mathcal{B}$ such that $x \in B_i \subseteq X_i$. By Lemma \ref{lem:coverFiniteIntersectionOfBasis}, this means that there exists some set~$B \in \mathcal{B}$ such that 
\begin{equation*}
    x \in B \subseteq \bigcap_{i = 1}^nB_i \subseteq \bigcap_{i = 1}^nX_i.    
\end{equation*}
Therefore, $\bigcap_{i = 1}^nX_i \in \mathcal{T}$. So this set $\mathcal{T}$ generated by $\mathcal{B}$ is indeed a topology on $X$. 

The following proposition further shows that the topology generated by a basis $\mathcal{B}$ is the set of all possible unions of elements in $\mathcal{B}$:
\begin{probox}{\small Equivalent Construction of Topologies Generated from Bases}{equivalentConstruction}
    Let $X$ be any set. If $\mathcal{B}$ is a basis for a topology $\mathcal{T}$ on $X$, then 
    \begin{equation*}
        \mathcal{T} = \left\{\bigcup_{A \in \mathcal{V}} A  \colon \mathcal{V}\in \mathcal{P}\left(\mathcal{B}\right)\right\}.
    \end{equation*}
    \tcblower
    \begin{proof}
        Denote 
        \begin{equation*}
            \mathcal{T}_{\mathcal{B}} \coloneqq \left\{U \subseteq X \colon \textrm{for any } u \in U, \textrm{ there exists some } B \in \mathcal{B} \textrm{ such that } u \in B \subseteq U\right\}.
        \end{equation*}
        It suffices to prove that $\mathcal{T} = \mathcal{T}_{\mathcal{B}}$. Take any $T \in \mathcal{T}$, then there exists some $V \in \mathcal{P}\left(\mathcal{B}\right)$ such that $T = \bigcup_{A \in \mathcal{V}}A$. This means that for every $t \in T$, there exists some $B_t \in \mathcal{V}$ such that $t \in B_t \subseteq T$. Therefore, $T \in \mathcal{T}_{\mathcal{B}}$. Conversely, for any $S \in \mathcal{T}_{\mathcal{B}}$, there exists some $B_s \in \mathcal{B}$ for each $s \in S$ such that $s \in B_s$. Denote $U \coloneqq \left\{B_s \colon s \in S\right\} \in \mathcal{P}\left(\mathcal{B}\right)$, then it is clear that $S \subseteq \bigcup_{B \in U}B$. Since $B_s \subseteq S$ for each $s \in S$, we have $\bigcup_{B \in U}B \subseteq S$, which implies that $S = \bigcup_{B \in U}B$. This means that $S \in \mathcal{T}$. Therefore, $\mathcal{T} \subseteq \mathcal{T}_{\mathcal{B}}$ and $\mathcal{T}_{\mathcal{B}} \subseteq \mathcal{T}$, which means that~$\mathcal{T} = \mathcal{T}_{\mathcal{B}}$.
    \end{proof}
\end{probox}
Next, we define a special topology in Euclidean spaces using open balls.
\begin{dfnbox}{Standard Topology}{standardTopology}
    For any $\mathbfit{x} = \left(x_1, x_2, \cdots, x_n\right) \in \R^n$ and any $r > 0$. Denote the Euclidean open ball centred at $\mathbfit{x}$ with radius $r$ by
    \begin{equation*}
        B_{r}\left(\mathbfit{x}\right) \coloneqq \left\{\mathbfit{y} = \left(y_1, y_2, \cdots, y_n\right) \in \R^n \colon \sqrt{\sum_{i = 1}^{n}\left(x_i - y_i\right)^2} < r\right\}
    \end{equation*} 
    The {\color{red} \textbf{standard topology}} on $\R^n$ is the set generated by the basis
    \begin{equation*}
        \mathcal{B} \coloneqq \left\{B_{r}\left(\mathbfit{x}\right) \colon \mathbfit{x} \in \R^n, r \in \R^+\right\}.
    \end{equation*}
\end{dfnbox}
It may be helpful to actually show that this set $\mathcal{B}$ is indeed a basis of a topology on $\R^n$. The fact that $\mathcal{B}$ is a cover for $\R^n$ is trivial enough. Take any $\mathbfit{x} \in \R^n$ and balls $B_{\alpha}\left(\mathbfit{x}_1\right), B_{\beta}\left(\mathbfit{x}_2\right) \in \mathcal{B}$ such that $\mathbfit{x} \in B_{\alpha}\left(\mathbfit{x}_1\right) \cap B_{\beta}\left(\mathbfit{x}_2\right)$ (the existence of these $2$ balls is again trivial enough). Take 
\begin{equation*}
    r = \min\left\{\alpha - \norm{\mathbfit{x} - \mathbfit{x}_1}, \beta - \norm{\mathbfit{x} - \mathbfit{x}_2}\right\}.
\end{equation*}
Clearly, $r > 0$ and $\mathbfit{x} \in B_r\left(\mathbfit{x}\right)$, so we are done.

Now, we discuss the analogue of the subset relation in topologies.
\begin{dfnbox}{Fineness and Coarseness}{fine}
    Let $\mathcal{T}$ and $\mathcal{T}'$ be topologies on some set $X$. We say that $\mathcal{T}$ is {\color{red} \textbf{finer}} than $\mathcal{T}'$, or equivalently, that $\mathcal{T}'$ is {\color{red} \textbf{coarser}} than $\mathcal{T}$, if $\mathcal{T}' \subseteq \mathcal{T}$.
\end{dfnbox}
Observe that any topology of $X$ must be a subset of $\mathcal{P}\left(X\right)$, which is the discrete topology on~$X$, so the discrete topology is the finest topology on a set.
\begin{notebox}
    \begin{remark}
        For any basis $\mathcal{B}$ for a topology on $X$, the topology generated by $\mathcal{B}$ is the coarsest topology containing $\mathcal{B}$.
    \end{remark}
\end{notebox}
The above remark is easy to verify. Let $\mathcal{T}$ be any topology on $X$ with $\mathcal{B} \subseteq \mathcal{T}$ and $\mathcal{T}_{\mathcal{B}}$ be the topology generated by $\mathcal{B}$. For any $T \in \mathcal{T}_{\mathcal{B}}$, by Proposition \ref{pro:equivalentConstruction}, there exists some $V \subseteq \mathcal{B}$ such that $T = \bigcup_{A \in \mathcal{V}}A$. Note that $A \in \mathcal{T}$ for all $A \in \mathcal{V}$, so by Definition \ref{dfn:topology}, $T \in \mathcal{T}$ and so $\mathcal{T}_{\mathcal{B}} \subseteq \mathcal{T}$ as desired.

This motivates us to consider fineness in terms of bases.
\begin{probox}{Fineness in Terms of Bases}{fineBasis}
    Let $\mathcal{B}$ and $\mathcal{B}'$ generate topologies $\mathcal{T}$ and $\mathcal{T}'$ respectively on $X$. $\mathcal{T}'$ is finer than $\mathcal{T}$ if and only if for every $B \in \mathcal{B}$ and any $x \in B$, there exists some $B_x \in \mathcal{B}'$ such that~$x \in B_x \subseteq B$.
    \tcblower
    \begin{proof}
        Suppose that $\mathcal{T}'$ is finer than $\mathcal{T}$, then $\mathcal{T} \subseteq \mathcal{T}'$. Take any $B \in \mathcal{B}$, then by Proposition \ref{pro:equivalentConstruction}, $B \in \mathcal{T}$, which means that $B \in \mathcal{T}'$. Since $\mathcal{B}'$ is a basis for $\mathcal{T}'$, by Definition \ref{dfn:basis} for any $x \in B$, there exists some $B_x \in \mathcal{B}'$ such that $x \in B_x \subseteq B$.
        \\\\
        Suppose conversely that for every $B \in \mathcal{B}$ and any $x \in B$, there is some $B_x \in \mathcal{B}'$ such that $x \in B' \subseteq B$. Take any $T \in \mathcal{T}$, for each $x \in T$, by Definition \ref{dfn:basis} there exists some $B \in \mathcal{B}$ such that $x \in B \subseteq T$, and so we can find some~$B_x \in \mathcal{B}'$ such that~$x \in B_x \subseteq B \subseteq T$, so $T \in \mathcal{T}'$. Therefore, $\mathcal{T} \subseteq \mathcal{T}'$ and so $\mathcal{T}'$ is finer than~$\mathcal{T}$.
    \end{proof}
\end{probox}
Recall that every basis of a topology on $X$ is an open cover of $X$ consisting only of subsets of $X$. Therefore, the union of the elements in the basis is essentially $X$ itself. This motivates us to propose another way to generate a topology on a set.
\begin{dfnbox}{Sub-basis}{subbasis}
    A {\color{red} \textbf{sub-basis}} of $X$ is a collection $\mathcal{S} \subseteq \mathcal{P}\left(X\right)$ such that $\bigcup_{A \in \mathcal{S}}A = X$.
\end{dfnbox}
\begin{notebox}
    \begin{remark}
        Every basis is a sub-basis.
    \end{remark}
\end{notebox}
For an arbitrary set $X$, let $\mathcal{S}$ be a sub-basis and denote the collection of all finite subsets of~$\mathcal{P}\left(\mathcal{S}\right)$ as $\mathcal{F}_{\mathcal{S}}$. Define 
\begin{equation*}
    \mathcal{U}_{\mathcal{S}} \coloneqq \left\{\bigcap_{A \in F}A \colon F \in \mathcal{F}_{\mathcal{S}}\right\}
\end{equation*}
to be the collection of all finite intersections of sets in $\mathcal{S}$. The topology generated by a sub-basis of $X$ is given by 
\begin{equation*}
    \mathcal{T} \coloneqq \left\{\bigcup_{A \in \mathcal{V}}A  \colon \mathcal{V}\subseteq \mathcal{U}_{\mathcal{S}}\right\}.
\end{equation*}
We shall show that $\mathcal{T}$ is indeed a topology on $X$ by considering the following proposition:
\begin{probox}{Finite Intersections of Sets in a Sub-basis Form a Basis}{finiteIntersectionBasis}
    Let $\mathcal{S}$ be a sub-basis for a set $X$ and let $\mathcal{U}_{\mathcal{S}}$ be the set of all finite intersections of sets in $\mathcal{S}$, then $\mathcal{U}_{\mathcal{S}}$ is a basis of a topology on $X$.
    \tcblower
    \begin{proof}
        Take any $x \in X$. By Definition \ref{dfn:subbasis}, we have $x \in \bigcup_{A \in \mathcal{S}}A$. Therefore, there exists some $A \in \mathcal{S} \subseteq \mathcal{P}\left(X\right)$ such that $x \in A$. For any $x \in X$ and $B_1, B_2 \in \mathcal{U}_{\mathcal{S}}$ such that~$x \in B_1 \cap B_2$, notice that $B_1 \cap B_2$ is a finite intersection of sets in $\mathcal{S}$, so $B_1 \cap B_2 \in \mathcal{U}_{\mathcal{S}}$. Therefore, by Definition \ref{dfn:basis}, $\mathcal{U}_{\mathcal{S}}$ is a basis. 
    \end{proof}
\end{probox}
With Propositions \ref{pro:finiteIntersectionBasis} and \ref{pro:equivalentConstruction}, it is clear that $\mathcal{T}$ as constructed above is a topology on $X$.
\section{Metric Spaces}
\begin{dfnbox}{Metric}{metric}
    A {\color{red} \textbf{metric}} on a set $S$ is a function $d \colon S \times S \to \R$ such that:
    \begin{itemize}
        \item $d\left(x, y\right) \geq 0$ for all $x, y \in S$ (positivity);
        \item $d\left(x, y\right) = 0$ if and only if $x = y$ (definiteness);
        \item $d\left(x, y\right) = d\left(x, y\right)$ for all $x, y \in S$ (symmetry);
        \item $d\left(x, y\right) \leq d\left(x, z\right) + d\left(y, z\right)$ for all $x, y, z \in S$ (triangular inequality).
    \end{itemize}
\end{dfnbox}
\begin{notebox}
    \begin{remark}
        A metric is sometimes also called a \textit{distance function}.
    \end{remark}
\end{notebox}
A metric generalises the notion of distance in Euclidean spaces. We can weaken the above axioms to arrive at the following definition:
\begin{dfnbox}{Pseudo-metric}{pseudometric}
    A {\color{red} \textbf{pseudo-metric}} on a set $S$ is a function $d \colon S \times S \to \R$ such that:
    \begin{itemize}
        \item $d\left(x, y\right) \geq 0$ for all $x, y \in S$ (positivity);
        \item $d\left(x, x\right) = 0$ for all $x \in S$;
        \item $d\left(x, y\right) = d\left(x, y\right)$ for all $x, y \in S$ (symmetry);
        \item $d\left(x, y\right) \leq d\left(x, z\right) + d\left(y, z\right)$ for all $x, y, z \in S$ (triangular inequality).
    \end{itemize}
\end{dfnbox}
The key difference between a pseudo-metric and a metric is that a pseudo-metric only requires that every element is at $0$ distance away from itself, whereas a metric requires that every element is \textbf{the only element} that is at $0$ distance away from itself.

By dropping the requirement on symmetry, we obtain the following definition:
\begin{dfnbox}{Quasi-metric}{quasimetric}
    A {\color{red} \textbf{quasi-metric}} on a set $S$ is a function $d \colon S \times S \to \R$ such that:
    \begin{itemize}
        \item $d\left(x, y\right) \geq 0$ for all $x, y \in S$ (positivity);
        \item $d\left(x, y\right) = 0$ if and only if $x = y$ (definiteness);
        \item $d\left(x, y\right) \leq d\left(x, z\right) + d\left(y, z\right)$ for all $x, y, z \in S$ (triangular inequality).
    \end{itemize}
\end{dfnbox}
We equip a set with a metric to generalise the Euclidean spaces.
\begin{dfnbox}{Metirc Space}{metricSpace}
    A {\color{red} \textbf{metric space}} $\left(S, d\right)$ is a set $S$ together with a metric $d$ on $S$.    
\end{dfnbox}
The most basic example of a metric is the \textit{discrete metric} defined by 
\begin{equation*}
    d\left(x, y\right) = \begin{cases}
        1 & \quad \textrm{if } x \neq y \\
        0 & \quad \textrm{if } x = y
    \end{cases}
\end{equation*}
over any set $X$, which essentially is just a characteristic function.

Recall that in an inner product space $\left(V, g\right)$ over some field $\F$, we can define the length of any~$\mathbfit{v} \in V$ as 
\begin{equation*}
    \norm{\mathbfit{v}} = \sqrt{g\left(\mathbfit{v}, \mathbfit{v}\right)}.
\end{equation*}
This length function induces a metric over $V$ given by 
\begin{equation*}
    d\left(\mathbfit{x}, \mathbfit{y}\right) = \norm{\mathbfit{x - y}}.
\end{equation*}
In the Euclidean space $\R^n$, a usual definition for distance is
\begin{equation*}
    d_2\left(\mathbfit{x}, \mathbfit{y}\right) = \left[\sum_{i = 1}^{n}\left(y_i - x_i\right)^2\right]^{\frac{1}{2}}.
\end{equation*}
Note that $\left(\R^n, d_2\right)$ is a metric space, where $d_2$ is known as the \textit{Euclidean distance}. In general, we can prove that for any $p \in \N^+$,
\begin{equation*}
    d_p\left(\mathbfit{x}, \mathbfit{y}\right) = \left[\sum_{i = 1}^{n}\norm{y_i - x_i}^p\right]^{\frac{1}{p}}
\end{equation*}
is a metric over $\F^n$ for any inner product space $\left(\F^n, g\right)$ where $\F$ is a field, known as the $L^p$\textit{-norm}. Furthermore, notice that
\begin{equation*}
    \max_{i \in \N^+, i \leq n}\norm{y_i - x_i}^p \leq \sum_{i = 1}^{n}\norm{y_i - x_i}^p \leq n\max_{i \in \N^+, i \leq n}\norm{y_i - x_i}^p.
\end{equation*}
Taking the $p$-th root on all three parts, we have
\begin{equation*}
    \max_{i \in \N^+, i \leq n}\norm{y_i - x_i} \leq \left[\sum_{i = 1}^{n}\norm{y_i - x_i}^p\right]^{\frac{1}{p}} \leq n^{\frac{1}{p}}\max_{i \in \N^+, i \leq n}\norm{y_i - x_i}.
\end{equation*}
By Squeeze Theorem, this allows us to define
\begin{equation*}
    d_\infty\left(\mathbfit{x}, \mathbfit{y}\right) = \lim_{p \to \infty}d_p(\mathbfit{x}, \mathbfit{y}) = \max_{i \in \N^+, i \leq n}\norm{y_i - x_i}.
\end{equation*}
$d_\infty\left(\mathbfit{x}, \mathbfit{y}\right)$ can be alternatively written as $\norm{\mathbfit{x - y}}_\infty$, which is known as the \textit{infinite norm}.

The $p$-adic numbers can be defined from the following lemma:
\begin{lembox}{$p$-adic Numbers}{padic}
    Let $p$ be any prime number. For all $x \in \Q \setminus \left\{0\right\}$, there exists a unique $k \in \Z$ such that
    \begin{equation*}
        x = \frac{p^kr}{s}, \qquad r, s \in \Z
    \end{equation*}
    with $p \not\mid r, s$ and $s \neq 0$.
\end{lembox}
The $p$-\textit{adic norm} is defined as 
\begin{equation*}
    \abs{x}_p = \begin{cases}
        p^{-k} & \quad\textrm{if } x = \frac{p^kr}{s} \\
        0 & \quad\textrm{if } x = 0
    \end{cases},
\end{equation*}
which induces a metric over $\Q$ defined by 
\begin{equation*}
    d\left(x, y\right) =\abs{x - y}_p.
\end{equation*}
We can show that the $p$-adic metric satisfies 
\begin{equation*}
    d\left(x, z\right) \leq \max\left\{d\left(x, y\right), d\left(y, z\right)\right\}
\end{equation*}
for all $x, y, z \in \Q$. Such a metric is known as an \textit{ultra-metric}.

Given any metric space, the metric will induce a distance between subsets of the space.
\begin{dfnbox}{Distance between Subsets}{distSubset}
    Let $\left(X, d\right)$ be a metric space and $A, B \subseteq X$ be non-empty. The {\color{red} \textbf{distance}} between $A$ and $B$ is defined as 
    \begin{equation*}
        d\left(A, B\right) \coloneqq \inf \left\{d\left(x, y\right) \colon \left(x, y\right) \in A \times B\right\}.
    \end{equation*}
\end{dfnbox}
Additionally, we may wish to define a measure for the size of a subset in a metric space.
\begin{dfnbox}{Diameter}{diam}
    Let $\left(X, d\right)$ be a metric space. The {\color{red} \textbf{diameter}} of a set $A \subseteq X$ is defined as 
    \begin{equation*}
        \mathrm{diam}\left(A\right) \coloneqq \sup\left\{d\left(x, y\right) \colon \left(x, y\right) \in A \times A\right\}.
    \end{equation*}
    The set $A$ is {\color{red} \textbf{bounded}} if $\mathrm{diam}\left(A\right)$ is finite.
\end{dfnbox}
The name ``diameter'' is not a coincidence with the diameter of a graph. Specifically, if we consider a graph $G = \left(V, E\right)$, the pair $\left(V, d\right)$ forms a metric space with $d\left(u, v\right)$ being the usual distance between two vertices in $G$ defined as the size of the shortest $u$-$v$ path in $G$. It is clear that $d$ is indeed a metric.

Now, let us consider the subgraph $H \subseteq G$ induced by any $U \subseteq V$ and check the eccentricity for $H$, i.e.,
\begin{align*}
    \epsilon\left(u\right) = \max\left\{d_H\left(u, u'\right) \colon u' \in U\right\} & \qquad\textrm{for all } u \in U.
\end{align*}
Now, the diameters for $H$ can be computed as 
\begin{align*}
    \mathrm{diam}\left(H\right) & = \max\left\{\epsilon\left(u\right) \colon u \in U\right\} \\
    & = \sup\left\{d_H\left(u, u'\right) \colon \left(u, u'\right) \in U \times U\right\},
\end{align*}
and this obviously agrees with Definition \ref{dfn:diam}!

Recall that in Definition \ref{dfn:standardTopology}, we use Euclidean open balls to construct a basis for a topology on $\R^n$. We can generalise this idea in any metric space.
\begin{probox}{Metric Induces a Basis}{metricBasis}
    Let $\left(X, d\right)$ be a metric space. Define 
    \begin{equation*}
        B_r\left(x\right) \coloneqq \left\{y \in X \colon d\left(x, y\right) < r\right\},
    \end{equation*}
    then collection
    \begin{equation*}
        \mathcal{B}_d \coloneqq \left\{B_r\left(x\right) \colon x \in X, r \in \R^+\right\}
    \end{equation*}
    is a basis for a topology on $X$.
    \tcblower
    \begin{proof}
        Notice that for any $x \in X$, we have $x \in B_1\left(x\right) \in \mathcal{B}_d$. Let $B_p\left(x_1\right), B_q\left(x_2\right) \in \mathcal{B}_d$ be such that $x \in B_p\left(x\right) \cap B_q\left(x\right)$. Take $k = \min\left\{p - d\left(x, x_1\right), q - d\left(x, x_2\right)\right\}$, then clearly $k > 0$ and we can find $B_k\left(x\right) \subseteq B_p\left(x\right) \cap B_q\left(x\right)$ such that $x \in B_k\left(x\right) \in \mathcal{B}_d$. Therefore, $\mathcal{B}_d$ is a basis for a topology on $X$.
    \end{proof}
\end{probox}
Since we can obtain a basis from a metric, it follows naturally that we can generate a topology using this induced basis.
\begin{dfnbox}{Metrisable Topology}{metrisable}
    Let $\left(X, d\right)$ be a metric space. A topology $\mathcal{T}$ on $X$ is {\color{red} \textbf{metrisable}}, or {\color{red} \textbf{induced}} by $d$, if it is generated by $\mathcal{B}_d$
\end{dfnbox}
We can verify that the discrete topology $\mathcal{P}\left(X\right)$ is induced by the discrete metric. Let the discrete metric on $X$ be $\chi$, then it is easy to see that 
\begin{equation*}
    B_r\left(x\right) = \begin{cases}
        \left\{x\right\} & \quad\textrm{if } 0 < r \leq 1 \\
        X & \quad\textrm{if } r > 1
    \end{cases}.
\end{equation*}
Therefore, 
\begin{equation*}
    \mathcal{B}_{\chi} = \left\{X\right\} \cup \bigl\{\left\{x\right\} \colon x \in X\bigr\}.
\end{equation*}
Let $\mathcal{T}_{\chi}$ be the topology on $X$ generated by $\mathcal{B}_{\chi}$, then it suffices to prove that $\mathcal{P}\left(X\right) \subseteq \mathcal{T}_{\chi}$. Take any $U \in \mathcal{P}\left(X\right)$, then for any $u \in U$, we have $u \in \left\{u\right\} \subseteq U$. Clearly $\left\{u\right\} \in \mathcal{B}_{\chi}$, so $\mathcal{T}_{\chi} = \mathcal{P}\left(X\right)$ is the discrete topology indeed.

In particular, for Euclidean spaces, the following result extends Definition \ref{dfn:standardTopology}:
\begin{probox}{Every $L^p$-metric Generates the Standard Topology}{LpMetricGenerateStandardTopology}
    Let $\mathcal{T}$ be the standard topology on $\R^n$, then $\mathcal{T}$ is induced by any $L^p$-metric $d_p$.
    \tcblower
    \begin{proof}
        For any $p \in \N^+$, notice that 
        \begin{equation*}
            \max_{i \in \N^+, i \leq n}\norm{y_i - x_i}^p \leq \sum_{i = 1}^{n}\norm{y_i - x_i}^p \leq n\max_{i \in \N^+, i \leq n}\norm{y_i - x_i}^p.
        \end{equation*}
        Taking the $p$-th root yields
        \begin{equation*}
            \max_{i \in \N^+, i \leq n}\norm{y_i - x_i} \leq \left[\sum_{i = 1}^{n}\norm{y_i - x_i}^p\right]^{\frac{1}{p}} \leq n^{\frac{1}{p}}\max_{i \in \N^+, i \leq n}\norm{y_i - x_i}.
        \end{equation*}
        This means that 
        \begin{equation*}
            d_{\infty}\left(\mathbfit{x}, \mathbfit{y}\right) \leq d_p\left(\mathbfit{x}, \mathbfit{y}\right) \leq n^{\frac{1}{p}}d_{\infty}\left(\mathbfit{x}, \mathbfit{y}\right).
        \end{equation*}
        Let $\mathcal{T}_0$ and $\mathcal{T}_p$ be topologies on $\R^n$ generated by $\mathcal{B}_{d_{\infty}}$ and $\mathcal{B}_{d_p}$ respectively. Take any~$T \in \mathcal{T}_p$, then for any $\mathbfit{t} \in T$, there is some $B_{r}\left(\mathbfit{t}'\right) \in \mathcal{B}_{d_{p}}$ such that~$\mathbfit{t} \in B_{r}\left(\mathbfit{t}'\right) \subseteq T$. Take some $\ell = \frac{1}{2}\abs{r - d_p\left(\mathbfit{t}, \mathbfit{t}'\right)}$, then we have found $B_{\ell}\left(\mathbfit{t}\right) \in \mathcal{B}_{d_p}$ such that 
        \begin{equation*}
            \mathbfit{t} \in B_{\ell}\left(\mathbfit{t}\right) \subseteq B_{r}\left(\mathbfit{t}'\right) \subseteq T.
        \end{equation*}
        Take $k = \frac{1}{2}\ell n^{-\frac{1}{p}}$ and consider 
        \begin{equation*}
            B_{k}\left(\mathbfit{t}\right) \coloneqq \left\{\mathbfit{y} \in \R^n \colon d_{\infty}\left(\mathbfit{t}, \mathbfit{y}\right) < k\right\} \in \mathcal{B}_{d_{\infty}}.
        \end{equation*}
        Notice that for each $\mathbfit{y} \in B_{k}\left(\mathbfit{t}\right)$, we have 
        \begin{equation*}
            d_p\left(\mathbfit{t}, \mathbfit{y}\right) \leq n^{\frac{1}{p}}d_{\infty}\left(\mathbfit{t}, \mathbfit{y}\right) < \ell,
        \end{equation*}
        so $\mathbfit{t} \in B_{k}\left(\mathbfit{t}\right) \subseteq B_{\ell}\left(\mathbfit{t}\right) \subseteq T$. This implies that $T \in \mathcal{T}_0$ and so $\mathcal{T}_p \subseteq \mathcal{T}_0$. By a similar argument, one may check that $\mathcal{T}_0 \subseteq \mathcal{T}_p$. Therefore, $\mathcal{T}_0 = \mathcal{T}_p$ for any $p \in \N^+$. Note that by Definition \ref{dfn:standardTopology}, $\mathcal{T}$ is generated by $\mathcal{B}_{d_2}$, which means that $\mathcal{T} = \mathcal{T}_2 = \mathcal{T}_0 = \mathcal{T}_p$ for any $p \in \N^+$. Therefore, $\mathcal{T}$ is induce by any $L^p$-metric $d_p$.
    \end{proof}
\end{probox}
The fact that
\begin{equation*}
    d_{\infty}\left(\mathbfit{x}, \mathbfit{y}\right) \leq d_p\left(\mathbfit{x}, \mathbfit{y}\right) \leq n^{\frac{1}{p}}d_{\infty}\left(\mathbfit{x}, \mathbfit{y}\right)
\end{equation*}
means that all $L^p$-metrics are equivalent over the same space.
\section{Subspace Topologies}
\begin{dfnbox}{Subspace Topology}{subTopology}
    Let $\left(Y, \mathcal{T}_Y\right)$ be a topological space and $X \subseteq Y$ be some subset. The collection
    \begin{equation*}
        \mathcal{T}_X \coloneqq \left\{U \cap X \colon U \in \mathcal{T}_Y\right\}
    \end{equation*}
    is the {\color{red} \textbf{subspace topology}} on $X$.
\end{dfnbox}
We may check that $\mathcal{T}_X$ defined as such is indeed a topology on $X$. First, by taking $U = \varnothing$ and~$U = Y$ respectively, we know that $\varnothing, X \in \mathcal{T}_X$. For any $U \in \mathcal{T}_Y$, we have $Y \setminus U \in \mathcal{T}_Y$ and so 
\begin{equation*}
    X \setminus \left(U \cap X\right) = \left(Y \setminus U\right) \cap X \in \mathcal{T}_X.
\end{equation*}
For any $\mathcal{V} \subseteq \mathcal{T}_X$, we define a subset $\mathcal{U_{\mathcal{V}}} \subseteq \mathcal{T}_Y$ such that for each $V \in \mathcal{V}$ there is a unique~$U_V \in \mathcal{U}_{\mathcal{V}}$ such that $V = U_V \cap X$. Then, 
\begin{align*}
    \bigcup_{A \in \mathcal{V}}A & = \bigcup_{B \in \mathcal{U}_{\mathcal{V}}}\left(B \cap X\right) \\
    & = \left(\bigcup_{B \in \mathcal{U}_{\mathcal{V}}}B\right) \cap X \\
    & \in \mathcal{T}_X.
\end{align*}
Let $X_1, X_2, \cdots, X_n \in \mathcal{T}_X$ and define $X_i = U_i \cap X$ where $U_i \in \mathcal{T}_Y$ for $i = 1, 2, \cdots, n$, then 
\begin{align*}
    \bigcap_{i = 1}^nX_i & = \bigcap_{i = 1}^n\left(U_i \cap X\right) \\
    & = \left(\bigcap_{i = 1}^nU_i\right) \cap X \\
    & \in \mathcal{T}_X.
\end{align*}
So $\mathcal{T}_X$ is really a topology on $X$. Intuitively, the following holds:
\begin{probox}{Basis for a Subspace}{subspaceBasis}
    Let $\left(Y, \mathcal{T}_Y\right)$ be a topological space and $\mathcal{T}_X$ be the subspace topology on some $X \subseteq Y$. If~$\mathcal{B}_Y$ is a basis of $\mathcal{T}_Y$, then 
    \begin{equation*}
        \mathcal{B}_X \coloneqq \left\{B \cap X \colon B \in \mathcal{B}_Y\right\}
    \end{equation*}
    is a basis of $\mathcal{T}_X$.
    \tcblower
    \begin{proof}
        We first prove that $\mathcal{B}_X$ is a basis. Take any $x \in X \subseteq Y$. Note that there exists some $B \in \mathcal{B}_Y$ such that $x \in B$. Take $B \cap X \in \mathcal{B}_X$, then $x \in B \cap X$. For any $B_1, B_2 \in \mathcal{B}_X$ with~$x \in B_1 \cap B_2$, we write $B_1 \coloneqq B_1' \cap X$ and $B_2 \coloneqq B_2' \cap X$ where $B_1', B_2' \in \mathcal{B}_Y$, then we have $x \in B_1' \cap B_2'$. This means that there is some $B \in \mathcal{B}_Y$ such that~$x \in B \subseteq B_1' \cap B_2'$. Write $B' \coloneqq B \cap X \in \mathcal{B}_X$, then for each $b \in B'$, we know that $b \in B_1' \cap B_2'$ and $b \in X$, which implies that $b \in B_1 \cap B_2$. Therefore, $x \in B' \subseteq B_1 \cap B_2$. This means that $\mathcal{B}_X$ is a basis of a topology on $X$. 
        \\\\
        We then prove that $\mathcal{T}_X$ is generated by $\mathcal{B}_X$. Let $\mathcal{T}$ be the topology generated by~$\mathcal{B}_X$. By Proposition \ref{pro:equivalentConstruction}, we have 
        \begin{equation*}
            \mathcal{T} = \left\{\bigcup_{A \in \mathcal{V}}A \colon \mathcal{V} \subseteq \mathcal{B}_X\right\}.
        \end{equation*}
        Similarly, we can write 
        \begin{equation*}
            \mathcal{T}_Y = \left\{\bigcup_{A \in \mathcal{V}}A \colon \mathcal{V} \subseteq \mathcal{B}_Y\right\}.
        \end{equation*}
        Take any $T \in \mathcal{T}_X$, then there exists some $\mathcal{V} \subseteq \mathcal{B}_Y$ such that
        \begin{align*}
            T & = \left(\bigcup_{A \in \mathcal{V}}A\right) \cap X \\
            & = \bigcup_{A \in \mathcal{V}}A \cap X \\
            & \in \mathcal{T}.
        \end{align*}
        Therefore, $\mathcal{T}_X \subseteq \mathcal{T}$. Conversely, take any $T' \in \mathcal{T}$, there exists some $\mathcal{U} \subseteq \mathcal{B}_Y$ such that
        \begin{align*}
            T' & = \bigcup_{B \in \mathcal{U}}\left(B \cap X\right) \\
            & = \left(\bigcup_{B \in \mathcal{U}}B\right) \cap X \\
            & \in \mathcal{T}_X.
        \end{align*}
        Therefore, $\mathcal{T} \subseteq \mathcal{T}_X$ and so $\mathcal{T}_X = \mathcal{T}$.
    \end{proof}
\end{probox}
The following result shows that open sets in subspaces remain open in the superspace:
\begin{probox}{Superspace Preserve Open Sets}{preserveOpen}
    Let $\left(Y, \mathcal{T}_Y\right)$ be a topological space. If $X \subseteq Y$ is open in $Y$ and $U \subseteq X$ is open in $X$, then~$U$ is open in $Y$.
    \tcblower
    \begin{proof}
        Let $\mathcal{T}_X$ be the subspace topology on $X$. Since $U$ is open in $X$, we have $U \in \mathcal{T}_X$. By Definition \ref{dfn:subTopology}, there exists some $V \in \mathcal{T}_Y$ such that $U = V \cap X$. However, $U \subseteq X$, so $U = V \in \mathcal{T}_Y$, which means that $U$ is open in $Y$.
    \end{proof}
\end{probox}
We can do a similar manipulation with metric spaces and induce a metric on a subspace.
\begin{dfnbox}{Subspace Metric}{subspaceMetric}
    Let $\left(X, d\right)$ be a metric space. The {\color{red} \textbf{subspace metric}} of some $A \subseteq X$ is the restriction of $d$ to $A$, denoted as 
    \begin{equation*}
        d_A\left(x, y\right) = d\left(x, y\right), \qquad \textrm{for all } x, y \in A.
    \end{equation*}
\end{dfnbox}
Naturally, the following result is true:
\begin{probox}{Subspace Metric Induces Subspace Topology}{subMetricSubTopology}
    Let $\left(X, d\right)$ be a metric space. The topology induced by the subspace metric $d_A$ on some subspace $A \subseteq X$ is the subspace topology on $A$.
    \tcblower
    \begin{proof}
        Let $\mathcal{T}_d$ and $\mathcal{T}_{d_A}$ be topologies induced by $d$ on $X$ with basis $\mathcal{B}_d$ and by $d_A$ on $A$ with basis $\mathcal{B}_{d_A}$ respectively. Let $\mathcal{T}_A$ be the subspace topology on $A$ with basis $\mathcal{B}_A$. Take any $B_A \in \mathcal{B}_A$, then there exists $B_r\left(x\right) \in \mathcal{B}_d$ such that $B_A = B_r\left(x\right) \cap A$. For any~$y \in B_A$, consider the ball 
        \begin{equation*}
            B_{r'}\left(y\right) \coloneqq \left\{z \in A \colon d_A\left(z, y\right) < r'\right\} \in \mathcal{B}_{d_A}.
        \end{equation*}
        Note that $y \in B_{r'}\left(y\right)$, so by Proposition \ref{pro:fineBasis}, we have $\mathcal{T}_A \subseteq \mathcal{T}_{d_A}$. Conversely, for any~$B_r\left(x\right) \in \mathcal{B}_{d_A}$, there exists some $B_{r'}\left(x\right) \in \mathcal{B}_d$ such that $B_r\left(x\right) \subseteq B_{r'}\left(x\right)$. Notice that $B_r\left(x\right) \subseteq A$, so for any $y \in B_r\left(x\right)$, we have $y \in B_{r'}\left(x\right) \cap A \in \mathcal{B}_A$. Therefore, by Proposition \ref{pro:fineBasis}, $\mathcal{T}_{d_A} \subseteq \mathcal{T}_A$ and so $\mathcal{T}_{d_A} = \mathcal{T}_A$.
    \end{proof}
\end{probox}
\section{Closed Sets}
\begin{dfnbox}{Closed Set}{closedSet}
    Let $\left(X, \mathcal{T}\right)$ be a topological space. A subset $A \subseteq X$ is {\color{red} \textbf{closed}} if $X \setminus A \in \mathcal{T}$.
\end{dfnbox}
A set might be open and closed simultaneously. For example, every set $X$ is both open and closed in itself.
\begin{probox}{\small Arbitrary Intersection and Finite Union of Closed Sets Are Closed}{intersectionClosed}
    Let $\left(X, \mathcal{T}\right)$ be a topological space, then 
    \begin{enumerate}
        \item if $\mathcal{G} \coloneqq \left\{G_{\alpha} \colon \alpha \in I\right\}$ is a family of closed set in $X$ with respect to some index set $I$, then $\bigcap_{\alpha \in I}G_{\alpha}$ is closed in $X$;
        \item if $G_1, G_2, \cdots, G_n$ are closed in $X$, then $\bigcup_{i = 1}^nG_i$ is closed in $X$.
    \end{enumerate}
    \tcblower
    \begin{proof}
        Notice that
        \begin{align*}
            X \setminus \bigcap_{\alpha \in I}G_{\alpha} & = \bigcup_{\alpha \in G} X \setminus G_{\alpha}.
        \end{align*}
        Since $X \setminus G_{\alpha}$ is open in $X$ for all $\alpha \in I$, this means that $X \setminus \bigcap_{\alpha \in I}G_{\alpha}$ is open in $X$, and so $\bigcap_{\alpha \in I}G_{\alpha}$ is closed in $X$. Notice also that 
        \begin{align*}
            X \setminus \bigcup_{i = 1}^nG_i & = \bigcap_{i = 1}^nX \setminus G_i.
        \end{align*}
        By a similar argument $\bigcup_{i = 1}^nG_i$ is closed in $X$.
    \end{proof}
\end{probox}
The following proposition justifies the fact that intersecting a closed set with a subspace produces a closed set in that subspace:
\begin{probox}{Closed Sets in Subspace Topology}{subClosed}
    Let $Y \subseteq X$, then $A \subseteq Y$ is closed in $Y$ if and only if there exists some closed set $G \subseteq X$ such that $A = G \cap Y$.
    \tcblower
    \begin{proof}
        Suppose that $A$ is closed in $Y$, then $Y \setminus A$ is open in $Y$. Therefore, there exists some open set $B \subseteq X$ such that $Y \setminus A = B \cap Y$. Take $G \coloneqq X \setminus B$, then 
        \begin{align*}
            G \cap Y & = A.
        \end{align*}
        Suppose conversely that there exists some closed set $G \subseteq X$ such that $A = G \cap Y$. Consider 
        \begin{equation*}
            Y \setminus \left(G \cap Y\right) = (X \setminus G) \cap Y.
        \end{equation*}
        Notice that $X \setminus G$ is open in $X$, so $Y \cap A$ is open in $Y$, i.e., $A$ is closed in $Y$.
    \end{proof}
\end{probox}
The following result is analogous to Proposition \ref{pro:preserveOpen}:
\begin{probox}{Superspace Preserves Closed Sets}{preserveClosed}
    If $Y \subseteq X$ is closed in $X$ and $A \subseteq Y$ is closed in $Y$, then $A$ is closed in $X$.
    \tcblower
    \begin{proof}
        Consider $X \setminus A = X \setminus Y \cup Y \setminus A$. Since $Y$ is closed in $X$, this means that $X \setminus Y$ is open in $X$. Note also that $Y \setminus A$ is open in $Y$. By Proposition \ref{pro:preserveOpen}, $Y \setminus A$ is open in $X$. Therefore, $X \setminus A$ is open in $X$ and so $A$ is closed in $X$.
    \end{proof}
\end{probox}
Closed sets help define the notion ``interior of a set''.
\begin{dfnbox}{Interior, Closure and Boundary}{intCl}
    Let $\left(X, \mathcal{T}\right)$ be a topological space and $A \subseteq X$. The {\color{red} \textbf{interior}} of $A$ is 
    \begin{equation*}
        \mathring{A} \coloneqq \bigcup_{\substack{U \in \mathcal{T}\\U \subseteq A}}A \cap U.
    \end{equation*}
    The {\color{red} \textbf{closure}} of $A$ is 
    \begin{equation*}
        \widebar{A} \coloneqq \bigcap_{\substack{X \setminus G \in \mathcal{T}\\A \subseteq G}}G.
    \end{equation*}
    The {\color{red} \textbf{boundary}} of $A$ is 
    \begin{equation*}
        \partial A = \widebar{A} \setminus \mathring{A}.
    \end{equation*}
\end{dfnbox}
We interpret the above definition as follows: $\mathring{A}$ is the union of all open sets contained by $A$. Moreover, $\widebar{A}$ is the smallest closed set in $X$ which contains $A$. To see this, let $C \subseteq X$ be any closed set in $X$ containing $A$. Take any $a \in \widebar{A}$, then since $a$ is contained by all closed sets containing $A$, it is clear that $a \in C$, which implies that $\widebar{A} \subseteq C$.
\begin{notebox}
    \begin{remark}
        \quad
        \begin{enumerate}
            \item $\mathring{A} \subseteq A \subseteq \widebar{A}$.
            \item $\mathring{A} = A$ if and only if $A$ is open in $X$.
            \item $\widebar{A} = A$ if and only if $A$ is closed in $X$.
        \end{enumerate}
    \end{remark}
\end{notebox}
We would like to discuss the properties of closure. The following definition is useful:
\begin{dfnbox}{Limit Point}{limPt}
    Let $\left(X, \mathcal{T}\right)$ be a topological space. For any $A \subseteq X$, a point $x \in X$ is a {\color{red} \textbf{limit point}} of $A$ if for every open set $U \subseteq X$ containing $x$,
    \begin{equation*}
        \left(A \setminus \left\{x\right\}\right) \cap U \neq \varnothing.
    \end{equation*}
\end{dfnbox}
Now, we propose two properties for the closure:
\begin{probox}{Properties of Closure}{clProps}
    Let $\left(X, \mathcal{T}\right)$ be a topological space. For any $A \subseteq X$,
    \begin{enumerate}
        \item $x \in \widebar{A}$ if and only if for any open set $U \subseteq X$ containing $x$, $U \cap A \neq \varnothing$;
        \item if $A'$ is the set of limit points of $A$, then $\widebar{A} = A \cup A'$.
    \end{enumerate}
    \tcblower
    \begin{proof}
        We will prove the first statement by considering the contrapositive, i.e., we prove that $x \in X \setminus \widebar{A}$ if and only if there exists some open set $U \subseteq X$ containing $x$ such that $U \cap A = \varnothing$. The ``if'' direction is trivial because $X \setminus \widebar{A}$ is open in $X$ such that~$\left(X \setminus \widebar{A}\right) \cap A = \varnothing$. Take $U \subseteq X$ to be an open set in $X$ with $x \in U$ and $U \cap A = \varnothing$. This means that $x \notin A \subseteq \widebar{A}$. Therefore, $x \in X \setminus \widebar{A}$.
        \\\\
        Take any $a \in A'$, then for every open set $U \subseteq X$ with $a \in U$, we have 
        \begin{equation*}
            U \cap A \supseteq U \cap \left(A \setminus \left\{a\right\}\right) \neq \varnothing.
        \end{equation*}
        Therefore, $A \cup A' \subseteq \widebar{A}$. Take any $x \in \widebar{A}$, we shall prove that if $x \notin A'$, then $x \in A$. Since $x$ is not a limit point of $A$, there exists some open set $V \subseteq X$ containing $x$ such that $\left(A \setminus \left\{x\right\}\right) \cap V = \varnothing$. However, $x \in \widebar{A}$ implies that $V \cap A \neq \varnothing$, so $x \in A$.
    \end{proof}
\end{probox}
The notion of limit points also leads to the definition of convergence. Before that, we shall define the notion of \textit{neighbourhood}. 
\begin{dfnbox}{Neighbourhood}{neighbourhood}
    Let $\left(X, \mathcal{T}\right)$ be a topological space. An open set $U \subseteq X$ is called a {\color{red} \textbf{neighbourhood}} of some $x \in X$ if $x \in U$.
\end{dfnbox}
Intuitively, we think of the statement $x_i \to x$ as the fact that no matter how small a neighbourhood we choose for $x$, there is always a consecutive infinite subsequence of the $x_i$'s which falls in this neighbourhood.
\begin{dfnbox}{Convergence}{convergence}
    A sequence $\left\{x_i\right\}_{i = 1}^{\infty}$ of points in a topological space $\left(X, \mathcal{T}\right)$ {\color{red} \textbf{converges}} to $x \in X$ if for any neighbourhood $U \subseteq X$ containing $x$, there exists some $N \in \N^+$ such that $x_k \in U$ for all $k > N$, denoted as $x_i \to x$. $x$ is said to be the {\color{red} \textbf{limit}} of $\left\{x_i\right\}_{i = 1}^{\infty}$.
\end{dfnbox}
It is important to distinguish between limit and limit points. For example, consider the constant sequence $\left\{1\right\}_{i = 1}^{\infty}$. Clearly, $x_i \to 1$ but one may check that $1$ is not a limit point for this sequence.

In a metric space, we can make use of the metric to describe convergence in a more quantitative way.
\begin{thmbox}{Convergence in Metric Spaces}{metricSpaceConverge}
    Let $\left(X, d\right)$ be a metric space. A sequence $\left\{x_i\right\}_{i = 1}^{\infty}$ in $X$ converges to $x$ if and only if for every~$\epsilon > 0$, there exists some $N \in \N^+$ such that $d\left(x_i, x\right) < \epsilon$ for all $i > N$.
    \tcblower
    \begin{proof}
        Suppose that $x_i \to x$ as $i \to \infty$. For all $\epsilon > 0$, take the open ball $B_{\epsilon}\left(x\right) \subseteq X$. Clearly, $B_{\epsilon}\left(x\right)$ is a neighbourhood of $x$. By Definition \ref{dfn:convergence}, there exists some $N \in \N^+$ such that $x_i \in B_{\epsilon}\left(x\right)$ for all $i > N$, i.e., $d\left(x_i, x\right) < \epsilon$ for all $i > N$. Conversely, suppose that for every $\epsilon > 0$, there exists some $N \in \N^+$ such that $d\left(x_i, x\right) < \epsilon$ for all $i > N$. Let~$U \subseteq X$ be any neighbourhood containing $x$. Note that $U$ is open in~$X$, so by Theorem \ref{pro:equivalentConstruction}, there exists some open ball $B_r\left(x\right) \subseteq U$ such that $x \in B_r\left(x\right)$. Therefore, there exists some $M \in \N^+$ such that $d\left(x_i, x\right) < r$, i.e., $x_i \in B_r\left(x\right) \subseteq U$, for all $i > M$. Therefore, $x_i \to x$.
    \end{proof}
\end{thmbox}
\section{Continuity}
\begin{dfnbox}{Continuous Map}{continuous}
    Let $X$ and $Y$ be topological spaces. A map $f \colon X \to Y$ is {\color{red} \textbf{continuous}} if for any open set $U \subseteq Y$, the pre-image $f^{-1}\left(U\right)$ is open in $X$.
\end{dfnbox}
Suppose $\mathcal{T}_X$ and $\mathcal{T}_Y$ are topologies on $X$ and $Y$ respectively. The above definition basically says that for all $U \in \mathcal{T}_Y$, we have $f^{-1}\left(U\right) \in \mathcal{T}_X$. The following proposition gives an equivalent definition for continuity in terms of sub-bases.
\begin{probox}{Equivalent Definition of Continuity}{equivDefContinuity}
    If $\mathcal{S}$ is a sub-basis for a topology on some set $Y$, then for any topological space $X$, a map~$f \colon X \to Y$ is continuous if and only if $f^{-1}\left(S\right)$ is open in $X$ for any $S \in \mathcal{S}$.
    \tcblower
    \begin{proof}
        Suppose that $f$ is continuous. Note that any $S \in \mathcal{S}$ is open in $Y$, so by Definition \ref{dfn:continuous}, $f^{-1}\left(S\right)$ is open in $X$. Suppose conversely that $f^{-1}\left(S\right)$ is open in $X$ for any~$S \in \mathcal{S}$. Take any open set $U \subseteq Y$. By Propositions \ref{pro:subspaceBasis} and \ref{pro:equivalentConstruction}, there exists finite subsets $\mathcal{U}_i \subseteq \mathcal{P}\left(\mathcal{S}\right)$ where $i \in I$ for some index set $I$ such that 
        \begin{equation*}
            U = \bigcup_{i \in I}\left(\bigcap_{S \in \mathcal{U}_i}S\right).
        \end{equation*}
        Therefore,
        \begin{align*}
            f^{-1}\left(U\right) & = \bigcup_{i \in I}\left(\bigcap_{S \in \mathcal{U}_i}f^{-1}\left(S\right)\right),
        \end{align*}
        which is clearly open in $X$. Therefore, $f$ is continuous.
    \end{proof}
\end{probox}
A trivial example for continuous maps is the \textit{constant map} $f \colon X \to Y$ such that $f\left(x\right) = y_0$ for some fixed $y_0 \in Y$. This is simply because 
\begin{equation*}
    f^{-1}\left(U\right) = \begin{cases}
        X & \quad\textrm{if } y_0 \in U \\
        \varnothing & \quad\textrm{otherwise}
    \end{cases}.
\end{equation*}
The following result should be very intuitive:
\begin{probox}{Composition Preserves Continuity}{compoContinuity}
    Let $X, Y, Z$ be topological spaces. If $f \colon X \to Y$ and $g \colon Y \to Z$ are continuous maps, then $g \circ f$ is continuous.
\end{probox}
It is also clear that for any topological space $X$, the \textit{inclusion map} $f \colon A \to X$ for any $A \subseteq X$ such that $f\left(a\right) = a$ is continuous. Analogously, if $f \colon X \to Y$ is continuous, then the restriction $f\rvert_A \colon A \to Y$ for any subspace $A \subseteq X$ is also continuous.
\begin{probox}{Properties of Continuous Maps}{continuousMapProps}
    Let $X$ and $Y$ be topological spaces. For any map $f \colon X \to Y$, the followings are equivalent:
    \begin{enumerate}
        \item $f$ is continuous;
        \item for all $A \subseteq X$, $f\left(\widebar{A}\right) \subseteq \widebar{f\left(A\right)}$;
        \item for any closed set $B \subseteq Y$, $f^{-1}\left(B\right)$ is closed in $X$;
        \item for any $x \in X$ and any open set $V \subseteq Y$ with $f\left(x\right) \in V$, there exists an open set~$U \subseteq X$ such that $x \in U$ and $f\left(U\right) \subseteq V$.
    \end{enumerate}
    \tcblower
    \begin{proof}
        Suppose that $f$ is continuous. Note that $f\left(A\right) \subseteq \widebar{f\left(A\right)}$, so $A \subseteq f^{-1}\left(\widebar{f\left(A\right)}\right)$. Since $\widebar{f\left(A\right)}$ is closed in $Y$, so by Definition \ref{dfn:continuous}, 
        \begin{equation*}
            f^{-1}\left(Y \setminus \widebar{f\left(A\right)}\right) = X \setminus f^{-1}\left(\widebar{f\left(A\right)}\right)
        \end{equation*}
        is open in $X$. Therefore, $f^{-1}\left(\widebar{f\left(A\right)}\right)$ is closed in $X$. By Proposition \ref{pro:clProps}, $\widebar{A} = A \cup A'$ where $A'$ is the set of limit points of $A$. We claim that $A' \subseteq f^{-1}\left(\widebar{f\left(A\right)}\right)$. Suppose on contrary that there exists some $a \in A' \setminus f^{-1}\left(\widebar{f\left(A\right)}\right)$. Since $X \setminus f^{-1}\left(\widebar{f\left(A\right)}\right)$ is open, by Definition \ref{dfn:limPt}, 
        \begin{equation*}
            \left(A \setminus \left\{a\right\}\right) \cap \Biggl(X \setminus f^{-1}\left(\widebar{f\left(A\right)}\right)\Biggr) \neq \varnothing,
        \end{equation*}
        which is a contradiction because $A \setminus \left\{a\right\} \subseteq f^{-1}\left(\widebar{f\left(A\right)}\right)$. Therefore, $\widebar{A} \subseteq f^{-1}\left(\widebar{f\left(A\right)}\right)$, which means that $f\left(\widebar{A}\right) \subseteq \widebar{f\left(A\right)}$.
        \\\\
        Suppose that $f\left(\widebar{A}\right) \subseteq \widebar{f\left(A\right)}$ for any $A \subseteq X$. For any closed set $B \subseteq Y$, we have $B = \widebar{B}$. Notice that 
        \begin{equation*}
            f\left(\widebar{f^{-1}\left(B\right)} \right) \subseteq \widebar{f\left(f^{-1}\left(B\right)\right)} = B = f\left(f^{-1}\left(B\right)\right),
        \end{equation*}
        so $\widebar{f^{-1}\left(B\right)} \subseteq f^{-1}\left(B\right)$. This implies that $\widebar{f^{-1}\left(B\right)} \subseteq f^{-1}\left(B\right)$, and so $f^{-1}\left(B\right)$ is closed in~$X$.
        \\\\
        Suppose that $f^{-1}\left(B\right)$ is closed in $X$ for any closed set $b \subseteq Y$. Take any $x \in X$ and any open set $V \subseteq Y$ with $f\left(x\right) \in V$. Since $Y \setminus V$ is closed in $Y$, $f^{-1}\left(Y \setminus V\right) = X \setminus f^{-1}\left(V\right)$ is closed in $X$. Therefore, $f^{-1}\left(V\right)$ is open in $X$. It is clear that $x \in f^{-1}\left(V\right)$ and~$f\left(f^{-1}\left(V\right)\right) \subseteq V$.
    \end{proof}
\end{probox}
Next, we introduce a lemma which specifies a methodology to construct a continuous map from two different continuous maps.
\begin{lembox}{Pasting Lemma}{pasting}
    Let $X$ and $Y$ be topological spaces such that $X = A \cup B$ for some closed sets $A$ and $B$. If~$f \colon A \to Y$ and $g \colon B \to Y$ are continuous and $f\left(x\right) = g\left(x\right)$ for all $x \in A \cap B$, then the function $h \colon X \to Y$ defined by 
    \begin{equation*}
        h\left(x\right) = \begin{cases}
            f\left(x\right) & \quad\textrm{if } x \in A \\
            g\left(x\right) & \quad\textrm{if } x \in B
        \end{cases}.
    \end{equation*}
    \tcblower
    \begin{proof}
        Let $U \subseteq Y$ be any open set. Then it is clear that $h^{-1}\left(U\right) = f^{-1}\left(U\right) \cup g^{-1}\left(U\right)$. Since $f$ and $g$ are continuous, both $f^{-1}\left(U\right)$ and $g^{-1}\left(U\right)$ are open in $X$, so it follows that $h^{-1}\left(U\right)$ is open in $X$. Therefore, $h$ is continuous.
    \end{proof}
\end{lembox}
Observe that the continuity of a function $f \colon X \to Y$ actually depends on our choice of topologies on $X$ and $Y$. On the other hand, this also means that every function $f$ could induce a topology on $X$ such that it is continuous.
\begin{dfnbox}{Pull-Back Topology}{pullback}
    Let $\mathcal{T}_Y$ be a topology on $Y$ and let $f \colon X \to Y$. The {\color{red} \textbf{pull-back topology}} on $X$ is defined as 
    \begin{equation*}
        \mathcal{T}_X \coloneqq \left\{f^{-1}\left(U\right) \colon U \in \mathcal{T}_Y\right\}.
    \end{equation*}
\end{dfnbox}
Note that the pull-back topology is the coarsest topology on $X$ such that $f$ is a continuous map. To verify this, let $\mathcal{T}$ be any topology on $X$ such that $f$ is continuous. Take any $T \in \mathcal{T}_X$, then there exists some $U \in \mathcal{T}_Y$ such that $T = f^{-1}\left(U\right)$. However, this means that $T \in \mathcal{T}$ since $f$ is continuous with respect to $\mathcal{T}$. This shows that $\mathcal{T}_X \subseteq \mathcal{T}$.
\begin{dfnbox}{Uniform Continuity}{uniformContinuity}
    Let $\left(X, d_X\right)$ and $\left(Y, d_Y\right)$ be metric spaces. A function $f \colon X \to Y$ is {\color{red} \textbf{uniformly continuous}} on $X$ if for all $\epsilon > 0$, there exists some $\delta > 0$ such that $d_Y\bigl(f\left(x\right), f\left(y\right)\bigr) < \epsilon$ whenever~$d_X\left(x, y\right) < \delta$.
\end{dfnbox}
Essentially, uniform continuity describes a phenomenon where the choice of $\delta$ is irrelevant to the point in the function's domain.

We wish to use the following proposition to characterise all uniformly continuous functions:
\begin{probox}{Uniform Continuity Characterisation}{characteriseUniformContinuity}
    Let $\left(X, d_X\right)$ and $\left(Y, d_Y\right)$ be metric spaces. A function $f \colon X \to Y$ is uniformly continuous if and only if for any sequences $\left\{x_i\right\}_i^{\infty}$ and $\left\{y_i\right\}_{i}^{\infty}$ in $X$ such that $\lim_{i \to \infty}d_X\left(x_i, y_i\right) = 0$, we have $\lim_{i \to \infty}d_Y\bigl(f\left(x_i\right), f\left(y_i\right)\bigr) = 0$.
    \tcblower
    \begin{proof}
        It suffices to prove the ``if'' direction only because the other direction is trivial from Definition \ref{dfn:uniformContinuity}. We shall consider the contrapositive statement. Suppose that there exist sequences $\left\{x_i\right\}_i^{\infty}$ and $\left\{y_i\right\}_{i}^{\infty}$ with $\lim_{i \to \infty}d_X\left(x_i, y_i\right) = 0$ such that~$\lim_{i \to \infty}d_Y\bigl(f\left(x_i\right), f\left(y_i\right)\bigr) \neq 0$, then for all $\delta > 0$, there exists some $N \in \N^+$ such that $d_X\left(x_i, y_i\right) > 0$ for all $i > N$. However, notice that there exists some $\epsilon > 0$ such that for all~$M \in \N^+$, there exists some $m > M$ with $d_Y\bigl(f\left(x_m\right), f\left(y_m\right)\bigr) \geq \epsilon$. This means that for any $\epsilon > 0$, we can find some $k$ such that for all $\delta > 0$, we have~$d_X\left(x_k, y_k\right) < \delta$ but $d_Y\bigl(f\left(x_k\right), f\left(y_k\right)\bigr) \geq \epsilon$. By Definition \ref{dfn:uniformContinuity}, this implies that $f$ is not uniformly continuous.
    \end{proof}
\end{probox}
Recall that previously, we have defined convergence for sequences. In particular, we can consider a sequence of functions. Note that, if we say that $\left\{f_n\right\}_{n = 1}^{\infty}$ converges to some function $f$, it might have two different interpretations. First, it might be the case where for any fixed $x$, we have $f_n\left(x\right)$ converging to $f\left(x\right)$; second it is also possible that the values of $f_n$ and $f$ becomes infinitesimally close when $n$ is large over all possible $x$. We will formalise this distinction as follows:
\begin{dfnbox}{Point-wise and Uniform Convergence}{pointwiseAndUniform}
    Let $\left\{f_n\right\}_{n = 1}^{\infty}$ be a sequence of maps from $X$ to a metric space $\left(Y, d\right)$. We say that $\left\{f_n\right\}_{n = 1}^{\infty}$ {\color{red} \textbf{converges point-wisely}} to $f \colon X \to Y$ if for any $x \in X$, $\lim_{n \to \infty}f_n\left(x\right) = f\left(x\right)$, and that $\left\{f_n\right\}_{n = 1}^{\infty}$ {\color{red} \textbf{converges uniformly}} to $f \colon X \to Y$ if for any $\epsilon > 0$, there exists some~$N \in \N^+$ such that for all $n \geq N$ and any $x \in X$, $d\bigl(f_n\left(x\right), f\left(x\right)\bigr) < \epsilon$.
\end{dfnbox}
The following proposition gives a test for continuity using convergence of functional sequences:
\begin{probox}{Uniform Convergence Implies Continuity}{uniformThenContinuous}
    Let $\left\{f_n\right\}_{n = 1}^{\infty}$ be a sequence of maps from a metric space $\left(X, d_X\right)$ to a metric space $\left(Y, d_Y\right)$. If $\left\{f_n\right\}_{n = 1}^{\infty}$ converges uniformly to $f \colon X \to Y$, then $f$ is continuous.
    \tcblower
    \begin{proof}
        Since $\left\{f_n\right\}_{n = 1}^{\infty}$ converges uniformly to $f$, for any $x \in X$ and any $\epsilon > 0$, there exists some $N \in \N^+$ such that $d_Y\bigl(f_n\left(x\right), f\left(x\right)\bigr) < \frac{\epsilon}{3}$ for all $n > N$. Since $f_n$ is continuous, for each $x_0 \in X$ there exists some $\delta > 0$ such that $d_Y\bigl(f_n\left(x_0\right), f_n\left(x\right)\bigr) < \frac{\epsilon}{3}$ whenever~$d_X\left(x_0, x\right) < \delta$. Notice that for all $x \in X$ such that $d_X\left(x_0, x\right) < \delta$, we have 
        \begin{align*}
            d_Y\bigl(f\left(x_0\right), f\left(x\right)\bigr) & \leq d_Y\bigl(f\left(x_0\right), f_n\left(x_0\right)\bigr)  + d_Y\bigl(f_n\left(x_0\right), f_n\left(x\right)\bigr) + d_Y\bigl(f_n\left(x\right), f\left(x\right)\bigr) \\
            & < \epsilon.
        \end{align*}
        Therefore, $f$ is continuous on $X$.
    \end{proof}
\end{probox}
In other words, a function is continuous if there exists some sequence of continuous functions which converges to it uniformly.
\section{Product Space}
Since topological spaces are essentially sets, we can consider the product of multiple spaces.
\begin{dfnbox}{Projection}{proj}
    Let $\left\{X_{\alpha}\right\}_{\alpha \in \Lambda}$ be a sequence of non-empty sets, consider the product space $\prod_{\alpha \in \Lambda}X_{\alpha}$. The {\color{red} \textbf{projection}} on the $\beta$-th factor is 
    \begin{equation*}
        \pi_{X_{\beta}} \colon \prod_{\alpha \in \Lambda}X_{\alpha} \to X_{\beta}
    \end{equation*}
    such that $\pi_{X_{\beta}}\left(\mathbfit{x}\right) = x_{\beta}$.
\end{dfnbox}
Observe that essentially, the projection maps a vector in the product space to some element in the space representing the $\beta$-th dimension.

Note that for any $\beta \in \Lambda$ and any $U \subseteq X_{\beta}$, we have
\begin{equation*}
    \pi_{X_{\beta}}^{-1}\left(U\right) = \left\{\mathbfit{x} \in \prod_{\alpha \in \Lambda}X_{\alpha} \colon x_{\beta} \in U\right\}.
\end{equation*}
In other words, the pre-image of projection for some set $U$ is the set of all vectors in the product space whose $\beta$-th component is in $U$.
\begin{dfnbox}{Product Topology}{prodTopo}
    Let $\left\{\left(X_{\alpha}, \mathcal{T}_{\alpha}\right)\right\}_{\alpha \in \Lambda}$ be a sequence of non-empty topological spaces. The {\color{red} \textbf{product topology}} is the topology generated by the sub-basis 
    \begin{equation*}
        \mathcal{S} \coloneqq \left\{\pi_{X_{\alpha}}^{-1}\left(U_{\alpha}\right) \colon \alpha \in \Lambda, U_{\alpha} \in \mathcal{T}_{\alpha}\right\}.
    \end{equation*}
\end{dfnbox}
Intuitively, the product topology is formed by all vectors whose every component is contained in some open set in one of the topological spaces. This might seem similar to collecting all products formed taking one arbitrary open set from each space and multiplying them together.
\begin{dfnbox}{Box Topology}{box}
    Let $\left\{\left(X_{\alpha}, \mathcal{T}_{\alpha}\right)\right\}_{\alpha \in \Lambda}$ be a sequence of non-empty topological spaces. The {\color{red} \textbf{box topology}} is the topology generated by the basis 
    \begin{equation*}
        \mathcal{B} \coloneqq \left\{\prod_{\alpha \in \Lambda}U_{\alpha} \colon U_{\alpha} \in \mathcal{T}_{\alpha}\right\}.
    \end{equation*}
\end{dfnbox}
\begin{notebox}
    \begin{remark}
        For finite product spaces, the product topology and the box topology are the same. However, they are different if the product is infinite.
    \end{remark}
\end{notebox}
We may check that the set $\mathcal{B}$ in Definition \ref{dfn:box} is indeed a basis. Take any $\mathcal{x} \in \prod_{\alpha \in \Lambda}X_{\alpha}$, then for each $\alpha \in \Lambda$, there exists some $U_{\alpha} \in \mathcal{T}_{\alpha}$ such that $x_{\alpha} \in U_{\alpha}$, which implies that 
\begin{equation*}
    \mathbfit{x} \in \prod_{\alpha \in \Lambda}U_{\alpha}.
\end{equation*}
If $V \coloneqq \prod_{\alpha \in \Lambda}U_{\alpha}$ and $U \coloneqq \prod_{\alpha \in \Lambda}U_{\alpha}$ are such that $\mathbfit{x} \in U \cap V$, then $x_{\alpha} \in U_{\alpha} \cap V_{\alpha}$ for each~$\alpha \in \Lambda$. Clearly, $U_{\alpha} \cap V_{\alpha} \in \mathcal{T}_{\alpha}$, so we can take $W \coloneqq \prod_{\alpha \in \Lambda}U_{\alpha} \cap V_{\alpha}$ such that 
\begin{equation*}
    \mathbfit{x} \in W \subseteq V \cap U.
\end{equation*}
We now make a claim that the projection from product topology is always continuous.
\begin{probox}{Product Topology Guarantees Continuous Projection}{continuousProj}
    Let $\left\{\left(X_{\alpha}, \mathcal{T}_{\alpha}\right)\right\}_{\alpha \in \Lambda}$ be a sequence of non-empty topological spaces. The product topology on $\prod_{\alpha \in \Lambda}X_{\alpha}$ is the coarsest topology such that $\pi_{X_{\alpha}}$ is continuous for all $\alpha \in \Lambda$.
    \tcblower
    \begin{proof}
        Let $\mathcal{S}$ be the sub-basis generating the product topology and let $U \subseteq X_{\alpha}$ be open, then $\pi^{-1}_{X_{\alpha}}\left(U\right) \in \mathcal{S}$ and therefore is open. Therefore, $\pi_{X_{\alpha}}$ is continuous with respect to the product topology.
        \\\\
        Let $\mathcal{T}$ be any topology on $\prod_{\alpha \in \Lambda}X_{\alpha}$ such that $\pi_{X_{\alpha}}$ is continuous, then for any open set $U \in \mathcal{T}$, the pre-image $\pi^{-1}_{X_{\alpha}}\left(U\right) \in \mathcal{S}$ must be open, which implies that $\mathcal{S} \subseteq \mathcal{T}$. Notice that this is equivalent to $\mathcal{T}$ containing 
        \begin{equation*}
            \mathcal{B} \coloneqq \left\{\bigcap_{\alpha \in \Lambda}\pi^{-1}_{X_{\alpha}}\left(U_{\alpha}\right) \colon \Lambda' \subseteq \Lambda \textrm{ is finite}, U_{\alpha} \in \mathcal{T}_{\alpha}\right\}.
        \end{equation*}
        However, $\mathcal{B}$ generates the product topology, which is the coarsest topology containing~$\mathcal{B}$, so $\mathcal{T}$ must be finer than the product topology.
    \end{proof}
\end{probox}
Analogously, we can construct continuous functions over the product space using continuous functions defined on each of the individual spaces.
\begin{probox}{Continuous of Functions over Product Spaces}{continuousOverProd}
    Let $\left\{\left(X_{\alpha}, \mathcal{T}_{\alpha}\right)\right\}_{\alpha \in \Lambda}$ be a sequence of non-empty topological spaces and let $Y$ be a topological space. For any $\alpha \in \Lambda$, define $f_{\alpha} \colon Y \to X_{\alpha}$, then $f \colon Y \to \prod_{\alpha \in \Lambda}X_{\alpha}$ defined by 
    \begin{equation*}
        f\left(y\right) = \bigl(f_{\alpha}\left(y\right)\bigr)_{\alpha \in \Lambda}
    \end{equation*}
    is continuous if and only if $f_{\alpha}$ is continuous for all $\alpha \in \Lambda$.
    \tcblower
    \begin{proof}
        Suppose that $f$ is continuous. Notice that for all $\alpha \in \Lambda$,
        \begin{equation*}
            f_{\alpha}\left(y\right) = \pi_{X_{\alpha}}\bigl(f\left(y\right)\bigr)
        \end{equation*}
        for all $y \in Y$. Therefore, $f_{\alpha} = \pi_{X_{\alpha}} \circ f$. By Proposition \ref{pro:continuousProj}, $\pi_{X_{\alpha}}$ is continuous, and so $f_{\alpha}$ is continuous for all $\alpha \in \Lambda$.
        \\\\
        Conversely, suppose that $f_{\alpha}$ is continuous for all $\alpha \in \Lambda$. Notice that the basis for the product topology is given by 
        \begin{equation*}
            \mathcal{B} \coloneqq \left\{\bigcap_{\alpha \in \Lambda}\pi^{-1}_{X_{\alpha}}\left(U_{\alpha}\right) \colon \Lambda' \subseteq \Lambda \textrm{ is finite}, U_{\alpha} \in \mathcal{T}_{\alpha}\right\}.
        \end{equation*}
        For any finite $\Lambda' \subseteq \Lambda$, we have 
        \begin{align*}
            f^{-1}\left(\bigcap_{\alpha \in \Lambda}\pi^{-1}_{X_{\alpha}}\left(U_{\alpha}\right)\right) & = \bigcap_{\alpha \in \Lambda}f^{-1}\left(\pi^{-1}_{X_{\alpha}}\left(U_{\alpha}\right)\right) \\
            & = \bigcap_{\alpha \in \Lambda}\left(\pi_{X_{\alpha}} \circ f\right)^{-1}\left(U_{\alpha}\right) \\
            & =  \bigcap_{\alpha \in \Lambda}f_{\alpha}^{-1}\left(U_{\alpha}\right),
        \end{align*}
        which is open. Therefore, $f$ is continuous.
    \end{proof}
\end{probox}
Notice that it is important to equip the product space with the product topology. Let $f \colon \R \to \R^{\N}$ such that $f\left(x\right) = \left(x, x, \cdots\right)$. We claim that $f$ is not continuous with respect to the box topology.

The following result should be common sense enough.
\begin{corbox}{Arithmetic of Continuous Functions}{arithmetic}
    Let $X$ be a topological space and let $f, g \colon X \to \R$ be continuous functions, then the functions $f + g$, $f - g$ and $fg$ are continuous. If $0 \notin g\left(X\right)$, then $\frac{f}{g}$ is continuous.
\end{corbox}
Intuitively, the subspace topology on product space should be the product of subspace topologies.
\begin{probox}{Subspace Topology on Product Spaces}{prodSubTopo}
    Let $\left(X, \mathcal{T}_X\right)$ and $\left(Y, \mathcal{T}_Y\right)$ be topological spaces. Let $A \subseteq X$, $B \subseteq Y$ and $\mathcal{T}_{A \times B}$ be the subspace topology on $A \times B$ induced by the product topology. If $\mathcal{T}_{A \times B}'$ be the product topology induced by the subspace topologies on $A$ and $B$, then $\mathcal{T}_{A \times B} = \mathcal{T}_{A \times B}'$.
    \tcblower
    \begin{proof}
        Let $\mathcal{T}_A$ and $\mathcal{T}_B$ be the subspace topologies on $A$ and $B$ respectively, then $\mathcal{T}_{A \times B}'$ is generated by the basis 
        \begin{equation*}
            \mathcal{B}' \coloneqq \left\{T_A \times T_B \colon T_A \in \mathcal{T}_A, T_B \in \mathcal{T}_B\right\}.
        \end{equation*}
        Let $\mathcal{T}_{X \times Y}$ be the product topology for $X \times Y$, then $\mathcal{T}_{X \times Y}$ is generated by 
        \begin{equation*}
            \mathcal{B}_{X \times Y} \coloneqq \left\{U_X \times U_Y \colon U_X \in \mathcal{T}_X, U_Y \in \mathcal{T}_Y\right\}.
        \end{equation*}
        Therefore, by Proposition \ref{pro:subspaceBasis}, $\mathcal{T}_{A \times B}$ is generated by 
        \begin{equation*}
            \mathcal{B} \coloneqq \left\{\left(U_X \times U_Y\right) \cap \left(A \times B\right) \colon U_X \in \mathcal{T}_X, U_Y \in \mathcal{T}_Y\right\}.
        \end{equation*}
        It is clear that $\mathcal{B} = \mathcal{B}'$ (???).
    \end{proof}
\end{probox}
\begin{probox}{Standard Topology Decomposition on Euclidean Spaces}{decompose}
    Let $n \in \Z^+$ and $n = \sum_{i = 1}^km_i$ with $m_i \in \Z^+$ for $i = 1, 2, \cdots, k$. Let $\mathcal{T}_i$ be the standard topology on $\R^{m_i}$ and $\mathcal{T}$ be the product topology induced by the $\mathcal{T}_i$'s, then $\mathcal{T}$ is the standard topology on $\R^n$.
    \tcblower
    \begin{proof}
        Let $\mathcal{B}_n$ be the standard basis generating the standard topology $\mathcal{T}_n$ over $\R^n$ for each $n \in \N^+$. Take any $B_r^{\left(n\right)}\left(\mathbfit{x}\right) \in \mathcal{B}_n$. For any $\mathbfit{y} \in B_r^{\left(n\right)}\left(\mathbfit{x}\right)$, define $t = r - \norm{\mathbfit{x - y}}$ and consider, for each $i = 1, 2, \cdots, k$,
        \begin{equation*}
            B_t^{\left(m_i\right)}\left(\pi_{\R^{m_i}}\left(\mathbfit{y}\right)\right) \in \mathcal{B}_{m_i} \subseteq \mathcal{T}_i
        \end{equation*}
        to be an open ball centred at the projection of $\mathbfit{y}$ to the $m_i$-th factor. Denote 
        \begin{equation*}
            B_t \coloneqq \prod_{i = 1}^{k}B_t^{\left(m_i\right)}\left(\pi_{\R^{m_i}}\left(\mathbfit{y}\right)\right).
        \end{equation*}
        Clearly, $B_t$ is contained by the basis for the product topology induced by the $\mathcal{T}_i$'s. For any $\mathbfit{p} \in B_t$, we have 
        \begin{equation*}
            \norm{\mathbfit{p - x}} \leq \norm{\mathbfit{p - y}} + \norm{\mathbfit{x - y}} < r.
        \end{equation*} 
        Therefore, $\mathbfit{y} \in B_t \subseteq B_r^{\left(n\right)}\left(\mathbfit{x}\right)$. By Proposition \ref{pro:fineBasis}, $\mathcal{T} \subseteq \mathcal{T}_n$. Conversely, take an arbitrary $U_i \in \mathcal{T}_i$ for each $i = 1, 2, \cdots, k$. Notice that for any $\mathbfit{x} \in \prod_{i = 1}^kU_i$, there is some open ball $B_{r_i}^{\left(m_i\right)}\left(\pi_{\R^{m_i}}\left(\mathbfit{x}\right)\right) \in \mathcal{B}_{m_i}$, centred at the projection of $\mathbfit{x}$ to the $m_i$-th factor, such that $B_{r_i}^{\left(m_i\right)}\left(\pi_{\R^{m_i}}\left(\mathbfit{x}\right)\right) \subseteq U_i$. Take
        \begin{equation*}
            r \coloneqq \min_{1 \leq i \leq k}r_i
        \end{equation*}
        and consider the open ball $B_r^{\left(n\right)}\left(\mathbfit{x}\right) \in \mathcal{B}_n$. For any $\mathbfit{y} \in B_r^{\left(n\right)}\left(\mathbfit{x}\right)$, consider 
        \begin{align*}
            \norm{\mathbfit{y - x}} & = \sum_{i = 1}^{k}\norm{\pi_{\R^{m_i}}\left(\mathbfit{y}\right) - \pi_{\R^{m_i}}\left(\mathbfit{x}\right)} < r \leq \min_{1 \leq i \leq k}r_i.
        \end{align*}
        Therefore, for every positive integer $i \leq k$, we have 
        \begin{equation*}
            \norm{\pi_{\R^{m_i}}\left(\mathbfit{y}\right) - \pi_{\R^{m_i}}\left(\mathbfit{x}\right)} < r_i,
        \end{equation*}
        which implies that $\pi_{\R^{m_i}}\left(\mathbfit{y}\right) \in B_{r_i}^{\left(m_i\right)}\left(\pi_{\R^{m_i}}\left(\mathbfit{x}\right)\right) \subseteq U_i$. Therefore, $\mathbfit{y} \in \prod_{i = 1}^{k}U_i$ and so $\mathcal{T}_n \subseteq \mathcal{T}$ by Proposition \ref{pro:fineBasis}. Therefore, $\mathcal{T} = \mathcal{T}_n$.
    \end{proof}
\end{probox}
\begin{dfnbox}{$L^p$-Metric}{Lp}
    Let $\left(X_1, d_{X_1}\right), \left(X_2, d_{X_2}\right), \cdots, \left(X_n, d_{X_n}\right)$ be metric spaces, we define 
    \begin{align*}
        d_1\bigl(\left(x_1, x_2, \cdots, x_n\right), \left(y_1, y_2, \cdots, y_n\right)\bigr) & \coloneqq \sum_{i = 1}^{n}d_{X_i}\left(x_i, y_i\right) \\
        d_{\infty}\bigl(\left(x_1, x_2, \cdots, x_n\right), \left(y_1, y_2, \cdots, y_n\right)\bigr) & \coloneqq \max_{1 \leq i \leq n}d_{X_i}\left(x_i, y_i\right).
    \end{align*}
\end{dfnbox}
\begin{probox}{Basis for Product Topology}{prodTopoBasis}
    Let $\left\{\left(X_i, \mathcal{T}_i\right)\right\}_{i = 1}^n$ be a sequence of topological spaces and let $\mathcal{B}_i$ be the basis for $\mathcal{T}_i$ for all~$i = 1, 2, \cdots, n$, then $\prod_{i = 1}^{n}\mathcal{B}_i$ is a basis for the product topology over $\prod_{i = 1}^nX_i$.
\end{probox}
\begin{probox}{Metric for Product Topology}{prodTopoMetric}
    Let $\left\{\left(X_i, d_{X_i}\right)\right\}_{i = 1}^n$ be a sequence of topological spaces, then $d_1$ and $d_{\infty}$ both induce the product topology over $\prod_{i = 1}^nX_i$.
\end{probox}
\begin{probox}{$\rho$-Metric Induces the Same Topology}{rho}
    Let $\left(X, d\right)$ be a metric space and define $\rho \colon X \times X \to \R$ by 
    \begin{equation*}
        \rho\left(x, y\right) \coloneqq \frac{d\left(x, y\right)}{1 + d\left(x, y\right)},
    \end{equation*}
    then $\rho$ is a metric on $X$ and $\mathrm{diam}_{\rho}\left(X\right) < 1$. Let $\mathcal{T}_{\rho}$ and $\mathcal{T}_d$ be the topologies on $X$ induced by $\rho$ and $d$ respectively, then $\mathcal{T}_{\rho} = \mathcal{T}_d$.
    \tcblower
    \begin{proof}
        We first show that $\rho$ is a metric. The non-negativity and symmetry are obvious, so it suffices to prove the triangle inequality. Consider 
        \begin{equation*}
            f\left(t\right) = \frac{t}{1 + t}.
        \end{equation*}
        One may check that $f$ is an increasing concave function with $f\left(0\right) = 0$. Therefore,
        \begin{equation*}
            \frac{b}{a + b}f\left(a + b\right) = \frac{a}{a + b}f\left(0\right) + \frac{b}{a + b}f\left(a + b\right) \leq f\left(b\right).
        \end{equation*}
        By symmetry, $\frac{a}{a + b}f\left(a + b\right) \leq f\left(a\right)$, and so $f\left(a\right) + f\left(b\right) \geq f\left(a + b\right)$. Therefore, for any $x, y, z \in X$, 
        \begin{align*}
            \rho\left(x, y\right) + \rho\left(y, z\right) & = f\bigl(d\left(x, y\right)\bigr) + f\bigl(d\left(y, z\right)\bigr) \\
            & \geq f\bigl(d\left(x, y\right) + d\left(y, z\right)\bigr) \\
            & \geq f\bigl(d\left(x, z\right)\bigr) \\
            & = \rho\left(x, z\right).
        \end{align*}
        It is clear that $\mathrm{diam}_{\rho}\left(X\right) = \max_{\left(x, y\right) \in X^2}\rho\left(x, y\right) < 1$. Notice that 
        \begin{align*}
            2\rho\left(x, y\right) & = \frac{2d\left(x, y\right)}{1 + d\left(x, y\right)} \\
            & > d\left(x, y\right) \\
            & > \rho\left(x, y\right)
        \end{align*}
        for all $x, y \in X$. Therefore, $\mathcal{T}_d = \mathcal{T}_{\rho}$.
    \end{proof}
\end{probox}
\begin{probox}{Product Topology on Infinite Product}{infProdTopo}
    Let $\left\{\left(X, d_{X_i}\right)\right\}_{i = 1}^{\infty}$ be a sequence of metric spaces and define $\rho_i \colon X_i \times X_i \to \R$ by 
    \begin{equation*}
        \rho_i\left(x, y\right) \coloneqq \frac{d_{X_i}\left(x, y\right)}{1 + d_{X_i}\left(x, y\right)},
    \end{equation*}
    then $d \colon \prod_{i = 1}^{\infty}X_i \times \prod_{i = 1}^{\infty}X_i \to \R$ defined by 
    \begin{equation*}
        d\left(\mathbfit{x}, \mathbfit{y}\right) \coloneqq \sup \left\{\frac{\rho_{i}\left(x_i, y_i\right)}{i} \colon i \in \Z^+\right\}
    \end{equation*}
    is a metric inducing the product topology on $\prod_{i = 1}^{\infty}X_i$.
    \tcblower
    \begin{proof}
        Define $\rho_i' = \frac{1}{i}\rho_i$. Let $\mathcal{B}_d$ be the basis induced by the metric $d$ and $\mathcal{B}_i$ be the basis induced by the metric $\rho_i'$. Take any $B_r^d\left(\mathbfit{x}\right) \in \mathcal{B}_d$. For any $\mathbfit{y} \in B_r^d\left(\mathbfit{x}\right)$, we have 
        \begin{equation*}
            \frac{\rho_i\left(x_i, y_i\right)}{i} \leq d\left(\mathbfit{x}, \mathbfit{y}\right) < r
        \end{equation*}
        for all $i \in \Z^+$. Therefore, $\mathbfit{y} \in B_r^{\rho_i'}$ and so $B_r^d\left(\mathbfit{x}\right) \subseteq \prod_{i = 1}^{\infty}B_r^{\rho_i'}$
        For any $\mathbfit{x} \in \prod_{i = 1}^{\infty}X_i$, Notice that for any $r > 0$, there exists some $N \in \N^+$ such that $\frac{1}{i} < \frac{r}{2}$ for all $i \geq N$. By Proposition \ref{pro:rho}, $\mathrm{diam}\left(\rho_i\right) < 1$ for all $i \in \N^+$, so 
        \begin{equation*}
            \mathrm{diam}\left(\frac{1}{i}\rho_i\right) = \frac{1}{i}\mathrm{diam}\left(\rho_i\right) < r
        \end{equation*}
        whenever $i \geq N$.  Therefore, $B_r^{\rho_i'}\left(\mathbfit{x}_i\right) = X_i$ for any $\mathbfit{x}_i \in X_i$. This means that for any $\mathbfit{x} \in \prod_{i = 1}^{\infty}X_i$, we can write 
        \begin{equation*}
            B_r^d\left(\mathbfit{x}\right) = \prod_{i = 1}^{N - 1}B_r
        \end{equation*}
    \end{proof}
\end{probox}
\section{Quotient Topological Spaces}
\begin{dfnbox}{Quotient Map}{quotient}
    Let $\left(X, \mathcal{T}_X\right)$ and $\left(Y, \mathcal{T}_Y\right)$ be topological spaces. A surjective map $p \colon X \to Y$ is a {\color{red} \textbf{quotient map}} if for all $V \subseteq Y$, $V \in \mathcal{T}_Y$ if and only if $p^{-1}\left(V\right) \subseteq X$ is open in $X$.
\end{dfnbox}
\begin{dfnbox}{Open and Closed Maps}{openAndClosed}
    A continuous map $f \colon X \to Y$ is {\color{red} \textbf{open}} if $f\left(U\right)$ is open in $Y$ for any open set $U \subseteq X$, and {\color{red} \textbf{closed}} if $f\left(V\right)$ is closed in $Y$ for any closed set $V \subseteq X$.
\end{dfnbox}
\begin{notebox}
    \begin{remark}
        If a surjective continuous map is open or closed, then it is a quotient map.
    \end{remark}
\end{notebox}
\begin{notebox}
    \begin{remark}
        Quotient map, open map and closed map are all preserved under composition.
    \end{remark}
\end{notebox}
\begin{dfnbox}{Saturated Set}{saturate}
    Let $f \colon X \to Y$ be a surjective continuous map. A set $A \subseteq X$ is {\color{red} \textbf{saturated}} with respect to $f$ if $A = f^{-1}\left(S\right)$ for some $S \subseteq Y$.
\end{dfnbox}
Equivalently, this means that $A = f^{-1}\bigl(f\left(A\right)\bigr)$.
\begin{probox}{Equivalent Definition for Quotient Maps}{equivQuotient}
    Let $f \colon X \to Y$ be a surjective continuous map, then $f$ is a quotient map if and only if $f\left(A\right)$ is open (closed) in $Y$ whenever $A$ is a saturated open (closed) set with respect to $f$ in $X$.
    \tcblower
    \begin{proof}
        Take any saturated open set $A \subseteq X$, then $A = f^{-1}\bigl(f\left(A\right)\bigr)$. By definition \ref{dfn:quotient}, since $f^{-1}\bigl(f\left(A\right)\bigr)$ is open, $f\left(A\right)$ must be open in $Y$. Conversely, suppose that $f\left(A\right)$ is open in $Y$ for all saturated open set $A \subseteq X$. Let $U \subseteq Y$ such that $f^{-1}\left(U\right)$ is open in~$X$. Notice that $f^{-1}\left(U\right)$ is saturated. Therefore, $f\bigl(f^{-1}\left(U\right)\bigr) = U$ is open in $Y$. Since $f$ is surjective, it is a quotient map.
    \end{proof}
\end{probox}
\begin{probox}{Restriction of Quotient Map to Saturated Open Sets}{restrictQuotient}
    If $f \colon X \to Y$ is a continuous quotient map and $A \subseteq X$ is a saturated open (closed) set with respect to $f$, then $\left.f\right\rvert_A \colon A \to f\left(A\right)$ is a quotient map.
    \tcblower
    \begin{proof}
        Let $B \subseteq A$ be open and saturated with respect to $\left.f\right\rvert_A$, then $B = \left.f\right\rvert_A^{-1}\bigl(\left.f\right\rvert_A\left(B\right)\bigr)$. Note that 
        \begin{equation*}
            \left.f\right\rvert_A\left(B\right) = f\left(B \cap A\right) = f\left(B\right)
        \end{equation*}
        for any $B \subseteq A$. Since $B$ is open,by Proposition \ref{pro:equivQuotient}, $f\left(B\right)$ is open. This implies that $\left.f\right\rvert_A$ is a quotient map.
    \end{proof}
\end{probox}
\begin{probox}{Existence and Uniqueness of Quotient Topology}{exist!quotientTopo}
    Let $p \colon X \to A$ be surjective for some $A \subseteq X$, then there exists a unique topology $\mathcal{T}_A$ such that $p$ is a quotient map.
    \tcblower
    \begin{proof}
        Define 
        \begin{equation*}
            \mathcal{T}_A \coloneqq \left\{U \subseteq A \colon p^{-1}\left(U\right) \subseteq X \textrm{ is open}\right\}.
        \end{equation*}
        Observe that $\varnothing, A \in \mathcal{T}_A$. Note that for any index set $\Lambda$ such that $U_{\alpha} \in \mathcal{T}_A$ for all $\alpha \in \Lambda$,
        \begin{equation*}
            p^{-1}\left(\bigcup_{\alpha \in \Lambda}U_{\alpha}\right) = \bigcup_{\alpha \in \Lambda}p^{-1}\left(U_{\alpha}\right)
        \end{equation*}
        is open is $X$, so $\bigcup_{\alpha \in \Lambda}U_{\alpha} \in \mathcal{T}_A$. Similarly, one may check that $\bigcup_{i = 1}^nU_{i} \in \mathcal{T}_A$ for any $n \in \N^+$ such that $U_i \in \mathcal{T}_A$ for all $i = 1, 2, \cdots, n$. Let $\mathcal{T}'$ be any topology on $A$ such that $p$ is a quotient map. For any $U \subseteq A$, notice that $U \in \mathcal{T}'$ if and only if $p^{-1}\left(U\right) \subseteq X$ is open, i.e., $U \in \mathcal{T}'$ if and only if $U \in \mathcal{T}_A$. Therefore, $\mathcal{T}' = \mathcal{T}_A$, which means that $\mathcal{T}_A$ is unique.
    \end{proof}
\end{probox}
\begin{dfnbox}{Quotient Space}{quotientSpace}
    Let $X$ be a topological space and $X^*$ be a partition of $X$. Let $p \colon X \to X^*$ be a surjective map such that $x \in p\left(x\right)$. If $\mathcal{T}^*$ is the quotient topology on $X^*$, then $\left(X^*, \mathcal{T}^*\right)$ is the {\color{red} \textbf{quotient space}} of $X$.
\end{dfnbox}
\chapter{Characterisation of Topological Spaces}
In the previous chapter, we have discussed various fundamental concepts and properties of general topological spaces. In this chapter, the main focus is to characterise a number of topological spaces and categorise them into different types.
\section{Hausdorff Spaces}
\begin{dfnbox}{$T_1$ and $T_2$ Spaces}{T1T2}
    Let $X$ be a topological space. $X$ is {\color{red} \textbf{$T_1$}} if for any $x, y \in X$ with $x \neq y$, there exists an open set $U \subseteq X$ such that $x \in U$ but $y \notin U$. $X$ is {\color{red} \textbf{$T_2$}} or {\color{red} \textbf{Hausdorff}} if for any $x, y \in X$ with $x \neq y$, there exist open neighbourhoods $U, V \subseteq X$ with $x \in U$ and $y \in V$ such that $U \cap V = \varnothing$.
\end{dfnbox}
We can check that every metric space is Hausdorff, as for any $x \neq y$, we can always take 
\begin{equation*}
    r \coloneqq \frac{1}{3}d\left(x, y\right),
\end{equation*}
and so the $r$-neighbourhoods of $x$ and $y$ will always be disjoint. It can also be easily checked that the discrete topology of any set is Hausdorff. 

Next, we prove that the co-finite topology is always $T_1$.
\begin{probox}{Co-finite Topology Is $T_1$}{T1Cofinite}
    Let $X$ be a topological space with the co-finite topology $\mathcal{T}$, then $X$ is $T_1$. If $X$ is finite, then~$X$ is Hausdorff.
\end{probox}
Intuitively, if we fix any point $x$ in a $T_1$ space, then we can find some open sets to cover every point in the space other than $x$, which means that the isolated singleton $\left\{x\right\}$ is closed.
\begin{probox}{Characterisation of $T_1$ Spaces}{T1Space}
    A topological space $X$ is $T_1$ if and only if for all $x \in X$, the singleton set $\left\{x\right\}$ is closed.
    \tcblower
    \begin{proof}
        Suppose that $X$ is $T_1$ and take any $x \in X$. Let $y \in X \setminus \left\{x\right\}$ be any point. Notice that there exists some open set $V_y \subseteq X$ such that $y \in V_y$ but $x \notin V_y$. Clearly, 
        \begin{equation*}
            X \setminus \left\{x\right\} = \bigcup_{y \in X \setminus \left\{x\right\}}V_y
        \end{equation*}
        is open, and so $\left\{x\right\}$ is closed. Conversely, suppose that for all $x \in X$, the set $\left\{x\right\}$ is closed. For any $x \in X$, take any $y \in X$ with $x \neq y$. Notice that $y \in X \setminus \left\{x\right\}$, which is open. Therefore, $X$ is $T_1$.
    \end{proof}
\end{probox}
Notice that by Definition \ref{dfn:T1T2}, all $T_2$ spaces are $T_1$, so the following corollary is easy to show:
\begin{corbox}{Finite Sets in Metric Spaces Are Closed}{finiteSetInMetric}
    Let $X$ be a metric space, then all finite sets in $X$ are closed.
\end{corbox}
\section{First Countable Spaces}
Consider a random point $x$ in some topological space. Suppose that $U$ is an open set containing $x$. Intuitively, we can reduce $U$ by taking an open subset of $U$ containing $x$. We do this repeatedly until we find some minimal open subset of $U$ containing $x$. Now, we can perform this operation to every open set containing $x$ in the space and collect all these minimal open subsets. The countability of this collection offers a different criterion of space categorisation.
\begin{dfnbox}{Countable Basis}{countableBasis}
    Let $X$ be a topological space. For all $x \in X$, a {\color{red} \textbf{countable basis}} of $x$ is a countable collection $\mathcal{B}$ of open sets in $X$ containing $x$ such that for any open set $Y \subseteq X$ containing~$x$, there exists some $B \in \mathcal{B}$ with $B \subseteq Y$. 
\end{dfnbox}
Now we define \textit{first countable spaces} as follows:
\begin{dfnbox}{First Countable Space}{1stCountable}
    A topological space $X$ is {\color{red} \textbf{first countable}} if every $x \in X$ has a countable basis.
\end{dfnbox}
We claim that all metric spaces are first countable. Let $X$ be a metric space and take any $x \in X$, then we can take its countable basis as 
\begin{equation*}
    \mathcal{B} \coloneqq \left\{B_{\frac{1}{i}}\left(x\right) \colon i \in \Z^+\right\}.
\end{equation*}
Consider some uncountable set $Y$ equipped with the co-finite topology. For any $y \in Y$, suppose it has a countable basis $\mathcal{B}$. Note that in the co-finite topology, a set is open if and only if its complement is finite. Therefore, there exists some finite set $F_i \subseteq Y$ such that there exists some $B_i \in \mathcal{B}$ such that $Y \setminus F_i = B_i$. Notice that 
\begin{equation*}
    Y \setminus \left(\left\{y\right\} \cup \bigcup_{i = 1}^{\infty}F_i\right) \neq \varnothing.
\end{equation*}
Take some $z \in Y \setminus \left(\left\{y\right\} \cup \bigcup_{i = 1}^{\infty}F_i\right)$, then $U \coloneqq Y \setminus \left\{z\right\}$ is open such that $y \in U$. Note that for all $B_i \in \mathcal{B}$, we have $z \in B_i$. This means that $B_i \not\subseteq U$ for all $i \in \Z^+$ because otherwise $y \in U$ which is not possible. Therefore, $\mathcal{B}$ is not a countable basis.

Suppose that $x$ is a random point in a topological space with a countable basis $\mathcal{B}$. Fix any~$B \in \mathcal{B}$, then $x \in B$. Notice that for any open set $U$ containing $x$, it is not possible that $U$ and $B$ are disjoint. Therefore, there exists some open subset of $B$ which contains $x$. This motivates the following natural question: can we always re-construct the countable basis such that for any pair of sets in the basis, one must be a subset of the other?
\begin{probox}{Construction of Countable Basis}{constructCountable}
    Let $x \in X$ be a point with a countable basis $\mathcal{B}$, then there exists some countable basis $\mathcal{B}'$ of $x$ such that for any $B_1, B_2 \in \mathcal{B}'$, either $B_1 \subseteq B_2$ or $B_2 \subseteq B_1$.
    \tcblower
    \begin{proof}
        Take any $B_1, B_2 \in \mathcal{B}$ such that $B_1 \not\subseteq B_2$ and $B_2 \not\subseteq B_1$, then $B_1 \cap B_2$ is open and $x \in B_1 \cap B_2$. Therefore, for any open set $U$ containing $B_2$, clearly $B_1 \cap B_2 \subseteq U$. Therefore, we can repeat this process until we obtain a countable basis $\mathcal{B}'$ such that for any $B_1', B_2' \in \mathcal{B}'$, we have either $B_1' \subseteq B_2'$ or $B_2' \subseteq B_1'$.
    \end{proof}
\end{probox}
Based on this small proposition, we introduce the following result to help us identify spaces which are not first countable:
\begin{probox}{Closure Characterisation of First Countable Spaces}{characterise1stCountableClosure}
    Let $X$ be a topological space and $A \subseteq X$. If there exists a sequence $\left\{x_i\right\}_{i \in \N^+} \subseteq A$ such that $x_i \to x$, then $x \in \widebar{A}$. The converse is true if $X$ is first countable.
    \tcblower
    \begin{proof}
        Suppose that there exists a sequence $\left\{x_i\right\}_{i \in \N^+} \subseteq A$ such that $x_i \to x$. Notice that $\widebar{A} = A \cup A'$, then it suffices to prove that $x \notin A$ implies $x \in A'$. Suppose that~$x \notin A$. Note that for any open set $U \subseteq X$ such that $x \in U$, there exists some $N \in \N$ such that $x_n \in U$ for all $n \geq N$. In particular, this means that~$x_N \in U \cap \left(A \setminus \left\{x\right\}\right)$. Therefore, $x$ is a limit point of $A$.
        \\\\
        Conversely, suppose that $x \in \widebar{A}$. Since $X$ is first countable, $x$ has some countable basis $\mathcal{B} \coloneqq \left\{B_i \colon i \in \N^+\right\}$. By Proposition \ref{pro:constructCountable}, we can assume that $B_j \subseteq B_i$ whenever $j \geq i$. For each $i \in \N^+$, take 
        \begin{equation*}
            x_i \in \bigcap_{j = 1}^{i}\left(B_j \cap A\right),
        \end{equation*}
        then $\left\{x_i\right\}_{i = \N^+}$ is a sequence in $A$. Let $U$ be any open set containing $x$, then there exists some $B_N \in \mathcal{B}$ such that $x \in B_N \subseteq U$. However, notice that for all $i \geq N$, we have~$x_i \in B_N \subseteq U$, and so $x_i \to x$.
    \end{proof}
\end{probox}
An alternative proposition makes use of continuous maps:
\begin{probox}{Continuous Map Characterisation of First Countable Spaces}{characterise1stCountableContinuous}
    Let $X$ be a topological space. If $f \colon X \to Y$ is continuous, then for any sequence $\left\{x_i\right\}_{i \in \N^+}$ with $x_i \to x$, then $f\left(x_i\right) \to f\left(x\right)$. The converse is true if $X$ is first countable.
    \tcblower
    \begin{proof}
        Suppose that $f \colon X \to Y$ is continuous. Let $U \subseteq Y$ be an open set such that $f\left(x\right) \in U$, then $x \in f^{-1}\left(U\right)$, which is open. Note that there exists some $N \in \N^+$ such that $x_i \in f^{-1}\left(U\right)$ for all $i \geq N$, i.e., $f\left(x_i\right) \in U$ for all $i \geq N$. Therefore, $f\left(x_i\right) \to f\left(x\right)$.
        \\\\
        Conversely, suppose that for any sequence $\left\{x_i\right\}_{i \in \N^+}$ with $x_i \to x$, then $f\left(x_i\right) \to f\left(x\right)$. Take any $A \subseteq X$ with $x \in \widebar{A}$. Since $X$ is first countable, by Proposition \ref{pro:characterise1stCountableClosure}, there exists a sequence $\left\{y_i\right\}_{i \in \N^+}$ in $A$ such that $y_i \to x$. This means that $f\left(y_i\right) \to f\left(x\right)$. Since $f\left(x_i\right) \in f\left(A\right)$, this means that $f\left(x\right) \in \widebar{f\left(A\right)}$, and so $f\left(\widebar{A}\right) \subseteq \widebar{f\left(A\right)}$. By Proposition \ref{pro:continuousMapProps}, $f$ is continuous.
    \end{proof}
\end{probox}
\section{Compactness}
We next introduce the notion of compact spaces, which are an attempt to generalise the notion of \textbf{closed and bounded} subsets in Euclidean spaces. The idea here is to define compactness as ``having no missing end point''.

First, notice that given any topological space, we can always partition the space into a collection of open sets in the space.
\begin{dfnbox}{Open Cover}{openCover}
    Let $X$ be a topological space. An {\color{red} \textbf{open cover}} is a collection of open sets $\left\{U_{\alpha}\right\}_{\alpha \in \Lambda}$ such that 
    \begin{equation*}
        \bigcup_{\alpha \in \Lambda}U_{\alpha} = X.
    \end{equation*}
\end{dfnbox}
The intuition here is that if a space is not compact, i.e., it can extend indefinitely without an end, then there is a way to partition the space into infinitely many open subsets. On the other hand, if the space has a clearly defined boundary, then every possible partition must be finite.
\begin{dfnbox}{Compact Space}{compact}
    Let $X$ be a topological space. $X$ is {\color{red} \textbf{compact}} if every open cover of $X$ contains a finite sub-cover for $X$.
\end{dfnbox}
\begin{notebox}
    \begin{remark}
        $Y \subseteq X$ is a compact subspace if and only if for every collection $\mathcal{U}$ of open sets in $Y$ such that $Y \subseteq \bigcup_{U \in \mathcal{U}}U$, there exists a finite sub-collection $\mathcal{U}' \subseteq \mathcal{U}$ such that~$Y \subseteq \bigcup_{U \in \mathcal{U}'}U$.
    \end{remark}
\end{notebox}
Note that previously we have shown that a closed set contains all of its limit points. Therefore, it is intuitive that if a closed set is contained by a compact space, this closed set should be compact as well.
\begin{probox}{Closed Subspaces of Compact Spaces Are Compact}{closedCompact}
    Every closed subspace of a compact space is compact.
    \tcblower
    \begin{proof}
        Let $X$ be a compact topological space and $Y \subseteq X$ be a closed subspace. Let~$\left\{U_{\alpha}\right\}_{\alpha \in \Lambda}$ be an open cover of $Y$ with $U_{\alpha} \subseteq X$ for all $\alpha \in \Lambda$, then $\left\{U_{\alpha}\right\}_{\alpha \in \Lambda} \cup \left\{X \setminus Y\right\}$ is an open cover for $X$ because $Y$ is closed. Since $X$ is compact, we can find a finite sub-cover $\mathcal{U} \subseteq \left\{U_{\alpha}\right\}_{\alpha \in \Lambda} \cup \left\{X \setminus Y\right\}$. Clearly, $\mathcal{U} \setminus \left\{X \setminus Y\right\} \subseteq \left\{U_{\alpha}\right\}_{\alpha \in \Lambda}$ is a finite sub-cover for $Y$. Therefore, $Y$ is compact. 
    \end{proof}
\end{probox}
One may be tempted to think that compact subspaces must be closed. However, this is not true in general. Consider $X$ to be an infinite topological space equipped with the co-finite topology. For any $Y \subseteq X$, let $\mathcal{U}$ be an open cover for $Y$. Take any $U \in \mathcal{U}$, then clearly $Y \setminus U$ is finite. For each $y \in Y \setminus U$, there exists some $U_y \in \mathcal{U}$ such that $y \in U_y$. Therefore, 
\begin{equation*}
    \left\{U\right\} \cup \left\{U_y \colon y \in Y \setminus U\right\}
\end{equation*}
is a finite sub-cover for $Y$. Therefore, every subset of $X$ is compact. However, $Y \subseteq X$ is closed if and only if $Y$ is finite.

To conclude closed-ness from compactness, we require some additional assumptions.
\begin{probox}{Compact Subspaces of Hausdorff Spaces Are Closed}{compactClosed}
    Every compact subspace of a Hausdorff space is closed.
\end{probox}
Intuitively, a continuous map should send a compact set to a compact image because open sets, and hence open covers, are preserved by the map.
\begin{probox}{Continuous Map Induces Compact Image}{ContinuousCompact}
    Let $f \colon X \to Y$ be continuous, if $X$ is compact, then $f\left(X\right)$ is compact.
    \tcblower
    \begin{proof}
        Let $\left\{U_{\alpha}\right\}_{\alpha \in \Lambda}$ be an open cover for $f\left(X\right)$. Note that $\left\{f^{-1}\left(U_{\alpha}\right)\right\}_{\alpha \in \Lambda}$ is an open cover for $X$. Since $X$ is compact, there exists some finite $I \subseteq \Lambda$ such that $\left\{f^{-1}\left(U_i\right)\right\}_{i \in I}$ is a finite sub-cover for $X$. Clearly, $\left\{U_i\right\}_{i \in I}$ is a finite sub-cover for $f\left(X\right)$ and so $f\left(X\right)$ is compact.
    \end{proof}
\end{probox}

\begin{probox}{Sets in Co-finite Topology Are Compact}{compactCofinite}
    Let $X$ be a topological space equipped with the co-finite topology, then for every $U \subseteq X$, $U$ is compact and $U$ is closed if and only if $U$ is finite.
\end{probox}
\begin{lembox}{Tube Lemma}{tube}
    Let $X$ be a topological space and $Y$ be a compact topological space. If $N \subseteq X \times Y$ is an open set that contains $\left\{\left(x_0, y\right) \colon y \in Y \right\}$, then $N$ contains $W \times Y$ for some open $W \subseteq X$ that contains $x_0$.
    \tcblower
    \begin{proof}
        Let $\mathcal{B}$ be the basis generating the product topology. Define 
        \begin{equation*}
            \mathcal{A} \coloneqq \left\{U \times V \in \mathcal{B} \colon \subseteq N, x_0 \in U\right\}.
        \end{equation*}
        Since $N$ is open, then for any $y \in Y$, there exists some $U_y \times V_y \in \mathcal{B}$ such that 
        \begin{equation*}
            \left(x_0, y\right) \in U_y \times V_y \subseteq N.
        \end{equation*}
        Therefore, $\mathcal{A}$ is an open cover for $\left\{x_0\right\} \times Y$. Note that $\left\{x_0\right\} \times Y$ is compact because of the compactness of $Y$. Therefore, there exists a finite sub-cover
        \begin{equation*}
            \left\{U_1 \times V_1, \cdots, U_n \times V_n\right\} \subseteq \mathcal{A}
        \end{equation*}
        of $\left\{x_0\right\} \times Y$ where $\bigcup_{i = 1}^nV_i = Y$. Take 
        \begin{equation*}
            W \coloneqq \bigcap_{i = 1}^nU_i \subseteq X,
        \end{equation*}
        then $W$ is open and $x_0 \in W$. Since $W \subseteq U_i$ for all $i = 1, 2, \cdots, n$, we have 
        \begin{equation*}
            W \times Y = \bigcup_{i = 1}^n\left(W \times V_i\right) \subseteq \bigcup_{i = 1}^n\left(U_i \times V_i\right) \subseteq N.
        \end{equation*}
    \end{proof}
\end{lembox}
Applying the Tube Lemma to Cartesian products yields the following result:
\begin{corbox}{Compactness of Cartesian Product}{compactCartProd}
    If $X$ and $Y$ are compact topological spaces, then $X \times Y$ is compact.
    \tcblower
    \begin{proof}
        Let $\mathcal{A}$ be an open cover of $X \times Y$, then $\mathcal{A}$ is also an open cover of $\left\{x\right\} \times Y$ for any~$x \in X$. Since $Y$ is compact, there exists some finite subset $\mathcal{A}_x \subseteq \mathcal{A}$ such that 
        \begin{equation*}
            \left\{x\right\} \times Y \subseteq \bigcup_{A \in \mathcal{A}_x}A \coloneqq N_x.
        \end{equation*}
        Note that $N_x$ is open in $X \times Y$. By Lemma \ref{lem:tube}, there exists some open subset $W_x \subseteq X$ such that $x \in W_x$ and $W_x \times Y \subseteq N_x$. Clearly, $\left\{W_x \coloneqq x \in X\right\}$ is an open cover for $X$. Since $X$ is compact, this means that there exists $x_1, x_2, \cdots, x_n \in X$ such that 
        \begin{equation*}
            X = \bigcup_{i = 1}^nW_{x_i}.
        \end{equation*}
        Take $\mathcal{A}' \coloneqq \bigcup_{i = 1}^n\mathcal{A}_{x_i} \subseteq \mathcal{A}$ which is finite, then 
        \begin{align*}
            X \times Y & \subseteq \bigcup_{i = 1}^n\left(W_{x_i} \times Y\right) \\
            & \subseteq \bigcup_{i = 1}^nN_{x_i} \\
            & = \bigcup_{i = 1}^n\bigcup_{A \in \mathcal{A}_{x_i}}A \\
            & = \bigcup_{A \in \mathcal{A}'}A.
        \end{align*}
        Therefore, $\mathcal{A}'$ is a finite sub-cover for $X \times Y$ and so $X \times Y$ is compact.
    \end{proof}
\end{corbox}
\begin{dfnbox}{Finite Intersection Property}{finiteIntersect}
    Let $X$ be some set. A collection $\mathcal{G} \subseteq \mathcal{P}\left(X\right)$ has the {\color{red} \textbf{finite intersection property}} if for any finite sub-collection $\mathcal{G}' \subseteq \mathcal{G}$, we have 
    \begin{equation*}
        \bigcap_{G \in \mathcal{G}'}G \neq \varnothing.
    \end{equation*}
\end{dfnbox}
\begin{probox}{Characterisation of Compactness}{compactFiniteIntersect}
    Let $X$ be a topological space, then $X$ is compact if and only if for any collection of closed sets $\mathcal{G} \subseteq \mathcal{P}\left(X\right)$ with the finite intersection property, we have 
    \begin{equation*}
        \bigcap_{G \in \mathcal{G}}G \neq \varnothing.
    \end{equation*}
    \tcblower
    \begin{proof}
        Suppose that for any collection $\mathcal{G} \subseteq \mathcal{P}\left(X\right)$ of closed sets in $X$ with the finite intersection property, we have $\bigcap_{G \in \mathcal{G}} \neq \varnothing$. Let $\mathcal{U}$ be an open cover of $X$ and consider 
        \begin{equation*}
            \mathcal{G} \coloneqq \left\{X \setminus U \colon U \in \mathcal{U}\right\}.
        \end{equation*}
        Note that $\mathcal{G}$ is a collection of closed sets and 
        \begin{equation*}
            \bigcap_{G \in \mathcal{G}}G = \bigcap_{U \in \mathcal{U}}X \setminus U = X \setminus \bigcup_{U \in \mathcal{U}}U = \varnothing.
        \end{equation*}
        This implies that $\mathcal{G}$ does not have the finite intersection property, i.e., there exists open sets $U_1, U_2, \cdots, U_n \in \mathcal{U}$ such that 
        \begin{equation*}
            \bigcap_{i = 1}^nX \setminus U_i = X \setminus \bigcup_{i = 1}^nU_i = \varnothing.
        \end{equation*}
        This means that $\left\{U_1, U_2, \cdots, U_n\right\} \subseteq \mathcal{U}$ is a finite sub-cover for $X$ and so $X$ is compact. Conversely suppose $X$ is compact and let $\mathcal{G}$ be a collection of closed sets in $X$ such that 
        \begin{equation*}
            \bigcap_{G \in \mathcal{G}}G = \varnothing.
        \end{equation*}
        Define 
        \begin{equation*}
            \mathcal{U} \coloneqq \left\{X \setminus G \colon G \in \mathcal{G}\right\},
        \end{equation*}
        then clearly 
        \begin{equation*}
            \bigcup_{U \in \mathcal{U}} = X \setminus \bigcap_{G \in \mathcal{G}}G = X,
        \end{equation*}
        so there exists $G_1, G_2, \cdots, G_n \in \mathcal{G}$ such that 
        \begin{equation*}
            \bigcup_{i = 1}^nX \setminus G_i = X \setminus \bigcap_{i = 1}^nG_i = X.
        \end{equation*}
        Therefore, $\mathcal{G}$ does not have the finite intersection property.
    \end{proof}
\end{probox}
An intuitive example of a collection with finite intersection property is \textbf{nested closed intervals} in Euclidean spaces. This can be generalised as the following corollary:
\begin{corbox}{Intersection of Nested Closed Sets in Compact Spaces}{nestedClosedInCompact}
    Let $X$ be a compact topological space and $\left\{G_i\right\}_{i = 1}^{\infty}$ be a sequence of closed sets in $X$ such that $G_{i + 1} \subseteq G_i$ for all $i \in \N^+$, then 
    \begin{equation*}
        \bigcap_{i = 1}^{\infty}G_i \neq \varnothing.
    \end{equation*}
\end{corbox}
Lastly, we discuss the cardinality of compact spaces. Intuitively, a space is unlikely to be countable if its points are \textbf{dense}. To visualise the density of points, it might be helpful to think of \textbf{sparse points} as points around which there is some \textbf{empty space}.
\begin{dfnbox}{Isolated Point}{isolatePt}
    A point $x$ in a topological space $X$ is {\color{red} \textbf{isolated}} if $\left\{x\right\}$ is open in $X$.
\end{dfnbox}
We can establish an analogous connection between the density of points distributed in a space and the number of isolated points in the space. Naturally, a space contains ``fewer'' points if there are lots of isolated points.
\begin{probox}{Characterisation of Uncountable Topological Spaces}{characteriseUncountable}
    Let $X$ be a non-empty compact Hausdorff space. If $X$ has no isolated point, then $X$ is uncountable.
    \tcblower
    \begin{proof}
        Suppose on contrary that $X$ is countable, then we can write 
        \begin{equation*}
            X \coloneqq \left\{x_i \colon i \in I\right\}
        \end{equation*}
        for some $I \subseteq \N^+$. We consider the following lemma:
        \begin{lembox}{Non-isolated Points in Hausdorff Spaces}{closureNonIso}
            Let $X$ be a Hausdorff space. If $U \subseteq X$ is open and non-empty and $x \in X$ is not isolated, then there exists some open and non-empty subset $V \subseteq U$ with $x \notin \widebar{V}$.
            \tcblower
            \begin{proof}
                Note that $U \setminus \left\{x\right\}$ is open and non-empty because otherwise $\left\{x\right\}$ is open. Fix some $y \in U \setminus \left\{x\right\}$, then since $X$ is Hausdorff, we can find some disjoint open sets $W_{x}, W_{y} \subseteq X$ such that $x \in W_{x}$ and $y \in W_{y}$. Take $V \coloneqq W_y \cap U \subseteq U$, then $V$ is non-empty and open. Note that $X \setminus W_{x}$ is a closed set containing $V$, so $\widebar{V} \subseteq X \setminus W_{x_i}$. This means that $x \notin \widebar{V}$.
            \end{proof}
        \end{lembox}
        By Lemma \ref{lem:closureNonIso}, there exists some open set $V_1 \subseteq X$ such that $x_1 \notin \widebar{V_1}$. For each $i = 2, 3, \cdots$, we can find some open set $V_i \subseteq V_{i - 1}$ such that $x_i \notin \widebar{V_i}$. By Corollary \ref{cor:nestedClosedInCompact}, we know that 
        \begin{equation*}
            \bigcap_{i \in I}\widebar{V_i} \neq \varnothing.
        \end{equation*}
        However, for every $x_i \in X$, there exists some $V_i$ such that $x_i \notin \widebar{V_i}$, which is a contradiction.
    \end{proof}
\end{probox}
\subsection{Limit Point and Sequential Compactness}
\begin{dfnbox}{Limit Point Compactness}{limPtCompact}
    A topological space $X$ is {\color{red} \textbf{limit point compact}} if every infinite subset of $X$ has a limit point in $X$.
\end{dfnbox}
It is possible for a non-compact space to be limit point compact. However, a compact space must be limit point compact.
\begin{probox}{Compactness Implies Limit Point Compactness}{compactToLimPtCompact}
    Any compact topological space is limit point compact.
    \tcblower
    \begin{proof}
        Let $A \subseteq X$ be a subset without any limit point in $X$, then it suffices to prove that $A$ is finite. Notice that $A = \widebar{A}$, so $A$ is closed in $X$. By Proposition \ref{pro:closedCompact}, $A$ is compact. For every $a \in A$, since $a$ is not a limit point, by Definition \ref{dfn:limPt} there exists some open set $U_a \subseteq X$ with $a \in U_a$ but 
        \begin{equation*}
            \left(A \setminus \left\{a\right\}\right) \cap U_a = \varnothing.
        \end{equation*}
        Clearly, this means that $U_a \cap A = \left\{a\right\}$. Therefore, $\left\{a\right\}$ is open in $A$. This means that 
        \begin{equation*}
            \mathcal{A} \coloneqq \bigl\{\left\{a\right\} \colon a \in A\bigr\}
        \end{equation*}
        is an open cover of $A$. Since $A$ is compact, there exists $a_1, a_2, \cdots, a_n \in A$ such that 
        \begin{equation*}
            A \subseteq \bigcup_{i = 1}^n\left\{a_i\right\},
        \end{equation*}
        and so $A$ is finite.
    \end{proof}
\end{probox}
\begin{dfnbox}{Sequential Compactness}{seqCompact}
    A topological space $X$ is {\color{red} \textbf{sequentially compact}} if every sequence in $X$ has a convergent subsequence.
\end{dfnbox}
\begin{probox}{Sequential Compactness Implies Limit Point Compactness}{seqCompactToLimPtCompact}
    Any sequentially compact topological space is limit point compact, but the converse is not true.
\end{probox}
\begin{dfnbox}{Lebesgue Number}{LebesgueNum}
    Let $\mathcal{U}$ be an open cover for a metric space $X$, then $\delta > 0$ is called a {\color{red} \textbf{Lebesgue number}} for $\mathcal{U}$ if for all $S \subseteq X$ with $\mathrm{diam}\left(S\right) < \delta$, there exists some $U \in \mathcal{U}$ such that $S \subseteq U$. 
\end{dfnbox}
The existence of Lebesgue number is not guaranteed in general. However, it can be shown that Lebesgue number always exists for sequentially compact spaces.
\begin{probox}{Sequentially Compact Spaces Guarantee Lebesgue Number}{seqCompactLeb}
    Every open cover of a sequentially compact metric space has a Lebesgue number.
\end{probox}
\begin{dfnbox}{Totally Bounded Space}{totalBound}
    A metric space $X$ is {\color{red} \textbf{totally bounded}} if for all $\epsilon > 0$, there exists a finite cover of $X$ by open balls of radius $\epsilon$.
\end{dfnbox}
\begin{probox}{Sequentially Compactness Implies Total Boundedness}{seqCompactTotalBound}
    Every sequentially compact metrisable topological space is totally bounded.
\end{probox}
A nice thing about Propositions \ref{pro:seqCompactLeb} and \ref{pro:seqCompactTotalBound} is that they apply to any compact space in general, which is a result of the following equivalence:
\begin{probox}{Compactness of Metrisable Spaces}{metricCompact}
    If $X$ is a metrisable topological space, then the followings are equivalent:
    \begin{enumerate}
        \item $X$ is compact;
        \item $X$ is limit point compact;
        \item $X$ is sequentially compact.
    \end{enumerate}
\end{probox} 
\begin{probox}{Compact Spaces Induce Uniform Continuity}{compactUniContinuous}
    Let $\left(X, d_X\right)$ and $\left(Y, d_Y\right)$ be metric spaces and $f \colon X \to Y$ be continuous. If $X$ is compact, then $f$ is uniformly continuous.
\end{probox}
\end{document}