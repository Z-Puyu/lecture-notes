\documentclass[math]{amznotes}
\usepackage[utf8]{inputenc}
\usepackage{amsmath}
\usepackage{amsfonts}
\usepackage{graphicx}
\usepackage{tikz}
\usepackage{etoolbox}

\graphicspath{ {./images/} }
\geometry{
    a4paper,
    headheight = 1.5cm
}

\patchcmd{\chapter}{\thispagestyle{plain}}
{\thispagestyle{fancy}}{}{}

\theoremstyle{remark}
\newtheorem*{claim}{Claim}
\newtheorem*{remark}{Remark}
\newtheorem{case}{Case}

\newcommand{\map}[3]{#1: #2 \rightarrow #3} % Mapping
\newcommand{\image}[2]{#2\left[#1\right]} % Image
\newcommand{\preimage}[2]{#2\left[#1\right]^{-1}} % Pre-image
\newcommand{\eval}[3]{\left. #1\right\rvert_{#2 = #3}} % Evaluated at

\DeclareMathOperator*{\argmax}{argmax}
\DeclareMathOperator*{\argmin}{argmin}

\begin{document}
\fancyhead[L]{
    Set Theory
}
\fancyhead[R]{
    Lecture Notes
}
\tableofcontents

\chapter{Sets and Classes}
\section{Classes}
\textit{Russell's Paradox} states the following:
\begin{genbox}{Russell's Paradox}
    Let $X$ be the set of all sets which do not contain themselves, i.e.,
    \begin{equation*}
        X = \left\{S \colon S \notin S\right\}.
    \end{equation*}
    Now consider $X$. If $X \in X$, it means that $X$ contains itself and should not be a member of $X$, i.e., $X \in X \implies X \notin X$. If $X \notin X$, it means that $X$ does not contain itself and therefore should be a member of $X$, i.e. $X \notin X \implies X \in X$. Hence, we have a paradox and such a set $X$ does not exist.
\end{genbox}
However, in some cases it is still useful to consider the ``set'' of all sets for practical reasons. Therefore, we introduce the notion of a \textit{class} to avoid Russell's Paradox. 
\begin{dfnbox}{Class}{class}
    Let $\phi$ be some formula and $\mathbfit{u}$ be a vector, the collection
    \begin{displaymath}
        \mathbb{C} = \left\{X \colon \phi(X, \mathbfit{u})\right\}
    \end{displaymath}
    is called a {\color{red} \textbf{class}} of all sets satisfying $\phi(X, \mathbfit{u})$, where $\mathbb{C}$ is said to be {\color{red} \textbf{definable}} from $\mathbfit{u}$. Equivalently, we say that
    \begin{displaymath}
        X \in \mathbb{C} \iff \phi(X, \mathbfit{u}).
    \end{displaymath}
    In particular, if $\mathbb{C} = \left\{X \colon \phi(X)\right\}$, i.e., $\phi$ only has one free variable, then we say that $\mathbb{C}$ is {\color{red} \textbf{definable}}.
\end{dfnbox}
\begin{notebox}
    \begin{remark}
        It is easy to see that every set $X$ is a class given by $\left\{x \colon x \in X\right\}$.
    \end{remark}
\end{notebox}
Intuitively, two classes are equal if they contain exactly the same members. We are able to give the following rigorous version of the notion of equality:
\begin{dfnbox}{Equality between Classes}{classEquality}
    Let $\mathbb{C} = \left\{X \colon \phi(X, \mathbfit{u})\right\}$ and $\mathbb{D} = \left\{X \colon \psi(X, \mathbfit{v})\right\}$, we say that $\mathbb{C} = \mathbb{D}$ if for all $X$,
    \begin{displaymath}
        \phi(X, \mathbfit{u}) \iff \psi(X, \mathbfit{u}).
    \end{displaymath}
\end{dfnbox}
There are clearly two types of classes --- the ones which are also sets and the ones which are not. Formally, this is put as follows:
\begin{dfnbox}{Proper Class}{properClass}
    A class $\mathbb{C}$ is said to be a {\color{red} \textbf{proper class}} if $\mathbb{C} \neq X$ for all sets $X$.
\end{dfnbox}
Like sets, we can define subclasses:
\begin{dfnbox}{Subclass}{subclass}
    Let $\mathbb{A}$ and $\mathbb{B}$ be classes. We say that $\mathbb{A}$ is a {\color{red} \textbf{subclass}} of $\mathbb{B}$ if every member of $\mathbb{A}$ is also a member of $\mathbb{B}$, i.e.,
    \begin{displaymath}
        \mathbb{A} \subseteq \mathbb{B} \iff (X \in \mathbb{A} \implies X \in \mathbb{B}).
    \end{displaymath}
\end{dfnbox}
We shall also define the operations applicable to classes:
\begin{dfnbox}{Intersection, Union and Difference}{classOps}
    Let $\mathbb{A}$ and $\mathbb{B}$ be classes. The {\color{red} \textbf{intersection}}, {\color{red} \textbf{union}} and {\color{red} \textbf{difference}} between $\mathbb{A}$ and $\mathbb{B}$ are given by
    \begin{align*}
        \mathbb{A \cap B} & \coloneqq \left\{X \colon X \in \mathbb{A} \wedge X \in\ \mathbb{B}\right\}, \\
        \mathbb{A \cup B} & \coloneqq \left\{X \colon X \in \mathbb{A} \vee X \in\ \mathbb{B}\right\}, \\
        \mathbb{A - B} & \coloneqq \left\{X \colon X \in \mathbb{A} \wedge X \notin\ \mathbb{B}\right\}
    \end{align*}
    respectively.
\end{dfnbox}
Finally, we shall introduce the universal class:
\begin{dfnbox}{Universal Class}{univClass}
    The {\color{red} \textbf{universal class}} is the class of all sets, denoted by
    \begin{displaymath}
        V \coloneqq \left\{X \colon X = X\right\}.
    \end{displaymath}
\end{dfnbox}
\begin{notebox}
    \begin{remark}
        It is easy to prove that the universal class is \textbf{unique}.
    \end{remark}
\end{notebox}

\chapter{Axiomatic Set Theory}
\section{Axioms of Zermelo-Fraenkel (ZF)}
In Na\"\i ve Set Theory, we define a set as ``a collection of mathematical objects which satisfy certain definable properties''. However, such a definition is problematic (e.g. it leads to the Russell's Paradox). Thus, instead of viewing a set as a clearly defined mathematical object, we can think a set as an object entirely defined by a set of axioms to which it complies. In this sense, we avoid paradoxes by making the notion of a set undefined but only specify rigorously the axioms a set must satisfy. The following sections discuss each of the axioms in ZF set theory.
\section{Extensionality}
\begin{axibox}{Extensionality}{extensionAxi}
    Let $X$ and $Y$ be sets, then $X = Y$ if for all $u$, $u \in X$ if and only if $u \in Y$.
\end{axibox}
An immediate result from Axiom \ref{axi:extensionAxi} is that there exists a set $X$ such that $X = X$, i.e. every set equals itself. Moreover, we can also prove the following:
\begin{thmbox}{The Empty Set}{emptySet}
    The set which has no elements is unique.
    \tcblower   
    \begin{proof}
        Let $X$ be a set with no elements. Note that this means that for all $u$, $u \notin X$.
        \\\\
        Let $Y$ be another set. Note that the statement $u \in X \implies u \in Y$ is vacuously true. Suppose that $Y$ has no elements, then similarly for all $u$, the statement~$u \in Y \implies u \in X$ is also vacuously true. 
        \\\\
        Therefore, for all $u$, we have proven that $u \in X$ if and only if $u \in Y$. By Axiom \ref{axi:extensionAxi}, this means that $X = Y$, i.e. the set with no elements is unique.
    \end{proof}
\end{thmbox}
This set with no elements is known as the {\color{red} \textbf{empty set}}, denoted by $\varnothing$.
\section{Pairing}
\begin{axibox}{Pairing}{pairAxi}
    For all $u$ and $v$, there exists a set $X$ such that for all $z$, $z \in X$ if and only if $z = u$ or~$z = v$.
\end{axibox}
\begin{notebox}
    \begin{remark}
        Note that Axiom \ref{axi:pairAxi} essentially says that given any sets $u$ and $v$, there exists a set whose elements are exactly $u$ and $v$.
    \end{remark}
\end{notebox}
This allows us to formally define the notion of a \textit{pair} as follows:
\begin{dfnbox}{Pair}{pair}
    For all $a$, $b$, the {\color{red} \textbf{pair}} $\{a, b\}$ is defined to be the set $C$ such that for all $x$, $x \in C$ if and only if $x = a$ or $x = b$.
\end{dfnbox}
\begin{notebox}
    \begin{remark}
        In particular, we can define the {\color{red} \textbf{singleton}} $\{a\}$ to be the pair $\{a, a\}$.
    \end{remark}
\end{notebox}
Furthermore, given any $a$ and $b$, we can prove by Extensionality that the the pair $\{a, b\}$ is unique:
\begin{thmbox}{Uniqueness of Pairs}{uniquePairs}
    For all $a$, $b$, the pair $\{a, b\}$ is unique.
    \tcblower   
    \begin{proof}
        Let $C \coloneqq \{a, b\}$ and $D \coloneqq \{a, b\}$. Suppose $x \in C$, then $x = a$ or $x = b$, which means $x \in D$. Similarly, suppose $y \in D$, we can prove that $y \in C$. Therefore, for all~$x$, we have $x \in C$ if and only if $x \in D$. By Axiom \ref{axi:extensionAxi}, this means that $C = D$, i.e., the pair $\{a, b\}$ is unique.
    \end{proof}
\end{thmbox}
We can further define the notion of an \textit{ordered pair}:
\begin{dfnbox}{Ordered Pair}{orderedPair}
    For all $a$ and $b$, the {\color{red} \textbf{ordered pair}} $(a, b)$ is defined to be the set $\left\{\{a\}, \{a, b\}\right\}$.
\end{dfnbox}
Again, one can use Extensionality to prove that such an ordered pair is always unique and that $(a, b) = (c, d)$ if and only if $a = c$ and $b = d$. The notions of pair and ordered pair can be extended to ordered and un-ordered $n$-tuples, which will have similar properties as we have proven as above. Recursively, we can write the following definition:
\begin{dfnbox}{Ordered $n$-tuple}{nTuple}
    The {\color{red} \textbf{$n$-tuple}} is defined as 
    \begin{displaymath}
        (a_1, a_2, \cdots, a_n) = ((a_1, a_2, \cdots, a_{n - 1}), a_n).
    \end{displaymath}
\end{dfnbox}
By Extensionality, we can similarly prove that two ordered $n$-tuples $(a_1, a_2, \cdots, a_n)$ and $(b_1, b_2, \cdots, b_n)$ if and only if $a_i = b_i$ for $i = 1, 2, \cdots, n$.
\section{Separtion}
\begin{axibox}{Axiom Schema of Separation}{schemaSeparation}
    If $P$ is a property with parameter $p$, then for all $X$ and $p$ there exists a set 
    \begin{displaymath}
        Y \coloneqq \left\{u \in X \colon P(u, p)\right\}.
    \end{displaymath}
\end{axibox}
The above axiom justifies our set-builder notation
\begin{displaymath}
    \left\{x \colon \varphi(x, \mathbfit{p})\right\},
\end{displaymath}
where $\varphi$ is some formula and $\mathbfit{p}$ is an ordered $n$-tuple of parameters.

Alternatively, we can write Axiom Schema \ref{axi:schemaSeparation} in the following form:

    Let $\mathbb{C} = \left\{u \colon \varphi(u, \mathbfit{p})\right\}$ be a class, then for all sets $X$ there exists a set $Y$ such that $\mathbb{C} \cap X = Y$.

    Consequently, the intersection and the difference between two sets is a set, which can be defined as
    \begin{displaymath}
        X \cap Y \coloneqq \left\{u \in X \colon u \in Y\right\} \quad \textrm{and} \quad X - Y \coloneqq \left\{u \in X \colon u \notin Y\right\}. 
    \end{displaymath}
    Suppose that there exists some set $X$ such that $X = X$, we can use Separtion to define the empty set as
    \begin{displaymath}
        \varnothing \coloneqq \left\{u \colon u \neq u\right\}.
    \end{displaymath}
    We shall define other notions related to Separation Axioms:
    \begin{dfnbox}{Disjoint}{disjoint}
        Two sets $X$ and $Y$ are called {\color{red} \textbf{disjoint}} if $X \cap Y = \varnothing$.
    \end{dfnbox}
    \begin{dfnbox}{Unary Intersection}{unaryIntersect}
        Let $\mathbb{C}$ be a non-empty class of sets, we define the {\color{red} \textbf{unary intersection}} of $\mathbb{C}$ to be
        \begin{displaymath}
            \bigcap \mathbb{C} \coloneqq \left\{u \colon u \in X \textrm{ for all } X \in C\right\}.
        \end{displaymath}
    \end{dfnbox}   
    Note that the unary intersection helps us define the intersection of two sets as 
    \begin{equation*}
        X \cap Y = \bigcap\{X, Y\}.
    \end{equation*} 
\section{Union}
\begin{axibox}{Axiom of Union}{unionAxi}
    For all $X$, there exists a set $Y = \bigcup X$ whose elements are all the elements of all elements of $X$, i.e.
    \begin{displaymath}
        Y \coloneqq \left\{u \in U \colon U \in X\right\}.
    \end{displaymath}
\end{axibox}
\begin{notebox}
    \begin{remark}
        We often call $\bigcup X$ the {\color{red} \textbf{unary union}} of $X$.
    \end{remark}
\end{notebox}
The unary union defines the union of two sets as 
\begin{equation*}
    X \cup Y = \bigcup\{X, Y\}.
\end{equation*}
One can prove that union between sets is \textbf{associative}. In general, we can also see that
\begin{equation*}
    \{a_1, a_2, \cdots, a_n\} = \bigcup_{i = 1}^n \{a_i\}.
\end{equation*}
In addition, we can also define the notion of \textit{symmetric difference}:
\begin{dfnbox}{Symmetric Difference}{symDif}
    The {\color{red} \textbf{symmetric difference}} between two sets $X$ and $Y$ is defined as 
    \begin{displaymath}
        X \triangle Y \coloneqq \left\{u \colon u \in X \cup Y, u \notin X \cap Y\right\} = (X - Y) \cup (Y - X).
    \end{displaymath}
\end{dfnbox}
\section{Power Set}
\begin{axibox}{Axiom of Power Set}{axiPowerSet}
    For all $X$, there exists a set $Y = \mathcal{P}(X)$, known as the {\color{red} \textbf{power set}} of $X$, such that
    \begin{equation*}
        Y \coloneqq \left\{U \colon U \subseteq X\right\}.
    \end{equation*}
\end{axibox}
This allows us to define the notion of the \textit{Cartesian product} (or simply the \textit{product}) of two sets:
\begin{dfnbox}{Cartesian Product}{cartesianProd}
    Let $X$ and $Y$ be sets. The {\color{red} \textbf{Cartesian product}} of $X$ and $Y$ is defined as the set
    \begin{displaymath}
        X \times Y \coloneqq \left\{(x, y) \colon x \in X, y \in Y\right\}.
    \end{displaymath}
\end{dfnbox}
\begin{notebox}
    \begin{remark}
        Note that $X \times Y$ is a set because $X \times Y \subseteq \mathcal{P}(X \cup Y)$.
    \end{remark}
\end{notebox}
The above offers a new way to define $n$-tuples, as we can define Cartesian products of countably many sets recursively.
\begin{dfnbox}{Cartesian Product of Countably Many Sets}
    Let $n \in \N^+$ and let $X$ be a set, we define
    \begin{displaymath}
        X^n \coloneqq \prod_{i = 1}^n X = \left(\prod_{i = 1}^{n - 1} X\right) \times X.
    \end{displaymath}
\end{dfnbox}
\subsection{Relations}
Colloquially, we may want to express the idea that a collection of $n$ objects are related by some rules. Observe that such a \textit{relation} between $n$ objects can be precisely abstracted as an ordered $n$-tuple, which motivates the following definition:
\begin{dfnbox}{Relation}{relation}
    An {\color{red} \textbf{$n$-ary relation}} $R$ is a set of $n$-tuples. We say that $R$ is an $n$-ary relation on $X$ if $R \subseteq X^n$. Conventionally, to say that $x_1, x_2, \cdots, x_n$ are related by the rules defined by $R$, we use the notation $R(x_1, x_2, \cdots, x_n)$. Note that this notation is equivalent to 
    \begin{displaymath}
        (x_1, x_2, \cdots, x_n) \in R.
    \end{displaymath}
\end{dfnbox}
\begin{notebox}
    \begin{remark}
        In the case where $R$ is a binary relation, we can also use the notation $xRy$ to express that $(x, y) \in R$.
    \end{remark}
\end{notebox}
If $R$ is a binary relation, then we define the \textit{domain} of $R$ to be
\begin{displaymath}
    \mathrm{dom}(R) = \left\{u \colon \exists v \textrm{s.t.} (u, v) \in R\right\},
\end{displaymath}
and the \textit{range} of $R$ to be
\begin{displaymath}
    \mathrm{ran}(R) = \left\{v \colon \exists u \textrm{s.t.} (u, v) \in R\right\}.
\end{displaymath}
Note that
\begin{displaymath}
    \mathrm{dom}(R) \subseteq \bigcup \left(\bigcup R\right) \quad \textrm{and} \quad \mathrm{ran}(R) \subseteq \bigcup \left(\bigcup R\right),
\end{displaymath}
so the domain and range of a relation are sets. Additionally, we define the \textit{field} of $R$ to be the set
\begin{displaymath}
    \mathrm{field}(R) = \mathrm{dom}(R) \cup \mathrm{ran}(R).
\end{displaymath}
\subsection{Functions}
Given a binary relation $R$, we can see $R$ as a \textbf{mapping} which corresponds each $u \in \mathrm{dom}(R)$ with some $v \in \mathrm{ran}(R)$. From this, we are able to derive the following definition for a \textit{function}:
\begin{dfnbox}{Function}{func}
    Let $X$ be a set. A binary relation $f$ on $X$ is a {\color{red} \textbf{function}} if $(x, y) \in f$ and $(x, z) \in f$ implies that $y = z$, i.e. for all $x \in X$ there exists a unique $y$ such that $(x, y) \in f$. This unique $y$ is called the {\color{red} \textbf{value}} of $f$ at $x$. We may use the notations
    \begin{displaymath}
        y = f(x) \quad \textrm{or} \quad f \colon x \mapsto y
    \end{displaymath}
    to express that $(x, y) \in f$.
\end{dfnbox}
\begin{notebox}
    \begin{remark}
        If $\mathrm{dom}(f) = X^n$, we also say that $f$ is an {\color{red} \textbf{$n$-nary function}} on $X$.
    \end{remark}
\end{notebox}
We denote a function $f$ from $X$ to $Y$ by
\begin{displaymath}
    f \colon X \to Y,
\end{displaymath}
where $\mathrm{dom}(f) = X$ and $\mathrm{ran}(f) \subseteq Y$. The set of all functions from $X$ to $Y$ is denoted as $Y^X$, which is a set because
\begin{displaymath}
    Y^X \subseteq \mathcal{P}(X \times Y).
\end{displaymath}
If $\mathrm{ran}(f) = Y$, we say that $f$ is \textit{onto} $Y$ or that $f$ is \textit{surjective}. A function $f$ is \textit{one-to-one} or \textit{injective} if
\begin{displaymath}
    f(x) = f(y) \implies x = y.
\end{displaymath}
Additionally, we may call the function $f \colon X^n \to X$ an \textit{$n$-nary operation} on $X$.

We may also define new functions from some existing function(s).
\begin{dfnbox}{Restriction}{Restriction}
    Let $f$ be a function. The {\color{red} \textbf{restriction}} of $f$ to a set $X$ is defined to be the function  
    \begin{displaymath}
        \restrict{f}{X} \coloneqq \left\{(x, y) \in f \colon x \in X\right\}.
    \end{displaymath}
\end{dfnbox}
\begin{dfnbox}{Extension}{extension}
    Let $f$, $g$ be functions. $g$ is called an {\color{red} \textbf{extension}} of $f$ if $f \subseteq g$, i.e.,
    \begin{displaymath}
        \mathrm{dom}(f) \subseteq \mathrm{dom}(g) \quad \textrm{and} \quad g(x) = f(x) \quad \textrm{for all } x \in \mathrm{dom}(f).
    \end{displaymath}
\end{dfnbox}
\begin{dfnbox}{Composition}{composition}
    Let $f$ and $g$ be functions such that $\mathrm{ran}(g) \subseteq \mathrm{dom}(f)$. The {\color{red} \textbf{composition}} of $f$ and $g$ is the function denoted by $f \circ g$ with $\mathrm{dom}(f \circ g) = \mathrm{dom}(g)$ such that
    \begin{displaymath}
        (f \circ g)(x) = f\left(g(x)\right) \quad \textrm{for all } x \in \mathrm{dom}(g).
    \end{displaymath}
\end{dfnbox}
Note that a function provides a mapping from one set to another set, and so we can define the notion of an \textit{image}.
\begin{dfnbox}{Image and Inverse Image}{image}
    Let $f$ be a function and $X$ be a set. The {\color{red} \textbf{image}} of $X$ by $f$ is the set
    \begin{displaymath}
        \left\{y \colon \exists x \in X \textrm{ s.t. } y = f(x)\right\},
    \end{displaymath}
    denoted by $f[X]$. The {\color{red} \textbf{inverse image}} of $X$ by $f$ is the set
    \begin{displaymath}
        \left\{x \colon f(x) \in X\right\},
    \end{displaymath}
    denoted by $f^{-1}[X]$.
\end{dfnbox}
\begin{notebox}
    \begin{remark}
        Trivially, if $X \cap \mathrm{dom}(f) = \varnothing$, then $f[X] = \varnothing$.
    \end{remark}
\end{notebox}
For injections, we can also define their \textit{inverses}.
\begin{dfnbox}{Inverse}{invFunc}
    Let $f$ be an injective function, then we denote the {\color{red} \textbf{inverse}} of $f$ by $f^{-1}$, which is defined by
    \begin{displaymath}
        f^{-1}(x) = y \quad \textrm{if and only if } x = f(y). 
    \end{displaymath}
\end{dfnbox}
The above definitions for functions can be applied similarly with respect to classes.
\begin{axibox}{Axiom of Infinity}{axiInf}
    There exists an infinite set.
\end{axibox}
\begin{axibox}{Axiom Schema of Replacement}{schemaReplacement}
    If a class $F$ is a function, then for all $X$ there exists a set $Y = F(X) = \left\{F(x) \colon x \in X\right\}$.
\end{axibox}
\begin{axibox}{Axiom of Regularity}{axiReg}
    For every non-empty set $X$, there exists some $Y \in X$ such that $Y \cap X = \varnothing$.
\end{axibox}
\begin{notebox}
    \begin{remark}
        Axiom \ref{axi:axiReg} is sometimes known as the {\color{red} \textbf{Axiom of Foundation}}. A direct result from it is that for all sets $X$, there exists some $x \in X$ such that $x \not\subseteq X$.
    \end{remark}
\end{notebox}
Furthermore, we can use Axiom \ref{axi:axiReg} to prove the following seemingly trivial result:
\begin{thmbox}{}{}
    There is no set $A$ such that $A \in A$.
    \tcblower   
    \begin{proof}
        If $A = \varnothing$, it is immediate that $A \notin A$ by definition.
        \\\\
        Suppose that there exists a non-empty set $A$ such that $A \in A$. Note that $A \in \{A\}$, so 
        \begin{equation*}
            A \cap \{A\} = A.
        \end{equation*}
        However, by Axiom \ref{axi:axiReg}, since $A$ is the only member of $\{A\}$, we have 
        \begin{equation*}
            A \cap \{A\} = \varnothing,
        \end{equation*}
        which is a contradiction. Therefore, there exists no set $A$ such that $A \in A$.
    \end{proof}
\end{thmbox}
Additionally, we also introduce the Axiom of Choice:
\begin{axibox}{Axiom of Choice}{axiChoice}
    For every $X$ with $\varnothing \notin X$, there exists a {\color{red} \textbf{choice function}}
    \begin{displaymath}
        f \colon X \to \bigcup X
    \end{displaymath}
    such that for all $S \in X$, we have $f(S) \in S$.
\end{axibox}
\begin{notebox}
    \begin{remark}
        Essentially, the choice function maps every set which is a member of some family of sets to one and only one element in that set.
    \end{remark}
\end{notebox}
\end{document}