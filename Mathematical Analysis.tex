\documentclass[math]{amznotes}
\usepackage[utf8]{inputenc}
\usepackage{amsmath}
\usepackage{amsfonts}
\usepackage{graphicx}
\usepackage{tikz}
\usepackage{etoolbox}

\graphicspath{ {./images/} }
\geometry{
    a4paper,
    headheight = 1.5cm
}

\patchcmd{\chapter}{\thispagestyle{plain}}
{\thispagestyle{fancy}}{}{}

\theoremstyle{remark}
\newtheorem*{claim}{Claim}
\newtheorem*{remark}{Remark}
\newtheorem{case}{Case}

\newcommand{\map}[3]{#1: #2 \rightarrow #3} % Mapping
\newcommand{\image}[2]{#2\left[#1\right]} % Image
\newcommand{\preimage}[2]{#2\left[#1\right]^{-1}} % Pre-image
\newcommand{\eval}[3]{\left. #1\right\rvert_{#2 = #3}} % Evaluated at

\DeclareMathOperator*{\argmax}{argmax}
\DeclareMathOperator*{\argmin}{argmin}

\begin{document}
\fancyhead[L]{
    Mathematical Analysis I
}
\fancyhead[R]{
    Lecture Notes
}
\tableofcontents

\chapter{The Real Numbers}
\section{Fields}
\begin{dfnbox}{Field}{field}
    A set $F$ with two binary operations, namely addition and multiplication, is called a {\color{red} \textbf{field}} if it satisfies the following axioms:
    \begin{enumerate}
        \item $\forall a, b \in F$, $a +_F b = b +_F a$.
        \item $\forall a, b, c \in F$, $(a +_F b) +_F c = a +_F (b +_F c)$.
        \item $\exists 0_F \in F$ such that $\forall a \in F$, $0_F +_F a = a +_F 0_F = a$.
        \item $\forall a \in F$, $\exists a' \in F$ such that $a +_F a' = 0_F$.
        \item $\forall a, b \in F$, $a \cdot_F b = b \cdot a$.
        \item $\forall a, b, c \in F$, $(a \cdot_F b) \cdot c = a \cdot_F (b \cdot_F c)$.
        \item $\forall a, b, c \in F$, $a \cdot_F (b +_F c) = a \cdot_F b +_F a \cdot_F b$ and $(a +_F b) \cdot_F c = a \cdot_F c +_F b \cdot_F c$.
        \item $\exists 1_F \in F$ such that $\forall a \in F$, $1_F \cdot_F a = a \cdot_F 1_F = a$.
        \item $\forall a \in F$, $\exists a' \in F$ such that $a \cdot_F a' = 1_F$.
    \end{enumerate}
\end{dfnbox}
If we denote addition by ``$+_F$'' and multiplication by ``$\cdot_F$'' or ``$\times_F$'', then we can denote the field over $F$ by $(F, +_F, \cdot_F)$ or $(F, +_F, \times_F)$.

Among the commonly used number sets, one may check that $\R$, $\Q$ and $\C$ are fields, while $\N$ and $\Z$ are not.
\subsection{Ordered Fields}
\begin{dfnbox}{Total Order}{totalOrder}
    A {\color{red} \textbf{total order}} on a set $X$ is a binary relation $\leq$ over $X$ such that for all $a, b, c \in X$:
    \begin{enumerate}
        \item $a \leq a$ (reflexive).
        \item $a \leq b$ and $b \leq c$ implies $a \leq c$ (transitive).
        \item $a \leq b$ and $b \leq a$ implies $a = b$ (antisymmetric).
        \item either $a \leq b$ or $b \leq a$ (strongly connected).
    \end{enumerate}
\end{dfnbox}
\begin{dfnbox}{Strict Total Order}{strictTotalOrder}
    A {\color{red} \textbf{strict total order}} on a set $X$ is a binary relation $<$ over $X$ such that for all $a, b, c \in X$:
    \begin{enumerate}
        \item $a \not < a$ (irreflexive).
        \item $a < b$ implies $b < a$ (asymmetric).
        \item $a < b$ and $b < c$ implies $a < c$ (transitive).
        \item if $a \neq b$, then either $a < b$ or $b < a$ (connected).
    \end{enumerate}
\end{dfnbox}
It is easy to see that the real numbers form the ordered fields $(\R, +, \times, \leq)$ and $(\R, +, \times, <)$. Note that this means $\R$ satisfies trichotomy. If we choose any $x \in \R$, then exactly one of~$x = 0$, $x > 0$ and $x < 0$ is true. Therefore, we can define that if $x \in \R$ and $x > 0$, then $x$ is said to be positive. This leads to the following axiomatic results:
\begin{enumerate}
    \item If $a$ and $b$ are both positive, then $a + b$ is positive;
    \item If $a$ and $b$ are both positive, then $ab$ is positive;
    \item For any $a \in \R$, either $a = 0$, $a$ is positive, or $-a$ is positive.
\end{enumerate}
Note that $a < b$ if and only if $b - a$ is positive. So the trichotomy of $\R$ guarantees that for any $a, b \in \R$, either $a = b$, $a < b$ or $b < a$ (i.e., $a > b$).

\section{Properties of $\R$}
We can derive a few obvious minor results based on the field properties of $\R$:
\begin{enumerate}
    \item If $a, b \in \R$, then $-ab + ab = 0$;
    \item For all $a \in \R$ with $a \neq 0$, $a^2 > 0$;
    \item If $a \in \R$ is such that $0 \leq a < \epsilon$ for all $\epsilon \in \R^+$, then $a = 0$;
    \item If $a < b$, then $a + c < b + c$ for all $c \in \R$.
    \item If $a < b$, then $ac < bc$ for all $c \in \R^+$ and $ac > bc$ for all $c \in \R^-$.
    \item For all $a \in \R$, $a^2 > 0$.
\end{enumerate}
We may consider the following interesting proposition:
\begin{probox}{}{infSmall}
    If $a \in \R$ is such that $0 \leq a < \epsilon$ for all $\epsilon \in \R^+$, then $a = 0$.
    \tcblower
    \begin{proof}
        Suppose on contrary that $a > 0$, then we can take $\epsilon_0 = \frac{a}{2}$. Note that $\epsilon_0 \in \R^+$ but $\epsilon_0 < a$, which is a contradiction. So $a = 0$.
    \end{proof}
\end{probox}
The above essentially asserts that \textbf{a non-negative real number is strictly less than any positive real number if and only if it is $0$}.

The properties of $\R$ also enables us to manipulate inequalities based on the following trivial results:
\begin{enumerate}
    \item If $ab > 0$, then $a$ and $b$ are either both positive or both negative;
    \item If $ab < 0$, then exactly one of them is positive and exactly one of them is negative.
\end{enumerate}
We shall introduce a few well-known inequalities.
\begin{thmbox}{Bernoulli's Inequality}{burnoulliIneq}
    If $x > -1$, then $(1 + x)^n \geq 1 + nx$ for all $n \in \N$.
    \tcblower
    \begin{proof}
        The case where $n = 0$ is trivial.
        \\\\
        Suppose that $(1 + x)^k \geq 1 + kx$ for some $k \in \N$, consider
        \begin{align*}
            (1 + x)^{k + 1} & = (1 + x)(1 + x)^k \\
            & \geq (1 + x)(1 + kx) \\
            & = 1 + (k + 1)x + kx^2 \\
            & \geq 1 + (k + 1)x.
        \end{align*}
        Therefore, $(1 + x)^n \geq 1 + nx$ for all $n \in \N$.
    \end{proof}
\end{thmbox}
\begin{thmbox}{AM-GM-HM Inequality}{AM-GM-HM}
    Let $n \in \N^+$ and let $a_1, a_2, \cdots, a_n$ be positive real numbers, then
    \begin{equation*}
        \frac{n}{\sum_{i = 1}^{n}\frac{1}{a_i}} \leq \left(\prod_{i = 1}^{n}a_i\right)^{\frac{1}{n}} \leq \frac{\sum_{i = 1}^{n}a_i}{n}.
    \end{equation*}
\end{thmbox}
\subsection{Absolute Value}
Given any real number $x$, intuitively we sense that $x$ possesses a certain ``distance'' from $0$. This distance can be formalised as follows:
\begin{dfnbox}{Absolute Value}{abs}
    Let $x \in \R$, the {\color{red} \textbf{absolute value}} of $x$ is defined as
    \begin{displaymath}
        \abs{x} = \begin{cases}
            x \quad & \textrm{if } x > 0 \\
            0 \quad & \textrm{if } x = 0 \\
            -x \quad & \textrm{if } x < 0
        \end{cases}.
    \end{displaymath}
\end{dfnbox}
We have some trivial properties about the absolute value:
\begin{enumerate}
    \item For all $a, b \in \R$, $\abs{ab} = \abs{a}\abs{b}$;
    \item For all $a \in \R$, $\abs{a}^2 = a^2$;
    \item If $c \geq 0$, then $\abs{a} \leq c$ if and only if $-c \leq a \leq c$ for all $a \in \R$;
    \item For all $a \in \R$, $-\abs{a} \leq a \leq \abs{a}$.
\end{enumerate}
Using these basic properties, we can prove the following results:
\begin{thmbox}{Triangle Inequality}{triIneq}
    For all $a, b \in \R$, $\abs{a + b} \leq \abs{a} + \abs{b}$.
\end{thmbox}
\begin{corbox}{Extended Triangle Inequality}{triIneqEx}
    For all $a, b \in \R$, $\abs{\abs{a} - \abs{b}} \leq \abs{a - b}$ and $\abs{a - b} \leq \abs{a} + \abs{b}$.
\end{corbox}
\begin{corbox}{Generalised Triangle Inequality}{triIneqGen}
    For all $a_1, a_2, \cdots, a_n \in \R$, 
    \begin{equation*}
        \abs{\sum_{i = 1}^{n}a_i} \leq \sum_{i = 1}^{n}\abs{a_i}.
    \end{equation*}
\end{corbox}
Analogously, if $\abs{x}$ represents the ``distance'' between $x$ and $0$, then by a simple translation we can see that $\abs{x - a}$ represents the ``distance'' between $x$ and $a$. Thus, we can have the following definition:
\begin{dfnbox}{Neighbourhood}{neighbourhood}
    Let $a \in \R$ and $\epsilon \in \R^+$. The {\color{red} \textbf{$\epsilon$-neighbourhood}} of $a$ is defined to be the set
    \begin{displaymath}
        V_\epsilon(a) \coloneqq \left\{x \in \R \colon \abs{x - a} < \epsilon\right\}.
    \end{displaymath}
\end{dfnbox}
Note that $x \in V_\epsilon(a)$ if and only if $-\epsilon < x - a < \epsilon$ or $a - \epsilon < x < a + \epsilon$. Which leads to the following interesting result:
\begin{probox}{}{xIsA}
    For any $a \in \R$, if $x \in V_\epsilon(a)$ for all $\epsilon \in \R^+$, then $x = a$.
    \tcblower
    \begin{proof}
        Note that this essentially means that $\abs{x - a} < \epsilon$ for all $\epsilon \in \R^+$. By Proposition \ref{pro:infSmall}, we have $\abs{x - a} = 0$ and therefore $x = a$.
    \end{proof}
\end{probox}

\subsection{The Completeness Property of $\R$}
Intuitively, there are no ``gaps'' among the real numbers, i.e., if you take any two real numbers, between them there is nothing else than other real numbers. Therefore, we say that $\R$ is \textit{complete}. This is in contrast with $\Q$ where there are gaps in between any two rational numbers (because there always exists some irrational numbers in between).

In this section, we probe into how the completeness of $\R$ can be established, and how the real numbers themselves can be constructed. To do that, we first establish the notion of \textit{boundedness}.

\begin{dfnbox}{Boundedness}{bound}
    Let $S \subseteq \R$. We say that $S$ is:
    \begin{itemize}
        \item {\color{red} \textbf{bounded above}} if there exists some $u \in R$ (known as the {\color{red} \textbf{upper bound}} of $S$) such that $u \geq s$ for all $s \in S$; 
        \item {\color{red} \textbf{bounded below}} if there exists some $v \in R$ (known as the {\color{red} \textbf{lower bound}} of $S$) such that $v \leq s$ for all $s \in S$; 
        \item {\color{red} \textbf{bounded}} if $S$ has both an upper bound and a lower bound;
        \item {\color{red} \textbf{unbounded}} either if $S$ has no upper bound or if $S$ has no lower bound;     
    \end{itemize}
\end{dfnbox}
\begin{notebox}
    \begin{remark}
        Note that $S$ is bounded if and only if there is some $M \geq 0$ such that $\abs{s} \leq M$ for all $s \in S$.
    \end{remark}
\end{notebox}

\chapter{Sequences and Series}
\section{Sequences}
\begin{dfnbox}{Sequence}{seq}
    A {\color{red} \textbf{sequence}} in $\R$ is a real-valued function $X \colon \N^+ \to \R$. Where $X(n)$ is called the $n$-th {\color{red} \textbf{term}} of the sequence.
\end{dfnbox}
\begin{dfnbox}{Convergence of Sequences}{seqConverge}
    A sequence $(x_n)$ in $\R$ is said to be {\color{red} \textbf{convergent}} to $x$ if for all $\epsilon > 0$, there exists some $N \in \N$ such that whenever $n > N$, $\abs{x_n - x} < \epsilon$. $x$ is known as the {\color{red} \textbf{limit}} of $(x_n)$, denoted as
    \begin{equation*}
        \lim_{n \to \infty}x_n = x.
    \end{equation*}
\end{dfnbox}
\subsection{Subsequences}
\begin{dfnbox}{Peak Point}{peakPt}
    Let $(x_n)$ be a sequence in $\R$, $x_m$ is called a {\color{red} \textbf{peak}} if for all $n \in \N$ with $n > m$, $x_m \geq x_n$.
\end{dfnbox}
\begin{thmbox}{Existence of Monotone Subsequences}{existMonoSubseq}
    Every infinite sequence has an infinite monotone subsequence.
    \tcblower
    \begin{proof}
        Let $(x_n)$ be any sequence in $\R$. We consider the following cases:
        \\\\
        \textit{Case 1.} $(x_n)$ has infinitely many peak points, so there exists infinitely many $m_1, m_2, \cdots \in \N$ such that $m_j > m_i$ whenever $j > i$. Therefore, the subsequence $(x_{m_n})$ is a monotone decreasing sequence.
        \\\\
        \textit{Case 2.} $(x_n)$ has finitely many peak points, so there exists $m_1, m_2, \cdots, m_k\in \N$.
    \end{proof}
\end{thmbox}
\begin{thmbox}{Bolzano-Weierstrass Theorem}{bolzanoWeierstrass}
    Every bounded sequence has a convergent subsequence.
\end{thmbox}
\begin{dfnbox}{Cauchy Sequence}{cauchySeq}
    A sequence $(x_n)$ is said to be a {\color{red} \textbf{Cauchy sequence}} if for every $\epsilon > 0$, there exists some $H \in \N$ such that for all $n, m \in \N$ with $n, m \geq H$, $\abs{x_n - x_m} < \epsilon$.
\end{dfnbox}

\begin{thmbox}{Cauchy Convergence Criterion}{cauchyConvergeCri}
    A sequence in $\R$ is convergent if and only if it is a Cauchy sequence.
    \tcblower
    \begin{proof}
        Let $(x_n)$ be a sequence in $\R$. Suppose that $(x_n)$ converges to $x$, then for all $\epsilon > 0$, there exists some $N \in \N$ such that whenever $n > N$, $\abs{x_n - x} < \frac{\epsilon}{2}$. Therefore, for all $m, n > N$, we have
        \begin{align*}
            \abs{x_m - x_n} & = \abs{x_m - x - x_n + x} \\
            & \leq \abs{x_m - x} + \abs{x_n - x} \\
            & = \frac{\epsilon}{2} + \frac{\epsilon}{2} \\
            & = \epsilon,
        \end{align*}
        and so $(x_n)$ is a Cauchy sequence.
        \\\\
        Suppose conversely that $(x_n)$ is a Cauchy sequence on $\R$. We consider the following lemma:
        \begin{lembox}{Boundedness of Cauchy Sequences}{cauchySeqBound}
            A Cauchy sequence in $\R$ is bounded.
            \tcblower
            \begin{proof}
                Let $(x_n)$ be a Cauchy sequence, then by Definition \ref{dfn:cauchySeq} there is some $H \in \N$ such that for all natural numbers $n \geq H$, $\abs{x_n - x_H} < 1$. By Corollary \ref{cor:triIneqEx}, we have
                \begin{equation*}
                    \abs{\abs{x_n} - \abs{x_H}} \leq \abs{x_n - x_H} < 1,
                \end{equation*}
                and so $\abs{x_n} < \abs{x_H} + 1$. Take 
                \begin{equation*}
                    m = \max\left\{\abs{x_1}, \abs{x_2}, \cdots, \abs{x_H}, \abs{x_H} + 1\right\},
                \end{equation*}
                then $\abs{x_n} < m$ for all $n \in \N^+$.
            \end{proof}
        \end{lembox}
        Therefore, by Theorem \ref{thm:bolzanoWeierstrass} there exists a subsequence $(x_{m_n})$ which converges to some $x \in \R$. Thus there exists some $M \in \N$ such that whenever $m_{n} > M$, $\abs{x_{m_n} - x} < \frac{\epsilon}{2}$ for all $\epsilon > 0$. By Definition \ref{dfn:cauchySeq}, there exists some $N \in \N$ such that $\abs{x_n - x_{m_n}} < \frac{\epsilon}{2}$ for all $\epsilon > 0$ and for all $n, m_n > N$. Take $K = \max\{M, N\}$, then whenever $n > K$, there is some $m_n > K$ such that
        \begin{align*}
            \abs{x_n - x} & = \abs{x_n - x_{m_n} + x_{m_n} - x} \\
            & \leq \abs{x_n - x_{m_n}} + \abs{x_{m_n} - x} \\
            & = \frac{\epsilon}{2} + \frac{\epsilon}{2} \\
            & = \epsilon.
        \end{align*}
        Therefore, $\lim_{n \to \infty}x_n = x$.
    \end{proof}
    
\end{thmbox}
\end{document}