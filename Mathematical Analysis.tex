\documentclass[math]{amznotes}
\usepackage[utf8]{inputenc}
\usepackage{amsmath}
\usepackage{amsfonts}
\usepackage{graphicx}
\usepackage{tikz}
\usepackage{etoolbox}

\graphicspath{ {./images/} }
\geometry{
    a4paper,
    headheight = 1.5cm
}

\patchcmd{\chapter}{\thispagestyle{plain}}
{\thispagestyle{fancy}}{}{}

\theoremstyle{remark}
\newtheorem*{claim}{Claim}
\newtheorem*{remark}{Remark}
\newtheorem{case}{Case}

\newcommand{\map}[3]{#1: #2 \rightarrow #3} % Mapping
\newcommand{\image}[2]{#2\left[#1\right]} % Image
\newcommand{\preimage}[2]{#2\left[#1\right]^{-1}} % Pre-image
\newcommand{\eval}[3]{\left. #1\right\rvert_{#2 = #3}} % Evaluated at

\DeclareMathOperator*{\argmax}{argmax}
\DeclareMathOperator*{\argmin}{argmin}

\begin{document}
\fancyhead[L]{
    Mathematical Analysis I
}
\fancyhead[R]{
    Lecture Notes
}
\tableofcontents

\chapter{The Real Numbers}
\section{Fields}
\begin{dfnbox}{Field}{field}
    A set $F$ with two binary operations, namely addition and multiplication, is called a {\color{red} \textbf{field}} if it satisfies the following axioms:
    \begin{enumerate}
        \item $\forall a, b \in F$, $a + b = b + a$.
        \item $\forall a, b, c \in F$, $(a + b) + c = a + (b + c)$.
        \item $\exists 0_F \in F$ such that $\forall a \in F$, $0_F + a = a + 0_F = a$.
        \item $\forall a \in F$, $\exists a' \in F$ such that $a + a' = 0_F$.
        \item $\forall a, b \in F$, $a \cdot b = b \cdot a$.
        \item $\forall a, b, c \in F$, $(a \cdot b) \cdot c = a \cdot (b \cdot c)$.
        \item $\forall a, b, c \in F$, $a \cdot (b + c) = a \cdot b + a \cdot b$ and $(a + b) \cdot c = a \cdot c + b \cdot c$.
        \item $\exists 1_F \in F$ such that $\forall a \in F$, $1_F \cdot a = a \cdot 1_F = a$.
        \item $\forall a \in F$, $\exists a' \in F$ such that $a \cdot a' = 1_F$.
    \end{enumerate}
\end{dfnbox}
If we denote addition by ``$+_F$'' and multiplication by ``$\cdot_F$'' or ``$\times_F$'', then we can denote the field over $F$ by $(F, +_F, \cdot_F)$ or $(F, +_F, \times_F)$.
\subsection{Ordered Fields}
\begin{dfnbox}{Total Order}{totalOrder}
    A {\color{red} \textbf{total order}} on a set $X$ is a binary relation $R$ over $X$ such that for all $a, b, c \in X$:
    \begin{enumerate}
        \item $aRa$.
        \item $aRb$ and $bRc$ implies $aRc$.
        \item $aRb$ and $bRa$ implies $a = b$.
        \item $aRb$ or $bRa$.
    \end{enumerate}
\end{dfnbox}
\begin{dfnbox}{Strict Total Order}{strictTotalOrder}
    A {\color{red} \textbf{strict total order}} on a set $X$ is a binary relation $R$ over $X$ such that for all $a, b, c \in X$:
    \begin{enumerate}
        \item $(a, a) \notin R$.
        \item $aRb$ implies $(b, a) \notin R$.
        \item $aRb$ and $bRc$ implies $aRc$.
        \item trichotomy is satisfied.
    \end{enumerate}
\end{dfnbox}
Define a field $(\R, +, \times)$ together with the total order $\leq$ (or the strict total order $<$), then clearly this field is an ordered field, which is known as the real numbers.
\begin{thmbox}{}{}
    If $a, b \in \R$, then $-ab + ab = 0$. 
\end{thmbox}
\begin{thmbox}{}{}
    For all $a \in \R$ with $a \neq 0$, $a^2 > 0$.
\end{thmbox}
\begin{thmbox}{}{}
    If $a \in \R$ is such that $0 \leq a < \epsilon$ for all $\epsilon \in \R^+$, then $a = 0$.
\end{thmbox}
\begin{thmbox}{Bernoulli's Inequality}{burnoulliIneq}
    If $x > -1$, then $(1 + x)^n \geq 1 + nx$ for all $n \in \N$.
\end{thmbox}
\end{document}